\documentclass{article}
\usepackage[space,fancyhdr,fntef]{ctexcap}
\usepackage[namelimits,sumlimits,nointlimits]{amsmath}
\usepackage[bottom=25mm,top=25mm,left=25mm,right=15mm,centering]{geometry}
\usepackage{xcolor}
\usepackage{arydshln}%234页,虚线表格宏包
\usepackage{mathdots}%反对角省略号
\pagestyle{fancy} \fancyhf{}
\fancyhead[OL]{~~~班序号:\hfill 学院:\hfill 学号:\hfill 姓名:王松年~~~ \thepage}
%\usepackage{parskip}
%\usepackage{indentfirst}
\usepackage{graphicx}%插图宏包,参见手册318页
\begin{document}

\newcounter{num} \renewcommand{\thenum}{\arabic{num}.} \newcommand{\num}{\refstepcounter{num}\text{\thenum}}

\hphantom{~~}\hfill {\zihao{3}\heiti 期中试卷剩余部分习题} \hfill\hphantom{~~}

\hphantom{~~}\hfill {\zihao{4}\heiti 群文件《期中$\&$期末试题》} \hfill\hphantom{~~}

\num 期中2016-2017 一4.

设$f(x)=ax^{2}+bx+c$,$A$为$n$阶方阵,定义$f(A)=aA^{2}+bA+cI$,如果$
A=
\begin{bmatrix}
  1 & 0 & 0 & 0 \\
  0 & 1& 0& 0 \\
  2 & 0 & 1& 0\\
   0 & 0 &0 &1
\end{bmatrix},f(x)=x^{2}-x-1,
$则$f(A)=$\underline{\hphantom{~~~~~~~~~~}}。\\

\num 期中2017-2018 二1.判断是否成立并给出理由。

设$A,B$为同阶对称方阵,则$AB$一定是对称矩阵;\\

\num 期中2017-2018 二4.

设2阶矩阵$A=
\begin{bmatrix}
  a & b \\
  c & d
\end{bmatrix}
$,若$A$与所有的2阶矩阵均可以交换,则$a=d,b=c=0$。\\

\num 期中2018-2019 二2.

设$A$是$n$阶实对称矩阵,如果$A^{2}=0$。证明$A=0$.并举例说明,如果$A$不是实对称矩阵,上述命题不正确。\\

\end{document}  