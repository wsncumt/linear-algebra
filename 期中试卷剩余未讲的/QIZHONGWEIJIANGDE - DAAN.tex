\documentclass{article}
\usepackage[space,fancyhdr,fntef]{ctexcap}
\usepackage[namelimits,sumlimits,nointlimits]{amsmath}
\usepackage[bottom=25mm,top=25mm,left=25mm,right=15mm,centering]{geometry}
\usepackage{xcolor}
\usepackage{arydshln}%234页,虚线表格宏包
\usepackage{mathdots}%反对角省略号
\pagestyle{fancy} \fancyhf{}
\fancyhead[OL]{~~~班序号:\hfill 学院:\hfill 学号:\hfill 姓名:王松年~~~ \thepage}
%\usepackage{parskip}
%\usepackage{indentfirst}
\usepackage{graphicx}%插图宏包,参见手册318页
\begin{document}
\newenvironment{jie}{\kaishu\zihao{-5}\color{blue}{\noindent\em 解:}\par}{\hfill $\diamondsuit$\par}

\newenvironment{zhengming}{\kaishu\zihao{-5}\color{blue}{\noindent\em 证明:}\par}{\hfill $\diamondsuit$\par}
\newcounter{num} \renewcommand{\thenum}{\arabic{num}.} \newcommand{\num}{\refstepcounter{num}\text{\thenum}}

\hphantom{~~}\hfill {\zihao{3}\heiti 期中试卷剩余部分习题} \hfill\hphantom{~~}

\hphantom{~~}\hfill {\zihao{4}\heiti 群文件《期中$\&$期末试题》} \hfill\hphantom{~~}

\num 期中2016-2017 一4.

设$f(x)=ax^{2}+bx+c$,$A$为$n$阶方阵,定义$f(A)=aA^{2}+bA+cI$,如果$
A=
\begin{bmatrix}
  1 & 0 & 0 & 0 \\
  0 & 1& 0& 0 \\
  2 & 0 & 1& 0\\
   0 & 0 &0 &1
\end{bmatrix},f(x)=x^{2}-x-1,
$则$f(A)=$\underline{\hphantom{~~~~~~~~~~}}。

\begin{jie}
由题得:
\begin{equation*}
A=E+\begin{bmatrix}
  0 & 0 & 0 & 0 \\
  0 & 0& 0& 0 \\
  2 & 0 & 0& 0\\
   0 & 0 &0 &0
\end{bmatrix}=E+B
\end{equation*}
可以看出:$B^2=0$.所以$A^2=(E+B)^2=E+B^2+2EB=E+2B$.

所以$f(A)=A^2-A-E=E+2B-E-B-E=B-E=
\begin{bmatrix}
  -1 & 0 & 0 & 0 \\
  0 & -1& 0& 0 \\
  2 & 0 & -1& 0\\
   0 & 0 &0 &-1
\end{bmatrix}
$
\end{jie}

\num 期中2017-2018 二1.判断是否成立并给出理由。

设$A,B$为同阶对称方阵,则$AB$一定是对称矩阵;

\begin{jie}
不成立。理由:

以2阶对称矩阵为例:(式中:$a,b,c,x,y,z$为任意实数。)
\begin{equation*}
A=\begin{bmatrix}
  a & b \\
  b & c
\end{bmatrix}~~~B=
\begin{bmatrix}
  x & y \\
  y & z
\end{bmatrix}~~AB=
\begin{bmatrix}
  ax+by & \textcolor[rgb]{1.00,0.00,0.00}{ay+bz}\\
  \textcolor[rgb]{1.00,0.00,0.00}{bx+cy} & by+cz
\end{bmatrix}
\end{equation*}
可以看出,由于$a,b,c,x,y,z$取值的任意性,所以$ay+bz\neq by+cz$。 (可以取$a=1,b=2,c=3,x=3,y=4,z=5$实际验证一下。)
\end{jie}

\num 期中2017-2018 二4.判断是否成立并给出理由。

设2阶矩阵$A=
\begin{bmatrix}
  a & b \\
  c & d
\end{bmatrix}
$,若$A$与所有的2阶矩阵均可以交换,则$a=d,b=c=0$。

\begin{jie}
成立,理由如下:

取任意二阶矩阵($x,y,z,w$为任意实数):$
B=
\begin{bmatrix}
  x & y \\
  z & w
\end{bmatrix}
$,则
\begin{gather*}
AB=\begin{bmatrix}
  a & b \\
  c & d
\end{bmatrix}
\begin{bmatrix}
  x & y \\
  z & w
\end{bmatrix}=
\begin{bmatrix}
  ax+bz & ay+bw \\
  cx+dz & cy+dw
\end{bmatrix}\\
BA=
\begin{bmatrix}
  x & y \\
  z & w
\end{bmatrix}
\begin{bmatrix}
  a & b \\
  c & d
\end{bmatrix}=
\begin{bmatrix}
  xa+cy & xb+yd \\
  az+cw & bz+dw
\end{bmatrix}
\end{gather*}
若$A$与$B$可交换,则有$AB=BA$,即:
\begin{gather}
ax+bz=xa+cy  \\ 
ay+bw=xb+yd\\
cx+dz=az+cw\\
cy+dw=bz+dw
\end{gather}
由(1)式和(4)式得:$bz=cy$,因为$z$和$y$为任意数,所以$b=c=0$,代入(2)式和(3)式:$ay=dy,az=dz$,所以$a=d$。
\end{jie}

\num 期中2018-2019 二2.

设$A$是$n$阶实对称矩阵,如果$A^{2}=0$。证明$A=0$.并举例说明,如果$A$不是实对称矩阵,上述命题不正确。

\begin{jie}
证明:
(用定义)。

举例:对于二阶矩阵$
A=\begin{bmatrix}
0& 0\\
1& 0
\end{bmatrix}
$,$A^2=0$,但$A$不是对称矩阵。
\end{jie}

\end{document}  