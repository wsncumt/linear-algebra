\documentclass{article}
\usepackage[space,fancyhdr,fntef]{ctexcap}
\usepackage[namelimits,sumlimits,nointlimits]{amsmath}
\usepackage[bottom=25mm,top=25mm,left=25mm,right=15mm,centering]{geometry}
\usepackage{xcolor}
\usepackage{arydshln}%234页,虚线表格宏包
\pagestyle{fancy} \fancyhf{}
\fancyhead[OL]{~~~班序号:\hfill 学院:\hfill 学号:\hfill 姓名:王松年~~~ \thepage}
%\usepackage{parskip}
%\usepackage{indentfirst}
\usepackage{graphicx}%插图宏包,参见手册318页
\usepackage{mathdots}%反对角省略号
\usepackage{extarrows}%等号上加文字
\begin{document}

\newcounter{num} \renewcommand{\thenum}{\arabic{num}.} \newcommand{\num}{\refstepcounter{num}\text{\thenum}}

\newenvironment{jie}{\kaishu\zihao{-5}\color{blue}{\noindent\em 解:}\par}{\hfill $\diamondsuit$\par}

\newenvironment{zhengming}{\kaishu\zihao{-5}\color{blue}{\noindent\em 证明:}\par}{\hfill $\diamondsuit$\par}

\hphantom{~~}\hfill {\zihao{3}\heiti 第七次习题课} \hfill\hphantom{~~}

\hphantom{~~}\hfill {\zihao{4}\heiti 群文件《期中$\&$期末试题》} \hfill\hphantom{~~}

{\heiti \zihao{4} 期末试题}

\num 期末2015-2016 一3.

设$\alpha_{1},\alpha_{2},\alpha_{3}$是非齐次线性方程组$Ax=b$的解,若$\sum\limits_{i=1}^{3}c_{i}\alpha_{i}$也是$Ax=b$的解,则$\sum\limits_{i=1}^{3}c_{i}=$\underline{\hphantom{~~~~~~~~~~}}。

\begin{jie}
由题得:$A\alpha_{1}=b,A\alpha_{2}=b,A\alpha_{3}=b$,$A\sum\limits_{i=1}^{3}c_{i}\alpha_{i}=A(c_{1}\alpha_{1}+c_{2}\alpha_{2}+c_{3}\alpha_{3})=b$.

所以$A\alpha_{1}+A\alpha_{2}+A\alpha_{3}=3b$,即$A\left(\dfrac{1}{3}\alpha_{1}+\dfrac{1}{3}\alpha_{2}+\dfrac{1}{3}\alpha_{3}\right)=b=A(c_{1}\alpha_{1}+c_{2}\alpha_{2}+c_{3}\alpha_{3})$,所以$\sum\limits_ {i=1}^{3}c_{i}=\dfrac{1}{3}+\dfrac{1}{3}+\dfrac{1}{3}=1$。
\end{jie}

\num 期末2015-2016 一5.

任意3维实列向量都可以由向量组$\alpha_{1}=(1,0,1)^{T},\alpha_{2}=(1,-2,3)^{T}\alpha_{3}=(t,1,2)^{T}$线性表示,则$t$应满足条件\underline{\hphantom{~~~~~~~~~~}}。

\begin{jie}
任意3维实列向量都可以由向量组$\alpha_{1},\alpha_{2},\alpha_{3}$线性表示,则$e_{1}=[1,0,0]^T,e_2=[0,1,0]^T,e_3=[0,0,1]^T$也可由$\alpha_ {1},\alpha_{2},\alpha_{3}$线性表示,而$e_1,e_2,e_3$可以表示任意三维实列向量,即向量组$\alpha_ {1},\alpha_{2},\alpha_{3}$和$e_1,e_2,e_3$可以相互线性表示,所以$r(\alpha_ {1},\alpha_{2},\alpha_{3})=r(e_1,e_2,e_3)=3$.所以$|\alpha_ {1},\alpha_{2},\alpha_{3}|=2t-6\neq0$,即$t\neq3$。
\end{jie}

\num 期末2015-2016 四2.

设向量组$\alpha_{1},\alpha_{2},\alpha_{3}$线性无关,向量$\beta$可由$\alpha_{1},\alpha_{2},\alpha_{3}$线性表示,向量$\gamma$不能由$\alpha_{1},\alpha_{2},\alpha_{3}$线性表示,证明向量组$\alpha_{1},\alpha_{2},\alpha_{3},\beta+\gamma$线性无关。

\begin{zhengming}
向量$\beta$可由$\alpha_{1},\alpha_{2},\alpha_{3}$线性表示,即存在一组不全为0的$k_{i}(1\leq i\leq3)$,使得
\begin{equation*}
 \beta=k_ {1}\alpha_{1}+k_{2}\alpha_{2}+k_{3}\alpha_{3}\tag{$1$}
\end{equation*}

反证:假设向量组$\alpha_{1},\alpha_{2},\alpha_{3},\beta+\alpha$线性相关。则存在一组不全为0的$l_{i}(1\leq i\leq3)$和$l$,使得
\begin{equation*}
l_ {1}\alpha_{1}+l_{2}\alpha_{2}+l_{3}\alpha_{3}+l(\beta+\gamma)=0\tag{$2$}
\end{equation*}

若$l=0$,则$l_ {1}\alpha_{1}+l_{2}\alpha_{2}+l_{3}\alpha_{3}+l(\beta+\gamma)=l_ {1}\alpha_{1}+l_{2}\alpha_{2}+l_{3}\alpha_{3}=0$,此时$\alpha_{1},\alpha_{2},\alpha_{3}$线性相关,与题中的条件矛盾,所以$l\neq0$。所以$(1)$式可变形为
\begin{equation*}
\beta+\gamma=-\frac{1}{l}(l_ {1}\alpha_{1}+l_{2}\alpha_{2}+l_{3}\alpha_{3})
\end{equation*}
代入$(1)$式:
\begin{equation*}
  \gamma=-\frac{1}{l}(l_ {1}\alpha_{1}+l_{2}\alpha_{2}+l_{3}\alpha_{3})-k_ {1}\alpha_{1}+k_{2}\alpha_{2}+k_{3}\alpha_{3}
\end{equation*}

可以看出此时$\gamma$可以由$\alpha_{1},\alpha_{2},\alpha_{3}$线性表示,与题目矛盾,所以假设错误,即向量组$\alpha_{1},\alpha_{2},\alpha_{3},\beta+\gamma$线性无关。
\end{zhengming}

\num 期末2016-2017 四1.

已知$\alpha_{1},\alpha_{2},\alpha_{3}$是线性无关的向量组,若$\alpha_{1},\alpha_{2},\alpha_{3},\beta$线性相关,证明$\beta$可以由$\alpha_{1},\alpha_{2},\alpha_{3}$线性表示并且表示方法唯一。

\begin{zhengming}
$\alpha_{1},\alpha_{2},\alpha_{3},\beta$线性相关,则存在一组不全为0的$l_{i}(1\leq i\leq3)$和$l$,使得
\begin{equation*}
l_ {1}\alpha_{1}+l_{2}\alpha_{2}+l_{3}\alpha_{3}+l\beta=0\tag{$1$}
\end{equation*}
若$l=0$,则$l_ {1}\alpha_{1}+l_{2}\alpha_{2}+l_{3}\alpha_{3}+l\beta=l_ {1}\alpha_{1}+l_{2}\alpha_{2}+l_{3}\alpha_{3}=0$,此时$\alpha_{1},\alpha_{2},\alpha_{3}$线性相关,与题中的条件矛盾,所以$l\neq0$。所以$(1)$式可变形为
\begin{equation*}
\beta=-\frac{1}{l}(l_ {1}\alpha_{1}+l_{2}\alpha_{2}+l_{3}\alpha_{3})
\end{equation*}
即$\beta$可以由$\alpha_{1},\alpha_{2},\alpha_{3}$线性表示。

$\beta$可以由$\alpha_{1},\alpha_{2},\alpha_{3}$线性表示,不妨设任意两组不全为0的数$m_{i},n_i,(1\leq i\leq3)$,使得
\begin{gather*}
\beta=m_{1}\alpha_{1}+m_{2}\alpha_{2}+m_{3}\alpha_{3}\tag{$2$}\\
\beta=n_{1}\alpha_{1}+n_{2}\alpha_{2}+n_{3}\alpha_{3}\tag{$3$}
\end{gather*}
$(2)$式减$(3)$式:$0=(m_{1}-n{1})\alpha_{1}+(m_{2}-n{2})\alpha_{2}+(m_{3}-n{3})\alpha_{3}$,因为$\alpha_{1},\alpha_{2},\alpha_{3}$线性无关,所以有
$m_{1}-n{1}=0,m_{2}-n{2}=0,m_{3}-n{3}=0$,即$m_{1}=n{1},m_{2}=n{2},m_{3}=n{3}$,由于$m_{i}$和$n_{i}$的任意性,所以可证得表示方法唯一。
\end{zhengming}

\num 期末2017-2018 三3.

已知$\alpha_ {1}=(1,4,0,2)^{T},\alpha_{2}=(2,7,1,3)^{T},\alpha_{3}=(0,1,-1,a)^{T}$及$\beta_{4}=(3,10,b,4)^{T}$.

(1)$a,b$为何值时,$\beta$不能表示成$\alpha_{1},\alpha_{2},\alpha_{3}$的线性组合?

(2)$a,b$为何值时,$\beta$可由$\alpha_{1},\alpha_{2},\alpha_{3}$线性表示?并写出该表达式。

\begin{jie}
记$A=[\alpha_{1},\alpha_{2},\alpha_{3}]$,则
\begin{align*}
[A|\beta]=&
\left[
\begin{array}{c:c}
\begin{matrix}
1 & 2 & 0 \\
  4 & 7 & 1 \\
  0 & 1 & -1\\
  2& 3&a
\end{matrix}&
\begin{matrix}
3  \\
10\\
b\\
4
\end{matrix}
\end{array}
\right]
\xrightarrow{\substack{r_{2}-4r_{1}\\ r_{4}-2r_{1}}}
{
\left[
\begin{array}{c:c}
\begin{matrix}
1 & 2 & 0 \\
  0 & -1 & 1 \\
  0 & 1 & -1\\
  0& -1&a
\end{matrix}&
\begin{matrix}
3  \\
-2\\
b\\
2
\end{matrix}
\end{array}
\right]
}
\xrightarrow{\substack{r_{3}+r_{2}\\ r_{4}-r_{2}}}
{
\left[
\begin{array}{c:c}
\begin{matrix}
1 & 2 & 0 \\
  0 & -1 & 1 \\
  0 & 0 & 0\\
  0& 0&a-1
\end{matrix}&
\begin{matrix}
3  \\
-2\\
b-2\\
0
\end{matrix}
\end{array}
\right]
}\\
\xrightarrow{r_{3}\leftrightarrow r_{2}}&
{
\left[
\begin{array}{c:c}
\begin{matrix}
1 & 2 & 0 \\
  0 & -1 & 1 \\
   0& 0&a-1\\
 0 & 0 & 0
\end{matrix}&
\begin{matrix}
3  \\
-2\\
0\\
b-2
\end{matrix}
\end{array}
\right]
}
\end{align*}

(1)可以看出$b\neq2,a\in R$时,$Ax=\beta$无解,即$\beta$不能表示成$\alpha_{1},\alpha_{2},\alpha_{3}$的线性组合。

(2)$b=2$时,$\beta$可由$\alpha_{1},\alpha_{2},\alpha_{3}$线性表示。

当$a\neq1$,$r(A)=r(A,\beta)=3$,此时:$Ax=\beta$有唯一解,即$\beta$可由$\alpha_{1},\alpha_{2},\alpha_{3}$线性表示的方法唯一。
\begin{align*}
\xrightarrow{r_{3}\times \frac{1}{a-1}}
{
\left[
\begin{array}{c:c}
\begin{matrix}
1 & 2 & 0 \\
  0 & -1 & 1 \\
   0& 0&1\\
 0 & 0 & 0
\end{matrix}&
\begin{matrix}
3  \\
-2\\
0\\
0
\end{matrix}
\end{array}
\right]
}
\xrightarrow{r_2-r_{3}}
{
\left[
\begin{array}{c:c}
\begin{matrix}
1 & 2 & 0 \\
  0 & -1 & 0 \\
   0& 0&1\\
 0 & 0 & 0
\end{matrix}&
\begin{matrix}
3  \\
-2\\
0\\
0
\end{matrix}
\end{array}
\right]
}\xrightarrow{r_1+2r_{2}}
{
\left[
\begin{array}{c:c}
\begin{matrix}
1 & 0 & 0 \\
  0 & -1 & 0 \\
   0& 0&1\\
 0 & 0 & 0
\end{matrix}&
\begin{matrix}
-1  \\
-2\\
0\\
0
\end{matrix}
\end{array}
\right]
}
\end{align*}
此时$Ax=\beta$的解为$x_{1}=-1,x_{2}=2,x_{3}=0$,所以$\beta=x_{1}\alpha_{1}+x_{2}\alpha_{2}+x_{3}\alpha_{3}=-\alpha_{1}+2\alpha_{2}+0\alpha_{3}=-\alpha_{1}+2\alpha_{2}$.

$a=1$时$r(A,\beta)=r(A)=2<3$,所以$\beta$可由$\alpha_{1},\alpha_{2},\alpha_{3}$线性表示的方法唯一。
\begin{equation*}
\xrightarrow{r_1+2r_{2}}
{
\left[
\begin{array}{c:c}
\begin{matrix}
1 & 0 & 2 \\
  0 & -1 & 1 \\
   0& 0&0\\
 0 & 0 & 0
\end{matrix}&
\begin{matrix}
-1  \\
-2\\
0\\
0
\end{matrix}
\end{array}
\right]}
\end{equation*}
所以解得$x_{1}=-1-2x_{3},x_{2}=x_{3}+2$,令$x_{3}=k,k\in R$,则$\beta=x_ {1}\alpha_{1}+x_{2}\alpha_{2}+x_{3}\alpha_{3}=-(1+2k)\alpha_{1}+(2+k)\alpha_{2}+k\alpha_{3}$
\end{jie}

\num 期末2018-2019 一2.

已知向量组$\alpha_{1}=(1,3,1),\alpha_{2}=(0,1,1),\alpha_{3}=(1,4,k)$线性无关,则实数$k$满足的条件是\underline{\hphantom{~~~~~~~~~~}}。

\begin{jie}
$\alpha_{1},\alpha_{2},\alpha_{3}$线性无关,即$r(\alpha_{1},\alpha_{2},\alpha_{3})=3$,记$A=(\alpha_{1},\alpha_{2},\alpha_{3})$,则$|A|\neq0$
\begin{equation*}
  |A|=\begin{vmatrix}
       1& 0 & 1\\
       3 & 1 & 4\\
       1 & 1 & k
\end{vmatrix}\xlongequal{c_{3}-c_{1}}
\begin{vmatrix}
1& 0 & 0\\
3 & 1 & 1\\
1 & 1 & k-1
\end{vmatrix}=k-2\neq0~~~\Rightarrow k\neq 2
\end{equation*}
\end{jie}

\num 期末2018-2019 一6.

设3维列向量组$\alpha_{1},\alpha_{2},\alpha_{3}$线性无关,3阶方阵$A$满足$A\alpha_{1}=-\alpha_{1},A\alpha_{2}=\alpha_{2},A\alpha_{3}=\alpha_{2}+\alpha_{3}$。则行列式$|A|=$\underline{\hphantom{~~~~~~~~~~}}。

\begin{jie}
由题得:$A\alpha_ {1}=-\alpha_{1},A\alpha_{2}=\alpha_{2},A\alpha_{3}=\alpha_{2}+\alpha_{3}$所以
\begin{equation*}
A(\alpha_{1}~\alpha_{2}~\alpha_{3})=(-\alpha_{1}~~\alpha_{2}~~\alpha_{2}+\alpha_{3})
\end{equation*}
即
\begin{gather*}
|A(\alpha_{1}~\alpha_{2}~\alpha_{3})|=|A|\cdot|\alpha_{1}~\alpha_{2}~\alpha_{3}|=|-\alpha_{1}~~\alpha_{2}~~\alpha_{2}+\alpha_{3}|\\
|-\alpha_{1}~~\alpha_{2}~~\alpha_{2}+\alpha_{3}|\xlongequal{c_{3}-c_{2}}|-\alpha_{1}~~\alpha_{2}~~\alpha_{3}|=-|\alpha_{1}~\alpha_{2}~\alpha_{3}|\\
|A|=-1
\end{gather*}
\end{jie}
\end{document}  