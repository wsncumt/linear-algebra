\documentclass{article}
\usepackage[space,fancyhdr,fntef]{ctexcap}
\usepackage[namelimits,sumlimits,nointlimits]{amsmath}
\usepackage[bottom=25mm,top=25mm,left=25mm,right=15mm,centering]{geometry}
\usepackage{xcolor}
\usepackage{arydshln}%234页,虚线表格宏包
\usepackage{mathdots}%反对角省略号
\pagestyle{fancy} \fancyhf{}
\fancyhead[OL]{~~~班序号:\hfill 学院:\hfill 学号:\hfill 姓名:王松年~~~ \thepage}
%\usepackage{parskip}
%\usepackage{indentfirst}
\usepackage{graphicx}%插图宏包,参见手册318页
\begin{document}

\newcounter{num} \renewcommand{\thenum}{\arabic{num}.} \newcommand{\num}{\refstepcounter{num}\text{\thenum}}

\hphantom{~~}\hfill {\zihao{3}\heiti 第七次习题课} \hfill\hphantom{~~}

\hphantom{~~}\hfill {\zihao{4}\heiti 群文件《期中$\&$期末试题》} \hfill\hphantom{~~}

{\heiti \zihao{4} 期末试题}

\num 期末2015-2016 一3.

设$\alpha_{1},\alpha_{2},\alpha_{3}$是非齐次线性方程组$Ax=b$的解,若$\sum\limits_{i=1}^{3}c_{i}\alpha_{i}$也是$Ax=b$的解,则$\sum\limits_{i=1}^{3}c_{i}=$\underline{\hphantom{~~~~~~~~~~}}。\\

\num 期末2015-2016 一5.

任意3维实列向量都可以由向量组$\alpha_{1}=(1,0,1)^{T},\alpha_{2}=(1,-2,3)^{T}\alpha_{3}=(t,1,2)^{T}$线性表示,则$t$应满足条件\underline{\hphantom{~~~~~~~~~~}}。\\

\num 期末2015-2016 四2.

设向量组$\alpha_{1},\alpha_{2},\alpha_{3}$线性无关,向量$\beta$可由$\alpha_{1},\alpha_{2},\alpha_{3}$线性表示,向量$\gamma$不能由$\alpha_{1},\alpha_{2},\alpha_{3}$线性表示,证明向量组$\alpha_{1},\alpha_{2},\alpha_{3},\beta+\gamma$线性无关。\\

\num 期末2016-2017 四1.

已知$\alpha_{1},\alpha_{2},\alpha_{3}$是线性无关的向量组,若$\alpha_{1},\alpha_{2},\alpha_{3},\beta$线性相关,证明$\beta$可以由$\alpha_{1},\alpha_{2},\alpha_{3}$线性表示并且表示方法唯一。\\

\num 期末2017-2018 三3.

已知$\alpha_ {1}=(1,4,0,2)^{T},\alpha_{2}=(2,7,1,3)^{T},\alpha_{3}=(0,1,-1,a)^{T}$及$\beta_{4}=(3,10,b,4)^{T}$.

(1)$a,b$为何值时,$\beta$不能表示成$\alpha_{1},\alpha_{2},\alpha_{3}$的线性组合?

(2)$a,b$为何值时,$\beta$可由$\alpha_{1},\alpha_{2},\alpha_{3}$线性表示?并写出该表达式。\\

\num 期末2018-2019 一2.

已知向量组$\alpha_{1}=(1,3,1),\alpha_{2}=(0,1,1),\alpha_{3}=(1,4,k)$线性无关,则实数$k$满足的条件是\underline{\hphantom{~~~~~~~~~~}}。\\

\num 期末2018-2019 一6.

设3维列向量组$\alpha_{1},\alpha_{2},\alpha_{3}$线性无关,3阶方阵$A$满足$A\alpha_{1}=-\alpha_{1},A\alpha_{2}=\alpha_{2},A\alpha_{3}=\alpha_{2}+\alpha_{3}$。则行列式$|A|=$\underline{\hphantom{~~~~~~~~~~}}。\\

\end{document}  