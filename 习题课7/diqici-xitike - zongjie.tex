\documentclass{article}
\usepackage[space,fancyhdr,fntef]{ctexcap}
\usepackage[namelimits,sumlimits,nointlimits]{amsmath}
\usepackage[bottom=25mm,top=25mm,left=25mm,right=15mm,centering]{geometry}
\usepackage{xcolor}
\usepackage{paralist}%列表宏包
\usepackage{arydshln}%234页,虚线表格宏包
\pagestyle{fancy} \fancyhf{}
\fancyhead[OL]{~~~班序号:\hfill 学院:\hfill 学号:\hfill 姓名:王松年~~~ \thepage}
%\usepackage{parskip}
%\usepackage{indentfirst}
\usepackage{graphicx}%插图宏包,参见手册318页
\begin{document}

\newcounter{num} \renewcommand{\thenum}{\arabic{num}.} \newcommand{\num}{\refstepcounter{num}\text{\thenum}}

\hphantom{~~}\hfill {\zihao{3}\heiti 第七次习题课} \hfill\hphantom{~~}

\hphantom{~~}\hfill {\zihao{4}\heiti 知识点} \hfill\hphantom{~~}


\num 向量:只有一行或一列的矩阵,一般用$\alpha,\beta,\gamma$表示。

\num 向量空间:由所有的$n$维(行)列向量组成的集合称为$n$维向量空间。

\num 两个$n$维向量相等,当且仅当他们各个对应分量相等。

\num  零向量:每个分量都为0。

\num 子空间:$H$是$R^n$的一个非空子集,如果满足以下条件:

(1).零向量属于$H$。

(2).$H$对加法封闭。即$\alpha,\beta\in H~~\rightarrow~~\alpha+\beta\in H$。

(3).$H$对数乘封闭。即$\alpha\in H,k\in R~~\rightarrow~~k\alpha\in H$。

称$H$是$R^n$的一个子空间。

\num 线性组合。

\num $\beta$可以由$\alpha_{1},\alpha_{2},\cdots,\alpha_{n}$线性表出当且仅当$r(\alpha_{1},\alpha_{2},\cdots,\alpha_{n})=r(\alpha_{1},\alpha_{2},\cdots,\alpha_{n},\beta)$。

\num 向量组的等价:$\alpha_{1},\alpha_{2},\cdots,\alpha_{s}$和$\beta_{1},\beta_{2},\cdots,\beta_{t}$可以相互线性表示,则称$\alpha_{1},\alpha_{2},\cdots,\alpha_{s}$和$\beta_{1},\beta_{2},\cdots,\beta_{t}$等价。$r(\alpha_{1},\alpha_{2},\cdots,\alpha_{s})=r(\alpha_{1},\alpha_{2},\cdots,\alpha_{s},\beta_{1},\beta_{2},\cdots,\beta_{t})=r(\beta_{1},\beta_{2},\cdots,\beta_{t})$

性质:

(1)自反性:向量组与自身等价。

(2)对称性。

(3)传递性。

\num 线性相关:对于向量组$\alpha_{1},\alpha_{2},\cdots,\alpha_{n}$,如果$x_{1}\alpha_{1}+x_{2}\alpha_{2}+\cdots+x_{n}\alpha_{n}=0$有不全为0的解,则称向量组$\alpha_{1},\alpha_{2},\cdots,\alpha_{n}$线性相关($r(A)<n$),否则称$\alpha_{1},\alpha_{2},\cdots,\alpha_{n}$线性无关($r(A)=n$)。

(1)$n$个$n$维列向量组线性无关当且仅当$|A|\neq0$。

(2)$m$个$n$($m>n$)维列向量组一定线性相关。

(3)两个向量线性相关,当且仅当两个向量成比例


\end{document}  