\documentclass{article}
\usepackage[space,fancyhdr,fntef]{ctexcap}
\usepackage[namelimits,sumlimits,nointlimits]{amsmath}
\usepackage[bottom=25mm,top=25mm,left=25mm,right=15mm,centering]{geometry}
\usepackage{xcolor}
\usepackage{arydshln}%234页,虚线表格宏包
\pagestyle{fancy} \fancyhf{}
\fancyhead[OL]{~~~班序号:\hfill 学院:\hfill 学号:\hfill 姓名:王松年~~~ \thepage}
%\usepackage{parskip}
%\usepackage{indentfirst}
\usepackage{graphicx}%插图宏包,参见手册318页
\usepackage{mathdots}%反对角省略号
\usepackage{extarrows}%等号上加文字
\begin{document}

\newcounter{num} \renewcommand{\thenum}{\arabic{num}.} \newcommand{\num}{\refstepcounter{num}\text{\thenum}}

\newenvironment{jie}{\kaishu\zihao{-5}\color{blue}{\noindent\em 解:}\par}{\hfill $\diamondsuit$\par}

\newenvironment{zhengming}{\kaishu\zihao{-5}\color{blue}{\noindent\em 证明:}\par}{\hfill $\diamondsuit$\par}

\hphantom{~~}\hfill {\zihao{3}\heiti 第六次习题课} \hfill\hphantom{~~}

\hphantom{~~}\hfill {\zihao{4}\heiti 群文件《期中$\&$期末试题》} \hfill\hphantom{~~}

{\heiti \zihao{4} 期中试题}

\num 期中2016-2017 一4.

设$f(x)=ax^{2}+bx+c$,$A$为$n$阶方阵,定义$f(A)=aA^{2}+bA+cI$,如果$
A=
\begin{bmatrix}
  1 & 0 & 0 & 0 \\
  0 & 1& 0& 0 \\
  2 & 0 & 1& 0\\
   0 & 0 &0 &1
\end{bmatrix},f(x)=x^{2}-x-1,
$则$f(A)=$\underline{\hphantom{~~~~~~~~~~}}。

\begin{jie}
由题得:
\begin{equation*}
A=E+\begin{bmatrix}
  0 & 0 & 0 & 0 \\
  0 & 0& 0& 0 \\
  2 & 0 & 0& 0\\
   0 & 0 &0 &0
\end{bmatrix}=E+B
\end{equation*}
可以看出:$B^2=0$.所以$A^2=(E+B)^2=E+B^2+2EB=E+2B$.

所以$f(A)=A^2-A-E=E+2B-E-B-E=B-E=
\begin{bmatrix}
  -1 & 0 & 0 & 0 \\
  0 & -1& 0& 0 \\
  2 & 0 & -1& 0\\
   0 & 0 &0 &-1
\end{bmatrix}
$
\end{jie}

\num 期中2017-2018 二1.判断是否成立并给出理由。

设$A,B$为同阶对称方阵,则$AB$一定是对称矩阵;

\begin{jie}
不成立。理由:

以2阶对称矩阵为例:(式中:$a,b,c,x,y,z$为任意实数。)
\begin{equation*}
A=\begin{bmatrix}
  a & b \\
  b & c
\end{bmatrix}~~~B=
\begin{bmatrix}
  x & y \\
  y & z
\end{bmatrix}~~AB=
\begin{bmatrix}
  ax+by & \textcolor[rgb]{1.00,0.00,0.00}{ay+bz}\\
  \textcolor[rgb]{1.00,0.00,0.00}{bx+cy} & by+cz
\end{bmatrix}
\end{equation*}
可以看出,由于$a,b,c,x,y,z$取值的任意性,所以$ay+bz\neq by+cz$。 (可以取$a=1,b=2,c=3,x=3,y=4,z=5$实际验证一下。)
\end{jie}

\num 期中2017-2018 二4.判断是否成立并给出理由。

设2阶矩阵$A=
\begin{bmatrix}
  a & b \\
  c & d
\end{bmatrix}
$,若$A$与所有的2阶矩阵均可以交换,则$a=d,b=c=0$。

\begin{jie}
成立,理由如下:

取任意二阶矩阵($x,y,z,w$为任意实数):$
B=
\begin{bmatrix}
  x & y \\
  z & w
\end{bmatrix}
$,则
\begin{gather*}
AB=\begin{bmatrix}
  a & b \\
  c & d
\end{bmatrix}
\begin{bmatrix}
  x & y \\
  z & w
\end{bmatrix}=
\begin{bmatrix}
  ax+bz & ay+bw \\
  cx+dz & cy+dw
\end{bmatrix}\\
BA=
\begin{bmatrix}
  x & y \\
  z & w
\end{bmatrix}
\begin{bmatrix}
  a & b \\
  c & d
\end{bmatrix}=
\begin{bmatrix}
  xa+cy & xb+yd \\
  az+cw & bz+dw
\end{bmatrix}
\end{gather*}
若$A$与$B$可交换,则有$AB=BA$,即:
\begin{gather}
ax+bz=xa+cy  \\
ay+bw=xb+yd\\
cx+dz=az+cw\\
cy+dw=bz+dw
\end{gather}
由(1)式和(4)式得:$bz=cy$,因为$z$和$y$为任意数,所以$b=c=0$,代入(2)式和(3)式:$ay=dy,az=dz$,所以$a=d$。
\end{jie}

\num 期中2018-2019 二2.

设$A$是$n$阶实对称矩阵,如果$A^{2}=0$,证明$A=0$.并举例说明,如果$A$不是实对称矩阵,上述命题不正确。

\begin{jie}
证明:
依题意设$A=
\begin{bmatrix}
  a_{11} & a_{12} & \cdots & a_{1n} \\
  a_{12} & a_{22} & \cdots & a_{2n} \\
  \vdots & \vdots & \ddots & \vdots \\
  a_{1n} & a_{2n} & \cdots & a_{nn}
\end{bmatrix}
$,所以$A^2$为:只看$A^2$对角线上的元素,$A^2$的第$k$行第$k$列的元素为$A$的第$k$行乘第$k$列:$a_{1k}^2+a_{2k}^2+\cdots+a_{kk}^2+a_{kk+1}^2+a_{kk+2}^2+\cdots+a_{kn}^2=0$,因为平方一定大于等于0,所以该式的每一项都为0,即$a_{1k},a_{2k},\cdots,a_{kk},a_{kk+1},a_{kk+2},\cdots,a_{kn}$为0.即$A$的第$k$行和第$k$列元素为0.\textcolor[rgb]{1.00,0.00,0.00}{(这一步看不懂的计算一下$A^2$的第一行第一列,第二行第二列的元素验证一下)}

因为$A^2$为0,即其对角线每个元素都为0,由上边的步骤可以推出$A$的每行每列元素都为0,即$A=0$。

举例:对于二阶矩阵$
A=\begin{bmatrix}
0& 0\\
1& 0
\end{bmatrix}
$,$A^2=0$,但$A$不是对称矩阵。
\end{jie}

\num 期中2015-2016 一2.

设$f(x)=
\begin{vmatrix}
  2x & x & 1 & 2\\
  1 & x & 1 & -1\\
  3 & 2 & x & 1\\
  1 & 1 & 1 & x\\
\end{vmatrix}
$,则$x^{3}$的系数为\underline{\hphantom{~~~~~~~~~~}}。

\begin{jie}
方法一:求出对应的行列式,然后写出$x^3$的系数。(此方法太过繁琐,容易出错,不推荐使用)

用$Matlab$计算出来的结果为:$f(x)=2x^4-x^3-7x^2+12x-8$.(仅供参考)
\\

方法二:使用定义,课本106-108页。

思路:使用行列式的定义来做。仅找出与$x^{3}$有关的项。这里取列按照自然排列,行由自己指定(也可以取行按照自然排列,列由自己指定)。

第一列中:取第一行,第二列第三列第四列无论怎么取都不可能构成$x^{3}$。

\textcolor[rgb]{1.00,0.00,0.00}{(注意:在行列式的定义式中,每一项中的几个元素必须来自不同的行数和列数,如:对于此题来说,列按自然排列,第一个元素取第一列中的第一行,那么第二个元素只能从剩下三列中的剩下三行来取)}

第一列中:取第二行,第二列取第一行,第三列取第三行,第四列取第四行。即$(-1)^{\tau(2134)}a_{\textcolor[rgb]{1.00,0.00,0.00}{2}\textcolor[rgb]{0.00,0.00,0.00}{1}}a_{\textcolor[rgb]{1.00,0.00,0.00}{1}\textcolor[rgb]{0.00,0.00,0.00}{2}}a_{\textcolor[rgb]{1.00,0.00,0.00}{3}\textcolor[rgb]{0.00,0.00,0.00}{3}}a_{\textcolor[rgb]{1.00,0.00,0.00}{4}\textcolor[rgb]{0.00,0.00,0.00}{4}}=(-1)^{1}1*x*x*x=-x^{3}$(注意下标,列(黑色)是自然排列,行(红色)是上边分析得来的)

第一列取第三行第四行都不能构成$x^{3}$。

\textcolor[rgb]{0.50,0.00,0.00}{
验证:$x^{2}$的系数
}

列取自然排列,行按下述几个取时构成$x^2$:$1324~~3124~~3214~~4231~~4132$
即:
\begin{align*}
&(-1)^{\tau(1324)}a_{11}a_{32}a_{23}a_{44}+(-1)^{\tau(3124)}a_{31}a_{12}a_{23}a_{44}+\\
&(-1)^{\tau(3214)}a_{31}a_{22}a_{13}a_{44}+(-1)^{\tau(4132)}a_{41}a_{12}a_{33}a_{24}+(-1)^{\tau(4231)}a_{41}a_{22}a_{33}a_{14}\\
=&(-4+3-3-1-2)x^2=-7x^2
\end{align*}
可以看出和方法1算的结果一样。
\end{jie}

\num 期中2015-2016 一5.

若$A$为4阶方阵,$A^{*}$为$A$的伴随矩阵,$|A|=\dfrac{1}{2}$,则$\left|\left(\dfrac{1}{4}A\right)^{-1}-A^{*}\right|=$\underline{\hphantom{~~~~~~~~~~}}。

\begin{jie}
$\left|\left(\dfrac{1}{4}A\right)^{-1}-A^{*}\right|=\left|\left(\dfrac{1}{4}A\right)^{-1}-|A|A^{-1}\right|=\left|4A^{-1}-\frac{1}{2}A^{-1}\right|=\left|\frac{7}{2}A^{-1}\right|=\left(\frac{7}{2}\right)^{4}|A|^{-1}=\frac{7^4}{8}$
\end{jie}

\num 期中2015-2016 一6.
设$A=
\begin{bmatrix}
  1 & 0 & 0\\
  1 & 1 & 0\\
  1 & 2 & 3
\end{bmatrix}
$,则$(A^*)^{-1}=$\underline{\hphantom{~~~~~~~~~~}}。

\begin{jie}
$A^*=|A|A^{-1}$,所以$(A^*)^{-1}=(|A|A^{-1})^{-1}=|A|^{-1}(A^{-1})^{-1}=|A|^{-1}A$

$|A|=1\times 1\times 3=3$
所以:$(A^*)^{-1}=\dfrac{1}{3}A=\begin{bmatrix}
  \frac{1}{3} & 0 & 0\\
  \frac{1}{3} & \frac{1}{3} & 0\\
  \frac{1 }{3}& \frac{2}{3} & 1
\end{bmatrix}$
\end{jie}

\num 期中2015-2016 三1.

设$A$可逆,且$A^{*}B=A^{-1}+B$,证明$B$可逆,当$A=
\begin{bmatrix}
  2 & 6 & 0 \\
  0 & 2 & 6\\
  0 & 0 & 2
\end{bmatrix}
$时,求$B$。

\begin{jie}
由题得:$A^{*}B=A^{-1}+B$,即$(A^*-E)B=A^{-1}$,两边同时左乘$A$得:$A(A^*-E)B=E$,所以$B$可逆,其逆矩阵为$A(A^*-E)=(|A|E-A)$.

由题得:$|A|=2\times2\times 2=8$
\begin{equation*}
B=(|A|E-A)^{-1}=\begin{bmatrix}
  6 & -6 & 0 \\
  0 & 6 & -6\\
  0 & 0 & 6
\end{bmatrix}^{-1}=\frac{1}{6}\begin{bmatrix}
  1 & 1 & 1 \\
  0 & 1 & 1\\
  0 & 0 & 1
\end{bmatrix}
\end{equation*}
\end{jie}

\num 期中2016-2017 一5.

若$A$为3阶方阵,$A^{*}$为$A$的伴随矩阵,$|A|=\dfrac{1}{2}$,则$\left|(3A)^{-1}-2A^{*}\right|=$\underline{\hphantom{~~~~~~~~~~}}。

\begin{jie}
$\left|(3A)^{-1}-2A^{*}\right|=\left|3^{-1}A^{-1}-2|A|A^{-1}\right|=\left|-\frac{2}{3}A^{-1}\right|=\left(-\frac{2}{3}\right)^{3}|A|^{-1}=-\frac{16}{27}$
\end{jie}

\num 期中2016-2017 二5.

若$\left(\dfrac{1}{4}A^{*}\right)^{-1}BA^{-1}=2AB+I$,且
$A=
\begin{bmatrix}
  2 & 0& 0 & 0 \\
  1 & 1 & 0& 0 \\
  0 & 0 & 2& 1\\
   0& 0 &0 &1
\end{bmatrix}
$,求$B$。

\begin{jie}
由题得:$A=
\begin{bmatrix}
  A_{11} & 0 \\
  0 & A_{22}
\end{bmatrix}
$,其中$A_{11}=\begin{bmatrix}
              2 & 0 \\
              1 & 1
            \end{bmatrix}$,
$A_{22}=
\begin{bmatrix}
  2 & 1 \\
  0 & 1
\end{bmatrix}
$,所以有:
\begin{equation*}
|A_{11}|=|A_{22}|=2\times1=2;~~|A|=|A_{11}|\cdot|A_{22}|=4;~~A_{11}^{-1}=
\begin{bmatrix}
  \frac{1}{2} & 0 \\
  -\frac{1}{2} & 1
\end{bmatrix};~~A_{22}^{-1}=
\begin{bmatrix}
  \frac{1}{2} & -\frac{1}{2} \\
   0& 1
\end{bmatrix};~~A^{-1}=\begin{bmatrix}
  A_{11}^{-1} & 0 \\
  0 & A_{22}^{-1}
\end{bmatrix}
\end{equation*}
\begin{gather*}
\left(\dfrac{1}{4}A^{*}\right)^{-1}BA^{-1}=2AB+I ~~~\Rightarrow~~~
4(|A|A^{-1})^{-1}BA^{-1}=2AB+I ~~~ \Rightarrow~~~
\frac{4}{|A|}ABA^{-1}=2AB+I\\
ABA^{-1}=2AB+I~~~\Rightarrow~~~
AB=2ABA+A~~~\Rightarrow~~~
AB(E-2A)=A~~~\Rightarrow~~~
B(E-2A)=I~~~\Rightarrow~~
B=(E-2A)^{-1}
\end{gather*}
\begin{equation*}
  B=(E-2A)^{-1}=\begin{bmatrix}
  -3 & 0& 0 & 0 \\
  -2 & -1 & 0& 0 \\
  0 & 0 & -3& -2\\
   0& 0 &0 &-1
\end{bmatrix}^{-1}=
\begin{bmatrix}
  -\frac{1}{3} & 0& 0 & 0 \\
  \frac{2}{3} & -1 & 0& 0 \\
  0 & 0 & -\frac{1}{3}& \frac{2}{3}\\
   0& 0 &0 &-1
\end{bmatrix}
\end{equation*}
\end{jie}

\num 期中2016-2017 三2.

已知$A=(a_{ij})$是三阶的非零矩阵,设$A_{ij}$是$a_{ij}$的代数余子式,且对任意的$i,j$有$A_{ij}+a_{ij}=0$,求$A$ 的行列式。

\begin{jie}
因为$A_{ij}+a_{ij}=0$,所以可以推出$A+(A^*)^T=0$。即$(A^*)^T=-A$.两边同时取行列式:

左边:$|(A^*)^T|=|A^*|=||A|A^{-1}|=|A|^n|A|^{-1}=|A|^{n-1}=|A|^2$

右边:$|-A|=(-1)^3|A|=-|A|$

所以$|A|^2=-|A|$,解得$|A|=-1$或$|A|=0$。

又因为$(A^*)^T=-A$,所以有$r((A^*)^T)=r(-A)$,即$r(A^*)=r(A)$,所以$r(A)=n=3$。\textcolor[rgb]{1.00,0.00,0.00}{注:此步看不懂的看课本121页的例3.3.23,记住这个例题的结论。}

$r(A)=3$,即满秩,满秩即(可逆$\&$行列式不为$0$),所以$|A|=-1$。
\end{jie}

\num 期中2017-2018 二2.判断是否正确并说明理由。

设$A,B$为$n$阶可逆方阵,则$(AB)^{*}=B^{*}A^{*}$.

\begin{jie}
正确,理由如下:

因为$A,B$为$n$阶可逆方阵,所以$AB$可逆,所以$(AB)^{*}=|AB|(AB)^{-1}=|A||B|B^{-1}A^{-1}=|A|B^{*}A^{-1}=B^{*}A^{*}$
\end{jie}

\num 期中2018-2019 一2.

设$A,B$为3阶矩阵,且$|A|=3,|B|=2$,$A^{*}$为$A$的伴随矩阵。

(1)若交换$A$的第一行与第二行得矩阵$C$,求$|CA^{*}|$;

\begin{jie}
交换交换$A$的第一行与第二行得矩阵$C$,所以$|C|=-|A|$,所以$|CA^{*}|=|C||A^{*}|=-|A|||A|A^{-1}|-|A||A|^{3}|A|^{-1}=-|A|^3=-27$
\end{jie}

\num 期中2018-2019 一3.

已知3阶矩阵$A$的逆矩阵$
A^{-1}=
\begin{bmatrix}
  1 & 1 & 1 \\
  1 & 2 & 1 \\
  2 & 1 & 3
\end{bmatrix}
$,试求伴随矩阵$A^{*}$的逆矩阵。

\begin{jie}
由伴随矩阵的性质:$A^{*}=|A|A^{-1}$,所以$(A^*)^{-1}=(|A|A^{-1})^{-1}=|A|^{-1}A=|A^{-1}|A$

由题得:
\begin{equation*}|A^{-1}|=
  \begin{vmatrix}
  1 & 1 & 1 \\
  1 & 2 & 1 \\
  2 & 1 & 3
  \end{vmatrix}\xlongequal{\substack{c_{2}-c_{1}\\c_{3}-c_{1}}}
  \begin{vmatrix}
  1 & 0 & 0 \\
  1 & 1 & 0 \\
  2 & -1 & 1
  \end{vmatrix}=1
\end{equation*}

\begin{align*}
[A^{-1}|E]=&
\left[
\begin{array}{c:c}
\begin{matrix}
1 & 1 & 1 \\
  1 & 2 & 1 \\
  2 & 1 & 3
\end{matrix}&
\begin{matrix}
1 & 0 & 0 \\
  0 & 1 & 0 \\
  0 & 0 & 1
\end{matrix}
\end{array}
\right]
\xrightarrow{\substack{r_{2}-r_{1}\\ r_{3}-2r_{1}}}
{
\left[
\begin{array}{c:c}
\begin{matrix}
1 & 1 & 1 \\
  0 & 1 & 0 \\
  0 & -1 & 1
\end{matrix}&
\begin{matrix}
1 & 0 & 0 \\
  -1 & 1 & 0 \\
  -2 & 0 & 1
\end{matrix}
\end{array}
\right]
}\xrightarrow{r_{3}+r_{2}}
{
\left[
\begin{array}{c:c}
\begin{matrix}
1 & 1 & 1 \\
  0 & 1 & 0 \\
  0 & 0 & 1
\end{matrix}&
\begin{matrix}
1 & 0 & 0 \\
  -1 & 1 & 0 \\
  -3 & 1 & 1
\end{matrix}
\end{array}
\right]
}\\
\xrightarrow{r_{1}-r_{3}}&
{
\left[
\begin{array}{c:c}
\begin{matrix}
1 & 1 & 0 \\
  0 & 1 & 0 \\
  0 & 0 & 1
\end{matrix}&
\begin{matrix}
4 & -1 & -1 \\
  -1 & 1 & 0 \\
  -3 & 1 & 1
\end{matrix}
\end{array}
\right]
}
\xrightarrow{r_{1}-r_{2}}
{
\left[
\begin{array}{c:c}
\begin{matrix}
1 & 0 & 0 \\
  0 & 1 & 0 \\
  0 & 0 & 1
\end{matrix}&
\begin{matrix}
5 & -2 & -1 \\
  -1 & 1 & 0 \\
  -3 & 1 & 1
\end{matrix}
\end{array}
\right]
}
\end{align*}
所以$(A^{*})^{-1}=|A^{-1}|A=
\begin{bmatrix}
  5 & -2 & -1\\
  -1 & 1 & 0\\
  -3 & 1 & 1
\end{bmatrix}
$
\end{jie}

\num 期中2018-2019 二1.

若$n$阶实矩阵$Q$满足$QQ^{T}=I$,则称$Q$为正交矩阵。设$Q$为正交矩阵,则

(1)$Q$的行列式为1或-1.

(2)当$|Q|=1$且$n$为奇数时,证明$|I-Q|=0$,其中$I$是$n$阶单位矩阵;

(3)$Q$的逆矩阵$Q^{-1}$和伴随矩阵$Q^{*}$都是正交矩阵。

\begin{zhengming}
(1)由题得:$|QQ^{T}|=|I|=1$,由行列式的性质:$|QQ^{T}|=|Q|\cdot|Q^T|,|Q^T|=|Q|$,所以$|QQ^{T}|=|Q|^2=1$,解得$|Q|=1$或$|Q|=-1$.

(2)$|I-Q|=|QQ^{T}-Q|=|Q|\cdot|Q^T-I|=|Q^T-I|=|(Q^T)^T-I^T|=|Q-I|=|-(I-Q)|=(-1)^{n}|I-Q|$,因为$n$为奇数,所以$(-1)^n=-1$,即$|I-Q|=-|I-Q|$,所以$|I-
Q|=0$。

(3)因为$QQ^{T}=I$,两边同时左乘$Q^{-1}$:$Q^{T}=Q^{-1}$,两边同时右乘$(Q^{T})^{-1}$:$I=Q^{-1}(Q^{T})^{-1}=Q^{-1}(Q^{-1})^{T}$,所以$Q^{-1}$是正交矩阵。

由伴随矩阵的性质:$Q^*=|Q|Q^{-1},(Q^T)^*=(Q^*)^T=|Q^T|(Q^{T})^{-1}=|Q|(Q^{-1})^{T}$,所以$Q^*(Q^*)^T=|Q|Q^{-1}|Q|(Q^{-1})^{T}=|Q|^{2}Q^{-1}(Q^{-1})^{T}$

由(1)得$|Q|^{2}=1$,所以有$|Q|^{2}Q^{-1}(Q^{-1})^{T}=Q^{-1}(Q^{-1})^{T}=I$,所以$Q^{*}$是正交矩阵。
\end{zhengming}

{\heiti \zihao{4} 期末试题}

\num 期末2014-2015 一2.

设
$
A=
\begin{bmatrix}
  1 & 2 & -2 \\
  2 & 5 &0\\
  3 & t&4
\end{bmatrix}
$,$B$为3阶非零矩阵且$AB=0$,则$t=$\underline{\hphantom{~~~~~~~~~~}}。

\begin{jie}
$B$为3阶非零矩阵且$AB=0$即$B$的非零列向量为$Ax=0$的解,即$Ax=0$有非零解,即$|A|=0$,把$|A|$按第三列展开。
\begin{equation*}
  |A|=
  \begin{vmatrix}
  1 & 2 & -2 \\
  2 & 5 &0\\
  3 & t&4
  \end{vmatrix}=-2\times(-1)^{1+3}
  \begin{vmatrix}
  2 & 5 \\
  3 & t
  \end{vmatrix}+4
  \begin{vmatrix}
  1 & 2  \\
  2 & 5
  \end{vmatrix}=-2(2t-15)+4=0~~~\Rightarrow~~~t=\frac{17}{2}
\end{equation*}
\end{jie}


\num 期末2014-2015 二.

设多项式$
f(x)=
\begin{vmatrix}
  2x & 3 & 1 & 2\\
  x & x & -2 & 1\\
  2 & 1 & x & 4\\
  x & 2 & 1 & 4x
\end{vmatrix}
$,分别求该多项式的三次项、常数项。

\begin{jie}
同第一题。$Matlab$算出的结果为:$f(x)=8x^{4}-14x^3+11x^2-53x+14$(作为参考)

取列为自然排列。分析得:行数按$2134$和$4231$排列时,对应的项为$x^3$。即
\begin{equation*}
(-1)^{\tau(2134)}a_{21}a_{12}a_{33}a_{44}+(-1)^{\tau(4231)}a_{41}a_{22}a_{33}a_{14}=(-12-2)x^{3}=-14x^3
\end{equation*}

同理,取列为自然排列。分析得行数按:$3142$、$3412$和$3421$排列时为常数项,即
\begin{equation*}
(-1)^{\tau(3142)}a_{31}a_{12}a_{43}a_{24}+(-1)^{\tau(3412)}a_{31}a_{42}a_{13}a_{24}+(-1)^{\tau(3421)}a_{31}a_{42}a_{23}a_{14}=-6+4+16=14
\end{equation*}
\end{jie}


\num 期末2014-2015 三.

设$A$的伴随矩阵
$
A^{*}=
\begin{bmatrix}
  2 & 0 & 0 & 0\\
  0 & 2 & 0 & 0\\
  1 & 0 & 2 & 0\\
  0 & -3 & 0 & 8
\end{bmatrix}
$,且$ABA^{-1}=BA^{-1}+3I$,求$B$。

\begin{jie}
由题得:$|A^*|=2\times2\times2\times8=64$
\begin{gather*}
ABA^{-1}=BA^{-1}+3I~~\Rightarrow~~AB=B+3A~~\Rightarrow~~A^*AB=A^*B+3A^*A\\
A^*A=|A|A^{-1}A=|A|I~~~|A^*|=||A|A^{-1}|=|A|^n|A|^{-1}=|A|^{n-1}=|A|^{4-1}=64~~~|A|=4
\end{gather*}
所以$4B=A^*B+3\times4~~~\Rightarrow~~~B=12(4-A^*)^{-1}$.

求逆的过程略。

最后的结果为:
\begin{equation*}
  \begin{bmatrix}
  6 & 0 & 0 & 0\\
  0 & 6 & 0 & 0\\
  3 & 0 & 6 & 0\\
  0 & 4.5 & 0 & -3
\end{bmatrix}
\end{equation*}
\end{jie}

\num 期末2014-2015 四.

$\lambda$为何值时,方程组$
\begin{cases}
 2x_{1}+\lambda x_{2}-x_{3}=1\\
 \lambda x_{1}-x_{2}+x_{3}=2\\
 4x_{1}+5 x_{2}-5x_{3}=-1
\end{cases}
$有无穷多组解?并在有无穷多解时,写出方程组的通解。

\begin{jie}
记$A=
\begin{bmatrix}
 2&\lambda &-1\\
 \lambda &-1&1\\
 4& 5 &-5
\end{bmatrix}
,B=\begin{bmatrix}
     1 \\ 2\\ -1
   \end{bmatrix}$,
   \begin{equation*}
   |A|=
   \begin{vmatrix}
    2&\lambda &-1\\
 \lambda &-1&1\\
 4& 5 &-5
   \end{vmatrix}
   \xlongequal{c_{3}+c_{2}}
  \begin{vmatrix}
    2&\lambda &\lambda-1\\
 \lambda &-1&0\\
 4& 5 &0
   \end{vmatrix}  =(\lambda-1)(5\lambda+4)
   \end{equation*}
   可以看出$\lambda\neq1$且$\lambda\neq-\dfrac{4}{5}$时即$|A|\neq0$时,方程有唯一解。

   $\lambda=1$时:
   \begin{align*}
 [A|B]=
 \left[
 \begin{array}{c:c}
\begin{matrix}
2 & 1 & -1 \\
  1 & -1 & 1 \\
  4 & 5 & -5
\end{matrix}&
\begin{matrix}
1  \\
 2\\
-1
\end{matrix}
\end{array}
\right]
\xrightarrow{\substack{r_{2}-\frac{1}{2}r_{1}\\ r_{3}-2r_{1}}}
{
 \left[
 \begin{array}{c:c}
\begin{matrix}
2 & 1 & -1 \\
 0 & -\frac{3}{2} & \frac{3}{2} \\
  0 & 3 & -3
\end{matrix}&
\begin{matrix}
1  \\
 \frac{3}{2}\\
-3
\end{matrix}
\end{array}
\right]
}
\xrightarrow{r_{3}+2r_{2}}
{
 \left[
 \begin{array}{c:c}
\begin{matrix}
2 & 1 & -1 \\
 0 & -\frac{3}{2} & \frac{3}{2} \\
  0 & 0 & 0
\end{matrix}&
\begin{matrix}
1  \\
 \frac{3}{2}\\
0
\end{matrix}
\end{array}
\right]
}
\xrightarrow{\text{中间略}}
{
 \left[
 \begin{array}{c:c}
\begin{matrix}
1 & 0 & 0 \\
 0 & 1 & 1 \\
  0 & 0 & 0
\end{matrix}&
\begin{matrix}
1  \\
1\\
0
\end{matrix}
\end{array}
\right]
}
   \end{align*}
所以$\lambda=1$时有无穷多解,$x_{1}=1,x_{2}=x_{3}-1$.

$\lambda=-\dfrac{4}{5}$时,$r(A)\neq r(A,B)$,此时无解。
\end{jie}

\num 期末2016-2017 一2.

设$A$的伴随矩阵$
A^{*}=
\begin{bmatrix}
  1 & 2 & 3 & 4\\
  0 & 2 & 3 & 4\\
  0 & 0 & 2 & 3\\
  0 & 0 & 0 & 2
\end{bmatrix}
$,则$r(A^{2}-2A)=$\underline{\hphantom{~~~~~~~~~~}}。

\begin{jie}
$|A^*|=|A|^{n-1}$得:$|A^*|=2^3=|A|^{3}$.所以$|A|=2\neq0$。即$A$可逆。

所以$r(A^{2}-2A)=r(A(A-2))=r(A-2)=r(|A|(A^*)^{-1}-2)=r((A^*)^{-1}-E)=3$.(求逆和秩的过程略)
\end{jie}

\num 期末2016-2017 二2.

设$
A=
\begin{bmatrix}
  1 & 2 & 3 \\
  0 & 1 & 3\\
  0 & 0 & 1
\end{bmatrix}
$,$B$为三阶矩阵,且满足方程$A^{*}BA=I+2A^{-1}B$,求矩阵$B$。

\begin{jie}
由题得:$|A|=1$,$A^*=|A|A^{-1}=A^{-1}$.对题中方程两边同时左乘$A$得:
\begin{align*}
BA&=A+2B\\
B&=A(A-2E)^{-1}=A=
\begin{bmatrix}
  1 & 2 & 3 \\
  0 & 1 & 3\\
  0 & 0 & 1
\end{bmatrix}
\begin{bmatrix}
  -1 & 2 & 3 \\
  0 & -1 & 3\\
  0 & 0 & -1
\end{bmatrix}^{-1}=\begin{bmatrix}
  -1 & -4 & -18 \\
  0 & -1 & -6\\
  0 & 0 & -1
\end{bmatrix}
\end{align*}

(求逆的过程略。$(A-2E)^{-1}=\begin{bmatrix}
  -1 & -2 & -9 \\
  0 & -1 & -3\\
  0 & 0 & -1
\end{bmatrix}$)
\end{jie}

\num 期末2017-2018 一3.

设$
A=
\begin{bmatrix}
  2 & 0 & 0 \\
  1 & 2 & 0 \\
  1 & 2 & 2
\end{bmatrix}
$,记$A*$是$A$的伴随矩阵,则$(A^{*})^{-1}=$\underline{\hphantom{~~~~~~~~~~}}。

\begin{jie}
$(A^ {*})^{-1}=(|A|A^{-1})^{-1}=\dfrac{A}{|A|}$,由题得:$|A|=8$
\end{jie}
\num 期末2018-2019 一1.

设$A$为5阶方阵满足$|A|=2$,$A^{*}$是$A$的伴随矩阵,则$|2A^{-1}A^{*}A^{T}|=$\underline{\hphantom{~~~~~~~~~~}}。

\begin{jie}
原式=
\begin{equation*}
2^{5}|A^{-1}|\cdot|A^*|\cdot|A^T|=2^{5}\cdot|A|^{-1}\cdot|A|^{5-1}\cdot|A|=2^9=512
\end{equation*}
\end{jie}

\num 期末2018-2019 一3.

设$A$为$m$阶阵,存在非零的$m$维列向量$B$,使$AB=0$的充分必要条件是\underline{\hphantom{~~~~~~~~~~}}。

\begin{jie}
$B$非零,说明$Ax=0$有非零解,由存在唯一性定理:$|A|=0$,或$r(A)<m$。
\end{jie}

\end{document}  