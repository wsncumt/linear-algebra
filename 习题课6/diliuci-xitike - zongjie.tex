\documentclass{article}
\usepackage[space,fancyhdr,fntef]{ctexcap}
\usepackage[namelimits,sumlimits,nointlimits]{amsmath}
\usepackage[bottom=25mm,top=25mm,left=25mm,right=15mm,centering]{geometry}
\usepackage{xcolor}
\usepackage{paralist}%列表宏包
\usepackage{arydshln}%234页,虚线表格宏包
\pagestyle{fancy} \fancyhf{}
\fancyhead[OL]{~~~班序号:\hfill 学院:\hfill 学号:\hfill 姓名:王松年~~~ \thepage}
%\usepackage{parskip}
%\usepackage{indentfirst}
\usepackage{graphicx}%插图宏包,参见手册318页
\begin{document}

\newcounter{num} \renewcommand{\thenum}{\arabic{num}.} \newcommand{\num}{\refstepcounter{num}\text{\thenum}}

\hphantom{~~}\hfill {\zihao{3}\heiti 第六次习题课} \hfill\hphantom{~~}

\hphantom{~~}\hfill {\zihao{4}\heiti 知识点} \hfill\hphantom{~~}

\num 逆序数的定义。奇排列,偶排列。

\num 行列式的定义:$\det A=\sum(-1)^{\tau(j_{1}j_{2}\cdots j_{n})}a_{1j_{1}}a_{2j_{2}}\cdots a_{nj_{n}}$。

\num $n$阶行列式$D=|a_{ij}|$的一般项可以记为$(-1)^{\tau(i_{1}i_{2}\cdots i_{n})+\tau(j_{1}j_{2}\cdots j_{n})}a_{i_{1}j_{1}}a_{i_{2}j_{2}}\cdots a_{i_{n}j_{n}}$

\num 设$A=(a_{ij})$是$n$阶方阵,则$\det A=\sum (-1)^ {\tau(i_{1}i_{2}\cdots i_{n})+\tau(j_{1}j_{2}\cdots j_{n})}a_{i_{1}j_{1}}a_{i_{2}j_{2}}\cdots a_{i_{n}j_{n}}$,其中$i_{1}i_{2}\cdots i_{n}$是一个取定的$n$级排列,$j_{1}j_{2}\cdots j_{n})$取遍$n$级排列。(或者$j_ {1}j_{2}\cdots j_{n}$是一个取定的$n$级排列,$i_{1}i_{2}\cdots i_{n})$取遍$n$级排列)

\num 伴随矩阵。性质:

(1)对任意$n$阶方阵:$AA^{*}=|A|E=A^{*}A$.$|AA^{*}|=||A|E|=|A|^{n}|E|=|A|^{n}$.

\num 设$A$是$n$阶方阵,若$|A|\neq0$,则称$A$非奇异。若$|A|=0$,则称$A$为奇异矩阵。

\num $n$阶矩阵$A$可逆的充要条件是$A$为非奇异矩阵。$A^{-1}=\dfrac{1}{|A|}A^{*}$。对于二阶可逆矩阵$A=
\begin{bmatrix}
  a & b \\
  c & d
\end{bmatrix},|A|\neq0
$,则$A^{-1}=\dfrac{1}{|A|}
\begin{bmatrix}
  d & -b \\
  -c & a
\end{bmatrix}$.

\num 克莱姆法则:$A_{n\times n}X=\beta$系数行列式$\det A\neq0$时,方程组$AX=\beta$恰有唯一解
\begin{equation*}
  x_{j}=\dfrac{\det A_{j}}{\det A},j=1,2,\cdots,n
\end{equation*}
其中$A_{j}$是把系数矩阵$A$的第$j$列$\alpha_{j}$换为常数项$\beta$后得到的方阵。
\end{document}  