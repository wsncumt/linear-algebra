\documentclass{article}
\usepackage[space,fancyhdr,fntef]{ctexcap}
\usepackage[namelimits,sumlimits,nointlimits]{amsmath}
\usepackage[bottom=25mm,top=25mm,left=25mm,right=15mm,centering]{geometry}
\usepackage{xcolor}
\usepackage{arydshln}%234页,虚线表格宏包
\pagestyle{fancy} \fancyhf{}
\fancyhead[OL]{~~~班序号:\hfill 学院:\hfill 学号:\hfill 姓名:王松年~~~ \thepage}
%\usepackage{parskip}
%\usepackage{indentfirst}
\usepackage{graphicx}%插图宏包,参见手册318页
\usepackage{mathdots}%反对角省略号
\usepackage{extarrows}%等号上加文字
\begin{document}

\newcounter{num} \renewcommand{\thenum}{\arabic{num}.} \newcommand{\num}{\refstepcounter{num}\text{\thenum}}

\newenvironment{jie}{\kaishu\zihao{-5}\color{blue}{\noindent\em 解:}\par}{\hfill $\diamondsuit$\par}

\newenvironment{zhengming}{\kaishu\zihao{-5}\color{blue}{\noindent\em 证明:}\par}{\hfill $\diamondsuit$\par}

\hphantom{~~}\hfill {\zihao{3}\heiti 第八次习题课} \hfill\hphantom{~~}

\hphantom{~~}\hfill {\zihao{4}\heiti 群文件《期中$\&$期末试题》} \hfill\hphantom{~~}

{\heiti \zihao{4} 期末试题}

\num 期末2014-2015 五.

设$
\alpha_{1}=
\begin{bmatrix}
1\\ 1 \\ 2\\ 3
\end{bmatrix},
\alpha_{2}=
\begin{bmatrix}
1\\ -1 \\ 1\\ 1
\end{bmatrix},
\alpha_{3}=
\begin{bmatrix}
1\\ 3 \\ 3\\ 5
\end{bmatrix},
\alpha_{4}=
\begin{bmatrix}
4\\ -2 \\ 5\\ 6
\end{bmatrix}
$.

(1)求向量组$\alpha_{1},\alpha_{2},\alpha_{3},\alpha_{4}$的秩与一个最大线性无关组;

(2)将其余向量用极大线性无关组线性表示。

\begin{jie}
\begin{align*}
(\alpha_{1},\alpha_{2},\alpha_{3},\alpha_{4})&=
\begin{bmatrix}
  1 & 1 & 1 & 4 \\
  1 & -1 & 3 & -2\\
  2 & 1 & 3 & 5\\
  3 & 1 & 5 & 6
\end{bmatrix}
\xrightarrow{\substack{r_{2}-r_{1}\\ r_{3}-2r_{1}\\ r_4-3r
-1}}
{
\begin{bmatrix}
  1 & 1 & 1 & 4 \\
  0 & -2 & 2 & -6\\
  0 & -1 & 1 & -3\\
    0 & -2 & 2 & -6
\end{bmatrix}
}
\xrightarrow{\substack{r_{3}-\frac{1}{2}r_{2}\\ r_{4}-r_{2}}}
{
\begin{bmatrix}
  1 & 1 & 1 & 4 \\
  0 & -2 & 2 & -6\\
  0 & 0 & 0 & 0\\
 0 & 0 & 0 & 0
\end{bmatrix}
}
\xrightarrow{\substack{r_{2}\times\frac{1}{2}}}
{
\begin{bmatrix}
  1 & 1 & 1 & 4 \\
  0 & 1 & -1 & 3\\
  0 & 0 & 0 & 0\\
 0 & 0 & 0 & 0
\end{bmatrix}
}\\
\xrightarrow{\substack{r_{1}-r_{2}}}&
{
\begin{bmatrix}
  1 & 0 & 2 & 1 \\
  0 & 1 & -1 & 3\\
  0 & 0 & 0 & 0\\
 0 & 0 & 0 & 0
\end{bmatrix}
}
\end{align*}

(1)$r(\alpha_{1},\alpha_{2},\alpha_{3},\alpha_{4})=2$,极大线性无关组为:$(\alpha_{1},\alpha_{2}),(\alpha_{1},\alpha_{3}),(\alpha_{1},\alpha_{4}),(\alpha_{2},\alpha_{3}),(\alpha_{2},\alpha_{4}),(\alpha_{3},\alpha_{4})$.\textcolor[rgb]{1.00,0.00,0.00}{(任写一个即可。)}

(2)取$(\alpha_ {1},\alpha_{2})$,由最简阶梯型矩阵可以看出:
\begin{equation*}
\begin{cases}
 \alpha_3=2\alpha_1-\alpha_2\\
  \alpha_4=\alpha_1+3\alpha_2
\end{cases}
\end{equation*}
\end{jie}

\num 期末2014-2015 七2.

设$X_{0}$是线性方程组$Ax=b~(b\neq0)$的一个解,$X_{1},X_{2}$是导出组$Ax=0$的一个基础解系。令$\xi_{0}=X_{0},\xi_{1}=X_{0}+X_{1},\xi_{2}=X_{0}+X_{2}$,证明:$\xi_{0},\xi_{1},\xi_{2}$线性无关。

\begin{zhengming}
设
\begin{equation*}
k_{0}\xi_0+k_1\xi_1+k_2\xi_2=0\tag{$1$}
\end{equation*}
要证明$\xi_{0},\xi_{1},\xi_{2}$线性无关,根据定义,只需证明$k_1=k_2=k_3=0$。

由题得:$AX_0=b,AX_1=AX_2=0$,因为$X_1,X_2$为$AX=0$的基础解系,所以$X_1,X_2$线性无关。

把题中条件代入(1)式:
\begin{equation*}
k_1X_0+k_2X_0+k_2X_1+k_3X_0+k_3X_2= (k_1+k_2+k_3)X_0+k_2X_1+k_3X_2=0\tag{$2$}
\end{equation*}
(2)式两边同时左乘矩阵$A$:
\begin{equation*}
(k_1+k_2+k_3)AX_0+k_2AX_1+k_3AX_2=(k_1+k_2+k_3)b=0
\end{equation*}
因为$b\neq0$,所以$k_1+k_2+k_3=0$.代入(2)式得:$k_2X_1+k_3X_2=0$,因为$X_1,X_2$线性无关,所以$k_2=k_3=0$。所以$k_1=0-k_2-k_3=0$。

所以$\xi_{0},\xi_{1},\xi_{2}$线性无关。

\end{zhengming}

\num 期末2015-2016 二3.

设向量组$\alpha_{1}=(1,-1,2,4),\alpha_{2}=(0,3,1,2),\alpha_{3}=(3,0,7,14),\alpha_{4}=(1,-1,2,0),\alpha_{5}=(2,1,5,6)$,求向量组的秩、极大线性无关组,并将其余向量由极大无关组线性表示出。

\begin{jie}
\begin{align*}
&(\alpha_1,\alpha_2,\alpha_3,\alpha_4)=
\begin{bmatrix}
  1 & 0 & 3 &1\\
  -1 & 3 & 0&-1\\
  2 & 1 & 7&2\\
  4 & 2 &14&0
\end{bmatrix}
\xrightarrow{\substack{r_{2}+r_{1}\\ r_3-2r_1 \\ r_4-4r_1}}
{
\begin{bmatrix}
  1 & 0 & 3 &1\\
  0 & 3 & 3&0\\
  0 & 1 & 1&0\\
  0 & 2 &2&-4
\end{bmatrix}
}
\xrightarrow{\substack{ r_3-\frac{1}{3}r_2 \\ r_4-\frac{2}{3}r_2}}
{
\begin{bmatrix}
  1 & 0 & 3 &1\\
  0 & 3 & 3&0\\
  0 & 0 & 0&0\\
  0 & 0 &0&-4
\end{bmatrix}
}
\xrightarrow{\substack{ r_2\times\frac{1}{3} \\ r_4\times\left(-\frac{1}{4}\right)}}
{
\begin{bmatrix}
  1 & 0 & 3 &1\\
  0 & 1 & 1&0\\
  0 & 0 & 0&0\\
  0 & 0 &0&1
\end{bmatrix}
}\\
&
\xrightarrow{\substack{ r_1-r_4}}
{
\begin{bmatrix}
  1 & 0 & 3 &0\\
  0 & 1 & 1&0\\
  0 & 0 & 0&0\\
  0 & 0 &0&1
\end{bmatrix}
}
\xrightarrow{\substack{ r_3\leftrightarrow r_4}}
{
\begin{bmatrix}
  1 & 0 & 3 &0\\
  0 & 1 & 1&0\\
  0 & 0 & 0&1\\
  0 & 0 &0&0
\end{bmatrix}
}
\end{align*}
所以该向量组的秩为$3$,极大线性无关组为$(\alpha_1,\alpha_2,\alpha_4),(\alpha_1,\alpha_3,\alpha_4),(\alpha_2,\alpha_3,\alpha_4)$.
由$(\alpha_1,\alpha_2,\alpha_4)$表示$\alpha_3$:由最简阶梯型矩阵可以看出$\alpha_3=3\alpha_1+\alpha_2+0\alpha_4=3\alpha_1+\alpha_2$。
\end{jie}

\num 期末2016-2017 一3.

已知线性方程组
$
\begin{cases}
 x_{1}+2x_{2}+x_{3}=2\\
 ax_{1}-x_{2}-2x_{3}=-3
\end{cases}
$与线性方程$ax_{2}+x_{3}=1$有公共的解,则$a$的取值范围为\underline{\hphantom{~~~~~~~~~~}}。\\

\num 期末2016-2017 二3.

设向量组$\alpha_{1}=(3,1,4,3)^{T},\alpha_{2}=(1,1,2,1)^{T},\alpha_{3}=(0,1,1,0)^{T},\alpha_{4}=(2,2,4,2)^{T}$,求向量组的所有的极大线性无关组。

\begin{jie}
\begin{align*}
&(\alpha_1,\alpha_2,\alpha_3,\alpha_4)=
\begin{bmatrix}
   3 &1 & 0 &2\\
  1 & 1 & 1&2\\
  4 & 2 & 1&4\\
   3 &1 & 0 &2
\end{bmatrix}
\xrightarrow{\substack{r_{2}\leftrightarrow r_{1}}}
{
\begin{bmatrix}
  1 & 1 & 1&2 \\
 3 &1 & 0 &2 \\
  4 & 2 & 1&4\\
   3 &1 & 0 &2
\end{bmatrix}
}
\xrightarrow{\substack{r_{4}- r_{2}}}
{
\begin{bmatrix}
  1 & 1 & 1&2 \\
 3 &1 & 0 &2 \\
  4 & 2 & 1&4\\
   0 &0 & 0 &0
\end{bmatrix}
}
\xrightarrow{\substack{r_{2}-3 r_{1}\\ r_3-4r_1}}
{
\begin{bmatrix}
  1 & 1 & 1&2 \\
 0 &-2 & -3 &-4 \\
 0 &-2 & -3 &-4\\
   0 &0 & 0 &0
\end{bmatrix}
}\\
&\xrightarrow{\substack{r_{3}-r_{2}}}
{
\begin{bmatrix}
  1 & 1 & 1&2 \\
 0 &-2 & -3 &-4 \\
0 &0 & 0 &0\\
   0 &0 & 0 &0
\end{bmatrix}
}
\end{align*}
所以该向量组的秩为$2$,极大线性无关组为$(\alpha_1,\alpha_2),(\alpha_1,\alpha_3),(\alpha_1,\alpha_4),(\alpha_2,\alpha_3),(\alpha_3,\alpha_4)$.
\end{jie}

\num 期末2017-2018 一4.

已知3阶方阵$A$的秩为2,设$\alpha_ {1}=(2,2,0)^{T},\alpha_{2}=(3,3,1)^{T}$是非齐次线性方程组$Ax=b$的解,则导出$Ax=0$的基础解系为\underline{\hphantom{~~~~~~~~~~}}。

\begin{jie}
因为$\alpha_ {1},\alpha_{2}$是非齐次线性方程组$Ax=b$的解,所以$A\alpha_1=b,A\alpha_2=b$,且$\alpha_ {1},\alpha_{2}$不相等,所以$\alpha_1-\alpha_2$是$AX=0$的基础解系。
\end{jie}

\num 期末2018-2019 二3.

设矩阵
$
A=
\begin{bmatrix}
  1 & 1 & 1 & 1\\
  0 & 1 & -1& b\\
  2 & 3 & a & 3\\
  3 & 5 &1 &5
\end{bmatrix}
$,$A^{*}$是$A$的伴随矩阵,求$r(A),r(A^{*})$和$A$的列向量组的极大线性无关组。

\begin{jie}
\begin{align*}
A\xrightarrow{\substack{r_{3}-2 r_{1}\\ r_4-3r_1}}
{
\begin{bmatrix}
 1 & 1 & 1 & 1\\
  0 & 1 & -1& b\\
  0 & 1& a-2 & 1\\
  0 & 2 &-2 &2
\end{bmatrix}
}\xrightarrow{\substack{r_{2}\leftrightarrow r_4}}
{
\begin{bmatrix}
 1 & 1 & 1 & 1\\
  0 & 2 &-2 &2\\
  0 & 1& a-2 & 1\\
  0 & 1 & -1& b
\end{bmatrix}
}\xrightarrow{\substack{r_{3}-\frac{1}{2} r_2 \\r_{4}-\frac{1}{2} r_2}}
{
\begin{bmatrix}
 1 & 1 & 1 & 1\\
  0 & 2 &-2 &2\\
  0 & 0& a-3 & 0\\
  0 & 0 & 0& b-1
\end{bmatrix}
}
\end{align*}
\begin{equation*}
r(A^*)=
\begin{cases}
n,~r(A)=n;\\
1,~r(A)=n-1\\
0,~r(A)<n-1.
\end{cases}
\end{equation*}
上式中:$n$为$A$的阶数。

记$A=(\alpha_1,\alpha_2,\alpha_3,\alpha_4)$

(1)$a=3$且$b=1$时:$r(A)=2<4-1=3$,$r(A^*)=0$。极大线性无关组:$(\alpha_1,\alpha_2),(\alpha_1,\alpha_3),(\alpha_1,\alpha_4),(\alpha_2,\alpha_3),(\alpha_3,\alpha_4)$.

(2)$a=3$且$b\neq1$时:$r(A)=3=4-1$,$r(A^*)=1$。极大线性无关组:$(\alpha_1,\alpha_2,\alpha_4),(\alpha_1,\alpha_3,\alpha_4),(\alpha_2,\alpha_3,\alpha_4)$.

(3)$a\neq3$且$b=1$时:$r(A)=3=4-1$,$r(A^*)=1$。极大线性无关组:$(\alpha_1,\alpha_2,\alpha_3),(\alpha_1,\alpha_3,\alpha_4),(\alpha_2,\alpha_3,\alpha_4)$.

(4)$a\neq3$且$b\neq1$时:$r(A)=4$,$r(A^*)=4$。极大线性无关组:$(\alpha_1,\alpha_2,\alpha_3,\alpha_4)$.
\end{jie}

\num 期末2019-2020 二2.

求线性方程组$
\begin{cases}
x_{1}+3x_{2}+2x_{3}+3x_{4}=0\\
2x_{1}+4x_{2}+x_{3}+3x_{4}=0\\
2x_{1}+4x_{2}+4x_{4}=0\\
\end{cases}
$的一个基础解系。

\begin{jie}
由题得:增广矩阵
\begin{align*}
A&=
\begin{bmatrix}
1 & 3 & 2 &3\\
2 & 4 & 1 & 3\\
2 & 4 & 0 & 4
\end{bmatrix}
\xrightarrow{\substack{r_{2}-2r_1\\ r_3-2r_1}}
{
\begin{bmatrix}
1 & 3 & 2 &3\\
0 & -2 & -3 & -3\\
0 & -2 & -4 & -2
\end{bmatrix}
}
\xrightarrow{\substack{r_{3}-r_2}}
{
\begin{bmatrix}
1 & 3 & 2 &3\\
0 & -2 & -3 & -3\\
0 & 0 & -1 & 1
\end{bmatrix}
}
\xrightarrow{\substack{r_{1}+2r_3 \\ r_2-3r_3}}
{
\begin{bmatrix}
1 & 3 & 0 & 5\\
0 & -2 & 0 & -6\\
0 & 0 & -1 & 1
\end{bmatrix}
}\\
&
\xrightarrow{\substack{r_{2}\times\left(-\frac{1}{2}\right) \\ r_3\times\left(-1\right)}}
{
\begin{bmatrix}
1 & 3 & 0 & 5\\
0 & 1 & 0 & 3\\
0 & 0 & 1 & -1
\end{bmatrix}
}\xrightarrow{\substack{r_{1}-3r_2}}
{
\begin{bmatrix}
1 & 0 & 0 & -4\\
0 & 1 & 0 & 3\\
0 & 0 & 1 & -1
\end{bmatrix}
}
\end{align*}
所以$x_{1}=4x_4,x_2=-3x_4,x_3=x_4$,令$x_4=1$,得基础解系:$\xi=[4,-3,1,1]^T$。
\end{jie}

\num 期末2019-2020 三1.

设向量组$\alpha_{1}=(1,-4,-3)^{T},\alpha_{2}=(-3,6,7)^{T},\alpha_{3}=(-4,-2,6)^{T},\alpha_{4}=(3,3,-4)^{T}$,求向量组的秩,并写出一个极大线性无关组,并将其余向量由极大无关组线性表示出。

\begin{jie}
由题得:
\begin{align*}
(\alpha_ {1},\alpha_{2},\alpha_{3},\alpha_{4})&=
\begin{bmatrix}
  1 & -3 & -4 & 3 \\
  -4 & 6 & -2 & 3\\
  -3 & 7 & 6 & -4
\end{bmatrix}
\xrightarrow{\substack{r_{2}+4r_1 \\ r_3+3r_1}}
{
\begin{bmatrix}
  1 & -3 & -4 & 3 \\
  0 & -6 & -18 & 15\\
  0 & -2 & -6 & 5
\end{bmatrix}
}
\xrightarrow{\substack{r_{3}-\frac{1}{3}r_2}}
{
\begin{bmatrix}
  1 & -3 & -4 & 3 \\
  0 & -6 & -18 & 15\\
  0 & 0 & 0 & 0
\end{bmatrix}
}\\ &\xrightarrow{\substack{r_{2}\times\left(-\frac{1}{6}\right)}}
{
\begin{bmatrix}
  1 & -3 & -4 & 3 \\
  0 & 1 & 3 & -2.5\\
  0 & 0 & 0 & 0
\end{bmatrix}
}
\xrightarrow{\substack{r_{1}+3r_2}}
{
\begin{bmatrix}
  1 & 0 & 5 & -4.5 \\
  0 & 1 & 3 & -2.5\\
  0 & 0 & 0 & 0
\end{bmatrix}
}
\end{align*}
所以$r(\alpha_ {1},\alpha_{2},\alpha_{3},\alpha_{4})=2$,极大线性无关组有两个向量:$(\alpha_ {1},\alpha_{2}),(\alpha_ {1}\alpha_{3}),(\alpha_ {1},\alpha_{4}),(\alpha_{2},\alpha_{3}),(\alpha_{2},\alpha_{4}),(\alpha_{3},\alpha_{4})$.\textcolor[rgb]{1.00,0.00,0.00}{(任写一个即可)}

以$(\alpha_ {1},\alpha_{2})$为例:$\alpha_3=5\alpha_1+3\alpha_2,\alpha_4=-4.5\alpha_1-2.5\alpha_2$。
\end{jie}
\end{document}  