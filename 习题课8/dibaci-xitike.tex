\documentclass{article}
\usepackage[space,fancyhdr,fntef]{ctexcap}
\usepackage[namelimits,sumlimits,nointlimits]{amsmath}
\usepackage[bottom=25mm,top=25mm,left=25mm,right=15mm,centering]{geometry}
\usepackage{xcolor}
\usepackage{arydshln}%234页,虚线表格宏包
\usepackage{mathdots}%反对角省略号
\pagestyle{fancy} \fancyhf{}
\fancyhead[OL]{~~~班序号:\hfill 学院:\hfill 学号:\hfill 姓名:王松年~~~ \thepage}
%\usepackage{parskip}
%\usepackage{indentfirst}
\usepackage{graphicx}%插图宏包,参见手册318页
\begin{document}

\newcounter{num} \renewcommand{\thenum}{\arabic{num}.} \newcommand{\num}{\refstepcounter{num}\text{\thenum}}

\hphantom{~~}\hfill {\zihao{3}\heiti 第八次习题课} \hfill\hphantom{~~}

\hphantom{~~}\hfill {\zihao{4}\heiti 群文件《期中$\&$期末试题》} \hfill\hphantom{~~}

{\heiti \zihao{4} 期末试题}

\num 期末2014-2015 五.

设$
\alpha_{1}=
\begin{bmatrix}
1\\ 1 \\ 2\\ 3
\end{bmatrix},
\alpha_{2}=
\begin{bmatrix}
1\\ -1 \\ 1\\ 1
\end{bmatrix},
\alpha_{3}=
\begin{bmatrix}
1\\ 3 \\ 3\\ 5
\end{bmatrix},
\alpha_{4}=
\begin{bmatrix}
4\\ -2 \\ 5\\ 6
\end{bmatrix}
$.

(1)求向量组$\alpha_{1},\alpha_{2},\alpha_{3},\alpha_{4}$的秩与一个最大线性无关组;

(2)将其余向量用极大线性无关组线性表示。\\

\num 期末2014-2015 七2.

设$X_{0}$是线性方程组$Ax=b~(b\neq0)$的一个解,$X_{1},X_{2}$是导出组$Ax=0$的一个基础解系。令$\xi_{0}=X_{0},\xi_{1}=X_{0}+X_{1},\xi_{2}=X_{0}+X_{2}$,证明:$\xi_{0},\xi_{1},\xi_{2}$线性无关。\\

\num 期末2015-2016 二3.

设向量组$\alpha_{1}=(1,-1,2,4),\alpha_{2}=(0,3,1,2),\alpha_{3}=(3,0,7,14),\alpha_{4}=(1,-1,2,0),\alpha_{5}=(2,1,5,6)$,求向量组的秩、极大线性无关组,并将其余向量由极大无关组线性表示出。\\

\num 期末2016-2017 一3.

已知线性方程组
$
\begin{cases}
 x_{1}+2x_{2}+x_{3}=2\\
 ax_{1}-x_{2}-2x_{3}=-3
\end{cases}
$与线性方程$ax_{2}+x_{3}=1$有公共的解,则$a$的取值范围为\underline{\hphantom{~~~~~~~~~~}}。\\

\num 期末2016-2017 二3.

设向量组$\alpha_{1}=(3,1,4,3)^{T},\alpha_{2}=(1,1,2,1)^{T},\alpha_{3}=(0,1,1,0)^{T},\alpha_{4}=(2,2,4,2)^{T}$,求向量组的所有的极大线性无关组。\\

\num 期末2017-2018 一4.

已知3阶方阵$A$的秩为2,设$\alpha_ {1}=(2,2,0)^{T},\alpha_{2}=(3,3,1)^{T}$是非齐次线性方程组$Ax=b$的解,则导出$Ax=0$的基础解系为\underline{\hphantom{~~~~~~~~~~}}。\\

\num 期末2018-2019 二3.

设矩阵
$
A=
\begin{bmatrix}
  1 & 1 & 1 & 1\\
  0 & 1 & -1& b\\
  2 & 3 & a & 3\\
  3 & 5 &1 &5
\end{bmatrix}
$,$A^{*}$是$A$的伴随矩阵,求$r(A),r(A^{*})$和$A$的列向量组的极大线性无关组。\\

\num 期末2019-2020 二2.

求线性方程组$
\begin{cases}
x_{1}+3x_{2}+2x_{3}+3x_{4}=0\\
2x_{1}+4x_{2}+x_{3}+3x_{4}=0\\
2x_{1}+4x_{2}+4x_{4}=0\\
\end{cases}
$的一个基础解系。\\

\num 期末2019-2020 三1.

设向量组$\alpha_{1}=(1,-4,-3)^{T},\alpha_{2}=(-3,6,7)^{T},\alpha_{3}=(-4,-2,6)^{T},\alpha_{4}=(3,3,-4)^{T}$,求向量组的秩,并写出一个极大线性无关组,并将其余向量由极大无关组线性表示出。\\
\end{document}  