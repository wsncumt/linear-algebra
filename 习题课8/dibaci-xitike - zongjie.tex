\documentclass{article}
\usepackage[space,fancyhdr,fntef]{ctexcap}
\usepackage[namelimits,sumlimits,nointlimits]{amsmath}
\usepackage[bottom=25mm,top=25mm,left=25mm,right=15mm,centering]{geometry}
\usepackage{xcolor}
\usepackage{paralist}%列表宏包
\usepackage{arydshln}%234页,虚线表格宏包
\pagestyle{fancy} \fancyhf{}
\fancyhead[OL]{~~~班序号:\hfill 学院:\hfill 学号:\hfill 姓名:王松年~~~ \thepage}
%\usepackage{parskip}
%\usepackage{indentfirst}
\usepackage{graphicx}%插图宏包,参见手册318页
\begin{document}

\newcounter{num} \renewcommand{\thenum}{\arabic{num}.} \newcommand{\num}{\refstepcounter{num}\text{\thenum}}

\hphantom{~~}\hfill {\zihao{3}\heiti 第七次习题课} \hfill\hphantom{~~}

\hphantom{~~}\hfill {\zihao{4}\heiti 知识点} \hfill\hphantom{~~}


\num 极大线性无关组:在不全为的向量组$\alpha_{1},\alpha_{2},\cdots,\alpha_{s}$中取出一组线性无关的向量$\alpha_{i_{1}},\alpha_{i_{2}},\cdots,\alpha_{i_{k}}$,若任意添加$\beta\in\{\alpha_{1},\cdots,\alpha_{s}\}$得到的向量组$\alpha_ {i_{1}},\alpha_{i_{2}},\cdots,\alpha_{i_{k}},\beta$是线性相关的,那么我们称$\alpha_ {i_{1}},\alpha_{i_{2}},\cdots,\alpha_{i_{k}}$是$\alpha_ {1},\alpha_{2},\cdots,\alpha_{s}$的一个极大线性无关组。向量组$\alpha_{1},\alpha_{2},\cdots,\alpha_{s}$的极大无关组所含向量的个数称为向量组的秩。

\num $\alpha_{i_{1}},\alpha_{i_{2}},\cdots,\alpha_{i_{k}}$是$\alpha_ {1},\alpha_{2},\cdots,\alpha_{s}$一个极大线性无关组的充分必要条件是$k=r(\alpha_{i_{1}},\alpha_{i_{2}},\cdots,\alpha_{i_{k}})=r(\alpha_ {1},\alpha_{2},\cdots,\alpha_{s})$.把$\alpha_ {1},\alpha_{2},\cdots,\alpha_{s}$化为阶梯型矩阵后,不全为0的行的首个非零元对应的向量放一起就得到的是极大线性无关组。

\num 向量组$\alpha_{1},\alpha_{2},\cdots,\alpha_{s}$中一部分向量(部分向量组)线性相关,则整个向量组线性相关。

推论:部分相关则整体相关,反之不成立。整体无关则部分无关,反之不成立。

\num 向量组的秩:向量组$\alpha_{1},\alpha_{2},\cdots,\alpha_{s}$的极大线性无关组所含向量的个数称为向量组的秩。记为$r(\alpha_{1},\alpha_{2},\cdots,\alpha_{s})$。

\num 向量组$\alpha_{1},\alpha_{2},\cdots,\alpha_{s}(s\geq 2)$线性相关当且仅当其中至少有一个向量是其余$s-1$个向量的线性组合。

\num 如何求$Ax=0$的通解?

(1)$(A,0)$化为最简阶梯型矩阵;

(2)找出自由变量,假设自由变量是$x_{r+1},x_{r+2},x_{n}$;

(3)写出基础解系(基础解系是解的极大线性无关组)

$\xi_{i}$:取$\xi_{i}=1$,其余自由变量取0得到的解

则通解为$c_{1}\xi_{1}+c_{2}x_{2}+\cdots+c_{n-r}\xi_{n-r}$,其中$c_{1},c_{2},\cdots,c_{n-r}$是任意常数。

\num 称齐次方程$Ax=0$是$Ax=\beta$的导出组。

(1)$\eta$是$Ax=\beta$的一个解,$\xi$是$Ax=0$的解,则$\eta+\xi$也是$Ax=\beta$的一个解。

(2)$\eta_{1},\eta_{2}$是$Ax=\beta$的解,则$\eta_{1}-\eta_{2}$是$Ax=0$的解。

(3)$\eta$是$Ax=\beta$的一个(特)解,则$Ax=\beta$的(通解)任意解都可以表示为$\eta+\xi$的形式,其中$\xi$是$Ax=0$的任意一个解。
\end{document}  