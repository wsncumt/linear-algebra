\documentclass{article}
\usepackage[space,fancyhdr,fntef]{ctexcap}
\usepackage[namelimits,sumlimits,nointlimits]{amsmath}
\usepackage[bottom=25mm,top=25mm,left=25mm,right=15mm,centering]{geometry}
\usepackage{xcolor}
\usepackage{arydshln}%234页,虚线表格宏包
\pagestyle{fancy} \fancyhf{}
\fancyhead[OL]{~~~班序号:\hfill 学院:\hfill 学号:\hfill 姓名:王松年~~~ \thepage}
%\usepackage{parskip}
%\usepackage{indentfirst}
\usepackage{graphicx}%插图宏包,参见手册318页
\usepackage{mathdots}%反对角省略号
\begin{document}

\newcounter{num} \renewcommand{\thenum}{\arabic{num}.} \newcommand{\num}{\refstepcounter{num}\text{\thenum}}

\hphantom{~~}\hfill {\zihao{3}\heiti 期中2015-2016} \hfill\hphantom{~~}

%期中2015-2016

%期中2015-2016 一1.
\num 已知$A=
\begin{bmatrix}
  3 & 0 & 4 & 0 \\
  2 & 2 & 2 & 2 \\
  0 & -7 & 0 & 0 \\
  5 & 3 & -2 & 2 \\
\end{bmatrix}
$,则代数余子式之和$A_{41}+A_{42}+A_{43}+A_{44}=$\underline{\hphantom{~~~~~~~~~~}}。\\

%期中2015-2016 一2.
\num 设$f(x)=
\begin{vmatrix}
  2x & x & 1 & 2\\
  1 & x & 1 & -1\\
  3 & 2 & x & 1\\
  1 & 1 & 1 & x\\
\end{vmatrix}
$,则$x^{3}$的系数为\underline{\hphantom{~~~~~~~~~~}}。\\

%期中2015-2016 一3.
\num 设$a,b,c$满足方程
$
\begin{vmatrix}
  1 & a & b & c\\
  a & 1 & 0 & 0\\
  b & 0 & 1 & 0\\
  c & 0 & 0 & 1\\
\end{vmatrix}=1
$,则$abc=$\underline{\hphantom{~~~~~~~~~~}}。\\

%期中2015-2016 一4.
\num 设$A=
\begin{bmatrix}
  2 & 0 \\
  1 & 4 
\end{bmatrix}
$,若$B=2BA-3I$,其中$I$为单位矩阵,则$|B|=$\underline{\hphantom{~~~~~~~~~~}}。\\

%期中2015-2016 一5.
\num 若$A$为4阶方阵,$A^{*}$为$A$的伴随矩阵,$|A|=\dfrac{1}{2}$,则$\left|\left(\dfrac{1}{4}A\right)^{-1}-A^{*}\right|=$\underline{\hphantom{~~~~~~~~~~}}。\\

%期中2015-2016 一6.
\num 设$A=
\begin{bmatrix}
  1 & 0 & 0\\
  1 & 1 & 0\\
  1 & 2 & 3
\end{bmatrix}
$,则$(A*)^{-1}=$\underline{\hphantom{~~~~~~~~~~}}。\\

%期中2015-2016 一7.
\num 设$A=
\begin{bmatrix}
  1 & 0 & 0 & 0\\
  0 & 1 & 0 & 0\\
  2 & 0 & 1 & 0\\
  0 & 0 & 0 & 1
\end{bmatrix},B=
\begin{bmatrix}
  1 & 1 & 2 & 3\\
  0 & 1 & 1 & -4\\
  1 & 2 & 3 & -1\\
  2 & 3 & -1 & -1
\end{bmatrix}
$,则$ABA^{-1}=$\underline{\hphantom{~~~~~~~~~~}}。\\

%期中2015-2016 一8.
\num 设$A,B$均为$n$阶方阵,$|A|=2$,且$AB$可逆,则$r(B)=$\underline{\hphantom{~~~~~~~~~~}}。\\

%期中2015-2016 二1.
\num 计算行列式$
\begin{vmatrix}
  5 & 0 & 4 & 2\\
  1 & -1 & 2 & 1\\
  4 & 1 & 2 & 0\\
  1 & 1 & 1 & 1\\
\end{vmatrix}
$。\\

%期中2015-2016 二2.
\num 计算行列式$
\begin{vmatrix}
  a^{2} & ab & b^{2}\\
  2a & a+b & 2b \\
  1 & 1 & 1 
\end{vmatrix}
$。\\

%期中2015-2016 二3.
\num 若$(2I-C^{-1}B)A^{T}=C^{-1}$,
$B=
\begin{bmatrix}
  1 & 2 & -3 & -2\\
  0 & 1 & 2 & -3\\
  0 & 0 & 1 & 2\\
  0 & 0 & 0 & 1\\
\end{bmatrix},C=
\begin{bmatrix}
  1 & 2 & 0 & 1\\
  0 & 1 & 2 & 0\\
  0 & 0 & 1 & 2\\
  0 & 0 & 0 & 1\\
\end{bmatrix}
$,求$A$。\\

%期中2015-2016 二4.
\num 设$
A=
\begin{bmatrix}
  1 & -1 & 0 \\
  0 & 1 & -1\\
  0 & 0 & 1 
\end{bmatrix},B=
\begin{bmatrix}
  2 & 1 & 3 \\
  0 & 2 & 1\\
  0 & 0 & 2
\end{bmatrix}
$,$A^{T}(BA^{-1}-I)^{T}X=B^{T}$,求$X$。\\

%期中2015-2016 二5.
\num
\\


%期中2015-2016 三1.
\num 设$A$可逆,且$A^{*}B=A^{-1}+B$,证明$B$可逆,当$A=
\begin{bmatrix}
  2 & 6 & 0 \\
  0 & 2 & 6\\
  0 & 0 & 2
\end{bmatrix}
$时,求$B$。\\

%期中2015-2016 三2.
\num 设$A$为$n$阶方阵,$AA^{T}=I$,$|A|<0$,证明:$|A+I|=0$。\\

\hphantom{~~}\hfill {\zihao{3}\heiti 期中2016-2017} \hfill\hphantom{~~}

%期中2016-2017 一1.
\num 设$M_{ij}$是$
\begin{vmatrix}
  0 & 4 & 0 \\
  2 & 2 & 2\\
  2 & 0 & 0
\end{vmatrix}
$的第$i$行第$j$列元素的余子式,则$M_{11}+M_{22}=$\underline{\hphantom{~~~~~~~~~~}}。\\

%期中2016-2017 一2.
\num 计算行列式$
\begin{vmatrix}
  1 & 1 & 1 & 1 \\
  1 & 2 & 4& 8 \\
  1 & 3 & 9& 27\\
   1 & 4 &16 &64
\end{vmatrix}
=$\underline{\hphantom{~~~~~~~~~~}}。\\

%期中2016-2017 一3.
\num 设方程组$
\begin{cases}
 2x_{1}-x_{2}+x_{3}=0\\
 x_{1}+kx_{2}-x_{3}=0\\
 kx_{1}+x_{2}+x_{3}=0
\end{cases}
$有非零解,则$k=$
\underline{\hphantom{~~~~~~~~~~}}。\\

%期中2016-2017 一4.
\num 设$f(x)=ax^{2}+bx+c$,$A$为$n$阶方阵,定义$f(A)=aA^{2}+bA+cI$,如果$
A=
\begin{bmatrix}
  1 & 0 & 0 & 0 \\
  0 & 1& 0& 0 \\
  2 & 0 & 1& 0\\
   0 & 0 &0 &1
\end{bmatrix},f(x)=x^{2}-x-1,
$则$f(A)=$\underline{\hphantom{~~~~~~~~~~}}。\\

%期中2016-2017 一5.
\num 若$A$为3阶方阵,$A^{*}$为$A$的伴随矩阵,$|A|=\dfrac{1}{2}$,则$\left|(3A)^{-1}-2A^{*}\right|=$\underline{\hphantom{~~~~~~~~~~}}。\\

%期中2016-2017 一6.
\num 
\\

%期中2016-2017 二1.
\num 计算行列式$
\begin{vmatrix}
  1 & 2 & 3 & 4 \\
  2 & 3 & 4& 1 \\
  3 & 4 & 1& 2\\
   4 & 1 &2 &3
\end{vmatrix}
$。\\

%期中2016-2017 二2.
\num 设$f(x)=
\begin{vmatrix}
  x-1 & 1 & -1 & 1 \\
  -1 & x+1 & -1& 1 \\
  -1 & 1 & x-1& 1\\
   -1 & 1 &-1 &x+1
\end{vmatrix}
$,求$f(x)=0$的根。\\

%期中2016-2017 二3.
\num 
\\

%期中2016-2017 二4.
\num 
\\

%期中2016-2017 二5.
\num 若$\left(\dfrac{1}{4}A^{*}\right)^{-1}BA^{-1}=2AB+I$,且
$A=
\begin{bmatrix}
  2 & 0& 0 & 0 \\
  1 & 1 & 0& 0 \\
  0 & 0 & 2& 1\\
   0& 0 &0 &1
\end{bmatrix}
$,求$B$。\\

%期中2016-2017 三1.
\num 设$A$满足$A^{2}-2A+4I=0$,证明$A+I$可逆,并求$(A+I)^{-1}$.\\

%期中2016-2017 三2.
\num 已知$A=(a_{ij})$是三阶的非零矩阵,设$A_{ij}$是$a_{ij}$的代数余子式,且对任意的$i,j$有$A_{ij}+a_{ij}=0$,求$A$ 的行列式。\\

\hphantom{~~}\hfill {\zihao{3}\heiti 期中2017-2018} \hfill\hphantom{~~}

%期中2017-2018 一1.
\num 计算行列式$
\begin{vmatrix}
  1 & 1 & 1 & 1 \\
  1 & 1 & 1& 2 \\
  1 & 1 & 3& 1\\
   1 & 4 &1 &1
\end{vmatrix}
.$\\

%期中2017-2018 一2.
\num 求方程$
\begin{vmatrix}
  1 & 2 & 1 & 1 \\
  1 & x & 2& 3 \\
  1 & 2 & x& 2\\
   0 & 0 &2 &x
\end{vmatrix}=0
$的根。\\

%期中2017-2018 一3.
\num 设$\gamma_{1},\gamma_{2},\gamma_{3},\gamma_{4}$及$\beta$均为4维列向量。4阶矩阵$A=[\gamma_ {1}~\gamma_{2}~\gamma_{3}~\gamma_{4}],A=[\beta~\gamma_{2}~\gamma_{3}~\gamma_{4}]$,若$\left|A\right|=2,\left|B\right|=3$,求

(1)$\left|A+B\right|$;

(2)$\left|A^{2}+AB\right|$;\\

%期中2017-2018 一4.
\num \\

%期中2017-2018 一5.
\num \\

%期中2017-2018 二1.
\num 设$A,B$为同阶对称方阵,则$AB$一定是对称矩阵;\\

%期中2017-2018 二2.
\num 设$A,B$为$n$阶可逆方阵,则$(AB)^{*}=B^{*}A^{*}$.\\

%期中2017-2018 二3.
\num 若$A^{2}=B^{2}$,则$A=B$或$A=-B$。\\

%期中2017-2018 二4.
\num 设2阶矩阵$A=
\begin{bmatrix}
  a & b \\
  c & d
\end{bmatrix}
$,若$A$与所有的2阶矩阵均可以交换,则$a=d,b=c=0$。\\

%期中2017-2018 二5.
\num 若$AB=I$且$BC=I$,其中$I$为单位矩阵,则$A=C$。\\

%期中2017-2018 二6.
\num 若$n$阶矩阵$A$满足$A^{3}=3A(A-I)$,则$I-A$可逆。\\

\hphantom{~~}\hfill {\zihao{3}\heiti 期中2018-2019} \hfill\hphantom{~~}

%期中2018-2019 一1.
\num 计算行列式$
\begin{vmatrix}
  b^{2}+c^{2} & c^{2}+a^{2} & a^{2}+b^{2} \\
  a & b & c \\
  a^{2} & b^{2} & c^{2} 
\end{vmatrix}
.$\\

%期中2018-2019 一2.
\num 设$A,B$为3阶矩阵,且$|A|=3,|B|=2$,$A^{*}$为$A$的伴随矩阵。

(1)若交换$A$的第一行与第二行得矩阵$C$,求$|CA^{*}|$;

(2)若$|A^{-1}+B|=2$,求$|A+B^{-1}|$.\\

%期中2018-2019 一3.
\num 已知3阶矩阵$A$的逆矩阵$
A^{-1}=
\begin{bmatrix}
  1 & 1 & 1 \\
  1 & 2 & 1 \\
  2 & 1 & 3
\end{bmatrix}
$,试求伴随矩阵$A^{*}$的逆矩阵。\\

%期中2018-2019 一4.
\num 设$n$阶行列式$D_{n}(n=1,2,\cdots):D_{1}=1,D_{2}=
\begin{vmatrix}
  1 & 1 \\
  1 & 1
\end{vmatrix},D_{3}=
\begin{vmatrix}
  1 & 1 & 0\\
  1 & 1 & 1\\
  0 & 1 & 1
\end{vmatrix},D_{4}=
\begin{vmatrix}
  1 & 1 & 0 & 0\\
  1 & 1 & 1 & 0\\
  0 & 1 & 1 & 1\\
  0 & 0 & 1 & 1
\end{vmatrix},\ldots\ldots,D_{n}=
\begin{vmatrix}
  1 & 1 & 0 & 0 & \cdots & 0\\
  1 & 1 & 1 & 0 & \cdots & 0\\
  0 & 1 & 1 & 1 & \cdots & 0\\
  \vdots & \vdots &\ddots & \ddots &\ddots &\vdots\\
  0 &\cdots & 0 & 1 & 1& 1\\
  0 &\cdots & 0 & 0 & 1& 1
\end{vmatrix}.
$

(1)给出$D_{n},D_{n-1},D_{n-2}$的关系;

(2)利用找到的递推关系及$D_{1}=1,D_{2}=0$,计算$D_{3},D_{4},\cdots,D_{8}$;

(3)求$D_{2018}$\\

%期中2018-2019 一5.
\num 已知矩阵$A=
\begin{bmatrix}
  1 &  0 & 0\\
  1 & 1 & 0\\
   1 & 1 & 1
\end{bmatrix},B=
\begin{vmatrix}
  0 & 1 & 1\\
  1 & 0 & 1 \\
  0 & 1 & 0
\end{vmatrix}
$,且矩阵$X$满足
\begin{equation*}
  AXA+BXB=AXB+BXA+I
\end{equation*}
其中$I$为3阶单位阵,求$X$。\\

%期中2018-2019 一6.
\num 

(2)当$A$和$B$等价时,求可逆矩阵$P$,使得$PA=B$。\\

%期中2018-2019 二1.
\num 若$n$阶实矩阵$Q$满足$QQ^{T}=I$,则称$Q$为正交矩阵。设$Q$为正交矩阵,则

(1)$Q$的行列式为1或-1.

(2)当$|Q|=1$且$n$为奇数时,证明$|I-Q|=0$,其中$I$是$n$阶单位矩阵;

(3)$Q$的逆矩阵$Q^{-1}$和伴随矩阵$Q^{*}$都是正交矩阵。\\

%期中2018-2019 二2.
\num 设$A$是$n$阶实对称矩阵,如果$A^{2}=0$。证明$A=0$.并举例说明,如果$A$不是实对称矩阵,上述命题不正确。\\

\hphantom{~~}\hfill {\zihao{3}\heiti 期末2014-2015} \hfill\hphantom{~~}

%期末2014-2015 一1.
\num 若已知行列式
$
\begin{vmatrix}
  1 & 3 & a \\
  5 & -1 &1\\
  3 & 2&1
\end{vmatrix}
$的代数余子式$A_{21}=1$,则$a=$\underline{\hphantom{~~~~~~~~~~}}。\\

%期末2014-2015 一2.
\num 设
$
A=
\begin{bmatrix}
  1 & 2 & -2 \\
  2 & 5 &0\\
  3 & t&4
\end{bmatrix}
$,$B$为3阶非零矩阵且$AB=0$,则$t=$\underline{\hphantom{~~~~~~~~~~}}。\\

%期末2014-2015 一3.
\num 设3阶方阵$A=(\alpha_{1},\alpha_{2},\alpha_{3})$的行列式$|A|=3$,矩阵$B=(\alpha_{2},2\alpha_{3},-\alpha_{1})$,则行列式$|A-B|=$\underline{\hphantom{~~~~~~~~~~}}。\\

%期末2014-2015 一4.
\num 已知3阶矩阵$A$的特征值为$-1,3,2$,$A^{*}$是$A$的伴随矩阵,则矩阵$A^{3}+2A^{*}$主对角线元素之和为\underline{\hphantom{~~~~~~~~~~}}。\\

%期末2014-2015 一5.
\num 已知实二次型$f(x_{1},x_{2},x_{3})=a(x_{1}^{2}+x_{2}^{2}+x_{3}^{2})+4x_{1}x_{2}+4x_{1}x_{3}+4x_{2}x_{2}$经正交变换$x=py$可化为标准形:$f=6y^{2}$,则$a=$\underline{\hphantom{~~~~~~~~~~}}。\\

%期末2014-2015 一6.
\num 设$(1,1,1)^{T}$是矩阵$
\begin{bmatrix}
  1 & 2 & 3 \\
  0 & a & 2\\
  2 & 2 & b
\end{bmatrix}
$的一个特征值,则$a-b=$\underline{\hphantom{~~~~~~~~~~}}。\\

%期末2014-2015 二.
\num 设多项式$
f(x)=
\begin{vmatrix}
  2x & 3 & 1 & 2\\
  x & x & -2 & 1\\
  2 & 1 & x & 4\\
  x & 2 & 1 & 4x
\end{vmatrix}
$,分别求该多项式的三次项、常数项。\\

%期末2014-2015 三.
\num 设$A$的伴随矩阵
$
A^{*}=
\begin{bmatrix}
  2 & 0 & 0 & 0\\
  0 & 2 & 0 & 0\\
  1 & 0 & 2 & 0\\
  0 & -3 & 0 & 8
\end{bmatrix}
$,且$ABA^{-1}=BA^{-1}+3I$,求$B$。\\

%期末2014-2015 四.
\num $\lambda$为何值时,方程组$
\begin{cases}
 2x_{1}+\lambda x_{2}-x_{3}=1\\
 \lambda x_{1}-x_{2}+x_{3}=2\\
 4x_{1}+5 x_{2}-5x_{3}=-1
\end{cases}
$有无穷多组解?并在有无穷多解时,写出方程组的通解。\\

%期末2014-2015 五.
\num 设$
\alpha_{1}=
\begin{bmatrix}
1\\ 1 \\ 2\\ 3
\end{bmatrix},
\alpha_{2}=
\begin{bmatrix}
1\\ -1 \\ 1\\ 1
\end{bmatrix},
\alpha_{3}=
\begin{bmatrix}
1\\ 3 \\ 3\\ 5
\end{bmatrix},
\alpha_{4}=
\begin{bmatrix}
4\\ -2 \\ 5\\ 6
\end{bmatrix}
$.

(1)求向量组$\alpha_{1},\alpha_{2},\alpha_{3},\alpha_{4}$的秩与一个最大线性无关组;

(2)将其余向量用极大线性无关组线性表示。\\

%期末2014-2015 六.
\num 设实二次型
\begin{equation*}
  f(x_{1},x_{2},x_{3})=X^{T}AX=ax_{1}^{2}+2x_{2}^{2}-2x_{3}^{2}+2bx_{1}x_{3}~~(b>0)
\end{equation*}
的矩阵$A$的特征值之和为$1$,特征值之积为-12。

(1)求$a,b$的值;

(2)利用正交变换将二次型$f$化为标准型,并写出所用正交变换。
\\

%期末2014-2015 七1.
\num 

(2)证明:矩阵$A+2I$可逆,并求$(A+2I)^{-1}$。\\

%期末2014-2015 七2.
\num 设$X_{0}$是线性方程组$Ax=b~(b\neq0)$的一个解,$X_{1},X_{2}$是导出组$Ax=0$的一个基础解系。令$\xi_{0}=X_{0},\xi_{1}=X_{0}+X_{1},\xi_{2}=X_{0}+X_{2}$,证明:$\xi_{0},\xi_{1},\xi_{2}$线性无关。\\

%期末2014-2015 八.
\num 设3阶方阵$A$的特征值-1,1对应的特征向量分别为$\alpha_{1},\alpha_{2}$,向量$\alpha_{3}$满足$A\alpha_{3}=\alpha_{2}+\alpha_{3}$.

(1)证明:$\alpha_{1},\alpha_{2},\alpha_{3}$线性无关;

(2)设$P=[\alpha_{1},\alpha_{2},\alpha_{3}]$,求$P^{-1}AP$。\\

\hphantom{~~}\hfill {\zihao{3}\heiti 期末2015-2016} \hfill\hphantom{~~}

%期末2015-2016 一1.
\num 行列式$
D=
\begin{vmatrix}
  1 & a & 0 & 0\\
  -1 & 2-a & a & 0\\
  0 & -2 & 3-a & a\\
  0 & 0 & -3 & 4-a
\end{vmatrix}=
$\underline{\hphantom{~~~~~~~~~~}}。\\

%期末2015-2016 一2.
\num  
\\

%期末2015-2016 一3.
\num 设$\alpha_{1},\alpha_{2},\alpha_{3}$是非齐次线性方程组$Ax=b$得到解,若$\sum\limits_{i=1}^{3}c_{i}\alpha_{i}$也是$Ax=b$得到解,则$\sum\limits_{i=1}^{3}c_{i}=$\underline{\hphantom{~~~~~~~~~~}}。\\

%期末2015-2016 一4.
\num 已知矩阵$
A=
\begin{bmatrix}
  3 & 2 & -1 \\
  a & -2 & 2\\
  3 & b & -1
\end{bmatrix}
$,若$\alpha=(1,-2,3)^{T}$是其特征向量,则$a+b=$\underline{\hphantom{~~~~~~~~~~}}。\\

%期末2015-2016 一5.
\num 任意3维实列向量都可以由向量组$\alpha_{1}=(1,0,1)^{T},\alpha_{2}=(1,-2,3)^{T}\alpha_{3}=(t,1,2)^{T}$线性表示,则$t$应满足条件\underline{\hphantom{~~~~~~~~~~}}。\\

%期末2015-2016 一6.
\num 若矩阵$
A=
\begin{bmatrix}
  1 & 1 & 2 \\
  1 & 2 & 3\\
  2 & 3 & \lambda
\end{bmatrix}
$正定,则$\lambda$满足的条件为\underline{\hphantom{~~~~~~~~~~}}。\\

%期末2015-2016 二1.
\num 若行列式$D=
\begin{vmatrix}
  1 & 2 & 3 & 4 \\
  0 & 3 & 4 & 6 \\
  3 & 4 & 1 & 2 \\
  2 & 2 & 2 & 2 
\end{vmatrix}
$,求$A_{11}+2A_{21}+A_{31}+2A_{41}$,其中$A_{ij}$为元素$a_{ij}$的代数余子式。\\

%期末2015-2016 二2.
\num 已知矩阵$X$满足方程$X
\begin{bmatrix}
  1 & 0 & -2 \\
  0 & 1 & 2 \\
  -1 & 0 & 3
\end{bmatrix}
=
\begin{bmatrix}
  -1 & 2 & 0 \\
  3 & 0 & 5
\end{bmatrix}$,求矩阵$X$。\\

%期末2015-2016 二3.
\num 设向量组$\alpha_{1}=(1,-1,2,4),\alpha_{2}=(0,3,1,2),\alpha_{3}=(3,0,7,14),\alpha_{4}=(1,-1,2,0),\alpha_{5}=(2,1,5,6)$,求向量组的秩、极大线性无关组,并将其余向量由极大无关组线性表示出。\\

%期末2015-2016 三1.
\num 
\\

%期末2015-2016 三2.
\num 设3阶实对称矩阵$A$的特征值为$\lambda_{1}=-1,\lambda_{2}=\lambda_{3}=1$,对应于$\lambda_{1}$的特征向量$\alpha_{1}=(0,1,1)^{T}$。

(1)求$A$对应于特征值1的特征向量;

(2)求$A$;

(3)求$A^{2016}$。\\

%期末2015-2016 三3.
\num 设$
A=
\begin{bmatrix}
  1 & 0 & 1 \\
  0 & 1 & 1 \\
  -1 & 0 & a \\
  0 & a & -1 
\end{bmatrix},A^{T}
$为矩阵$A$的转置,已知$r(A)=2$,且二次型$f(x)=x^{T}A^{T}Ax$.

(1)求$a$;

(2)写出二次型$f(x)$的矩阵$B=A^{T}A$;

(3)求正交变换$x=Qy$将二次型$f(x)$化为标准型,并写出所用的正交变换。\\

%期末2015-2016 四1.
\num 设$A$为$n$阶实对称矩阵,且满足$A^{2}-3A+2E=0$,其中$E$为单位矩阵,试证:

(1)$A+2E$可逆;

(2)$A$为正定矩阵。\\

%期末2015-2016 四2.
\num 设向量组$\alpha_{1},\alpha_{2},\alpha_{3}$线性无关,向量$\beta$可由$\alpha_{1},\alpha_{2},\alpha_{3}$线性表示,向量$\gamma$不能由$\alpha_{1},\alpha_{2},\alpha_{3}$线性表示,证明向量组$\alpha_{1},\alpha_{2},\alpha_{3},\beta+\alpha$线性无关。\\

\hphantom{~~}\hfill {\zihao{3}\heiti 期末2016-2017} \hfill\hphantom{~~}

%期末2016-2017 一1.
\num 行列式$
D=
\begin{vmatrix}
  1 & x & y & z\\
  x & 1 & 0 & 0\\
  y & 0 & 1 & 0\\
  z & 0 & 0 & 1
\end{vmatrix}=
$\underline{\hphantom{~~~~~~~~~~}}。\\

%期末2016-2017 一2.
\num 设$A$的伴随矩阵$
A^{*}=
\begin{bmatrix}
  1 & 2 & 3 & 4\\
  0 & 2 & 3 & 4\\
  0 & 0 & 2 & 3\\
  0 & 0 & 0 & 2
\end{bmatrix}
$,则$r(A^{2}-2A)=$\underline{\hphantom{~~~~~~~~~~}}。\\

%期末2016-2017 一3.
\num 已知线性方程组
$
\begin{cases}
 x_{1}+2x_{2}+x_{3}=2\\
 ax_{1}-x_{2}-2x_{3}=-3
\end{cases}
$与线性方程$ax_{2}+x_{3}=1$有公共的解,则$a$的取值范围为\underline{\hphantom{~~~~~~~~~~}}。\\

%期末2016-2017 一4.
\num 设$\alpha_{1}=(a,1,1)^{T},\alpha_{2}=(1,b,-1)^{T},\alpha_{3}=(1,-2,c)^{T}$是正交向量组,则$a+b+c=$\underline{\hphantom{~~~~~~~~~~}}。\\

%期末2016-2017 一5.
\num 设3阶实对称矩阵$A$的特征值分别为$1,2,3$对应的特征向量分别为$\alpha_ {1}=(1,1,1)^{T},\alpha_{2}=(2,-1,-1)^{T},\alpha_{3}$,则$A$的对应于特征值3的一个特征向量$\alpha_{3}=$\underline{\hphantom{~~~~~~~~~~}}。\\

%期末2016-2017 一5.
\num 设
$
B=
\begin{bmatrix}
  1 & 2 & 4 \\
  0 & 2 & 6\\
  0 & 0 & \lambda
\end{bmatrix}
$,已知二次型$f(x)=x^{T}Bx$是正定的,则$\lambda$的取值范围为\underline{\hphantom{~~~~~~~~~~}}。\\

%期末2016-2017 二1.
\num 若行列式$D=
\begin{vmatrix}
  1 & 2 & 3 & 4 \\
  0 & 3 & 4 & 6 \\
  0 & 4 & 1 & 2 \\
  0 & 2 & 2 & 2
\end{vmatrix}
$,求$A_{11}-2A_{21}+A_{31}-2A_{41}$,其中$A_{ij}$为元素$a_{ij}$的代数余子式。\\

%期末2016-2017 二2.
\num 设$
A=
\begin{bmatrix}
  1 & 2 & 3 \\
  0 & 1 & 3\\
  0 & 0 & 1
\end{bmatrix}
$,$B$为三阶矩阵,且满足方程$A^{*}BA=I+2A^{-1}B$,求矩阵$B$。\\

%期末2016-2017 二3.
\num 设向量组$\alpha_{1}=(3,1,4,3)^{T},\alpha_{2}=(1,1,2,1)^{T},\alpha_{3}=(0,1,1,0)^{T},\alpha_{4}=(2,2,4,2)^{T}$,求向量组的所有的极大线性无关组。\\

%期末2016-2017 三1.
\num 令$\alpha=(1,1,0)^{T}$,实对称矩阵$A=\alpha\alpha_{T}$.

(1)把矩阵$A$相似对角化;

(2)求$|6I-A^{2017}|$.\\

%期末2016-2017 三2.
\num 已知实对称矩阵$A
\begin{bmatrix}
  a & -1 & 4 \\
  -1 & 3 & b\\
  4 & b & 0
\end{bmatrix}
$与
$A
\begin{bmatrix}
  2 & ~ & ~ \\
  ~ & -4 & ~\\
  ~ & ~ & 5
\end{bmatrix}
$相似。

(1)求矩阵$A$化;

(2)求正交线性变换$x=Qy$,把二次型$f(x)=x^{T}Ax$化为标准型.\\

%期末2016-2017 三3.
\num 在对观测数据拟合的时候经常遇到线性方程组$Ax=b$是矛盾方程的情形,是没有解的。此时我们转而解$A^{T}Ax=A^{T}b$,我们称$A^{T}Ax=A^{T}b$是原线性方程组的正规方程组。称正规方程组的解为原方程组的最小二乘解。设
$
A=
\begin{bmatrix}
  1 & 1 & 0\\
  1 & 1 & 0\\
  1 & 0 & 1\\
  1 & 1 & 1
\end{bmatrix},b=
\begin{bmatrix}
1 \\ 3 \\ 8 \\ 2
\end{bmatrix}
$.

(1)证明$Ax=b$无解;

(2)求$Ax=b$的最小二乘解。\\

%期末2016-2017 四1.
\num 已知$\alpha_{1},\alpha_{2},\alpha_{3}$是线性无关的向量组,若$\alpha_{1},\alpha_{2},\alpha_{3},\beta$线性相关,证明$\beta$可以由$\alpha_{1},\alpha_{2},\alpha_{3}$线性表示并且表示方法唯一。\\

%期末2016-2017 四2.
\num 已知$A,B$是同阶实对称矩阵。

(1)证明如果$A\~{}B$,则$A\simeq B$,也就是相似一定合同;

(2)举例说明反过来不成立。\\

\hphantom{~~}\hfill {\zihao{3}\heiti 期末2017-2018} \hfill\hphantom{~~}

%期末2017-2018 一1.
\num 设$A_{ij}$是三阶行列式$
D=
\begin{vmatrix}
  2 & 2 & 2\\
  1 & 2 & 3\\
  4 & 5 & 6
\end{vmatrix}
$第$i$行第$j$列元素的代数余子式,则$A_{31}+A_{32}+A_{33}=$\underline{\hphantom{~~~~~~~~~~}}。\\

%期末2017-2018 一2.
\num 设
\\

%期末2017-2018 一3.
\num 设$
A=
\begin{bmatrix}
  2 & 0 & 0 \\
  1 & 2 & 0 \\
  1 & 2 & 2 
\end{bmatrix}
$,记$A*$是$A$的伴随矩阵,则$(A^{*})^{-1}=$\underline{\hphantom{~~~~~~~~~~}}。\\

%期末2017-2018 一4.
\num 已知3阶方阵$A$的秩为2,设$\alpha_ {1}=(2,2,0)^{T},\alpha_{2}=(3,3,1)^{T}$是非齐次线性方程组$Ax=b$的解,则导出$Ax=0$的基础解系为\underline{\hphantom{~~~~~~~~~~}}。\\

%期末2017-2018 一5.
\num 若3阶矩阵$A$相似于$B$,矩阵$A$的特征值是1,2,3那么行列式$|2B+I|=$\underline{\hphantom{~~~~~~~~~~}}。(其中$I$是3阶单位矩阵)\\

%期末2017-2018 一6.
\num 设二次型$f(x_{1},x_{2},x_{3})=2x_{1}^{2}+x_{2}^{2}+x_{3}^{2}+2x_{1}x_{2}+2tx_{2}x_{3}$的秩为2,则$t=$\underline{\hphantom{~~~~~~~~~~}}。\\

%期末2017-2018 二1.
\num 计算行列式$D=
\begin{vmatrix}
  3 & 1 & -1 & 2 \\
  -5 & 1 & 3 & -4 \\
  2 & 0 & 1 & -1\\
   1 & -5 & 3 & -3
\end{vmatrix}
.$\\

%期末2017-2018 二2.
\num 解矩阵方程$(2I-B^{-1}A)X^{T}=B^{-1}$,其中$I$是3阶单位矩阵,$X^{T}$是3阶矩阵$X$的转置矩阵,$A=
\begin{bmatrix}
  1 & 2 & -3 \\
  0 & 1 & 2 \\
  0 & 0 & 1
\end{bmatrix},B=
\begin{bmatrix}
  1 & 2 & 0 \\
  0 & 1 & 2 \\
  0 & 0 & 1
\end{bmatrix}
$.\\

%期末2017-2018 二3.
\num 
\\

%期末2017-2018 三1.
\num 设1为矩阵$A
\begin{bmatrix}
  1 & 2 & 3 \\
  x & 1 & -1 \\
  1 & 1 & x
\end{bmatrix}
$的特征值,其中$x>1$.

(1)求$x$及$A$的其他特征值。

(2)判断$A$能否对角化,若能对角化,写出相应的对角矩阵$A$。\\

%期末2017-2018 三2.
\num 设$f(x_ {1},x_{2},x_{3})=2x_{1}^{2}+2x_{2}^{2}+3x_{3}^{2}+2x_{1}x_{2}$。

(1)写出该二次型的矩阵$A$;

(2)求正交矩阵$Q$使得$Q^{T}AQ=Q^{-1}AQ$为对角型矩阵;

(3)给出正交变换,化该二次型为标准型。\\

%期末2017-2018 三2.
\num 已知$\alpha_ {1}=(1,4,0,2)^{T},\alpha_{2}=(2,7,1,3)^{T},\alpha_{3}=(0,1,-1,a)^{T}$及$\beta_{4}=(3,10,b,4)^{T}$.

(1)$a,b$为何值时,$\beta$不能表示成$\alpha_{1},\alpha_{2},\alpha_{3}$的线性组合?

(2)$a,b$为何值时,$\beta$可由$\alpha_{1},\alpha_{2},\alpha_{3}$线性表示?并写出该表达式。\\

%期末2017-2018 四1.
\num 设$A,B$均为$n$阶方阵,证明:若$A,B$相似则$|A|=|B|$,举例说明反过来不成立。\\

%期末2017-2018 四2.
\num 设$A$为$m\times n$实矩阵,证明$Ax=0$与$(A^{T}A)x=0$是同解方程,进一步得出$r(A)=r(A^{t}A)$。\\

\hphantom{~~}\hfill {\zihao{3}\heiti 期末2018-2019} \hfill\hphantom{~~}

%期末2018-2019 一1.
\num 设$A$为5阶方阵满足$|A|=2$,$A^{*}$是$A$的伴随矩阵,则$|2A^{-1}A^{*}A^{T}|=$\underline{\hphantom{~~~~~~~~~~}}。\\

%期末2018-2019 一2.
\num 已知向量组$\alpha_{1}=(1,3,1),\alpha_{2}=(0,1,1),\alpha_{3}=(1,4,k)$线性无关,则实数$k$满足的条件是\underline{\hphantom{~~~~~~~~~~}}。\\

%期末2018-2019 一3.
\num 
\\

%期末2018-2019 一4.
\num 设$A=(a_{ij})_{3\times 3}$,其特征值为$1,-1,2$,$A_{ij}$是元素$a_{ij}$的代数余子式,$A^{*}$是$A$的伴随矩阵,则$A^{*}$的主对角线元素之和即$A_{11}+A_{22}+A_{33}=$\underline{\hphantom{~~~~~~~~~~}}。\\

%期末2018-2019 一5.
\num 若二次型$f(x_ {1},x_{2},x_{3})=x_{1}^{2}+4x_{2}^{2}+4x_{3}^{2}+2tx_{1}x_{2}-2x_{1}x_{3}+4x_{2}x_{3}$正定,则$t$应满足\underline{\hphantom{~~~~~~~~~~}}。\\

%期末2018-2019 一6.
\num 设3维列向量组$\alpha_{1},\alpha_{2},\alpha_{3}$线性无关,3阶方阵$A$满足$A\alpha_{1}=-\alpha_{1},A\alpha_{2}=\alpha_{2},A\alpha_{3}=\alpha_{2}+\alpha_{3}$。则行列式$|A|=$\underline{\hphantom{~~~~~~~~~~}}。\\

%期末2018-2019 二1.
\num 已知$D=
\begin{vmatrix}
  1 & 1 & 1 & 1 \\
  -1 & 2 & 2 & 3 \\
  1 & 4 & 3 & 9 \\
  -1 & 8 & 5 & 27
\end{vmatrix}
$,求$A_{13}+A_{23}+A_{33}+A_{43}$,其中$A_{ij}$为元素$a_{ij}$的代数余子式。\\

%期末2018-2019 二2.
\num 已知
$
A=
\begin{bmatrix}
  1 & 3 & 1 \\
  1 & 1 & 0\\
  0 & 1 & 1
\end{bmatrix}
$,且$X$满足$AX=X+A$,求$X$。\\

%期末2018-2019 二3.
\num 设矩阵
$
A=
\begin{bmatrix}
  1 & 1 & 1 & 1\\
  0 & 1 & -1& b\\
  2 & 3 & a & 3\\
  3 & 5 &1 &5
\end{bmatrix}
$,$A^{*}$是$A$的伴随矩阵,求$r(A),r(A^{*})$和$A$的列向量组的极大线性无关组。\\

%期末2018-2019 三1.
\num 
\\

%期末2018-2019 三2.
\num 设实二次型$f(x_{1},x_{2},x_{3})=4x_{1}x_{2}-4x_{1}x_{3}+4x_{2}^{2}+8x_{2}x_{3}-3x_{3}^{2}$。

(1)写出该二次型的矩阵$A$;

(2)求正交矩阵$P$,使得$P^{-1}AP$为对角型矩阵;

(3)给出正交变换,将该二次型化为标准型;

(4)写出二次型的秩,正惯性指标和负惯性指标。\\

%期末2018-2019 四1.
\num 设$n$阶矩阵$A$满足$A^{2}+3A-4I=0$,其中$I$为$n$阶单位矩阵。

(1)证明:$A,A+3I$可逆,并求他们的逆;

(2)当$A\neq I$时,判断$A+4I$是否可逆并说明理由。
(4)写出二次型的秩,正惯性指标和负惯性指标。\\

%期末2018-2019 四2.
\num 若同阶矩阵$A$与$B$相似,即$A\~{}B$,证明$A^{2}\~{}B^{2}$。反过来结论是否成立并说明理由。\\

%期末2018-2019 四3.
\num 设$\lambda_{1},\lambda_{2}$是对应于$\lambda_{2}$的线性无关的特征向量,证明:向量组$\alpha_{11},\cdots,\alpha_{1s},\alpha_{21},\cdots,\alpha_{2t}$线性无关。\\

\hphantom{~~}\hfill {\zihao{3}\heiti 期末2019-2020} \hfill\hphantom{~~}

%期末2019-2020 一1.
\num 设$A$是3阶方阵,$E$是3阶单位矩阵,已知$A$的特征值为$1,1,2$,则$\left|\left(\left(\dfrac{1}{2}A\right)^{*}\right)^{-1}-2A^{-1}+E\right|= $\underline{\hphantom{~~~~~~~~~~}}。\\

%期末2019-2020 一2.
\num 
\\

%期末2019-2020 一3.
\num 记$A=
\begin{bmatrix}
  0 & 0 & 1 & 2 \\
  0 & 0 & 2 & 3 \\
  1 & 1 & 0 & 0  \\
  2& 3 & 0 & 0
\end{bmatrix}
$,则$A^{-1}$\underline{\hphantom{~~~~~~~~~~}}。\\

%期末2019-2020 一4.
\num
\\

%期末2019-2020 一5.
\num 已知$n$阶方阵$A$对应于特征值$\lambda$的全部的特征向量为$c\alpha$,其中$c$为非零常数,设$n$阶方阵$P$可逆,则$P^{-1}AP$对应于对应于特征值$\lambda$的全部的特征向量为\underline{\hphantom{~~~~~~~~~~}}。\\

%期末2019-2020 一6.
\num 已知实对称矩阵$A=
\begin{bmatrix}
  2 & 0 & 1 \\
  0 & 3 & 3\\
  1 & 3 & x
\end{bmatrix}
$的正惯性指数为3,则$x$的取值范围为\underline{\hphantom{~~~~~~~~~~}}。\\

%期末2019-2020 二1.
\num 设$
A=
\begin{bmatrix}
  0 & 1 & 0 \\
  0 & 0 & 1\\
  0 & 0 & 0
\end{bmatrix}
$.求满足$AX=XA$的全部的矩阵$X$。\\

%期末2019-2020 二2.
\num 求线性方程组$
\begin{cases}
x_{1}+3x_{2}+2x_{3}+3x_{4}=0\\
2x_{1}+4x_{2}+x_{3}+3x_{4}=0\\
2x_{1}+4x_{2}+4x_{4}=0\\
\end{cases}
$的一个基础解系。\\

%期末2019-2020 二3.
\num 记$2n$阶方阵$
A_{n}=
\begin{bmatrix}
  a_{n} & ~  & ~ & ~ & ~ & ~ & ~ & b_{n}\\
  ~ & a_{n-1}  & ~ & ~ & ~ & ~ & b_{n-1} & ~\\
  ~ & ~ & \ddots & ~ & ~ & \iddots  & ~ & ~\\
  ~ & ~ & ~ & a_{1}&b_{1} & ~ & ~ & ~\\
  ~ & ~ & ~ & c_{1}&d_{1} & ~ & ~ & ~\\
  ~ & ~ & \iddots & ~ & ~ & \ddots  & ~ & ~\\
  ~ & c_{n-1}  & ~ & ~ & ~ & ~ & d_{n-1} & ~\\
  c_{n} & ~  & ~ & ~ & ~ & ~ & ~ & d_{n}
\end{bmatrix}
$.

(1)求$|A_{1}|,|A_{2}|$

(2)求$|A_{n}|$。\\

%期末2019-2020 三1.
\num 设向量组$\alpha_{1}=(1,-4,-3)^{T},\alpha_{2}=(-3,6,7)^{T},\alpha_{3}=(-4,-2,6)^{T},\alpha_{4}=(3,3,-4)^{T}$,求向量组的秩,并写出一个极大线性无关组,并将其余向量由极大无关组线性表示出。\\

%期末2019-2020 三2.
\num 已知3阶方阵$
A=
\begin{bmatrix}
  -1 & a+2 & 0\\
  a-2 & 3 & 0\\
 8 & -8 & -1
\end{bmatrix}
$可以相似对角化且$A$得到特征方程有一个二重根,求$a$的值。其中$a\leq 0$.\\

%期末2019-2020 三3.
\num 设三元二次型$f(x_{1},x_{2},x_{3})=4x_{2}^{2}+4x_{3}^{2}-2x_{1}x_{2}+4x_{1}x_{3}$.

(1)写出该二次型的矩阵$A$;

(2)用正交变换$x=Qy$把该二次型化为标准型。\\

%期末2019-2020 四1.
\num 设$A$为$m$阶正定矩阵,$B$为$\times n$实矩阵,$B^{T}$为$B$的转置矩阵,试证:$B^{T}AB$为正定矩阵的充分必要条件是$B$的秩$r(B)=n$。\\

%期末2019-2020 四2.
\num 设$\alpha,\beta$是$n$维列向量,证明$r(\alpha\alpha^{T}+\beta\beta^{T})\leq 2$。


\end{document}  