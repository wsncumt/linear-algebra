\documentclass{article}
\usepackage[space,fancyhdr,fntef]{ctexcap}
\usepackage[namelimits,sumlimits,nointlimits]{amsmath}
\usepackage[bottom=25mm,top=25mm,left=25mm,right=15mm,centering]{geometry}
\usepackage{xcolor}
\usepackage{arydshln}%234页,虚线表格宏包
\pagestyle{fancy} \fancyhf{}
\fancyhead[OL]{~~~班序号:\hfill 学院:\hfill 学号:\hfill 姓名:王松年~~~ \thepage}
%\usepackage{parskip}
%\usepackage{indentfirst}
\usepackage{graphicx}%插图宏包,参见手册318页
\begin{document}

\newcounter{num} \renewcommand{\thenum}{\arabic{num}.} \newcommand{\num}{\refstepcounter{num}\text{\thenum}}

\hphantom{~~}\hfill {\zihao{3}\heiti 第四次习题课} \hfill\hphantom{~~}

\hphantom{~~}\hfill {\zihao{4}\heiti 群文件《期中$\&$期末试题》} \hfill\hphantom{~~}

{\heiti \zihao{4} 期中试题}

\num 2015-2016 一7.

设$A=
\begin{bmatrix}
  1 & 0 & 0 & 0\\
  0 & 1 & 0 & 0\\
  2 & 0 & 1 & 0\\
  0 & 0 & 0 & 1
\end{bmatrix},B=
\begin{bmatrix}
  1 & 1 & 2 & 3\\
  0 & 1 & 1 & -4\\
  1 & 2 & 3 & -1\\
  2 & 3 & -1 & -1
\end{bmatrix}
$,则$ABA^{-1}=$\underline{\hphantom{~~~~~~~~~~}}。\\

\num 期中2015-2016 一8.

设$A,B$均为$n$阶方阵,$|A|=2$,且$AB$可逆,则$r(B)=$\underline{\hphantom{~~~~~~~~~~}}。\\

\num 2015-2016 二3.

若$(2I-C^{-1}B)A^{T}=C^{-1}$,
$B=
\begin{bmatrix}
  1 & 2 & -3 & -2\\
  0 & 1 & 2 & -3\\
  0 & 0 & 1 & 2\\
  0 & 0 & 0 & 1\\
\end{bmatrix},C=
\begin{bmatrix}
  1 & 2 & 0 & 1\\
  0 & 1 & 2 & 0\\
  0 & 0 & 1 & 2\\
  0 & 0 & 0 & 1\\
\end{bmatrix}
$,求$A$。\\

\num 2015-2016 二4.

设$
A=
\begin{bmatrix}
  1 & -1 & 0 \\
  0 & 1 & -1\\
  0 & 0 & 1
\end{bmatrix},B=
\begin{bmatrix}
  2 & 1 & 3 \\
  0 & 2 & 1\\
  0 & 0 & 2
\end{bmatrix}
$,$A^{T}(BA^{-1}-I)^{T}X=B^{T}$,求$X$。\\

\num 2016-2017 二3.(\textcolor[rgb]{1.00,0.00,0.00}{第二次习题课讲过,自己翻看,不讲})

\num 2016-2017 三1.

设$A$满足$A^{2}-2A+4I=0$,证明$A+I$可逆,并求$(A+I)^{-1}$.\\

\num 2017-2018 一4.\textcolor[rgb]{1.00,0.00,0.00}{第二次习题课讲过,自己翻看,不讲})

\num 2017-2018 二5.

若$AB=I$且$BC=I$,其中$I$为单位矩阵,则$A=C$。\\

\num 2017-2018 二6.

若$n$阶矩阵$A$满足$A^{3}=3A(A-I)$,则$I-A$可逆。\\

\num 2018-2019 一5.

已知矩阵$A=
\begin{bmatrix}
  1 &  0 & 0\\
  1 & 1 & 0\\
   1 & 1 & 1
\end{bmatrix},B=
\begin{bmatrix}
  0 & 1 & 1\\
  1 & 0 & 1 \\
  0 & 1 & 0
\end{bmatrix}
$,且矩阵$X$满足
\begin{equation*}
  AXA+BXB=AXB+BXA+I
\end{equation*}
其中$I$为3阶单位阵,求$X$。\\


{\heiti \zihao{4} 期末试题}

\num 2014-2015 七1.设$A$为$n$阶矩阵,且$A^{2}-A-2I=0$。

(2)证明:矩阵$A+2I$可逆,并求$(A+2I)^{-1}$。\\

\num 2015-2016 二2.

已知矩阵$X$满足方程$X
\begin{bmatrix}
  1 & 0 & -2 \\
  0 & 1 & 2 \\
  -1 & 0 & 3
\end{bmatrix}
=
\begin{bmatrix}
  -1 & 2 & 0 \\
  3 & 0 & 5
\end{bmatrix}$,求矩阵$X$。\\

\num 2015-2016 四1.(1)

设$A$为$n$阶实对称矩阵,且满足$A^{2}-3A+2E=0$,其中$E$为单位矩阵,试证:

(1)$A+2E$可逆;\\

\num 2017-2018 二2.

解矩阵方程$(2I-B^{-1}A)X^{T}=B^{-1}$,其中$I$是3阶单位矩阵,$X^{T}$是3阶矩阵$X$的转置矩阵,$A=
\begin{bmatrix}
  1 & 2 & -3 \\
  0 & 1 & 2 \\
  0 & 0 & 1
\end{bmatrix},B=
\begin{bmatrix}
  1 & 2 & 0 \\
  0 & 1 & 2 \\
  0 & 0 & 1
\end{bmatrix}
$.\\

\num 2018-2019 二2.

已知
$
A=
\begin{bmatrix}
  1 & 3 & 1 \\
  1 & 1 & 0\\
  0 & 1 & 1
\end{bmatrix}
$,且$X$满足$AX=X+A$,求$X$。\\

\num 2018-2019 四1.

设$n$阶矩阵$A$满足$A^{2}+3A-4I=0$,其中$I$为$n$阶单位矩阵。

(1)证明:$A,A+3I$可逆,并求他们的逆;

(2)当$A\neq I$时,判断$A+4I$是否可逆并说明理由。

\num 2019-2020 一3.

记$A=
\begin{bmatrix}
  0 & 0 & 1 & 2 \\
  0 & 0 & 2 & 3 \\
  1 & 1 & 0 & 0  \\
  2& 3 & 0 & 0
\end{bmatrix}
$,则$A^{-1}$\underline{\hphantom{~~~~~~~~~~}}。\\

\end{document}  