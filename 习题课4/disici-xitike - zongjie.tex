\documentclass{article}
\usepackage[space,fancyhdr,fntef]{ctexcap}
\usepackage[namelimits,sumlimits,nointlimits]{amsmath}
\usepackage[bottom=25mm,top=25mm,left=25mm,right=15mm,centering]{geometry}
\usepackage{xcolor}
\usepackage{paralist}%列表宏包
\usepackage{arydshln}%234页,虚线表格宏包
\pagestyle{fancy} \fancyhf{}
\fancyhead[OL]{~~~班序号:\hfill 学院:\hfill 学号:\hfill 姓名:王松年~~~ \thepage}
%\usepackage{parskip}
%\usepackage{indentfirst}
\usepackage{graphicx}%插图宏包,参见手册318页
\begin{document}

\newcounter{num} \renewcommand{\thenum}{\arabic{num}.} \newcommand{\num}{\refstepcounter{num}\text{\thenum}}

\hphantom{~~}\hfill {\zihao{3}\heiti 第四次习题课} \hfill\hphantom{~~}

\hphantom{~~}\hfill {\zihao{4}\heiti 知识点} \hfill\hphantom{~~}

\num 可逆矩阵的定义。

设$A$是$n$阶\textcolor[rgb]{1.00,0.00,0.00}{方阵},若存在$n$阶矩阵$B$使得$AB=BA=E$,则称$A$可逆(或称$B$可逆),而称$B$为$A$的逆矩阵(或$A$为$B$的逆矩阵)。通常把$A$的逆矩阵表示为$A^{-1}$,即$AA^{-1}=A^{-1}A=E$。\\
\textcolor[rgb]{1.00,0.00,0.00}{注意}:
\begin{asparaenum}[(1)]
\item 只有$A$是方阵时,才讨论其可逆性;
\item 可逆矩阵具有左消去律和右消去律。即若$AB=AC$,且$A$可逆,那么$B=C$。同理$BA=CA$,且$A$可逆,那么那么$B=C$。
\item 若$A$可逆,则其逆矩阵一定唯一。
\item 对角矩阵可逆当且仅当主对角线元素全部\textcolor[rgb]{1.00,0.00,0.00}{不}为0.
\item 初等矩阵可逆,且其逆矩阵还是初等矩阵。
\end{asparaenum}

\num 逆矩阵的性质(以下所指的矩阵都可逆)
\begin{asparaenum}[(1)]
\item $(A^{-1})^{-1}=A$。
\item $(ABC)^{-1}=C^{-1}B^{-1}A^{-1}$。
\item $(A^{T})^{-1}=(A^{-1})^{T}$。
\item $(kA)^{-1}=\frac{1}{k}A^{-1}$。
\item $(A+B)^{-1}\neq A^{-1}+B^{-1}$
\end{asparaenum}

\num $n$阶方阵$A$可逆,当且仅当$r(A)=n$。即方阵满秩则可逆,可逆则满秩。(是充要条件)

\num $n$阶方阵$A$可逆当且仅当以下其中一条成立:
\begin{asparaenum}[(1)]
\item $r(A)=n$。
\item 线性方程组$AX=B$有唯一解。即线性方程组$AX=0$只有零解。
\item $A$与可逆矩阵$B$等价。
\item $A$的等价标准型为$E$。
\item $A$可以表示为一系列初等矩阵的乘积。
\item 存在矩阵$B$使得$AB=E$或者$BA=E$。
\item $A$行等价于$E$($A$列等价于$E$)。
\item $\det A\neq 0$。
\end{asparaenum}

\num 如何求$n$阶方阵$A$的逆矩阵(如果存在的话)。

将$(A|E_{n})$化为最简阶梯型矩阵,则最简阶梯型矩阵为$(E_{n}|A^{-1})$

\num 分块矩阵的逆:

设$A=
\begin{bmatrix}
  A_{11} & A_{12} \\
  0 & A_{22}
\end{bmatrix}
$,其中$A_{11},A_{22}$是方阵,则$A$可逆当且仅当$A_{11},A_{22}$都可逆,并且
\begin{equation*}
A^{-1}=
  \begin{bmatrix}
    A_{11}^{-1} & -A_{11}^{-1}A_{12}A_{22}^{-1} \\
    0 & A_{22}^{-1}
  \end{bmatrix}
\end{equation*}

设$A=
\begin{bmatrix}
  A_{11} & 0 \\
  A_{21} & A_{22}
\end{bmatrix}
$,其中$A_{11},A_{22}$是方阵,则$A$可逆当且仅当$A_{11},A_{22}$都可逆,并且
\begin{equation*}
A^{-1}=
  \begin{bmatrix}
    A_{11}^{-1} & 0 \\
    -A_{22}^{-1}A_{21}A_{11}^{-1} & A_{22}^{-1}
  \end{bmatrix}
\end{equation*}
\end{document}  