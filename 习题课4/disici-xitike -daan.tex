\documentclass{article}
\usepackage[space,fancyhdr,fntef]{ctexcap}
\usepackage[namelimits,sumlimits,nointlimits]{amsmath}
\usepackage[bottom=25mm,top=25mm,left=25mm,right=15mm,centering]{geometry}
\usepackage{xcolor}
\usepackage{arydshln}%234页,虚线表格宏包
\pagestyle{fancy} \fancyhf{}
\fancyhead[OL]{~~~班序号:\hfill 学院:\hfill 学号:\hfill 姓名:王松年~~~ \thepage}
%\usepackage{parskip}
%\usepackage{indentfirst}
\usepackage{graphicx}%插图宏包,参见手册318页
\begin{document}

\newcounter{num} \renewcommand{\thenum}{\arabic{num}.} \newcommand{\num}{\refstepcounter{num}\text{\thenum}}

\newenvironment{jie}{\kaishu\zihao{-5}\color{blue}{\noindent\em 解:}\par}{\hfill $\diamondsuit$\par}

\newenvironment{zhengming}{\kaishu\zihao{-5}\color{blue}{\noindent\em 证明:}\par}{\hfill $\diamondsuit$\par}

\hphantom{~~}\hfill {\zihao{3}\heiti 第四次习题课} \hfill\hphantom{~~}

\hphantom{~~}\hfill {\zihao{4}\heiti 群文件《期中$\&$期末试题》} \hfill\hphantom{~~}

{\heiti \zihao{4} 期中试题}

\num 2015-2016 一7.

设$A=
\begin{bmatrix}
  1 & 0 & 0 & 0\\
  0 & 1 & 0 & 0\\
  2 & 0 & 1 & 0\\
  0 & 0 & 0 & 1
\end{bmatrix},B=
\begin{bmatrix}
  1 & 1 & 2 & 3\\
  0 & 1 & 1 & -4\\
  1 & 2 & 3 & -1\\
  2 & 3 & -1 & -1
\end{bmatrix}
$,则$ABA^{-1}=$\underline{\hphantom{~~~~~~~~~~}}。

\begin{jie}
由题得:$A=
\begin{bmatrix}
  A_{11} & 0 \\
   A_{21}& A_{22}
\end{bmatrix}
$,其中$A_{11}=A_{22}=
\begin{bmatrix}
  1 & 0 \\
  0 & 1
\end{bmatrix}=E_{2}
,A_{21}=
\begin{bmatrix}
  2 & 0 \\
  0 & 0
\end{bmatrix}
$.所以
\begin{equation*}
  A^{-1}=
  \begin{bmatrix}
    E_{2} & 0 \\
    -E_{2}A_{21}E_{2} & E_{2}
  \end{bmatrix}=
  \begin{bmatrix}
    1 & 0 & 0 & 0\\
  0 & 1 & 0 & 0\\
  -2 & 0 & 1 & 0\\
  0 & 0 & 0 & 1
  \end{bmatrix}
\end{equation*}
所以
\begin{align*}
ABA^{-1}=\begin{bmatrix}
  1 & 0 & 0 & 0\\
  0 & 1 & 0 & 0\\
  2 & 0 & 1 & 0\\
  0 & 0 & 0 & 1
\end{bmatrix}
\begin{bmatrix}
  1 & 1 & 2 & 3\\
  0 & 1 & 1 & -4\\
  1 & 2 & 3 & -1\\
  2 & 3 & -1 & -1
\end{bmatrix}
  \begin{bmatrix}
    1 & 0 & 0 & 0\\
  0 & 1 & 0 & 0\\
  -2 & 0 & 1 & 0\\
  0 & 0 & 0 & 1
  \end{bmatrix}=\begin{bmatrix}
  1 & 1 & 2 & 3\\
  0 & 1 & 1 & -4\\
  3 & 4 & 7 & 5\\
  2 & 3 & -1 & -1
\end{bmatrix}
  \begin{bmatrix}
    1 & 0 & 0 & 0\\
  0 & 1 & 0 & 0\\
  -2 & 0 & 1 & 0\\
  0 & 0 & 0 & 1
  \end{bmatrix}=\begin{bmatrix}
  -3 & 1 & 2 & 3\\
  -2 & 1 & 1 & -4\\
  -11 & 4 & 7 & 5\\
  4 & 3 & -1 & -1
\end{bmatrix}
\end{align*}
\end{jie}

\num 期中2015-2016 一8.

设$A,B$均为$n$阶方阵,$|A|=2$,且$AB$可逆,则$r(B)=$\underline{\hphantom{~~~~~~~~~~}}。

\begin{jie}
课本97页命题3.2.4.

$|A|=2\neq0$,所以$A$可逆,又因为$AB$可逆,所以$B$可逆,即$B$满秩。$r(B)=n$
\end{jie}

\num 2015-2016 二3.

若$(2I-C^{-1}B)A^{T}=C^{-1}$,
$B=
\begin{bmatrix}
  1 & 2 & -3 & -2\\
  0 & 1 & 2 & -3\\
  0 & 0 & 1 & 2\\
  0 & 0 & 0 & 1\\
\end{bmatrix},C=
\begin{bmatrix}
  1 & 2 & 0 & 1\\
  0 & 1 & 2 & 0\\
  0 & 0 & 1 & 2\\
  0 & 0 & 0 & 1\\
\end{bmatrix}
$,求$A$。

\begin{jie}
由题得:$(2I-C^{-1}B)A^{T}=C^{-1}$,等式两边左乘$(2I-C^{-1}B)^{-1}$:
\begin{align*}
A^{T}&=(2I-C^{-1}B)^{-1}C^{-1}=[C(2I-C^{-1}B)]^{-1}\\
&=(2C-B)^{-1}
\end{align*}
\begin{equation*}
D=2C-B=2\begin{bmatrix}
  1 & 2 & 0 & 1\\
  0 & 1 & 2 & 0\\
  0 & 0 & 1 & 2\\
  0 & 0 & 0 & 1\\
\end{bmatrix}-\begin{bmatrix}
  1 & 2 & -3 & -2\\
  0 & 1 & 2 & -3\\
  0 & 0 & 1 & 2\\
  0 & 0 & 0 & 1\\
\end{bmatrix}=
\begin{bmatrix}
  1 & 2 & 3 & 4\\
  0 & 1 & 2 & 3\\
  0 & 0 & 1 & 2\\
  0 & 0 & 0 & 1\\
\end{bmatrix}
\end{equation*}
所以
\begin{align*}
[D|E_{4}]&\xrightarrow{\substack{r_{1}-4r_{4}\\ r_{2}-3r_{4}\\ r_{3}-2r_{4}}}
{\left[
\begin{array}{c:c}
\begin{matrix}
  1 & 2 & 3 & 0\\
  0 & 1 & 2 & 0\\
  0 & 0 & 1 & 0\\
  0 & 0 & 0 & 1\\
\end{matrix} &
\begin{matrix}
  1 & 0 & 0 & -4\\
  0 & 1 & 0 & -3\\
  0 & 0 & 1 & -2\\
  0 & 0 & 0 & 1\\
\end{matrix}
\end{array}
\right]
}\xrightarrow{\substack{r_{1}-3r_{3}\\ r_{2}-2r_{3}}}
{\left[
\begin{array}{c:c}
\begin{matrix}
  1 & 2 & 0 & 0\\
  0 & 1 & 0 & 0\\
  0 & 0 & 1 & 0\\
  0 & 0 & 0 & 1\\
\end{matrix} &
\begin{matrix}
  1 & 0 & -3 & 2\\
  0 & 1 & -2 & 1\\
  0 & 0 & 1 & -2\\
  0 & 0 & 0 & 1\\
\end{matrix}
\end{array}
\right]
}\\
&\xrightarrow{\substack{r_{1}-2r_{2}}}
{\left[
\begin{array}{c:c}
\begin{matrix}
  1 & 0 & 0 & 0\\
  0 & 1 & 0 & 0\\
  0 & 0 & 1 & 0\\
  0 & 0 & 0 & 1\\
\end{matrix} &
\begin{matrix}
  1 & -2 & 1 & 0\\
  0 & 1 & -2 & 1\\
  0 & 0 & 1 & -2\\
  0 & 0 & 0 & 1\\
\end{matrix}
\end{array}
\right]
}
\end{align*}
所以$A^{T}=D^{-1}=(2C-B)^{-1}=\begin{bmatrix}
  1 & -2 & 1 & 0\\
  0 & 1 & -2 & 1\\
  0 & 0 & 1 & -2\\
  0 & 0 & 0 & 1\\
\end{bmatrix}$
,$
  A=(A^{T})^{T}=\begin{bmatrix}
  1 & 0 & 0 & 0\\
  -2 & 1 & 0 & 0\\
  1 & -2 & 1 & 0\\
  0 & 1 & -2 & 1\\
\end{bmatrix}
.$
\end{jie}

\num 2015-2016 二4.

设$
A=
\begin{bmatrix}
  1 & -1 & 0 \\
  0 & 1 & -1\\
  0 & 0 & 1
\end{bmatrix},B=
\begin{bmatrix}
  2 & 1 & 3 \\
  0 & 2 & 1\\
  0 & 0 & 2
\end{bmatrix}
$,$A^{T}(BA^{-1}-I)^{T}X=B^{T}$,求$X$。

\begin{jie}
由题得:$A^{T}(BA^{-1}-I)^{T}=[(BA^{-1}-I)A]^{T}=(B-A)^{T}$,所以
\begin{equation*}
  X=((B-A)^{T})^{-1}B^{T}=((B-A)^{-1})^{T}B^{T}=[B(B-A)^{-1}]^{T}
\end{equation*}
\begin{equation*}
  B-A=\begin{bmatrix}
  2 & 1 & 3 \\
  0 & 2 & 1\\
  0 & 0 & 2
\end{bmatrix}-\begin{bmatrix}
  1 & -1 & 0 \\
  0 & 1 & -1\\
  0 & 0 & 1
\end{bmatrix}=
\begin{bmatrix}
  1 & 2 & 3 \\
  0 & 1 & 2\\
  0 & 0 & 1
\end{bmatrix}
\end{equation*}
(\textcolor[rgb]{0.00,1.00,0.50}{由上一题可以看出,B-A是上一题目中D的左上角三行三列的元素。其逆矩阵也应该是D逆矩阵左上角三行三列,这里直接用结论})
所以:
\begin{gather*}
( B-A)^{-1}=\begin{bmatrix}
  1 & -2 & 1 \\
  0 & 1 & -2 \\
  0 & 0 & 1
\end{bmatrix}~~~~~~
B( B-A)^{-1}=\begin{bmatrix}
  2 & 1 & 3 \\
  0 & 2 & 1\\
  0 & 0 & 2
\end{bmatrix}\begin{bmatrix}
  1 & -2 & 1 \\
  0 & 1 & -2 \\
  0 & 0 & 1
\end{bmatrix}=
\begin{bmatrix}
  2 & -3 & 3 \\
  0 & 2 & -3 \\
  0 & 0 & 2
\end{bmatrix}\\
X=\left[B(B-A)^{-1}\right]^{T}=
\begin{bmatrix}
  2 & 0 & 0 \\
  -3 & 2 & 0\\
  3 & -3 & 2
\end{bmatrix}
\end{gather*}
\end{jie}

\num 2016-2017 二3.第二次习题课讲过了。

\num 2016-2017 三1.

设$A$满足$A^{2}-2A+4I=0$,证明$A+I$可逆,并求$(A+I)^{-1}$.

\textcolor[rgb]{1.00,0.00,0.00}{思路:题目让证明谁可逆,就凑出这个表达式与某个表达式的乘积等于单位矩阵。}
\begin{zhengming}
由题得:$A^{2}-2A+4I=0$,所以:
\begin{gather*}
  A^{2}\textcolor[rgb]{1.00,0.00,0.00}{+A-A}-2A+4I=0\\
  A(A+I)-3A\textcolor[rgb]{1.00,0.00,0.00}{-3I+3I}+4I=0\\
  A(A+I)-3(A+I)=-7I\\
  (A-3I)(A+I)=-7I\\
  -\frac{1}{7}(A-3I)(A+I)=I
\end{gather*}
所以$A+I$可逆,$(A+I)^{-1}=-\dfrac{1}{7}(A-3I)$.
\end{zhengming}

\num 2017-2018 一4.第二次习题课讲过了。自己去翻看。

\num 2017-2018 二5.判断是否成立并给出理由。

若$AB=I$且$BC=I$,其中$I$为单位矩阵,则$A=C$。

\begin{jie}
成立。理由如下:

设$I$为n阶单位矩阵,那么由矩阵乘法的定义有如下关系:

$A$的列数等于$B$的行数;$A$的行数等于$I$的行数等于$n$;$B$的列数等于$I$的列数等于$n$;\\
$B$的列数等于$C$的行数;$B$的行数等于$I$的行数等于$n$;$C$的列数等于$I$的列数等于$n$.

即$A,B,C$均为$n$阶方阵,对于$n$阶方阵,如果$AB=I$,那么$A,B$可逆,所以$A$是$B$的逆矩阵,$C$是$B$的逆矩阵,由逆矩阵的唯一性可知$B$=$C$。
\end{jie}

\num 2017-2018 二6.判断是否成立并给出理由。

若$n$阶矩阵$A$满足$A^{3}=3A(A-I)$,则$I-A$可逆。

\begin{jie}
成立。理由如下:

由题得:$A^{3}=3A(A-I)$,即$-A^{3}+3A^{2}-3A=0$,$-A^{3}+3A^{2}-3A+I=I$,所以$(I-A)^{3}=I$,所以$I-A$可逆,其逆矩阵为$(I-A)^{-1}=(I-A)^{2}$
\end{jie}

\num 2018-2019 一5.

已知矩阵$A=
\begin{bmatrix}
  1 &  0 & 0\\
  1 & 1 & 0\\
   1 & 1 & 1
\end{bmatrix},B=
\begin{bmatrix}
  0 & 1 & 1\\
  1 & 0 & 1 \\
  0 & 1 & 0
\end{bmatrix}
$,且矩阵$X$满足
\begin{equation*}
  AXA+BXB=AXB+BXA+I
\end{equation*}
其中$I$为3阶单位阵,求$X$。

\begin{jie}
由题得:$AXA+BXB=AXB+BXA+I$所以:
\begin{gather*}
  AXA+BXB-AXB-BXA=I\\
  (A-B)XA+(B-A)XB=(A-B)XA-(A-B)XB=I \\
  (A-B)X(A-B)=I\\
  (A-B)X=I(A-B)^{-1}\\
  X=(A-B)^{-1}I(A-B)^{-1}=((A-B)^{-1})^{2}
\end{gather*}
\begin{align*}
 &C=A-B=\begin{bmatrix}
  1 &  0 & 0\\
  1 & 1 & 0\\
   1 & 1 & 1
\end{bmatrix}-
\begin{bmatrix}
  0 & 1 & 1\\
  1 & 0 & 1 \\
  0 & 1 & 0
\end{bmatrix}=\begin{bmatrix}
  1 &  -1 & -1\\
  0 & 1 & -1\\
   0 & 0 & 1
\end{bmatrix}\\
&[C|E]\xrightarrow{\substack{r_{1}+r_{3}\\ r_{2}+r_{3}}}
{\left[
\begin{array}{c:c}
\begin{matrix}
 1 &  -1 & 0\\
  0 & 1 & 0\\
   0 & 0 & 1
\end{matrix} &
\begin{matrix}
  1 & 0 & 1\\
  0 & 1 & 1\\
  0 & 0 & 1
\end{matrix}
\end{array}
\right]
}\xrightarrow{\substack{r_{1}+r_{3}\\ r_{2}+r_{3}}}
{\left[
\begin{array}{c:c}
\begin{matrix}
 1 &  0 & 0\\
  0 & 1 & 0\\
   0 & 0 & 1
\end{matrix} &
\begin{matrix}
  1 & 1 & 2\\
  0 & 1 & 1\\
  0 & 0 & 1
\end{matrix}
\end{array}
\right]
}
\end{align*}
所以$C^{-1}=\begin{bmatrix}
  1 & 1 & 2\\
  0 & 1 & 1\\
  0 & 0 & 1
\end{bmatrix},C^{2}=\begin{bmatrix}
  1 & 2 & 5\\
  0 & 1 & 2\\
  0 & 0 & 1
\end{bmatrix}$
\end{jie}


{\heiti \zihao{4} 期末试题}

\num 2014-2015 七1.设$A$为$n$阶矩阵,且$A^{2}-A-2I=0$。

(2)证明:矩阵$A+2I$可逆,并求$(A+2I)^{-1}$。

\begin{zhengming}
由题得:$A^{2}-A-2I=0$,所以
\begin{gather*}
  A^{2}\textcolor[rgb]{1.00,0.00,0.00}{+2A-2A}-A-2I=0 \\
  A(A+2I)-3A\textcolor[rgb]{1.00,0.00,0.00}{-6I+6I}-2I=0 \\
  A(A+2I)-3(A+2I)=(A-3I)(A+2I)=-4I\\
  -\frac{1}{4}(A-3I)(A+2I)=I
\end{gather*}
所以$A+2I$可逆,$(A+2I)^{-1}=-\dfrac{1}{4}(A-3I)$
\end{zhengming}

\num 2015-2016 二2.

已知矩阵$X$满足方程$X
\begin{bmatrix}
  1 & 0 & -2 \\
  0 & 1 & 2 \\
  -1 & 0 & 3
\end{bmatrix}
=
\begin{bmatrix}
  -1 & 2 & 0 \\
  3 & 0 & 5
\end{bmatrix}$,求矩阵$X$。

\begin{jie}
由题得:
\begin{align*}
\left[
\begin{array}{c:c}
\begin{matrix}
 1 & 0 & -2 \\
  0 & 1 & 2 \\
  -1 & 0 & 3
\end{matrix} &
\begin{matrix}
  1 & 0 & 0\\
  0 & 1 & 0\\
  0 & 0 & 1
\end{matrix}
\end{array}
\right]\xrightarrow{\substack{r_{3}+r_{1}}}
{\left[
\begin{array}{c:c}
\begin{matrix}
 1 & 0 & -2 \\
  0 & 1 & 2 \\
  0 & 0 & 1
\end{matrix} &
\begin{matrix}
  1 & 0 & 0\\
  0 & 1 & 0\\
  1 & 0 & 1
\end{matrix}
\end{array}
\right]
}\xrightarrow{\substack{r_{2}-2r_{3}\\ r_{1}+2r_{3}}}
{\left[
\begin{array}{c:c}
\begin{matrix}
 1 & 0 & 0 \\
  0 & 1 & 0 \\
  0 & 0 & 1
\end{matrix} &
\begin{matrix}
  3 & 0 & 2\\
  -2 & 1 & -2\\
  1 & 0 & 1
\end{matrix}
\end{array}
\right]
}
\end{align*}
所以$X=\begin{bmatrix}
  -1 & 2 & 0 \\
  3 & 0 & 5
\end{bmatrix}\begin{bmatrix}
  3 & 0 & 2\\
  -2 & 1 & -2\\
  1 & 0 & 1
\end{bmatrix}=\begin{bmatrix}
  -7 & 2 & -6 \\
  14 & 0 & 11
\end{bmatrix}$
\end{jie}

\num 2015-2016 四1.(1)

设$A$为$n$阶实对称矩阵,且满足$A^{2}-3A+2E=0$,其中$E$为单位矩阵,试证:

(1)$A+2E$可逆;

\begin{zhengming}
由题得:$A^{2}-3A+2E=0$,所以:
\begin{gather*}
  A^{2}\textcolor[rgb]{1.00,0.00,0.00}{+2A-2A}-3A+2E=0 \\
  A(A+2E)-5A\textcolor[rgb]{1.00,0.00,0.00}{-10E+10E}+2E=0 \\
  A(A+2E)-5(A+2E)=(A-5E)(A+2E)=-12E\\
  -\frac{1}{12}(A-5E)(A+2E)=E
\end{gather*}
所以$A+2E$可逆,$(A+2E)^{-1}=-\dfrac{1}{12}(A-5E)$
\end{zhengming}

\num 2017-2018 二2.

解矩阵方程$(2I-B^{-1}A)X^{T}=B^{-1}$,其中$I$是3阶单位矩阵,$X^{T}$是3阶矩阵$X$的转置矩阵,$A=
\begin{bmatrix}
  1 & 2 & -3 \\
  0 & 1 & 2 \\
  0 & 0 & 1
\end{bmatrix},B=
\begin{bmatrix}
  1 & 2 & 0 \\
  0 & 1 & 2 \\
  0 & 0 & 1
\end{bmatrix}
$.

\begin{jie}
由题得:$(2I-B^{-1}A)X^{T}=B^{-1}$,所以:
\begin{align*}
X^{T}&=(2I-B^{-1}A)^{-1}B^{-1}=[B(2I-B^{-1}A)]^{-1}\\
&=(2B-A)^{-1}
\end{align*}
\begin{align*}
&C=2B-A=\begin{bmatrix}
  1 & 2 & 3 \\
  0 & 1 & 2 \\
  0 & 0 & 1
\end{bmatrix}
\end{align*}
\textcolor[rgb]{0.50,1.00,0.00}{由第三题的计算结果有}
$C^{-1}=\begin{bmatrix}
  1 & -2 & 1 \\
  0 & 1 & -2 \\
  0 & 0 & 1
\end{bmatrix}$
所以$X=C^{T}=\begin{bmatrix}
  1 & 0 & 0 \\
  -2 & 1 & 0 \\
  1 & -2 & 1
\end{bmatrix}$
\end{jie}

\num 2018-2019 二2.

已知
$
A=
\begin{bmatrix}
  1 & 3 & 1 \\
  1 & 1 & 0\\
  0 & 1 & 1
\end{bmatrix}
$,且$X$满足$AX=X+A$,求$X$。

\begin{jie}
由题得:$AX=X+A$,所以$(A-E)X=A$,所以$X=(A-E)^{-1}A$
\begin{align*}
&B=A-E=\begin{bmatrix}
  0 & 3 & 1 \\
  1 & 0 & 0\\
  0 & 1 & 0
\end{bmatrix}\\
[B|E]&\xrightarrow{\substack{r_{1}\Leftrightarrow r_{2}}}
{\left[
\begin{array}{c:c}
\begin{matrix}
 1 & 0 & 0\\
   0 & 3 & 1 \\
  0 & 1 & 0
\end{matrix} &
\begin{matrix}
 0 & 1 & 0\\
  1 & 0 & 0\\
  0 & 0 & 1
\end{matrix}
\end{array}
\right]
}\xrightarrow{\substack{r_{2}\Leftrightarrow r_{3}}}
{\left[
\begin{array}{c:c}
\begin{matrix}
 1 & 0 & 0\\
  0 & 1 & 0  \\
  0 & 3 & 1
\end{matrix} &
\begin{matrix}
 0 & 1 & 0\\
0 & 0 & 1  \\
  1 & 0 & 0
\end{matrix}
\end{array}
\right]
}\xrightarrow{\substack{r_{3}-3 r_{2}}}
{\left[
\begin{array}{c:c}
\begin{matrix}
 1 & 0 & 0\\
  0 & 1 & 0  \\
  0 & 0 & 1
\end{matrix} &
\begin{matrix}
 0 & 1 & 0\\
0 & 0 & 1  \\
  1 & 0 & -3
\end{matrix}
\end{array}
\right]
}
\end{align*}
所以$(A-E)^{-1}=\begin{bmatrix}
 0 & 1 & 0\\
0 & 0 & 1  \\
  1 & 0 & -3
\end{bmatrix}$,$X=(A-E)^{-1}A=\begin{bmatrix}
 0 & 1 & 0\\
0 & 0 & 1  \\
  1 & 0 & -3
\end{bmatrix}\begin{bmatrix}
  1 & 3 & 1 \\
  1 & 1 & 0\\
  0 & 1 & 1
\end{bmatrix}=\begin{bmatrix}
  1 & 1 & 0\\
  0 & 1 & 1\\
  1 & 0 & -2
\end{bmatrix}$
\end{jie}

\num 2018-2019 四1.

设$n$阶矩阵$A$满足$A^{2}+3A-4I=0$,其中$I$为$n$阶单位矩阵。

(1)证明:$A,A+3I$可逆,并求他们的逆;

(2)当$A\neq I$时,判断$A+4I$是否可逆并说明理由。

\begin{jie}
(1)由题得:$A^{2}+3A-4I=0$,所以$A(A+3I)=4I$,所以$A,A+3I$可逆,$A$的逆为$\dfrac{1}{4}(A+3I)$,$A+3I$的逆为$\dfrac{1}{4}A$。

(2)不可逆,理由:

由题得:
$A^{2}+3A-4I=(A+4I)(A-I)=0$,假设$A+4I$可逆,则等式两端同时左乘$(A+4I)^{-1}$得$A-I=0$,即$A=I$与题目中$A\neq I$矛盾,所以假设不成立。即$A+4I$不可逆。
\end{jie}

\num 2019-2020 一3.

记$A=
\begin{bmatrix}
  0 & 0 & 1 & 2 \\
  0 & 0 & 2 & 3 \\
  1 & 1 & 0 & 0  \\
  2& 3 & 0 & 0
\end{bmatrix}
$,则$A^{-1}$\underline{\hphantom{~~~~~~~~~~}}。

\begin{jie}
由题得:$A=
\begin{bmatrix}
  0 & A_{12} \\
  A_{21} & 0
\end{bmatrix}
$,其中$
A_{12}=
\begin{bmatrix}
  1 & 2 \\
  2 & 3
\end{bmatrix},
A_{21}=
\begin{bmatrix}
  1 & 1 \\
  2 & 3
\end{bmatrix}
$,设$A$的逆矩阵为$A^{-1}=
\begin{bmatrix}
  a & b \\
  c & d
\end{bmatrix}
$,其中$a,b,c,d$为$2$阶方阵,则
\begin{equation*}
AA^{-1}=\begin{bmatrix}
  0 & A_{12} \\
  A_{21} & 0
\end{bmatrix}\begin{bmatrix}
  a & b \\
  c & d
\end{bmatrix}=\begin{bmatrix}
  A_{12}c &  A_{12}d\\
  A_{21}a & A_{21}b
\end{bmatrix}
=\begin{bmatrix}
  E_{2} & 0\\
  0 & E_{2}
\end{bmatrix}
\end{equation*}
所以:$c=A_{12}^{-1},d=0,a=0,b=A_{21}^{-1}$.
\begin{align*}
&[A_{12}|E_{2}]\xrightarrow{\substack{r_{2}-2r_{1}}}
{\left[
\begin{array}{c:c}
\begin{matrix}
  1 & 2 \\
  0 & -1
\end{matrix} &
\begin{matrix}
1& 0\\
-2 &1
\end{matrix}
\end{array}
\right]
}\xrightarrow{\substack{r_{1}+2r_{2}}}
{\left[
\begin{array}{c:c}
\begin{matrix}
  1 & 0 \\
  0 & -1
\end{matrix} &
\begin{matrix}
-3 & 2\\
-2 &1
\end{matrix}
\end{array}
\right]
}\xrightarrow{\substack{r_{2}\times(-1)}}
{\left[
\begin{array}{c:c}
\begin{matrix}
  1 & 0 \\
  0 & 1
\end{matrix} &
\begin{matrix}
-3 & 2\\
2 &-1
\end{matrix}
\end{array}
\right]
}\\
&[A_{21}|E_{2}]\xrightarrow{\substack{r_{2}-2r_{1}}}
{\left[
\begin{array}{c:c}
\begin{matrix}
  1 & 1 \\
  0 & 1
\end{matrix} &
\begin{matrix}
1& 0\\
-2 &1
\end{matrix}
\end{array}
\right]
}\xrightarrow{\substack{r_{1}-r_{2}}}
{\left[
\begin{array}{c:c}
\begin{matrix}
  1 & 0 \\
  0 & 1
\end{matrix} &
\begin{matrix}
3& -1\\
-2 &1
\end{matrix}
\end{array}
\right]
}
\end{align*}
所以$A^{-1}=\begin{bmatrix}
  a & b \\
  c & d
\end{bmatrix}=
\begin{bmatrix}
  0 & A_{21}^{-1} \\
  A_{12}^{-1} & 0
\end{bmatrix}=
\begin{bmatrix}
  0 & 0 & 3 & -1\\
  0 & 0 & -2 & 1 \\
  -3& 2 & 0 & 0  \\
  2& -1 & 0 & 0
\end{bmatrix}
$
\end{jie}
\end{document}  