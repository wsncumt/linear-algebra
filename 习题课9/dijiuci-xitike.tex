\documentclass{article}
\usepackage[space,fancyhdr,fntef]{ctexcap}
\usepackage[namelimits,sumlimits,nointlimits]{amsmath}
\usepackage[bottom=25mm,top=25mm,left=25mm,right=15mm,centering]{geometry}
\usepackage{xcolor}
\usepackage{arydshln}%234页,虚线表格宏包
\usepackage{mathdots}%反对角省略号
\pagestyle{fancy} \fancyhf{}
\fancyhead[OL]{~~~班序号:\hfill 学院:\hfill 学号:\hfill 姓名:王松年~~~ \thepage}
%\usepackage{parskip}
%\usepackage{indentfirst}
\usepackage{graphicx}%插图宏包,参见手册318页
\begin{document}

\newcounter{num} \renewcommand{\thenum}{\arabic{num}.} \newcommand{\num}{\refstepcounter{num}\text{\thenum}}

\hphantom{~~}\hfill {\zihao{3}\heiti 第九次习题课} \hfill\hphantom{~~}

\hphantom{~~}\hfill {\zihao{4}\heiti 群文件《期中$\&$期末试题》} \hfill\hphantom{~~}


{\heiti \zihao{4} 期末试题}

\num 期末2014-2015 一4.

已知3阶矩阵$A$的特征值为$-1,3,2$,$A^{*}$是$A$的伴随矩阵,则矩阵$A^{3}+2A^{*}$主对角线元素之和为\underline{\hphantom{~~~~~~~~~~}}。\\

\num 期末2014-2015 一6.

设$(1,1,1)^{T}$是矩阵$
\begin{bmatrix}
  1 & 2 & 3 \\
  0 & a & 2\\
  2 & 2 & b
\end{bmatrix}
$的一个特征值,则$a-b=$\underline{\hphantom{~~~~~~~~~~}}。\\



\num 期末2014-2015 八.

设3阶方阵$A$的特征值-1,1对应的特征向量分别为$\alpha_{1},\alpha_{2}$,向量$\alpha_{3}$满足$A\alpha_{3}=\alpha_{2}+\alpha_{3}$.

(1)证明:$\alpha_{1},\alpha_{2},\alpha_{3}$线性无关;

(2)设$P=[\alpha_{1},\alpha_{2},\alpha_{3}]$,求$P^{-1}AP$。\\

\num 期末2015-2016 一4.

已知矩阵$
A=
\begin{bmatrix}
  3 & 2 & -1 \\
  a & -2 & 2\\
  3 & b & -1
\end{bmatrix}
$,若$\alpha=(1,-2,3)^{T}$是其特征向量,则$a+b=$\underline{\hphantom{~~~~~~~~~~}}。\\

\num 期末2016-2017 三1.

令$\alpha=(1,1,0)^{T}$,实对称矩阵$A=\alpha\alpha_{T}$.

(1)把矩阵$A$相似对角化;

(2)求$|6I-A^{2017}|$.\\

\num 期末2017-2018 一5.

若3阶矩阵$A$相似于$B$,矩阵$A$的特征值是1,2,3那么行列式$|2B+I|=$\underline{\hphantom{~~~~~~~~~~}}。(其中$I$是3阶单位矩阵)\\

\num 期末2017-2018 三1.

设1为矩阵$A=
\begin{bmatrix}
  1 & 2 & 3 \\
  x & 1 & -1 \\
  1 & 1 & x
\end{bmatrix}
$的特征值,其中$x>1$.

(1)求$x$及$A$的其他特征值。

(2)判断$A$能否对角化,若能对角化,写出相应的对角矩阵$\Lambda$。\\


\num 期末2017-2018 四1.

设$A,B$均为$n$阶方阵,证明:若$A,B$相似则$|A|=|B|$,举例说明反过来不成立。\\

\num 期末2018-2019 一4.

设$A=(a_{ij})_{3\times 3}$,其特征值为$1,-1,2$,$A_{ij}$是元素$a_{ij}$的代数余子式,$A^{*}$是$A$的伴随矩阵,则$A^{*}$的主对角线元素之和即$A_{11}+A_{22}+A_{33}=$\underline{\hphantom{~~~~~~~~~~}}。\\

\num 期末2018-2019 四2.

若同阶矩阵$A$与$B$相似,即$A\~{}B$,证明$A^{2}\~{}B^{2}$。反过来结论是否成立并说明理由。\\

\num 期末2018-2019 四3.

设$\lambda_{1},\lambda_{2}$是$A$的两个互异的特征值,$\alpha_{11},\cdots,\alpha_ {1s}$是对应于$\lambda_{1}$的线性无关的特征向量,$\alpha_ {21},\cdots,\alpha_ {2t}$是对应于$\lambda_{2}$的线性无关的特征向量,证明:向量组$\alpha_{11},\cdots,\alpha_{1s},\alpha_{21},\cdots,\alpha_{2t}$线性无关。

\num 期末2019-2020 一1.

设$A$是3阶方阵,$E$是3阶单位矩阵,已知$A$的特征值为$1,1,2$,则$\left|\left(\left(\dfrac{1}{2}A\right)^{*}\right)^{-1}-2A^{-1}+E\right|= $\underline{\hphantom{~~~~~~~~~~}}。\\

\num 期末2019-2020 一5.

已知$n$阶方阵$A$对应于特征值$\lambda$的全部的特征向量为$c\alpha$,其中$c$为非零常数,设$n$阶方阵$P$可逆,则$P^{-1}AP$对应于特征值$\lambda$的全部的特征向量为\underline{\hphantom{~~~~~~~~~~}}。\\

\num 期末2019-2020 三2.

已知3阶方阵$
A=
\begin{bmatrix}
  -1 & a+2 & 0\\
  a-2 & 3 & 0\\
 8 & -8 & -1
\end{bmatrix}
$可以相似对角化且$A$得到特征方程有一个二重根,求$a$的值。其中$a\leq 0$.
\end{document}  