\documentclass{article}
\usepackage[space,fancyhdr,fntef]{ctexcap}
\usepackage[namelimits,sumlimits,nointlimits]{amsmath}
\usepackage[bottom=25mm,top=25mm,left=25mm,right=15mm,centering]{geometry}
\usepackage{xcolor}
\usepackage{arydshln}%234页,虚线表格宏包
\pagestyle{fancy} \fancyhf{}
\fancyhead[OL]{~~~班序号:\hfill 学院:\hfill 学号:\hfill 姓名:王松年~~~ \thepage}
%\usepackage{parskip}
%\usepackage{indentfirst}
\usepackage{graphicx}%插图宏包,参见手册318页
\usepackage{mathdots}%反对角省略号
\usepackage{extarrows}
\begin{document}

\newcounter{num} \renewcommand{\thenum}{\arabic{num}.} \newcommand{\num}{\refstepcounter{num}\text{\thenum}}

\newenvironment{jie}{\kaishu\zihao{-5}\color{blue}{\noindent\em 解:}\par}{\hfill $\diamondsuit$\par}

\newenvironment{zhengming}{\kaishu\zihao{-5}\color{blue}{\noindent\em 证明:}\par}{\hfill $\diamondsuit$\par}

\hphantom{~~}\hfill {\zihao{3}\heiti 第九次习题课} \hfill\hphantom{~~}

\hphantom{~~}\hfill {\zihao{4}\heiti 群文件《期中$\&$期末试题》} \hfill\hphantom{~~}

{\heiti \zihao{4} 期末试题}

\num 期末2014-2015 一4.

已知3阶矩阵$A$的特征值为$-1,3,2$,$A^{*}$是$A$的伴随矩阵,则矩阵$A^{3}+2A^{*}$主对角线元素之和为\underline{\hphantom{~~~~~~~~~~}}。

\begin{jie}
由题得:$|A|=\prod\limits_{i=1}^3\lambda_i=-1\times3\times2=-6$。所以$A^*$的特征值为:$\dfrac{|A|}{\lambda_i}$。由特征值的性质:

$A^ {3}+2A^{*}$的特征值为$\lambda_i^3+2\dfrac{|A|}{\lambda_i}$.所以
$A^ {3}+2A^{*}$主对角线元素之和为

\begin{equation*}
trace(A^ {3}+2A^{*})=\sum_{i=1}^{3}\left(\lambda_i^3+2\dfrac{|A|}{\lambda_i}\right)=36
\end{equation*}
\end{jie}

\num 期末2014-2015 一6.

设$(1,1,1)^{T}$是矩阵$
\begin{bmatrix}
  1 & 2 & 3 \\
  0 & a & 2\\
  2 & 2 & b
\end{bmatrix}
$的一个特征向量,则$a-b=$\underline{\hphantom{~~~~~~~~~~}}。

\begin{jie}
由特征向量的定义有
\begin{equation*}
A
\begin{bmatrix}
1\\ 1\\ 1
\end{bmatrix}=\lambda
\begin{bmatrix}
1\\ 1\\ 1
\end{bmatrix}~~~\Rightarrow~~~
\begin{bmatrix}
6\\ a+2\\ b+4
\end{bmatrix}=\begin{bmatrix}
\lambda\\ \lambda\\ \lambda
\end{bmatrix}~~~\Rightarrow~~~
\begin{cases}
\lambda=6\\
a=4\\
b=2
\end{cases}~~~\Rightarrow~~a-b=2
\end{equation*}
\end{jie}

\num 期末2014-2015 八.

设3阶方阵$A$的特征值-1,1对应的特征向量分别为$\alpha_{1},\alpha_{2}$,向量$\alpha_{3}$满足$A\alpha_{3}=\alpha_{2}+\alpha_{3}$.

(1)证明:$\alpha_{1},\alpha_{2},\alpha_{3}$线性无关;

(2)设$P=[\alpha_{1},\alpha_{2},\alpha_{3}]$,求$P^{-1}AP$。

\begin{jie}
由题得:$A\alpha_ {1}=-\alpha_ {1},A\alpha_{2}=\alpha_{2}$。

(1)设
\begin{equation*}
k_1\alpha_{1}+k_2\alpha_{2}+k_3\alpha_{3}=0\tag{$1$}
\end{equation*}
要证明$\alpha_{1},\alpha_{2},\alpha_{3}$线性无关,只需证明$k_1=k_2=k_3=0$,(1)式两边同左乘$A$:
\begin{equation*}
k_1A\alpha_{1}+k_2A\alpha_{2}+k_3A\alpha_{3}=-k_1\alpha_{1}+k_2\alpha_{2}+k_3(\alpha_{2}+\alpha_{3})=-k_1\alpha_{1}+(k_2+k_3)\alpha_{2}+k_3\alpha_{3}=0\tag{$2$}
\end{equation*}
(1)式减(2)式:$-2k_1\alpha_{1}-k_3\alpha_{2}=0$,因为$\alpha_1$和$\alpha_2$是分属于不同的特征值的特征向量,所以$\alpha_1$和$\alpha_2$线性无关。即$
\begin{cases}
-2k_1=0\\
-k_3=0
\end{cases}~\Rightarrow k_1=k_3=0
$,代入到$(1)$式:$k_2\alpha_{2}=0$,又因为特征向量不为0,所以$k_{2}=0$。

综上所述:$k_{1}=k_2=k_3=0$,所以$\alpha_{1},\alpha_{2},\alpha_{3}$线性无关。

(2)由题得:
\begin{equation*}
AP=[A\alpha_1,A\alpha_2,A\alpha_3]=[-\alpha_1,\alpha_2,\alpha_2+\alpha_3]=[\alpha_1,\alpha_2,\alpha_3]
\begin{bmatrix}
  -1 & 0 & 0 \\
  0 & 1 & 1\\
  0 & 0 & 1
\end{bmatrix}=P\Lambda
\end{equation*}
所以$\Lambda=P^-1AP=
\begin{bmatrix}
  -1 & 0 & 0 \\
  0 & 1 & 1\\
  0 & 0 & 1
\end{bmatrix}
$。
\end{jie}

\num 期末2015-2016 一4.

已知矩阵$
A=
\begin{bmatrix}
  3 & 2 & -1 \\
  a & -2 & 2\\
  3 & b & -1
\end{bmatrix}
$,若$\alpha=(1,-2,3)^{T}$是其特征向量,则$a+b=$\underline{\hphantom{~~~~~~~~~~}}。

\begin{jie}
设$\alpha$对应的特征值为$\lambda$。由特征向量的定义:
\begin{equation*}
A\alpha=\lambda\alpha~~~\Rightarrow~~~
\begin{bmatrix}
-4 \\ a+10 \\ -2b
\end{bmatrix}=
\begin{bmatrix}
\lambda \\ -2\lambda \\ 3\lambda
\end{bmatrix}~~~\Rightarrow ~~~
\begin{cases}
\lambda=-4\\
a=-2\\
b=6
\end{cases}~~~\Rightarrow~~~a+b=4
\end{equation*}
\end{jie}

\num 期末2016-2017 三1.

令$\alpha=(1,1,0)^{T}$,实对称矩阵$A=\alpha\alpha_{T}$.

(1)把矩阵$A$相似对角化;

(2)求$|6I-A^{2017}|$.

\begin{jie}
由题得:$A=\alpha\alpha_{T}=(1,1,0)^{T}(1,1,0)=
\begin{bmatrix}
  1 & 1 & 0 \\
  1 & 1 & 0 \\
   0 & 0 & 0 \\
\end{bmatrix}
$。所以
\begin{equation*}
|\lambda E-A|=
\begin{vmatrix}
  \lambda-1 & -1 & 0 \\
  -1 & \lambda-1 & 0 \\
   0 & 0 & \lambda
\end{vmatrix}=\lambda[(\lambda-1)^2-1]=0~~~\Rightarrow~~~\lambda_{1}=2,\lambda_{2}=\lambda_3=0
\end{equation*}

$\lambda_{1}=2$时:
\begin{align*}
[\lambda E-A]=
\begin{bmatrix}
  1 & -1 & 0 \\
  -1 & 1 & 0 \\
   0 & 0 & 2
\end{bmatrix}
\xrightarrow{\text{高斯消元,步骤略}}&
{
\begin{bmatrix}
  1 & -1 & 0 \\
  0 & 0 & 1 \\
   0 & 0 & 0
\end{bmatrix}
}~~~\Rightarrow~~~\alpha_{1}=
\begin{bmatrix}
  1  \\
   1 \\
    0
\end{bmatrix}
\end{align*}
可以看出,$\lambda_1=2$时:代数重数等于几何重数。

$\lambda_{2}=\lambda_3=0$时:
\begin{align*}
[\lambda E-A]=
\begin{bmatrix}
  -1 & -1 & 0 \\
  -1 & -1 & 0 \\
   0 & 0 & 0
\end{bmatrix}
\xrightarrow{\text{高斯消元,步骤略}}&
{
\begin{bmatrix}
  1 & 1 & 0 \\
  0 & 0 & 0 \\
   0 & 0 & 0
\end{bmatrix}
}~~~\Rightarrow~~~\alpha_{2}=
\begin{bmatrix}
  1  \\
   -1 \\
    0
\end{bmatrix},
\alpha_{2}=
\begin{bmatrix}
0  \\
0 \\
1
\end{bmatrix}
\end{align*}
可以看出,$\lambda_2=\lambda_2=0$时:代数重数等于几何重数。

所以$A$可以相似对角化:即存在可逆矩阵$P$,使得$P^{-1}AP=\Lambda=
\begin{bmatrix}
  0 &  &   \\
   &  0&   \\
   &  &  2
\end{bmatrix}
$,其中$P=
\begin{bmatrix}
  1 &  0&  1 \\
  -1 &  0&  1 \\
  0 &  1&  0
\end{bmatrix}
$

(2)由特征值的性质:$6I-A^{2017}$的特征值为:$6-\lambda_{i}^{2017}$,所以
\begin{equation*}
|6I-A^{2017}|=\prod_{i=1}^{3}(6-\lambda_ {i}^{2017})=36\times(6-2^{2017})
\end{equation*}
\end{jie}

\num 期末2017-2018 一5.

若3阶矩阵$A$相似于$B$,矩阵$A$的特征值是1,2,3那么行列式$|2B+I|=$\underline{\hphantom{~~~~~~~~~~}}。(其中$I$是3阶单位矩阵)

\begin{jie}
$A$相似于$B$,所以$A$与$B$的特征值相等。所以$2B+I$的特征值为$2\lambda_i+1$,所以$|2B+I|=\prod\limits_{i=1}^{3}(2\lambda_i+1)=105$
\end{jie}

\num 期末2017-2018 三1.

设1为矩阵$A=
\begin{bmatrix}
  1 & 2 & 3 \\
  x & 1 & -1 \\
  1 & 1 & x
\end{bmatrix}
$的特征值,其中$x>1$.

(1)求$x$及$A$的其他特征值。

(2)判断$A$能否对角化,若能对角化,写出相应的对角矩阵$\Lambda$。

\begin{jie}
设$\alpha_1$为特征值$1$对应的特征向量,所以$\alpha_1\neq 0$由题得:$A\alpha_1=\alpha_1$,即$(A-E)\alpha_1=0$,即$(A-E)x=0$有非零解。所以由存在唯一性定理:$|A-E|=0$,所以

\begin{equation*}
|A-E|=
\begin{vmatrix}
  0 & 2 & 3 \\
  x & 0 & -1 \\
  1 & 1 & x-1
\end{vmatrix}=-x
\begin{vmatrix}
 2 & 3 \\
1 & x-1
\end{vmatrix}+
\begin{vmatrix}
 2 & 3 \\
0 & -1
\end{vmatrix}=(2x-1)(x-2)=0
\end{equation*}
由题得:$x>1$,所以解得$x=2$。

(2)把$x=2$代入得:
\begin{align*}
|\lambda E-A|&=
\begin{vmatrix}
  \lambda-1 & -2 & -3 \\
  -2 & \lambda-1 & 1 \\
  -1 & -1 & \lambda-2
\end{vmatrix}\xlongequal{c_{1}-c_{2}}
\begin{vmatrix}
  \lambda+1 & -2 & -3 \\
  -(\lambda+1) & \lambda-1 & 1 \\
  0 & -1 & \lambda-2
\end{vmatrix}=(\lambda+1)
\begin{vmatrix}
  1 & -2 & -3 \\
  -1 & \lambda-1 & 1 \\
  0 & -1 & \lambda-2
\end{vmatrix}\xlongequal{r_{2}+r_{1}}
(\lambda+1)
\begin{vmatrix}
  1 & -2 & -3 \\
  0 & \lambda-3 & -2 \\
  0 & -1 & \lambda-2
\end{vmatrix}\\
&=(\lambda+1)[(\lambda-3)(\lambda-2)-2]=0
\end{align*}
解得:$\lambda_1=1,\lambda_2=4,\lambda_3=-1$。

因为$A$为三阶,并且有3个不同的特征值,所以可以相似对角化,
$
\Lambda=
\begin{bmatrix}
  1 & & \\
    & 4 &\\
    &&-1
\end{bmatrix}
$。\textcolor[rgb]{1.00,0.00,0.00}{(不唯一,只要对角线元素是这三个就可以)}
\end{jie}

\num 期末2017-2018 四1.

设$A,B$均为$n$阶方阵,证明:若$A,B$相似则$|A|=|B|$,举例说明反过来不成立。

\begin{zhengming}
若$A$与$B$相似,则依定义有:存在一个可逆矩阵$P$,使得$A=P^{-1}BP$,两边同时求行列式:
$|A|=|P^ {-1}BP|=|P^{-1}|\cdot|B|\cdot|P|=|B|\cdot|P^{-1}P|=|B|\cdot|E|=|B|$。

反过来描述:如果$|A|=|B|$,则$A$和$B$相似。

例如:$A
\begin{bmatrix}
  1 & 0\\
  0& 1
\end{bmatrix},|A|=1
,B=
\begin{bmatrix}
  1 & 1 \\
  0 & 1
\end{bmatrix},|B|=1
$,所以$|A|=|B|$,
但是:假设存在一个可逆矩阵$P$,$P^{-1}AP=P^{-1}EP=E\neq B$,即$|A|=|B|$,但是$A,B$不相似。
\end{zhengming}

\num 期末2018-2019 一4.

设$A=(a_{ij})_{3\times 3}$,其特征值为$1,-1,2$,$A_{ij}$是元素$a_{ij}$的代数余子式,$A^{*}$是$A$的伴随矩阵,则$A^{*}$的主对角线元素之和即$A_{11}+A_{22}+A_{33}=$\underline{\hphantom{~~~~~~~~~~}}。

\begin{jie}
由题得:$|A|=\prod\limits_{i=1}^{3}\lambda_{i}=-2$,所以$A^*$的特征值为$\dfrac{|A|}{\lambda_{i}}$,所以$A^{*}$的主对角线元素之和为$trace(A^*)=\sum\limits_{i=1}^{3}\dfrac{|A|}{\lambda_{i}}=-1$。
\end{jie}

\num 期末2018-2019 四2.

若同阶矩阵$A$与$B$相似,即$A\~{}B$,证明$A^{2}\~{}B^{2}$。反过来结论是否成立并说明理由。

\begin{zhengming}
若$A$与$B$相似,则依定义有:存在一个可逆矩阵$P$,使得$A=P^{-1}BP$,所以:
$A^2=P^ {-1}B\textcolor[rgb]{1.00,0.00,0.00}{P\cdot P^ {-1}}BP=P^ {-1}B^{2}P$。所以$A^{2}\~{}B^{2}$。

反过来描述:如果$A^{2}\~{}B^{2}$,则$A$和$B$相似。

不成立。理由如下:

例如:$A
\begin{bmatrix}
  1 & 0\\
  0& 1
\end{bmatrix},A^2=
\begin{bmatrix}
  1 & 0\\
  0& 1
\end{bmatrix}
,B=
\begin{bmatrix}
  0 & 1 \\
  1 & 0
\end{bmatrix},B^2=
\begin{bmatrix}
  1 & 0\\
  0& 1
\end{bmatrix}
$,所以$A^2=B^2$,由相似的性质$A^2\~{}B^{2}$
但是:假设存在一个可逆矩阵$P$,$P^{-1}AP=P^{-1}EP=E\neq B$,即$A^2\~{}B^{2}$,但是$A,B$不相似。
\end{zhengming}

\num 期末2018-2019 四3.

设$\lambda_{1},\lambda_{2}$是$A$的两个互异的特征值,$\alpha_{11},\cdots,\alpha_ {1s}$是对应于$\lambda_{1}$的线性无关的特征向量,$\alpha_ {21},\cdots,\alpha_ {2t}$是对应于$\lambda_{2}$的线性无关的特征向量,证明:向量组$\alpha_{11},\cdots,\alpha_{1s},\alpha_{21},\cdots,\alpha_{2t}$线性无关。

\begin{zhengming}
由题得:$A\alpha_ {1i}=\lambda_{1}\alpha_ {1i},(1\leq i\leq s)$,$A\alpha_ {2j}=\lambda_{1}\alpha_ {1j},(1\leq j\leq t)$。

设
\begin{equation*}
k_1\alpha_ {11}+\cdots+k_s\alpha_ {1s}+k_{s+1}\alpha_ {21}+\cdots+k_{s+t}\alpha_{2t}=0\tag{$1$}
\end{equation*}

要证明向量组$\alpha_{11},\cdots,\alpha_{1s},\alpha_{21},\cdots,\alpha_{2t}$线性无关,只需证明$k_1=k_2=\cdots=k_s=k_{s+1}=\cdots=k_{s+t}=0$即可。

在$(1)$式左边同乘$A$:
\begin{equation*}
  k_1A\alpha_ {11}+\cdots+k_sA\alpha_ {1s}+k_{s+1}A\alpha_ {21}+\cdots+k_{s+t}A\alpha_{2t}=\lambda_{1}(k_1\alpha_ {11}+\cdots+k_s\alpha_ {1s})+\lambda_{2}(k_{s+1}\alpha_ {21}+\cdots+k_{s+t}\alpha_{2t})=0\tag{$2$}
\end{equation*}
$(2)-\lambda_{2}(1)$得:$(\lambda_1-\lambda_2)(k_1\alpha_ {11}+\cdots+k_s\alpha_ {1s})=0$,因为$\lambda_{1},\lambda_{2}$是$A$的两个互异的特征值,所以$\lambda_1-\lambda_2\neq0$,所以$k_1\alpha_ {11}+\cdots+k_s\alpha_ {1s}=0$,又因为$\alpha_ {11},\cdots,\alpha_ {1s}$是对应于$\lambda_{1}$的线性无关的特征向量,所以:$k_1=k_2=\cdots=k_{s}=0$。代入到$(1)$式得:

$k_{s+1}\alpha_ {21}+\cdots+k_{s+t}\alpha_{2t}=0$,因为$\alpha_ {21},\cdots,\alpha_ {2t}$是对应于$\lambda_{2}$的线性无关的特征向量,所以$k_{s+1}=k_{s+2}=\cdots=k_{s+t}=0$

综上所述:$k_1=k_2=\cdots=k_{s}=k_{s+1}=k_{s+2}=\cdots=k_{s+t}=0$,所以向量组$\alpha_{11},\cdots,\alpha_{1s},\alpha_{21},\cdots,\alpha_{2t}$线性无关。
\end{zhengming}

\num 期末2019-2020 一1.

设$A$是3阶方阵,$E$是3阶单位矩阵,已知$A$的特征值为$1,1,2$,则$\left|\left(\left(\dfrac{1}{2}A\right)^{*}\right)^{-1}-2A^{-1}+E\right|= $\underline{\hphantom{~~~~~~~~~~}}。

\begin{jie}
由题得:$|A|=\prod\limits_{i=1}^{3}\lambda_{i}=2,A^*$的特征值为$\dfrac{|A|}{\lambda}=\dfrac{2}{\lambda}$.

由伴随矩阵的性质:$\left(\dfrac{1}{2}A\right)^*=\left(\dfrac{1}{2}\right)^{3-1}A^*=\dfrac{A^*}{4}$,所以$\left(\left(\dfrac{1} {2}A\right)^{*}\right)^{-1}-2A^{-1}+E $的特征值为
\begin{equation*}
\left(\dfrac{1} {4}\cdot\dfrac{2}{\lambda_{i}}\right)^{-1}-2\lambda_{i}^{-1}+1=2\lambda_{i}-\dfrac{2}{\lambda_i}+1
\end{equation*}
所以:
\begin{equation*}
  \left|\left(\left(\dfrac{1} {2}A\right)^{*}\right)^{-1}-2A^{-1}+E\right|=\prod_{i=1}^{3}\left(2\lambda_{i}-\dfrac{2}{\lambda_i}+1\right)=4
\end{equation*}
\end{jie}

\num 期末2019-2020 一5.

已知$n$阶方阵$A$对应于特征值$\lambda$的全部的特征向量为$c\alpha$,其中$c$为非零常数,设$n$阶方阵$P$可逆,则$P^{-1}AP$对应于特征值$\lambda$的全部的特征向量为\underline{\hphantom{~~~~~~~~~~}}。

\begin{jie}
由题得:$A(c\alpha)=\lambda (c\alpha)$等式两边同时左乘$P^{-1}$:
\begin{equation*}
P^{-1}A\textcolor[rgb]{1.00,0.00,0.00}{E}(c\alpha)=P^{-1}A\textcolor[rgb]{1.00,0.00,0.00}{PP^{-1}}(c\alpha)=(P^{-1}AP)(P^{-1}c\alpha)=\lambda (P^{-1}c\alpha)
\end{equation*}
所以$P^ {-1}AP$对应于特征值$\lambda$的全部的特征向量为$P^{-1}c\alpha=cP^{-1}\alpha$
\end{jie}

\num 期末2019-2020 三2.

已知3阶方阵$
A=
\begin{bmatrix}
  -1 & a+2 & 0\\
  a-2 & 3 & 0\\
 8 & -8 & -1
\end{bmatrix}
$可以相似对角化且$A$得到特征方程有一个二重根,求$a$的值。其中$a\leq 0$.

\begin{jie}
由题得:
\begin{align*}
|\lambda E-A|&=
\begin{vmatrix}
  \lambda+1 & -(a+2) & 0\\
  2-a & \lambda-3 & 0\\
 -8 & 8 & \lambda+1
\end{vmatrix}=(\lambda+1)[(\lambda-1)(\lambda-3)+(2+a)(2-a)]=(\lambda+1)[(\lambda-1)^2-a^2]=0\\
&\Rightarrow ~~\lambda_{1}=-1,\lambda_2=1+a,\lambda_{3}=1-a.
\end{align*}
依题意:有二重根且可以相似对角化且$a\leq0$.

讨论:

(1)$\lambda_1=\lambda_2$,即$-1=1+a,a=-2\leq 0$,此时$\lambda_{3}=1-a=3$,代入到${\lambda E-A}$得:
\begin{align*}
[\lambda E-A]=
\begin{bmatrix}
  \lambda+1 & 0 & 0\\
  4 & \lambda-3 & 0\\
 -8 & 8 & \lambda+1
\end{bmatrix}
\end{align*}
对于重根$-1$:
\begin{align*}
[\lambda E-A]=
\begin{bmatrix}
  0 & 0 & 0\\
  4 & -4 & 0\\
 -8 & 8 & 0
\end{bmatrix}
\rightarrow
\begin{bmatrix}
  1 & -1 & 0\\
  0 & 0 & 0\\
 0 & 0 & 0
\end{bmatrix}~~~\Rightarrow~~~\alpha_1=
\begin{bmatrix}
0\\
0\\
1
\end{bmatrix}~~\alpha_2=
\begin{bmatrix}
1\\
1\\
0
\end{bmatrix}
\end{align*}
对于根$3$:
\begin{align*}
[\lambda E-A]=
\begin{bmatrix}
  4 & 0 & 0\\
  4 & 0 & 0\\
 -8 & 8 & 4
\end{bmatrix}
\rightarrow
\begin{bmatrix}
  1 & 0 & 0\\
  0 & 2 & 1\\
 0 & 0 & 0
\end{bmatrix}~~~\Rightarrow~~~\alpha_3=
\begin{bmatrix}
0\\
-1\\
2
\end{bmatrix}
\end{align*}
可以看出$\alpha_{1},\alpha_{2},\alpha_3$线性无关,即$A$可相似对角化,即$a=-2$符合题意。

(2)$\lambda_1=\lambda_3$,即$-1=1-a,a=2>0$,不符合题意。

(3)$\lambda_{2}=\lambda=3$,即$1+a=1-a,a=0$。此时$\lambda_{2}=\lambda_3=1$.把$a=0$代入到$[\lambda E-A]$得:
\begin{equation*}
[\lambda E-A]=
\begin{bmatrix}
  \lambda+1 & -2 & 0\\
  2 & \lambda-3 & 0\\
 -8 & 8 & \lambda+1
\end{bmatrix}
\end{equation*}
对于重根$1$:
\begin{align*}
[\lambda E-A]=
\begin{bmatrix}
  2 & -2 & 0\\
  2 & -2 & 0\\
 -8 & 8 & 2
\end{bmatrix}
\rightarrow
\begin{bmatrix}
  1 & -1 & 0\\
  0 & 0 & 1\\
 0 & 0 & 0
\end{bmatrix}~~~\Rightarrow~~~\alpha_1=
\begin{bmatrix}
1\\
1\\
0
\end{bmatrix}
\end{align*}
对于重根$1$,其代数重数与几何重数不相等,所以不能相似对角化。

综上所述:$a=-2$.
\end{jie}

\end{document}  