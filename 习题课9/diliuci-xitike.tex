\documentclass{article}
\usepackage[space,fancyhdr,fntef]{ctexcap}
\usepackage[namelimits,sumlimits,nointlimits]{amsmath}
\usepackage[bottom=25mm,top=25mm,left=25mm,right=15mm,centering]{geometry}
\usepackage{xcolor}
\usepackage{arydshln}%234页,虚线表格宏包
\usepackage{mathdots}%反对角省略号
\pagestyle{fancy} \fancyhf{}
\fancyhead[OL]{~~~班序号:\hfill 学院:\hfill 学号:\hfill 姓名:王松年~~~ \thepage}
%\usepackage{parskip}
%\usepackage{indentfirst}
\usepackage{graphicx}%插图宏包,参见手册318页
\begin{document}

\newcounter{num} \renewcommand{\thenum}{\arabic{num}.} \newcommand{\num}{\refstepcounter{num}\text{\thenum}}

\hphantom{~~}\hfill {\zihao{3}\heiti 第六次习题课} \hfill\hphantom{~~}

\hphantom{~~}\hfill {\zihao{4}\heiti 群文件《期中$\&$期末试题》} \hfill\hphantom{~~}

{\heiti \zihao{4} 期中试题}

\num 期中2015-2016 一2.

设$f(x)=
\begin{vmatrix}
  2x & x & 1 & 2\\
  1 & x & 1 & -1\\
  3 & 2 & x & 1\\
  1 & 1 & 1 & x\\
\end{vmatrix}
$,则$x^{3}$的系数为\underline{\hphantom{~~~~~~~~~~}}。\\

\num 期中2015-2016 一5.

若$A$为4阶方阵,$A^{*}$为$A$的伴随矩阵,$|A|=\dfrac{1}{2}$,则$\left|\left(\dfrac{1}{4}A\right)^{-1}-A^{*}\right|=$\underline{\hphantom{~~~~~~~~~~}}。\\

\num 期中2015-2016 一6.
设$A=
\begin{bmatrix}
  1 & 0 & 0\\
  1 & 1 & 0\\
  1 & 2 & 3
\end{bmatrix}
$,则$(A*)^{-1}=$\underline{\hphantom{~~~~~~~~~~}}。\\

\num 期中2015-2016 三1.

设$A$可逆,且$A^{*}B=A^{-1}+B$,证明$B$可逆,当$A=
\begin{bmatrix}
  2 & 6 & 0 \\
  0 & 2 & 6\\
  0 & 0 & 2
\end{bmatrix}
$时,求$B$。\\

\num 期中2016-2017 一5.

若$A$为3阶方阵,$A^{*}$为$A$的伴随矩阵,$|A|=\dfrac{1}{2}$,则$\left|(3A)^{-1}-2A^{*}\right|=$\underline{\hphantom{~~~~~~~~~~}}。\\

\num 期中2016-2017 二5.

若$\left(\dfrac{1}{4}A^{*}\right)^{-1}BA^{-1}=2AB+I$,且
$A=
\begin{bmatrix}
  2 & 0& 0 & 0 \\
  1 & 1 & 0& 0 \\
  0 & 0 & 2& 1\\
   0& 0 &0 &1
\end{bmatrix}
$,求$B$。\\

\num 期中2017-2018 二2.

设$A,B$为$n$阶可逆方阵,则$(AB)^{*}=B^{*}A^{*}$.\\

\num 期中2018-2019 一2.

设$A,B$为3阶矩阵,且$|A|=3,|B|=2$,$A^{*}$为$A$的伴随矩阵。

(1)若交换$A$的第一行与第二行得矩阵$C$,求$|CA^{*}|$;\\

\num 期中2018-2019 一3.

已知3阶矩阵$A$的逆矩阵$
A^{-1}=
\begin{bmatrix}
  1 & 1 & 1 \\
  1 & 2 & 1 \\
  2 & 1 & 3
\end{bmatrix}
$,试求伴随矩阵$A^{*}$的逆矩阵。\\

\num 期中2018-2019 二1.

若$n$阶实矩阵$Q$满足$QQ^{T}=I$,则称$Q$为正交矩阵。设$Q$为正交矩阵,则

(1)$Q$的行列式为1或-1.

(2)当$|Q|=1$且$n$为奇数时,证明$|I-Q|=0$,其中$I$是$n$阶单位矩阵;

(3)$Q$的逆矩阵$Q^{-1}$和伴随矩阵$Q^{*}$都是正交矩阵。\\

{\heiti \zihao{4} 期末试题}

\num 期末2014-2015 二.

设多项式$
f(x)=
\begin{vmatrix}
  2x & 3 & 1 & 2\\
  x & x & -2 & 1\\
  2 & 1 & x & 4\\
  x & 2 & 1 & 4x
\end{vmatrix}
$,分别求该多项式的三次项、常数项。\\


\num 期末2014-2015 三.

设$A$的伴随矩阵
$
A^{*}=
\begin{bmatrix}
  2 & 0 & 0 & 0\\
  0 & 2 & 0 & 0\\
  1 & 0 & 2 & 0\\
  0 & -3 & 0 & 8
\end{bmatrix}
$,且$ABA^{-1}=BA^{-1}+3I$,求$B$。\\

\num 期末2016-2017 一2.

设$A$的伴随矩阵$
A^{*}=
\begin{bmatrix}
  1 & 2 & 3 & 4\\
  0 & 2 & 3 & 4\\
  0 & 0 & 2 & 3\\
  0 & 0 & 0 & 2
\end{bmatrix}
$,则$r(A^{2}-2A)=$\underline{\hphantom{~~~~~~~~~~}}。\\

\num 期末2016-2017 二2.

设$
A=
\begin{bmatrix}
  1 & 2 & 3 \\
  0 & 1 & 3\\
  0 & 0 & 1
\end{bmatrix}
$,$B$为三阶矩阵,且满足方程$A^{*}BA=I+2A^{-1}B$,求矩阵$B$。\\

\num 期末2017-2018 一3.

设$
A=
\begin{bmatrix}
  2 & 0 & 0 \\
  1 & 2 & 0 \\
  1 & 2 & 2
\end{bmatrix}
$,记$A*$是$A$的伴随矩阵,则$(A^{*})^{-1}=$\underline{\hphantom{~~~~~~~~~~}}。\\

\num 期末2018-2019 一1.

设$A$为5阶方阵满足$|A|=2$,$A^{*}$是$A$的伴随矩阵,则$|2A^{-1}A^{*}A^{T}|=$\underline{\hphantom{~~~~~~~~~~}}。\\

\end{document}  