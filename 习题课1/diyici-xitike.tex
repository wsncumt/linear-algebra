\documentclass{article}
\usepackage[space,fancyhdr,fntef]{ctexcap}
\usepackage[namelimits,sumlimits,nointlimits]{amsmath}
\usepackage[bottom=25mm,top=25mm,left=25mm,right=15mm,centering]{geometry}
\usepackage{xcolor}
\usepackage{arydshln}%234页,虚线表格宏包
\pagestyle{fancy} \fancyhf{}
\fancyhead[OL]{~~~班序号:\hfill 学院:\hfill 学号:\hfill 姓名:王松年~~~ \thepage}
%\usepackage{parskip}
%\usepackage{indentfirst}
\usepackage{graphicx}%插图宏包,参见手册318页
\begin{document}

\newcounter{num} \renewcommand{\thenum}{\arabic{num}.} \newcommand{\num}{\refstepcounter{num}\text{\thenum}}

\hphantom{~~}\hfill {\zihao{3}\heiti 第一次习题课} \hfill\hphantom{~~}

\hphantom{~~}\hfill {\zihao{4}\heiti 群文件《期中$\&$期末试题》} \hfill\hphantom{~~}

{\heiti \zihao{-3} 期末试题}

1.2014\~{}2015(双语)1.

Determine if the following systems are consistent.\\
\begin{flalign*}
  &(a).
  \begin{cases}
    x_{1}+x_{2}-x_{3}=1\\
    x_{2}=1\\
    x_{2}-2x_{3}=3
  \end{cases}
  ~~~~(b).
  \begin{cases}
    x_{1}+x_{2}+x_{3}=0\\
    x_{2}-2x_{3}=1\\
    x_{2}+2x_{3}=6
  \end{cases}
\end{flalign*}


2.2015\~{}2016~~三.1.

当$k$为何值时,线性方程组
$
\begin{cases}
kx_{1}+x_{2}+x_{3}=k-3\\
x_{1}+kx_{2}+x_{3}=-2\\
x_{1}+x_{2}+kx_{3}=-2
\end{cases}
$
有唯一解,无解和有无穷多解?当方程组有无穷多解时求出所有解。

3.2015\~{}2016~~三.1.

求线性方程组
$
\begin{cases}
 2x_{1}-x_{2}+4x_{3}-3x_{4}=-4\\
 x_{1}+x_{3}-x_{4}=-3\\
 3x_{1}+x_{2}+x_{3}=1\\
 7x_{1}+7x_{3}-3x_{4}=3\\
\end{cases}
$
的通解。

4.2018\~{}2019~~三.1.

设
$
\begin{cases}
\lambda x_{1}+x_{2}+x_{3}=\lambda-2\\
x_{1}+\lambda x_{2} +x_{3}=2\\
x_{1}+ x_{2} +\lambda x_{3}=2
\end{cases}
$
,$\lambda$为何值时,该方程组无解、唯一解、无穷解?并且在有唯一解时求出解;有无穷多解时,求出全部解并用向量表示。

5.2019\~{}2020~~一.4.

若线性方程组
$
\begin{cases}
x_{1}+x_{2}=-a_{1}\\
x_{2}+x_{3}=a_{2}\\
x_{3}+x_{4}=-a_{3}\\
x_{4}+x_{1}=a_{4}
\end{cases}
$
有解,$a_{1},a_{2},a_{3},a_{4}$应满足的条件是\underline{\hphantom{~~~~~~~~~~~~~~~}}。

6.期末2019-2020 二1.

设$
A=
\begin{bmatrix}
  0 & 1 & 0 \\
  0 & 0 & 1\\
  0 & 0 & 0
\end{bmatrix}
$.求满足$AX=XA$的全部的矩阵$X$。


{\heiti \zihao{-3} 期中试题}

2017\~{}2018~~二.3.判断命题是否成立并给出理由

若$A^{2}=B^{2}$,则$A=B$或$A=-B$。
%$X\stackrel{\substack{F\\F}}{\longrightarrow}Y$   %268页,281页
%$\xrightarrow{122222222222222223}$
\end{document}  