\documentclass{article}
\usepackage[space,fancyhdr,fntef]{ctexcap}
\usepackage[namelimits,sumlimits,nointlimits]{amsmath}
\usepackage[bottom=25mm,top=25mm,left=25mm,right=15mm,centering]{geometry}
\usepackage{xcolor}
\usepackage{arydshln}%234页,虚线表格宏包
\pagestyle{fancy} \fancyhf{}
\fancyhead[OL]{~~~班序号:\hfill 学院:\hfill 学号:\hfill 姓名:王松年~~~ \thepage}
%\usepackage{parskip}
%\usepackage{indentfirst}
\usepackage{graphicx}%插图宏包,参见手册318页
\begin{document}

\newcounter{num} \renewcommand{\thenum}{\arabic{num}.} \newcommand{\num}{\refstepcounter{num}\text{\thenum}}

\newenvironment{jie}{\kaishu\zihao{-5}\color{blue}{\noindent\em 解:}\par}{\hfill $\diamondsuit$\par}

\hphantom{~~}\hfill {\zihao{3}\heiti 第一次习题课} \hfill\hphantom{~~}

\hphantom{~~}\hfill {\zihao{4}\heiti 群文件《期中$\&$期末试题》} \hfill\hphantom{~~}

{\heiti \zihao{-3} 期末试题}


1.2014\~{}2015(双语)1.

Determine if the following systems are consistent.\\
\begin{flalign*}
  &(a).
  \begin{cases}
    x_{1}+x_{2}-x_{3}=1\\
    x_{2}=1\\
    x_{2}-2x_{3}=3
  \end{cases}
  ~~~~(b).
  \begin{cases}
    x_{1}+x_{2}+x_{3}=0\\
    x_{2}-2x_{3}=1\\
    x_{2}+2x_{3}=6
  \end{cases}
\end{flalign*}

\begin{jie}
(a)增广矩阵:
\begin{align*}
\left[
\begin{array}{c:c}
 \begin{matrix}
   1 & 1 & -1 \\
   0 & 1 &  0 \\
   0 & 1 & -2 \\
 \end{matrix} &
 \begin{matrix}
   1 \\
   1 \\
   3 \\
 \end{matrix}
\end{array}
\right]
& \xrightarrow{r_{3}-r_{2}}
{
\left[
\begin{array}{c:c}
 \begin{matrix}
   1 & 1 & -1 \\
   0 & 1 &  0 \\
   0 & 0 & -2 \\
 \end{matrix} &
 \begin{matrix}
   1 \\
   1 \\
   2 \\
 \end{matrix}
\end{array}
\right]
}
\xrightarrow{r_{3}\times\left(-\frac{1}{2}\right)}
{
\left[
\begin{array}{c:c}
 \begin{matrix}
   1 & 1 & -1 \\
   0 & 1 &  0 \\
   0 & 0 & 1 \\
 \end{matrix} &
 \begin{matrix}
   1 \\
   1 \\
   -1 \\
 \end{matrix}
\end{array}
\right]
}
\xrightarrow{r_{1}+r_{3}}
{
\left[
\begin{array}{c:c}
 \begin{matrix}
   1 & 1 & 0 \\
   0 & 1 & 0 \\
   0 & 0 & 1 \\
 \end{matrix} &
 \begin{matrix}
   0 \\
   1 \\
   -1 \\
 \end{matrix}
\end{array}
\right]
}\\
&\xrightarrow{r_{1}-r_{2}}
{
\left[
\begin{array}{c:c}
 \begin{matrix}
   1 & 0 & 0 \\
   0 & 1 &  0 \\
   0 & 0 & 1 \\
 \end{matrix} &
 \begin{matrix}
   -1 \\
   1 \\
   -1 \\
 \end{matrix}
\end{array}
\right]
}
\end{align*}
所以解得:$\mathbf{x}=\begin{bmatrix}
                   -1 \\
                   1 \\
                   -1
                 \end{bmatrix}$。

(b)增广矩阵:
\begin{align*}
\left[
\begin{array}{c:c}
 \begin{matrix}
   1 & 1 &  1 \\
   0 & 1 & -2 \\
   0 & 1 &  2 \\
 \end{matrix} &
 \begin{matrix}
   0 \\
   1 \\
   6 \\
 \end{matrix}
\end{array}
\right]
& \xrightarrow{r_{3}-r_{2}}
{
\left[
\begin{array}{c:c}
 \begin{matrix}
   1 & 1 &  1 \\
   0 & 1 & -2 \\
   0 & 0 &  4 \\
 \end{matrix} &
 \begin{matrix}
   0 \\
   1 \\
   5 \\
 \end{matrix}
\end{array}
\right]
}
\xrightarrow{r_{3}\times\left(\frac{1}{4}\right)}
{
\left[
\begin{array}{c:c}
 \begin{matrix}
   1 & 1 &  1 \\
   0 & 1 & -2 \\
   0 & 0 &  1 \\
 \end{matrix} &
 \begin{matrix}
   0 \\
   1 \\
  \frac{5}{4} \\
 \end{matrix}
\end{array}
\right]
}
\xrightarrow{\substack{r_{2}+2r_{3}\\ r_{1}-r_{3}}}
{
\left[
\begin{array}{c:c}
 \begin{matrix}
   1 & 1 & 0 \\
   0 & 1 & 0 \\
   0 & 0 & 1 \\
 \end{matrix} &
 \begin{matrix}
   -\frac{5}{4} \\
   \frac{14}{4} \\
   \frac{5}{4} \\
 \end{matrix}
\end{array}
\right]
}\\
&\xrightarrow{r_{1}-r_{2}}
{
\left[
\begin{array}{c:c}
 \begin{matrix}
   1 & 0 & 0 \\
   0 & 1 & 0 \\
   0 & 0 & 1 \\
 \end{matrix} &
 \begin{matrix}
   -\frac{19}{4} \\
   \frac{7}{2} \\
   \frac{5}{4} \\
 \end{matrix}
\end{array}
\right]
}
\end{align*}
所以解得:$\mathbf{x}=\begin{bmatrix}
                   -\frac{19}{4} \\
                   \frac{7}{2} \\
                   \frac{5}{4}
                 \end{bmatrix}$。
\end{jie}

2.2015\~{}2016~~三.1.

当$k$为何值时,线性方程组
$
\begin{cases}
kx_{1}+x_{2}+x_{3}=k-3\\
x_{1}+kx_{2}+x_{3}=-2\\
x_{1}+x_{2}+kx_{3}=-2
\end{cases}
$
有唯一解,无解和有无穷多解?当方程组有无穷多解时求出所有解。

\begin{jie}
增广矩阵
\begin{align*}
&\left[
\begin{array}{c:c}
\begin{matrix}
  k & 1 & 1 \\
  1 & k & 1 \\
  1 & 1 & k \\
\end{matrix}
&
\begin{matrix}
  k-3 \\
  -2 \\
  -2 \\
\end{matrix}
\end{array}
\right]
\xrightarrow{r_{1}\leftrightarrow r_{2}}
{
\left[
\begin{array}{c:c}
\begin{matrix}
  1 & k & 1 \\
  k & 1 & 1 \\
  1 & 1 & k \\
\end{matrix}
&
\begin{matrix}
  -2 \\
  k-3 \\
  -2 \\
\end{matrix}
\end{array}
\right]
}
\xrightarrow{\substack{r_{2}-kr_{1} \\ r_{3}-r_{1}}}
{
\left[
\begin{array}{c:c}
\begin{matrix}
  1 & k & 1 \\
  0 & 1-k^{2} & 1-k \\
  0 & 1-k & k-1 \\
\end{matrix}
&
\begin{matrix}
  -2 \\
  3(k-1) \\
  0 \\
\end{matrix}
\end{array}
\right]
}\\
\xrightarrow{r_{2}\leftrightarrow r_{3}} &
{
\left[
\begin{array}{c:c}
\begin{matrix}
  1 & k & 1 \\
  0 & 1-k & k-1 \\
  0 & 1-k^{2} & 1-k \\
\end{matrix}
&
\begin{matrix}
  -2 \\
  0 \\
  3(k-1) \\
\end{matrix}
\end{array}
\right]
}
\xrightarrow{r_{3}-(1+k)r_{2}}
{
\left[
\begin{array}{c:c}
\begin{matrix}
  1 & k & 1 \\
  0 & 1-k & k-1 \\
  0 & 0 & (1-k)(k+2) \\
\end{matrix}
&
\begin{matrix}
  -2 \\
  0 \\
  3(k-1) \\
\end{matrix}
\end{array}
\right]
}
\end{align*}
讨论:

(1)解不存在:即存在矛盾方程(增广矩阵主元列在最右列)。即对于$r_{3}$
\begin{equation*}
  \begin{cases}
    (1-k)(k+2)=0\\
    3(k-1)\neq 0
  \end{cases}~~~
  \Rightarrow~~~k=-2
\end{equation*}

(2)存在唯一解:主元列三个元素都不为0.即
\begin{equation*}
  \begin{cases}
    1\neq 0\\
    1-k\neq 0 \\
    (1-k)(k+2)\neq 0
  \end{cases}
  ~~~\Rightarrow~~~k\neq 1\text{且}k\neq -2
\end{equation*}
$k\neq 1\text{且}k\neq -2$,继续对阶梯矩阵进行初等行变换
\begin{align*}
 \xrightarrow{\substack{r_{2}\times\frac{1}{1-k}\\r_{3}\times\frac{1}{(k+2)(1-k)} }}
{
\left[
\begin{array}{c:c}
\begin{matrix}
  1 & k & 1 \\
  0 & 1 & -1 \\
  0 & 0 & 1 \\
\end{matrix}
&
\begin{matrix}
  -2 \\
  0 \\
  \frac{3}{(k+2)} \\
\end{matrix}
\end{array}
\right]
}
\xrightarrow{\substack{r_{2}+r_{3}\\r_{1}-r_{3} }}
{
\left[
\begin{array}{c:c}
\begin{matrix}
  1 & k & 0 \\
  0 & 1 & 0 \\
  0 & 0 & 1 \\
\end{matrix}
&
\begin{matrix}
  -2-\frac{3}{k+2} \\
  \frac{3}{k+2} \\
  \frac{3}{k+2} \\
\end{matrix}
\end{array}
\right]
}
\xrightarrow{r_{1}-kr_{2}}
{
\left[
\begin{array}{c:c}
\begin{matrix}
  1 & 0 & 0 \\
  0 & 1 & 0 \\
  0 & 0 & 1 \\
\end{matrix}
&
\begin{matrix}
  -\frac{5k+1}{k+2} \\
  \frac{3}{k+2} \\
  \frac{3}{k+2} \\
\end{matrix}
\end{array}
\right]
}
\end{align*}
所以方程组存在唯一解时:$k\neq 1$且$k\neq -2$,解为
\begin{equation*}
\mathbf{x}=
\begin{bmatrix}
 -\frac{5k+1}{k+2} \\
  \frac{3}{k+2} \\
  \frac{3}{k+2}
\end{bmatrix}
~,~~~k\neq1\text{且}k\neq -2
\end{equation*}

(3)存在无穷解:至少存在一个自由变量。由阶梯矩阵可以看出
\begin{equation*}
  \begin{cases}
    (k-1)(k+2)=0\\
    3(k-1)=0
  \end{cases}
  ~~~\Rightarrow~~~k=1
\end{equation*}
把$k=1$代入阶梯矩阵:
\begin{equation*}
\left[
  \begin{array}{c:c}
    \begin{matrix}
      1 & 1 & 1 \\
      0 & 0 & 0 \\
      0 & 0 & 0 \\
    \end{matrix}
     &
     \begin{matrix}
      -2 \\
      0 \\
      0\\
    \end{matrix}
  \end{array}
\right]
~~~\Rightarrow~~~
x=
\begin{bmatrix}
  -2-c_{1}-c_{2} \\
  c_{1} \\
  c_{2}
\end{bmatrix}
\end{equation*}
所以$x$的解为$x=
\begin{bmatrix}
  -2 \\
  0\\
   0
\end{bmatrix}+
\begin{bmatrix}
  -1 \\
  1\\
   0
\end{bmatrix}c_{1}+
\begin{bmatrix}
  -1 \\
  0\\
   1
\end{bmatrix}c_{1}
~~~c_{1},c_{2}\in R$。
\end{jie}

3.2017\~{}2018~~二.3.

求线性方程组
$
\begin{cases}
 2x_{1}-x_{2}+4x_{3}-3x_{4}=-4\\
 x_{1}+x_{3}-x_{4}=-3\\
 3x_{1}+x_{2}+x_{3}=1\\
 7x_{1}+7x_{3}-3x_{4}=3\\
\end{cases}
$
的通解。

\begin{jie}
增广矩阵
\begin{align*}
&\left[
\begin{array}{c:c}
 \begin{matrix}
   2 & -1 & 4 &-3 \\
   1 & 0 & 1 &-1 \\
   7 & 0 & 7 &-3\\
 \end{matrix}
 &
  \begin{matrix}
   -4 \\
   -3\\
   3\\
 \end{matrix}
\end{array}
\right]
\xrightarrow{\substack{r_{2}-\frac{1}{2}r_{1} \\ r_{3}-\frac{7}{2}r_{1}}}
{
\left[
\begin{array}{c:c}
 \begin{matrix}
   2 & -1 & 4 &-3 \\
   0 & \frac{1}{2} & -1 &\frac{1}{2} \\
   0 & \frac{7}{2} & -7 &\frac{15}{2}\\
 \end{matrix}
 &
  \begin{matrix}
   -4 \\
   -1\\
   17\\
 \end{matrix}
\end{array}
\right]
}
\xrightarrow{r_{3}-7r_{2}}
{
\left[
\begin{array}{c:c}
 \begin{matrix}
   2 & -1 & 4 &-3 \\
   0 & \frac{1}{2} & -1 &\frac{1}{2} \\
   0 &0& 0 &4\\
 \end{matrix}
 &
  \begin{matrix}
   -4 \\
   -1\\
   24\\
 \end{matrix}
\end{array}
\right]
}\\
\xrightarrow{\substack{r_{1}\times\frac{1}{2} \\ r_{2}\times 2 \\ r_{3}\times \frac{1}{4}}}&
{
\left[
\begin{array}{c:c}
 \begin{matrix}
   1 & -\frac{1}{2} & 2 &-\frac{3}{2} \\
   0 & 1 & -2 &1 \\
   0 &0& 0 &1\\
 \end{matrix}
 &
  \begin{matrix}
   -2 \\
   -2\\
   6\\
 \end{matrix}
\end{array}
\right]
}
\xrightarrow{\substack{r_{2}-r_{3}\\r_{1}+\frac{3}{2} r_{3}}}
{
\left[
\begin{array}{c:c}
 \begin{matrix}
   1 & -\frac{1}{2} & 2 &0 \\
   0 & 1 & -2 &0 \\
   0 &0& 0 &1\\
 \end{matrix}
 &
  \begin{matrix}
   7 \\
   -8\\
   6\\
 \end{matrix}
\end{array}
\right]
}
\xrightarrow{r_{1}+\frac{1}{2} r_{2}}
{
\left[
\begin{array}{c:c}
 \begin{matrix}
   1 & 0 & 1 &0 \\
   0 & 1 & -2 &0 \\
   0 &0& 0 &1\\
 \end{matrix}
 &
  \begin{matrix}
   3 \\
   -8\\
   6\\
 \end{matrix}
\end{array}
\right]
}
\end{align*}
由最简阶梯型矩阵可以看出:
\begin{equation*}
  x_{1}=3-x_{3}~~x_{2}=x_{3}-8~~x_{3}=x_{3}~~x_{4}=6
\end{equation*}
令$x_{3}=C,C\in R$,则
\begin{equation*}
x=
 \begin{bmatrix}
   3-C \\
   C-8 \\
   C\\
   6
 \end{bmatrix}
 =
  \begin{bmatrix}
   3 \\
   -8 \\
   0\\
   6
 \end{bmatrix}
 +
  \begin{bmatrix}
   -1 \\
   1 \\
   1\\
   0
 \end{bmatrix}C
 ~,~C\in R
\end{equation*}
\end{jie}

4.2018\~{}2019~~三.1.

设
$
\begin{cases}
\lambda x_{1}+x_{2}+x_{3}=\lambda-2\\
x_{1}+\lambda x_{2} +x_{3}=2\\
x_{1}+ x_{2} +\lambda x_{3}=2
\end{cases}
$
,$\lambda$为何值时,该方程组无解、唯一解、无穷解?并且在有唯一解时求出解;有无穷多解时,求出全部解并用向量表示。

\begin{jie}
增广矩阵
\begin{align*}
&\left[
\begin{array}{c:c}
\begin{matrix}
\lambda & 1 & 1 \\
1& \lambda & 1 \\
1 & 1&\lambda  \\
\end{matrix}
&
\begin{matrix}
\lambda-2 \\
2 \\
2 \\
\end{matrix}
\end{array}
\right]
\xrightarrow{r_{1}\Leftrightarrow r_{2}}
{
\left[
\begin{array}{c:c}
\begin{matrix}
1& \lambda & 1 \\
\lambda & 1 & 1 \\
1 & 1&\lambda  \\
\end{matrix}
&
\begin{matrix}
2 \\
\lambda-2 \\
2 \\
\end{matrix}
\end{array}
\right]
}
\xrightarrow{\substack{r_{2}-\lambda r_{1}\\ r_{3}- r_{1} }}
{
\left[
\begin{array}{c:c}
\begin{matrix}
1& \lambda & 1 \\
0 & 1-\lambda^{2} & 1-\lambda \\
0 & 1-\lambda &\lambda-1  \\
\end{matrix}
&
\begin{matrix}
2 \\
-\lambda-2 \\
0 \\
\end{matrix}
\end{array}
\right]
}\\
\xrightarrow{r_{2}\Leftrightarrow r_{3}}&
{
\left[
\begin{array}{c:c}
\begin{matrix}
1& \lambda & 1 \\
0 & 1-\lambda &\lambda-1  \\
0 & 1-\lambda^{2} & 1-\lambda \\
\end{matrix}
&
\begin{matrix}
2 \\
0 \\
-\lambda-2 \\
\end{matrix}
\end{array}
\right]
}
\xrightarrow{r_{3}-(\lambda+1) r_{2}}
{
\left[
\begin{array}{c:c}
\begin{matrix}
1& \lambda & 1 \\
0 & 1-\lambda &\lambda-1  \\
0 & 0 & (\lambda-1)(-2-\lambda) \\
\end{matrix}
&
\begin{matrix}
2 \\
0 \\
-\lambda-2 \\
\end{matrix}
\end{array}
\right]
}
\end{align*}
讨论:

(1)解不存在:即存在矛盾方程(增广矩阵主元列在最右列)。即对于$r_{3}$
\begin{equation*}
  \begin{cases}
    (\lambda-1)(-2-\lambda)=0\\
    -\lambda-2\neq 0
  \end{cases}~~~
  \Rightarrow~~~\lambda=1
\end{equation*}

(2)存在唯一解:主元列三个元素都不为0.即
\begin{equation*}
  \begin{cases}
    1\neq 0\\
    1-\lambda\neq 0 \\
    (\lambda-1)(-2-\lambda)\neq 0
  \end{cases}
  ~~~\Rightarrow~~~\lambda\neq 1\text{且}\lambda\neq -2
\end{equation*}
$\lambda\neq 1\text{且}\lambda\neq -2$,继续对阶梯矩阵进行初等行变换
\begin{equation*}
\xrightarrow{\substack{ r_{2}\times \frac{1}{1-\lambda}\\  r_{3}\times \frac{1}{(\lambda-1)(-2-\lambda)}}}
{
\left[
\begin{array}{c:c}
\begin{matrix}
1& \lambda & 1 \\
0 & 1 & -1  \\
0 & 0 & 1 \\
\end{matrix}
&
\begin{matrix}
2 \\
0 \\
\frac{1}{\lambda-1} \\
\end{matrix}
\end{array}
\right]
}
\xrightarrow{\substack{ r_{2}+r_{3}\\  r_{1}-r_{3}}}
{
\left[
\begin{array}{c:c}
\begin{matrix}
1& \lambda & 0 \\
0& 1 & 0  \\
0 & 0 & 1 \\
\end{matrix}
&
\begin{matrix}
\frac{2\lambda-3}{\lambda-1} \\
\frac{1}{\lambda-1} \\
\frac{1}{\lambda-1} \\
\end{matrix}
\end{array}
\right]
}
\xrightarrow{ r_{1}-\lambda r_{3}}
{
\left[
\begin{array}{c:c}
\begin{matrix}
1& 0 & 0 \\
0 & 1 & 0  \\
0 & 0 & 1 \\
\end{matrix}
&
\begin{matrix}
\frac{\lambda-3}{\lambda-1} \\
\frac{1}{\lambda-1} \\
\frac{1}{\lambda-1} \\
\end{matrix}
\end{array}
\right]
}
\end{equation*}
所以方程组存在唯一解时:$\lambda\neq 1$且$\lambda\neq -2$,解为
\begin{equation*}
\mathbf{x}=
\begin{bmatrix}
\frac{\lambda-3}{\lambda-1} \\
\frac{1}{\lambda-1} \\
\frac{1}{\lambda-1}
\end{bmatrix}
~,~~~\lambda\neq1\text{且}\lambda\neq -2
\end{equation*}

(3)存在无穷解:至少存在一个自由变量。由阶梯矩阵可以看出
\begin{equation*}
  \begin{cases}
    (\lambda-1)(-2-\lambda)=0\\
    -2-\lambda=0
  \end{cases}
  ~~~\Rightarrow~~~\lambda=-2
\end{equation*}
把$\lambda=-2$代入阶梯矩阵:
\begin{equation*}
\left[
\begin{array}{c:c}
\begin{matrix}
1& -2 & 1 \\
0 & 3 & -3  \\
0 & 0 & 0 \\
\end{matrix}
&
\begin{matrix}
2 \\
0 \\
0 \\
\end{matrix}
\end{array}
\right]
\xrightarrow{ r_{2}\times\frac{1}{3}}
{\left[
\begin{array}{c:c}
\begin{matrix}
1& -2 & 1 \\
0 & 1 & -1  \\
0 & 0 & 0 \\
\end{matrix}
&
\begin{matrix}
2 \\
0 \\
0 \\
\end{matrix}
\end{array}
\right]
}
\xrightarrow{ r_{1}+2r_{2}}
{\left[
\begin{array}{c:c}
\begin{matrix}
1& 0 & -1 \\
0 & 1 & -1  \\
0 & 0 & 0 \\
\end{matrix}
&
\begin{matrix}
2 \\
0 \\
0 \\
\end{matrix}
\end{array}
\right]
}
\end{equation*}
由最简阶梯型矩阵可以看出:
\begin{equation*}
  x_{1}=2+x_{3}~~x_{2}=x_{3}~~x_{3}=x_{3}
\end{equation*}
令$x_{3}=C,C\in R$,则
\begin{equation*}
x=
 \begin{bmatrix}
   2+C \\
   C \\
   C
 \end{bmatrix}
 =
  \begin{bmatrix}
   2 \\
   0 \\
   0
 \end{bmatrix}
 +
  \begin{bmatrix}
   1 \\
   1 \\
   1
 \end{bmatrix}C
 ~,~C\in R
\end{equation*}

\end{jie}
5.2019\~{}2020~~一.4.

若线性方程组
$
\begin{cases}
x_{1}+x_{2}=-a_{1}\\
x_{2}+x_{3}=a_{2}\\
x_{3}+x_{4}=-a_{3}\\
x_{4}+x_{1}=a_{4}
\end{cases}
$
有解,$a_{1},a_{2},a_{3},a_{4}$应满足的条件是\underline{~~\textcolor[rgb]{1.00,0.00,0.00}{$a_{1}+a_{2}-a_{3}+a_{4}=0$}~~}。

\begin{jie}
增广矩阵
\begin{align*}
&\left[
\begin{array}{c:c}
\begin{matrix}
1 & 1 & 0 & 0 \\
0 & 1 & 1 & 0 \\
0 & 0 & 1 & 1 \\
1 & 0 & 0 & 1 \\
\end{matrix}
&
\begin{matrix}
-a_{1} \\
a_{2} \\
-a_{3} \\
a_{4} \\
\end{matrix}
\end{array}
\right]
\xrightarrow{r_{4}-r_{1}}
{
\left[
\begin{array}{c:c}
\begin{matrix}
1 & 1 & 0 & 0 \\
0 & 1 & 1 & 0 \\
0 & 0 & 1 & 1 \\
0 & -1 & 0 & 1 \\
\end{matrix}
&
\begin{matrix}
-a_{1} \\
a_{2} \\
-a_{3} \\
a_{1}+a_{4} \\
\end{matrix}
\end{array}
\right]
}
\xrightarrow{r_{4}+r_{2}}
{
\left[
\begin{array}{c:c}
\begin{matrix}
1 & 1 & 0 & 0 \\
0 & 1 & 1 & 0 \\
0 & 0 & 1 & 1 \\
0 & 0 & 1 & 1 \\
\end{matrix}
&
\begin{matrix}
-a_{1} \\
a_{2} \\
-a_{3} \\
a_{1}+a_{2}+a_{4} \\
\end{matrix}
\end{array}
\right]
}\\
\xrightarrow{r_{4}-r_{3}}&
{
\left[
\begin{array}{c:c}
\begin{matrix}
1 & 1 & 0 & 0 \\
0 & 1 & 1 & 0 \\
0 & 0 & 1 & 1 \\
0 & 0 & 0 & 0 \\
\end{matrix}
&
\begin{matrix}
-a_{1} \\
a_{2} \\
-a_{3} \\
a_{1}+a_{2}-a_{3}+a_{4} \\
\end{matrix}
\end{array}
\right]
}
\end{align*}
若方程有解:$a_{1}+a_{2}-a_{3}+a_{4}=0$
\end{jie}
%\begin{align*}
%\left[
%\begin{array}{c:c}
%\begin{matrix}
%1 & 1 & 0 & 0 \\
%0 & 1 & 1 & 0 \\
%0 & 0 & 1 & 1 \\
%1 & 0 & 0 & 1 \\
%\end{matrix}
%&
%\begin{matrix}
%-a_{1} \\
%a_{2} \\
%-a_{3} \\
%a_{4} \\
%\end{array}
%\right]
%\xrightarrow{r_{4}-r_{1}}
%{
%\left[
%\begin{array}{c:c}
%\begin{matrix}
%1 & 1 & 0 & 0 \\
%0 & 1 & 1 & 0 \\
%0 & 0 & 1 & 1 \\
%0 & -1 & 0 & 1 \\
%\end{matrix}
%&
%\begin{matrix}
%-a_{1} \\
%a_{2} \\
%-a_{3} \\
%a_{1}+a_{4} \\
%\end{array}
%\right]
%}
%\xrightarrow{r_{4}+r_{2}}
%{
%\left[
%\begin{array}{c:c}
%\begin{matrix}
%1 & 1 & 0 & 0 \\
%0 & 1 & 1 & 0 \\
%0 & 0 & 1 & 1 \\
%0 & 0 & 1 & 1 \\
%\end{matrix}
%&
%\begin{matrix}
%-a_{1} \\
%a_{2} \\
%-a_{3} \\
%a_{1}+a_{2}+a_{4} \\
%\end{array}
%\right]
%}
%\end{align*}
6.期末2019-2020 二1.

设$
A=
\begin{bmatrix}
  0 & 1 & 0 \\
  0 & 0 & 1\\
  0 & 0 & 0
\end{bmatrix}
$.求满足$AX=XA$的全部的矩阵$X$。

\begin{jie}
设$X=
\begin{bmatrix}
  a &b& c \\
  d & e &f\\
  g&h&i
\end{bmatrix}
$,
\begin{gather*}
AX=\begin{bmatrix}
  0 & 1 & 0 \\
  0 & 0 & 1\\
  0 & 0 & 0
\end{bmatrix}\begin{bmatrix}
  a &b& c \\
  d & e &f\\
  g&h&i
\end{bmatrix}=
\begin{bmatrix}
d & e & f\\
g & h & i\\
0& 0 & 0
\end{bmatrix}\\
XA=
\begin{bmatrix}
  a &b& c \\
  d & e &f\\
  g&h&i
\end{bmatrix}\begin{bmatrix}
  0 & 1 & 0 \\
  0 & 0 & 1\\
  0 & 0 & 0
\end{bmatrix}=
\begin{bmatrix}
  0 & a & b\\
  0 & d & e\\
  0 & g & h 
\end{bmatrix}
\end{gather*}
$AX=XA$,即
\begin{align*}
\begin{bmatrix}
d & e & f\\
g & h & i\\
0& 0 & 0
\end{bmatrix}=\begin{bmatrix}
  0 & a & b\\
  0 & d & e\\
  0 & g & h
\end{bmatrix}~~~~\Rightarrow
\begin{cases}
d=0~~a=e\hphantom{=0}~~~b=f\\
g=0~~h=d=0~~i=e=a\\
0=0~~g=0\hphantom{=0}~~~h=0
\end{cases}
\end{align*}
所以$x=
\begin{bmatrix}
  a & b & c \\
  0& a & b\\
  0 & 0 & a
\end{bmatrix}
$,其中$a,b,c$是任意常数。
\end{jie}


{\heiti \zihao{-3} 期中试题}

2017\~{}2018~~二.3.判断命题是否成立并给出理由

若$A^{2}=B^{2}$,则$A=B$或$A=-B$。

\begin{jie}
判断是否成立:\textcolor[rgb]{1.00,0.00,0.00}{否}。

理由:
\begin{align*}
 A=\begin{bmatrix}
  1 & 0\\
  0 & 1
   \end{bmatrix}~~~
 A^{2}=\begin{bmatrix}
         1 & 0\\
         0 & 1
       \end{bmatrix}\\
 B=\begin{bmatrix}
  0 & 1\\
  1 & 0
   \end{bmatrix}~~~
 B^{2}=\begin{bmatrix}
         1 & 0\\
         0 & 1
       \end{bmatrix}
\end{align*}
可以看出$A^{2}=B^{2}$,但是$A\neq B$,$A\neq -B$
\end{jie}
%$X\stackrel{\substack{F\\F}}{\longrightarrow}Y$   %268页,281页
%$\xrightarrow{122222222222222223}$
\end{document}  