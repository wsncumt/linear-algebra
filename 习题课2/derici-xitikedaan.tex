\documentclass{article}
\usepackage[space,fancyhdr,fntef]{ctexcap}
\usepackage[namelimits,sumlimits,nointlimits]{amsmath}
\usepackage[bottom=25mm,top=25mm,left=25mm,right=15mm,centering]{geometry}
\usepackage{xcolor}
\usepackage{arydshln}%234页,虚线表格宏包
\pagestyle{fancy} \fancyhf{}
\fancyhead[OL]{~~~班序号:\hfill 学院:\hfill 学号:\hfill 姓名:王松年~~~ \thepage}
%\usepackage{parskip}
%\usepackage{indentfirst}
\usepackage{graphicx}%插图宏包,参见手册318页
\begin{document}

\newcounter{num} \renewcommand{\thenum}{\arabic{num}.} \newcommand{\num}{\refstepcounter{num}\text{\thenum}}

\newenvironment{jie}{\kaishu\zihao{-5}\color{blue}{\noindent\em 解:}\par}{\hfill $\diamondsuit$\par}

\hphantom{~~}\hfill {\zihao{3}\heiti 第二次习题课--答案} \hfill\hphantom{~~}

\hphantom{~~}\hfill {\zihao{4}\heiti 群文件《期中$\&$期末试题》} \hfill\hphantom{~~}


{\heiti \zihao{-3} 期中试题}

{\heiti \zihao{-3} ~}\\

2016\~{}2017~~二.3.

设$
A=
\begin{bmatrix}
  1& 0 &-1 \\
  1&3&0\\
  0&2&1
\end{bmatrix}
$,若矩阵$X$满足方程$AX+I=A^{2}+X$,求$X$。

\begin{jie}
方法一:\textcolor[rgb]{1.00,0.00,0.00}{视频25}

由题得:$AX+I=A^{2}+X$,所以$(A-I)X=A^{2}-I$.
\begin{equation*}
  A^{2}=
  \begin{bmatrix}
  1&-2&-2\\
  4&9&-1\\
  2&8&1
  \end{bmatrix}~~~A^{2}-I=
  \begin{bmatrix}
   0&-2&-2\\
  4&8&-1\\
  2&8&0
  \end{bmatrix}~~~A-I=
  \begin{bmatrix}
   0&0&-1\\
  1&2&0\\
  0&2&0
  \end{bmatrix}
\end{equation*}
增广矩阵
\begin{align*}
&\left[
  \begin{array}{c:c}
    \begin{matrix}
  0&0&-1\\
  1&2&0\\
  0&2&0\\
    \end{matrix}
    &
    \begin{matrix}
    0&-2&-2\\
  4&8&-1\\
  2&8&0
    \end{matrix}
  \end{array}\right]
  \xrightarrow{ r_{1}\Leftrightarrow r_{2}}
  {
  \left[
  \begin{array}{c:c}
    \begin{matrix}
  1&2&0\\
  0&0&-1\\
  0&2&0\\
    \end{matrix}
    &
    \begin{matrix}
    4&8&-1\\
    0&-2&-2\\
  2&8&0
    \end{matrix}
  \end{array}\right]
  }
  \xrightarrow{ r_{2}\Leftrightarrow r_{3}}
  {
  \left[
  \begin{array}{c:c}
    \begin{matrix}
  1&2&0\\
  0&2&0\\
  0&0&-1\\
    \end{matrix}
    &
    \begin{matrix}
    4&8&-1\\
  2&8&0\\
  0&-2&-2\\
    \end{matrix}
  \end{array}\right]
  }\\
\xrightarrow{\substack{  r_{2}\times\frac{1}{2}\\ r_{3}\times(-1)}}&
  {
  \left[
  \begin{array}{c:c}
    \begin{matrix}
  1&2&0\\
  0&1&0\\
  0&0&1\\
    \end{matrix}
    &
    \begin{matrix}
    4&8&-1\\
  1&4&0\\
  0&2&2\\
    \end{matrix}
  \end{array}\right]
  }
  \xrightarrow{r_{1}-2r_{2}}
  {
  \left[
  \begin{array}{c:c}
    \begin{matrix}
  1&0&0\\
  0&1&0\\
  0&0&1\\
    \end{matrix}
    &
    \begin{matrix}
    2&0&-1\\
  1&4&0\\
  0&2&2\\
    \end{matrix}
  \end{array}\right]
  }
\end{align*}
所以
\begin{equation*}
  x=\begin{bmatrix}
       2&0&-1\\
  1&4&0\\
  0&2&2\\
    \end{bmatrix}
\end{equation*}

方法二:由题得:$AX+I=A^{2}+X$,所以$(A-I)X=A^{2}-I=(A+I)(A-I)=(A-I)(A+I)$.
等式两边同时左乘$(A-I)^{-1}$:
\begin{equation*}
 (A-I)^{-1}(A-I)X=(A-I)^{-1}(A-I)(A+I) ~~~\Rightarrow~~~X=(A+I)
\end{equation*}
\textcolor[rgb]{1.00,0.00,0.00}{注:\\
1.单位矩阵乘任何矩阵都等于该矩阵的本身,任何矩阵乘单位矩阵都等于该矩阵的本身,即$AI=IA=A$。($I$为单位矩阵)\\
所以有
\begin{align*}
&A^ {2}-I=A^{2}-I^{2}\\
&(A+I)(A-I)=A^{2}-AI+IA-I^{2}=A^{2}-A+A-I^{2}=A^{2}-I^{2}\\
&(A-I)(A+I)=A^{2}+AI-IA-I^{2}=A^{2}+A-A-I^{2}=A^{2}-I^{2}
\end{align*}
2.可逆矩阵(方阵):若$AB=BA=I$,则$A$矩阵可逆,称$A$是$B$的逆矩阵(或$B$是$A$的逆矩阵)。通常把$A$的逆矩阵表示为$A^{-1}$,则$AA^{-1}=A^{-1}A=I$。
}
\end{jie}

{\heiti \zihao{-3} ~}\\
{\heiti \zihao{-3} ~}\\
{\heiti \zihao{-3} ~}\\
2016\~{}2017~~一.4.

设$
A=
\begin{bmatrix}
  3&2&2 \\
  0&1&1\\
  0&0&3
\end{bmatrix}
,~B=
\begin{bmatrix}
  1&0&0 \\
  0&0&0\\
  0&0&-1
\end{bmatrix}
,~AX+2B=BA+2X$,求$X^{2017}$。

\begin{jie}
方法一:

由题得:$(A-2I)X=B(A-2I)$
\begin{equation*}
A-2I=
\begin{bmatrix}
1&2&2\\
0&-1&1\\
0&0&1\\
\end{bmatrix}
~~~B(A-2I)=
\begin{bmatrix}
1&2&2\\
0&0&0\\
0&0&-1\\
\end{bmatrix}
\end{equation*}
增广矩阵
\begin{align*}
&
\left[
\begin{array}{c:c}
\begin{matrix}
1&2&2\\
0&-1&1\\
0&0&1\\
\end{matrix}
&
\begin{matrix}
1&2&2\\
0&0&0\\
0&0&-1\\
\end{matrix}
\end{array}
\right]
\xrightarrow{r_{1}\times(-1)}
{
\left[
\begin{array}{c:c}
\begin{matrix}
1&2&2\\
0&1&-1\\
0&0&1\\
\end{matrix}
&
\begin{matrix}
1&2&2\\
0&0&0\\
0&0&-1\\
\end{matrix}
\end{array}
\right]
}
\xrightarrow{\substack{  r_{1}-2r_{3}\\ r_{2}+r_{3}}}
{
\left[
\begin{array}{c:c}
\begin{matrix}
1&2&0\\
0&1&0\\
0&0&1\\
\end{matrix}
&
\begin{matrix}
1&2&4\\
0&0&-1\\
0&0&-1\\
\end{matrix}
\end{array}
\right]
}\\
\xrightarrow{r_{1}-2r_{2}}&
{
\left[
\begin{array}{c:c}
\begin{matrix}
1&0&0\\
0&1&0\\
0&0&1\\
\end{matrix}
&
\begin{matrix}
1&2&6\\
0&0&-1\\
0&0&-1\\
\end{matrix}
\end{array}
\right]
}
\end{align*}
解得
\begin{equation*}
X=
\begin{bmatrix}
1&2&6\\
0&0&-1\\
0&0&-1
\end{bmatrix}
\end{equation*}
计算得
\begin{align*}
&X^{2}=
X=
\begin{bmatrix}
1&2&-2\\
0&0&1\\
0&0&1
\end{bmatrix}\\
&X^{3}=X^{2}\cdot X=
\begin{bmatrix}
1&2&-2\\
0&0&1\\
0&0&1
\end{bmatrix}
\cdot
\begin{bmatrix}
1&2&6\\
0&0&-1\\
0&0&-1
\end{bmatrix}=
\begin{bmatrix}
1&2&6\\
0&0&-1\\
0&0&-1
\end{bmatrix}=X\\
&X^{3}=X\cdot X^{2}=
\begin{bmatrix}
1&2&6\\
0&0&-1\\
0&0&-1
\end{bmatrix}
\cdot
\begin{bmatrix}
1&2&-2\\
0&0&1\\
0&0&1
\end{bmatrix}=
\begin{bmatrix}
1&2&6\\
0&0&-1\\
0&0&-1
\end{bmatrix}=X
\end{align*}
可以看出:当$n$为奇数时,$X^{n}=X$,所以$X^{2017}=X$。

方法二:

\textcolor[rgb]{1.00,0.00,0.00}{若存在可逆矩阵$P$,使得$A=P^{-1}BP$,则称矩阵$A$与矩阵$B$相似,记为$A\~{}B$}

由题得:$(A-2I)X=B(A-2I)$,等式两边同时乘$(A-2I)^{-1}$得:$X=(A-2I)^{-1}B(A-2I)$,令$P=A-2I$,则$X=P^{-1}BP$.所以
\begin{equation*}
  X^{2}=(P^{-1}BP)^{2}=P^{-1}B\textcolor[rgb]{0.00,0.50,1.00}{PP^{-1}}BP=P^{-1}B\textcolor[rgb]{0.00,0.50,1.00}{I}BP=P^{-1}B^{2}P
\end{equation*}
同理可以推出:$X^{n}=P^{-1}B^{n}P$.
(由此我们可以得出一条结论:\textcolor[rgb]{1.00,0.00,0.00}{若$A\~{}B$,则$A^{n}=P^{-1}B^{n}P$},以后做题可直接使用)
\textcolor[rgb]{1.00,0.00,0.00}{对于任意对角矩阵$C=
\begin{bmatrix}
 c_{11} &~&~&~\\
 ~&c_{22}&~&~\\
 ~&~&\ddots&~\\
 ~&~&~&c_{nn}
\end{bmatrix}
$},可以计算得:$C^{2}=
\begin{bmatrix}
 c_{11}^{2} &~&~&~\\
 ~&c_{22}^{2}&~&~\\
 ~&~&\ddots&~\\
 ~&~&~&c_{nn}^{2}
\end{bmatrix}$,所以\textcolor[rgb]{1.00,0.00,0.00}{$C^{n}=
\begin{bmatrix}
 c_{11}^{n} &~&~&~\\
 ~&c_{22}^{n}&~&~\\
 ~&~&\ddots&~\\
 ~&~&~&c_{nn}^{n}
\end{bmatrix}$}
所以:
\begin{equation*}
B^{2017}=
\begin{bmatrix}
1^{2017}&0&0\\
0&0&0\\
0&0&(-1)^{2017}
\end{bmatrix}=
\begin{bmatrix}
1&0&0\\
0&0&0\\
0&0&-1
\end{bmatrix}=B
\end{equation*}
\textcolor[rgb]{0.50,0.00,0.00}{由于还没学逆矩阵的求法,这里先直接给结论,后边学到了逆矩阵的求法后再来计算这个P的逆矩阵}
\begin{equation*}
P=A-2I=
\begin{bmatrix}
1&2&2\\
0&-1&1\\
0&0&1\\
\end{bmatrix}~~~
P^{-1}=
\begin{bmatrix}
1 &2&-4\\
0&-1&1\\
0&0&1
\end{bmatrix}
\end{equation*}
所以:
\begin{align*}
X^{2017}&=P^{-1}B^{2017}P=P^{-1}BP=\begin{bmatrix}
1 &2&-4\\
0&-1&1\\
0&0&1
\end{bmatrix}
\begin{bmatrix}
  1&0&0 \\
  0&0&0\\
  0&0&-1
\end{bmatrix}
\begin{bmatrix}
1&2&2\\
0&-1&1\\
0&0&1\\
\end{bmatrix}\\
&=
\begin{bmatrix}
1&0&4\\
0&0&-1\\
0&0&-1\\
\end{bmatrix}
\begin{bmatrix}
1&2&2\\
0&-1&1\\
0&0&1\\
\end{bmatrix}=\begin{bmatrix}
1&2&6\\
0&0&-1\\
0&0&-1
\end{bmatrix}
\end{align*}
\end{jie}
{\heiti \zihao{-3} ~}\\
{\heiti \zihao{-3} ~}\\

{\heiti \zihao{-3} 考研例题}

  1.设$\alpha,\beta$是3维列向量,$\beta^{T}$是$\beta$的转置,如果$
  \alpha\beta^{T}=
  \begin{bmatrix}
    1 & -1 & 2 \\
    -2 & 2& -4\\
    3 & -3 & 6
  \end{bmatrix}
  $,则$\alpha^{T}\beta=\underline{~~~~~~9~~~~~}$。

\begin{jie}
$\alpha,\beta$是3维列向量,所以可以设
$\alpha=\begin{bmatrix}
          x_{1} & x_{2} & x_{3}
        \end{bmatrix}^{T},\beta=\begin{bmatrix}
          y_{1} & y_{2} & y_{3}
        \end{bmatrix}^{T}$
由题得:
\begin{equation*}
\alpha\beta^{T}=
\begin{bmatrix}
          x_{1}\\
          x_{2}\\
          x_{3}
        \end{bmatrix}
        \begin{bmatrix}
          y_{1} & y_{2} & y_{3}
        \end{bmatrix}=
        \begin{bmatrix}
          x_{1}y_{1} & x_{1}y_{2} & x_{1}y_{3}\\
          x_{2}y_{1} & x_{2}y_{2} & x_{2}y_{3}\\
          x_{3}y_{1} & x_{3}y_{2} & x_{3}y_{3}
        \end{bmatrix}=
\begin{bmatrix}
    1 & -1 & 2 \\
    -2 & 2& -4\\
    3 & -3 & 6
  \end{bmatrix}
\end{equation*}
所以:
\begin{equation*}
 \alpha^{T}\beta=
\begin{bmatrix}
x_{1} & x_{2} & x_{3}
\end{bmatrix}
\begin{bmatrix}
  y_{1} \\
  y_{2} \\
 y_{3}
\end{bmatrix}=x_{1}y_{1}+x_{2}y_{2}+x_{3}y_{3}=1+2+6=9
\end{equation*}

\textcolor[rgb]{1.00,0.00,0.00}{注:}

\textcolor[rgb]{1.00,0.00,0.00}{若$\alpha,\beta$是3维列向量,记:$A=\alpha\beta^{T}=\begin{bmatrix}
          x_{1}y_{1} & x_{1}y_{2} & x_{1}y_{3}\\
          x_{2}y_{1} & x_{2}y_{2} & x_{2}y_{3}\\
          x_{3}y_{1} & x_{3}y_{2} & x_{3}y_{3}
        \end{bmatrix}$,}
        \textcolor[rgb]{1.00,0.00,0.00}{
        $l=\alpha^{T}\beta=\beta^{T}\alpha=x_{1}y_{1}+x_{2}y_{2}+x_{3}y_{3}$,(可以看出,$l$是方阵$A$主对角线元素之和)则\\
        $A^{2}=(\alpha\beta^{T})(\alpha\beta^{T})=\alpha\beta^{T}\alpha\beta^{T}=\alpha(\beta^{T}\alpha)\beta^{T}=\alpha l \beta^{T}=l\alpha\beta^{T}=lA$\\
        所以可以推出$A^{n}=l^{n-1}A$,我们可以将此结论推广到任意n维列向量的情况(即$\alpha,\beta$是n维列向量,记:$A=\alpha\beta^{T},l=\alpha^{T}\beta=\beta^{T}\alpha$,$A^{n}=l^{n-1}A$)。
        }
\end{jie}

  2.若$
  A=
  \begin{bmatrix}
   0&0&0\\
   2&0&0\\
   1&3&0
  \end{bmatrix}
  $,则$A^{2}=\underline{~~~~~  \begin{bmatrix}
   0&0&0\\
   0&0&0\\
   6&0&0
  \end{bmatrix}~~~~},A^{3}=\underline{~~~~~~0~~~~~}$.

\begin{jie}
由矩阵的乘法:
\begin{equation*}
 A^{2}= \begin{bmatrix}
   0&0&0\\
   2&0&0\\
   1&3&0
  \end{bmatrix}  \begin{bmatrix}
   0&0&0\\
   2&0&0\\
   1&3&0
  \end{bmatrix}=
    \begin{bmatrix}
   0&0&0\\
   0&0&0\\
   6&0&0
  \end{bmatrix}
\end{equation*}
\begin{equation*}
 A^{3}=A^{2}A=    \begin{bmatrix}
   0&0&0\\
   0&0&0\\
   6&0&0
  \end{bmatrix}  \begin{bmatrix}
   0&0&0\\
   2&0&0\\
   1&3&0
  \end{bmatrix}=
    \begin{bmatrix}
   0&0&0\\
   0&0&0\\
   0&0&0
  \end{bmatrix}
\end{equation*}

\textcolor[rgb]{1.00,0.00,0.00}{注:由题我们可以得出如下结论:如果一个n阶方阵A对角线上以及对角线的一侧元素全为0,那么必有$A^{k}=0$,其中$k\geq n$。即A是下边的几种形状之一:
\begin{equation*}
\begin{bmatrix}
  0 &\textcolor[rgb]{0.00,0.50,1.00}{ a_{12}}&\textcolor[rgb]{0.00,0.50,1.00}{a_{13}}&\textcolor[rgb]{0.00,0.50,1.00}{\cdots}&\textcolor[rgb]{0.00,0.50,1.00}{a_{1n}} \\
  0& 0&\textcolor[rgb]{0.00,0.50,1.00}{a_{23}}&\textcolor[rgb]{0.00,0.50,1.00}{\cdots}&\textcolor[rgb]{0.00,0.50,1.00}{a_{2n}}\\
   0&0&0&\textcolor[rgb]{0.00,0.50,1.00}{\cdots}&\textcolor[rgb]{0.00,0.50,1.00}{a_{3n}}\\
  \vdots&\vdots&\vdots&\ddots&\textcolor[rgb]{0.00,0.50,1.00}{\vdots}\\
  0&0&0&\cdots&0
\end{bmatrix}
~~
\begin{bmatrix}
  0 & 0&0&\cdots&0 \\
  \textcolor[rgb]{0.00,0.50,1.00}{a_{21}}& 0&0&\cdots&0\\
   \textcolor[rgb]{0.00,0.50,1.00}{a_{31}}&\textcolor[rgb]{0.00,0.50,1.00}{a_{32}}&0&\cdots&0\\
  \textcolor[rgb]{0.00,0.50,1.00}{\vdots}&\textcolor[rgb]{0.00,0.50,1.00}{\vdots}&\textcolor[rgb]{0.00,0.50,1.00}{\vdots}&\ddots&\vdots\\
  \textcolor[rgb]{0.00,0.50,1.00}{a_{n1}}&\textcolor[rgb]{0.00,0.50,1.00}{a_{n2}}&\textcolor[rgb]{0.00,0.50,1.00}{a_{n3}}&\cdots&0
\end{bmatrix}
~~
\begin{bmatrix}
  0 & 0&0&\cdots&0 \\
 0& 0&0&\cdots&\textcolor[rgb]{0.00,0.50,1.00}{a_{2n}}\\
   0&0&0&\textcolor[rgb]{0.00,0.50,1.00}{\cdots}&\textcolor[rgb]{0.00,0.50,1.00}{a_{3n}}\\
  \vdots&\vdots&\textcolor[rgb]{0.00,0.50,1.00}{\vdots}&\textcolor[rgb]{0.00,0.50,1.00}{\vdots}&\textcolor[rgb]{0.00,0.50,1.00}{\vdots}\\
  0&\textcolor[rgb]{0.00,0.50,1.00}{a_{n2}}&\textcolor[rgb]{0.00,0.50,1.00}{a_{n3}}&\textcolor[rgb]{0.00,0.50,1.00}{\cdots}&\textcolor[rgb]{0.00,0.50,1.00}{a_{nn}}
\end{bmatrix}
~~
\begin{bmatrix}
 \textcolor[rgb]{0.00,0.50,1.00}{ a_{11}} & \textcolor[rgb]{0.00,0.50,1.00}{a_{12}}&\textcolor[rgb]{0.00,0.50,1.00}{a_{13}}&\textcolor[rgb]{0.00,0.50,1.00}{\cdots}&0 \\
 \textcolor[rgb]{0.00,0.50,1.00}{a_{21}}& \textcolor[rgb]{0.00,0.50,1.00}{a_{22}}&\textcolor[rgb]{0.00,0.50,1.00}{a_{23}}&\cdots&0\\
  \textcolor[rgb]{0.00,0.50,1.00}{ a_{31}}&\textcolor[rgb]{0.00,0.50,1.00}{a_{32}}&0&\cdots&0\\
  \textcolor[rgb]{0.00,0.50,1.00}{\vdots}&\vdots&\vdots&\vdots&\vdots\\
  0&0&0&\cdots&0
\end{bmatrix}
\end{equation*}
}
\end{jie}
  3.若$
  A=
  \begin{bmatrix}
   1 & 2 & 3\\
   0& 1&4\\
   0& 0&1
  \end{bmatrix}
  $,则$A^{n}=\underline{~~~~~~~~~~~~~}.$

\begin{jie}

由题得:
$
A=
\begin{bmatrix}
  1&2&3\\
  0&1&4\\
  0&0&1\\
\end{bmatrix}=
\begin{bmatrix}
1&0&0\\
  0&1&0\\
  0&0&1\\
\end{bmatrix}+
\begin{bmatrix}
  0&2&3\\
  0&0&4\\
  0&0&0\\
\end{bmatrix}=I+B
$\\
\textcolor[rgb]{1.00,0.00,0.00}{二项式定理:$(a+b)^{n}=C_{n}^{0}a^{0}b^{n}+C_{n}^{1}a^{1}b^{n-1}+C_{n}^{2}a^{2}b^{n-2}+\cdots+C_{n}^{n-1}a^{n-1}b^{n-(n-1)}+C_{n}^{n}a^{n}b^{n-n}$}

由第2题的结论可知:$B^{k}=0,k\geq3$。计算得:$B^{2}=\begin{bmatrix}
                                        0 & 0&8 \\
                                       0 & 0&0 \\  0 & 0&0 \\
                                      \end{bmatrix}$所以
\begin{align*}
A^{n}&=(I+B)^{n}=C_{n}^{0}I^{n}B^{0}+C_{n}^{1}I^{n-1}B^{1}+C_{n}^{2}I^{n-2}B^{2}+C_{n}^{3}I^{n-3}\textcolor[rgb]{1.00,0.00,0.00}{B^{3}}+\cdots+C_{n}^{0}I^{0}\textcolor[rgb]{1.00,0.00,0.00}{B^{n}}\\
&=I^{n}B^{0}+nI^{n-1}B^{1}+\frac{n(n-1)}{2}I^{n-2}B^{2}=I+nB+\frac{n(n-1)}{2}B^{2}\\
&=\begin{bmatrix}
1&0&0\\
  0&1&0\\
  0&0&1\\
\end{bmatrix}+n\begin{bmatrix}
  0&2&3\\
  0&0&4\\
  0&0&0\\
\end{bmatrix}+\frac{n(n-1)}{2}
\begin{bmatrix}
                                        0 & 0&8 \\
                                       0 & 0&0 \\  0 & 0&0 \\
                                      \end{bmatrix}\\
&=
\begin{bmatrix}
  1&2n&4n^{2}-n\\
  0&1&4n\\
  0&0&1\\
\end{bmatrix}
\end{align*}
\end{jie}
  4.设$A=
  \begin{bmatrix}
 3&1&0&0\\
 0&3&0&0\\
 0&0&3&9\\
 0&0&1&3
  \end{bmatrix}
  $,则$A^{n}=\underline{~~~~~~~~~~~~~}.$

\begin{jie}
经观察,我们可将矩阵按下列方式进行分块:
\begin{equation*}
A=
\begin{bmatrix}
3&1&0&0\\
 0&3&0&0\\
 0&0&3&9\\
 0&0&1&3
\end{bmatrix}
=
\begin{bmatrix}
  A_{11} & \mathbf{0} \\
  \mathbf{0} & A_{22}
\end{bmatrix}~~~
A_{11}=
\begin{bmatrix}
  3 & 1 \\
  0 & 3
\end{bmatrix}
~A_{22}=
\begin{bmatrix}
  3 & 9 \\
  1 & 3
\end{bmatrix}
\end{equation*}
$A$分块后是对角矩阵,所以:$A^{n}=\begin{bmatrix}
  A_{11}^{n} & \mathbf{0} \\
  \mathbf{0} & A_{22}^{n}
\end{bmatrix}$(注:这个结论是在上边期中测试的第二个题里的方法二推导出来的)\\
对于$A_{11}$:
\begin{align*}A_{11}&=
\begin{bmatrix}
  3 & 1 \\
  0 & 3
\end{bmatrix}=
\begin{bmatrix}
  3 & 0 \\
  0 & 3
\end{bmatrix}+
\begin{bmatrix}
  0 & 1 \\
  0 & 0
\end{bmatrix}=3I+B\\
&\text{和上一个题一样了,此时$B^{k}=0,k\geq2$}\\
A^{n}_{11}&=(3I+B)^{n}=C_{n}^{0}B^{0}(3I)^{n}+C_{n}^{1}B^{1}(3I)^{n-1}+C_{n}^{2}\textcolor[rgb]{1.00,0.00,0.00}{B^{2}}(3I)^{n-2}+\cdots+C_{n}^{n}\textcolor[rgb]{1.00,0.00,0.00}{B^{n}}(3I)^{0}\\
&=3^{n}I+3^{n-1}nB\\
&=\begin{bmatrix}
  3^{n} & n3^{n-1} \\
  0 & 3^{n}
\end{bmatrix}
\end{align*}
\textcolor[rgb]{1.00,0.00,0.00}{注:把一个矩阵按高斯消元法化为阶梯型矩阵后,非零行行数即为矩阵的行秩,对于一个方阵A,行秩等于列秩,用符号$r(A)$来表示秩。}

\textcolor[rgb]{1.00,0.00,0.00}{对于n维方阵$A$,如果$r(A)=1$,则$A$一定可以分解为一个列向量和一个行向量的乘积(此处只给结论,不做证明)。然后使用第一题得出的结论来计算$A^{n}$}

所以对于$A_{22}$,经过高斯消元法变换($r_{1}-3r_{2}$)后:$
\begin{bmatrix}
  3 & 9 \\
  0 & 0
\end{bmatrix}
$,可以看出$r(A_{22})=1$,符合红色字部分的描述,所以
\begin{align*}
 l&=3+3=6\\
 A_{22}^{n}&=l^{n-1}A=6^{n-1}
 \begin{bmatrix}
   3 & 9 \\
   1 & 3
 \end{bmatrix}
\end{align*}
所以:
\begin{equation*}
  A^{n}=
\begin{bmatrix}
  A_{11}^{n} & \mathbf{0} \\
  \mathbf{0} & A_{22}^{n}
\end{bmatrix}=
\begin{bmatrix}
  3^{n} & n\cdot3^{n-1}&0&0 \\
  0 & 3^{n}&0&0\\
  0&0&3\cdot6^{n-1}&9\cdot6^{n-1}\\
  0&0&6^{n-1}&3\cdot6^{n-1}
\end{bmatrix}
\end{equation*}
\end{jie}
  5.已知$
  A=
  \begin{bmatrix}
    2&0&1\\
    0&3&0\\
    2&0&2
  \end{bmatrix},
  B=
  \begin{bmatrix}
    1&0&0\\
    0&-1&0\\
    0&0&0
  \end{bmatrix}
  $,若$X$满足$AX+2B=BA+2X$,则$X^{4}=\underline{~~~~~~~~~~~~~}.$

  \begin{jie}
  步骤同上边期中测试题第二题的方法二,所以略。自己算一下。这里给出$P=A-2I$的逆矩阵$P^{-1}$:
  \begin{equation*}
    P^{-1}=
    \begin{bmatrix}
      0 & 0 & 0.5\\
      0 & 1 &0\\
      1&0&0
    \end{bmatrix}
  \end{equation*}
最后算出的结果是
\begin{equation*}
X^{4}=
\begin{bmatrix}
  0 & 0 &0 \\
  0 & 1&0 \\
  0&0&1
\end{bmatrix}
\end{equation*}
  \end{jie}
\end{document}  