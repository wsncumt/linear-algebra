\documentclass{article}
\usepackage[space,fancyhdr,fntef]{ctexcap}
\usepackage[namelimits,sumlimits,nointlimits]{amsmath}
\usepackage[bottom=25mm,top=25mm,left=25mm,right=15mm,centering]{geometry}
\usepackage{xcolor}
\usepackage{arydshln}%234页,虚线表格宏包
\pagestyle{fancy} \fancyhf{}
\fancyhead[OL]{~~~班序号:\hfill 学院:\hfill 学号:\hfill 姓名:王松年~~~ \thepage}
%\usepackage{parskip}
%\usepackage{indentfirst}
\usepackage{graphicx}%插图宏包,参见手册318页
\begin{document}

\newcounter{num} \renewcommand{\thenum}{\arabic{num}.} \newcommand{\num}{\refstepcounter{num}\text{\thenum}}

\hphantom{~~}\hfill {\zihao{3}\heiti 第二次习题课} \hfill\hphantom{~~}

\hphantom{~~}\hfill {\zihao{4}\heiti 群文件《期中$\&$期末试题》} \hfill\hphantom{~~}


{\heiti \zihao{-3} 期中试题}

{\heiti \zihao{-3} ~}\\

2016\~{}2017~~二.3.

设$
A=
\begin{bmatrix}
  1& 0 &-1 \\
  1&3&0\\
  0&2&1
\end{bmatrix}
$,若矩阵$X$满足方程$AX+I=A^{2}+X$,求$X$。

{\heiti \zihao{-3} ~}\\

2016\~{}2017~~一.4.

设$
A=
\begin{bmatrix}
  3&2&2 \\
  0&1&1\\
  0&0&3
\end{bmatrix}
,~B=
\begin{bmatrix}
  1&0&0 \\
  0&0&0\\
  0&0&-1
\end{bmatrix}
,~AX+2B=BA+2X$,求$X^{2017}$。

{\heiti \zihao{-3} ~}\\
{\heiti \zihao{-3} ~}\\

{\heiti \zihao{-3} 考研例题}

  1.设$\alpha,\beta$是3维列向量,$\beta^{T}$是$\beta$的转置,如果$
  \alpha\beta^{T}=
  \begin{bmatrix}
    1 & -1 & 2 \\
    -2 & 2& -4\\
    3 & -3 & 6
  \end{bmatrix}
  $,则$\alpha^{T}\beta=\underline{~~~~~~~~~~~~~}$。

  2.若$
  A=
  \begin{bmatrix}
   0&0&0\\
   2&0&0\\
   1&3&0
  \end{bmatrix}
  $,则$A^{2}=\underline{~~~~~~~~~~~~~},A^{3}=\underline{~~~~~~~~~~~~~}$.

  3.若$
  A=
  \begin{bmatrix}
   1 & 2 & 3\\
   0& 1&4\\
   0& 0&1
  \end{bmatrix}
  $,则$A^{n}=\underline{~~~~~~~~~~~~~}.$

  4.设$A=
  \begin{bmatrix}
 3&1&0&0\\
 0&3&0&0\\
 0&0&3&9\\
 0&0&1&3
  \end{bmatrix}
  $,则$A^{n}=\underline{~~~~~~~~~~~~~}.$

  5.已知$
  A=
  \begin{bmatrix}
    2&0&1\\
    0&3&0\\
    2&0&2
  \end{bmatrix},
  B=
  \begin{bmatrix}
    1&0&0\\
    0&-1&0\\
    0&0&0
  \end{bmatrix}
  $,若$X$满足$AX+2B=BA+2X$,则$X^{4}=\underline{~~~~~~~~~~~~~}.$
\end{document}  