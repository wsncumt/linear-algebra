\documentclass{article}
\usepackage[space,fancyhdr,fntef]{ctexcap}
\usepackage[namelimits,sumlimits,nointlimits]{amsmath}
\usepackage[bottom=25mm,top=25mm,left=25mm,right=15mm,centering]{geometry}
\usepackage{xcolor}
\usepackage{paralist}%列表宏包
\usepackage{arydshln}%234页,虚线表格宏包
\pagestyle{fancy} \fancyhf{}
\fancyhead[OL]{~~~班序号:\hfill 学院:\hfill 学号:\hfill 姓名:王松年~~~ \thepage}
%\usepackage{parskip}
%\usepackage{indentfirst}
\usepackage{graphicx}%插图宏包,参见手册318页
\begin{document}

\newcounter{num} \renewcommand{\thenum}{\arabic{num}.} \newcommand{\num}{\refstepcounter{num}\text{\thenum}}

\hphantom{~~}\hfill {\zihao{3}\heiti 第二次习题课} \hfill\hphantom{~~}

\hphantom{~~}\hfill {\zihao{4}\heiti 知识点} \hfill\hphantom{~~}

1.矩阵的转置。$(AB)^{T}=B^{T}A^{T}$

2.向量的内积。

3.方阵的幂。

4.矩阵多项式:设A是方阵,$f(x)=a_{0}+a_{1}x+\cdots+a_{n}x^{n}$,则定义$f(A)=a_{0}E+a_{1}A+\cdots+a_{n}A^{n}$.

如果$X$与$A$可交换,则$X$与$f(A)$可交换。

5.分块矩阵:分块矩阵的定义。分块矩阵的加减,数乘。

分块矩阵的乘法:设A、B都是分块矩阵,A的列数和B的行数相同,且A的列分法与B的行分法相同,则采用A的行分法和B的列分法对AB进行分块,看作分块矩阵$(AB)_{pq}=\sum_{j=1}^{t}A_{pj}B_{jp}$。\\
分块矩阵乘法常用的结果:
\begin{asparaenum}[(1)]
\item 若$A_{m\times n}=
\begin{bmatrix}
  \alpha_{1}\\
  \vdots\\
  \alpha_{m}
\end{bmatrix}
$,$
B_{n\times l}=\begin{bmatrix}
                \beta_{1},\cdots,\beta_{l}
              \end{bmatrix}
$,则$
AB=
\begin{bmatrix}
  \alpha_{1}\beta_{1} & \alpha_{1}\beta_{2} & \cdots & \alpha_{1}\beta_{l}\\
  \alpha_{2}\beta_{1} & \alpha_{2}\beta_{2} & \cdots & \alpha_{2}\beta_{l}\\
   \vdots & \vdots & \ddots & \vdots\\
   \alpha_{m}\beta_{1} & \alpha_{m}\beta_{2} & \cdots & \alpha_{m}\beta_{l}\\
\end{bmatrix}
$
\item 若$A_{m\times n},B_{n\times l}=\begin{bmatrix}\beta_{1}&\beta_{2}&\cdots&\beta_{l}\end{bmatrix}$,则有
\begin{equation*}
  AB=
  \begin{bmatrix}
    A\beta_{1} & A\beta_{2} & \cdots &A\beta_{l}
  \end{bmatrix}
\end{equation*}
\item 如果$AB=0$,显然$A\beta_{j}=0$,也就是$B$的列向量是$Ax=0$的解。
\item 设$A=\begin{bmatrix}\alpha_{1}&\alpha_{2}&\cdots&\alpha_{n}\end{bmatrix}$,
\begin{equation*}
  \begin{bmatrix}
 \alpha_{1}&\alpha_{2}&\cdots&\alpha_{n}
  \end{bmatrix}
  \begin{bmatrix}
 x_{1}\\
 x_{2}\\
 \vdots\\
 x_{n}
  \end{bmatrix}=
  \alpha_{1}x_{1}+\alpha_{2}x_{2}+\cdots+\alpha_{n}x_{n}
\end{equation*}
\end{asparaenum}

分块矩阵的转置:如果A是分块矩阵,则$A^{T}$的行采用与$A$的列相同的分法。即
\begin{equation*}
A=
  \begin{bmatrix}
    A_{11} & A_{12} & \cdots & A_{1t}\\
   A_{21} & A_{22} & \cdots & A_{2t}\\
   \vdots & \vdots & \ddots & \vdots\\
   A_{s1} & A_{s2} & \cdots & A_{st}
  \end{bmatrix}
  ~~~~A^{T}=
  \begin{bmatrix}
    A_{11}^{T} & A_{21}^{T} & \cdots & A_{s1}^{T} \\
    A_{12}^{T} & A_{22}^{T} & \cdots & A_{s2}^{T} \\
    \vdots & \vdots & \ddots & \vdots\\
    A_{1t}^{T} & A_{2t}^{T} & \cdots & A_{st}^{T} \\
  \end{bmatrix}
\end{equation*}

6.线性方程组、矩阵方程、向量方程组的转化。
\end{document}  