\documentclass{article}
\usepackage[space,fancyhdr,fntef]{ctexcap}
\usepackage[namelimits,sumlimits,nointlimits]{amsmath}
\usepackage[bottom=25mm,top=25mm,left=25mm,right=15mm,centering]{geometry}
\usepackage{xcolor}
\usepackage{arydshln}%234页,虚线表格宏包
\pagestyle{fancy} \fancyhf{}
\fancyhead[OL]{~~~班序号:\hfill 学院:\hfill 学号:\hfill 姓名:王松年~~~ \thepage}
%\usepackage{parskip}
%\usepackage{indentfirst}
\usepackage{graphicx}%插图宏包,参见手册318页
\usepackage{mathdots}%反对角省略号
\begin{document}

\newcounter{num} \renewcommand{\thenum}{\arabic{num}.} \newcommand{\num}{\refstepcounter{num}\text{\thenum}}

\newenvironment{jie}{\kaishu\zihao{-5}\color{blue}{\noindent\em 解:}\par}{\hfill $\diamondsuit$\par}

\newenvironment{zhengming}{\kaishu\zihao{-5}\color{blue}{\noindent\em 证明:}\par}{\hfill $\diamondsuit$\par}

\hphantom{~~}\hfill {\zihao{3}\heiti 第六次习题课} \hfill\hphantom{~~}

\hphantom{~~}\hfill {\zihao{4}\heiti 群文件《期中$\&$期末试题》} \hfill\hphantom{~~}

{\heiti \zihao{4} 期中试题}

\num 期中2015-2016 一2.

设$f(x)=
\begin{vmatrix}
  2x & x & 1 & 2\\
  1 & x & 1 & -1\\
  3 & 2 & x & 1\\
  1 & 1 & 1 & x\\
\end{vmatrix}
$,则$x^{3}$的系数为\underline{\hphantom{~~~~~~~~~~}}。

\begin{jie}
方法一:求出对应的行列式,然后写出$x^3$的系数。(此方法太过繁琐,容易出错,不推荐使用)

用$Matlab$计算出来的结果为:$f(x)=2x^4-x^3-7x^2+12x-8$.(仅供参考)
\\

方法二:使用定义,课本106-108页。

思路:使用行列式的定义来做。仅找出与$x^{3}$有关的项。这里取列按照自然排列,行由自己指定(也可以取行按照自然排列,列由自己指定)。

第一列中:取第一行,第二列第三列第四列无论怎么取都不可能构成$x^{3}$。

\textcolor[rgb]{1.00,0.00,0.00}{(注意:在行列式的定义式中,每一项中的几个元素必须来自不同的行数和列数,如:对于此题来说,列按自然排列,第一个元素取第一列中的第一行,那么第二个元素只能从剩下三列中的剩下三行来取)}

第一列中:取第二行,第二列取第一行,第三列取第三行,第四列取第四行。即$(-1)^{\tau(2134)}a_{\textcolor[rgb]{1.00,0.00,0.00}{2}\textcolor[rgb]{0.00,0.00,0.00}{1}}a_{\textcolor[rgb]{1.00,0.00,0.00}{1}\textcolor[rgb]{0.00,0.00,0.00}{2}}a_{\textcolor[rgb]{1.00,0.00,0.00}{3}\textcolor[rgb]{0.00,0.00,0.00}{3}}a_{\textcolor[rgb]{1.00,0.00,0.00}{4}\textcolor[rgb]{0.00,0.00,0.00}{4}}=(-1)^{1}1*x*x*x=-x^{3}$(注意下标,列(黑色)是自然排列,行(红色)是上边分析得来的)

第一列取第三行第四行都不能构成$x^{3}$。

\textcolor[rgb]{0.50,0.00,0.00}{
验证:$x^{2}$的系数
}

列取自然排列,行按下述几个取时构成$x^2$:$1324~~3124~~3214~~4231~~4132$
即:
\begin{align*}
&(-1)^{\tau(1324)}a_{11}a_{32}a_{23}a_{44}+(-1)^{\tau(3124)}a_{31}a_{12}a_{23}a_{44}+\\
&(-1)^{\tau(3214)}a_{31}a_{22}a_{13}a_{44}+(-1)^{\tau(4132)}a_{41}a_{12}a_{33}a_{24}+(-1)^{\tau(4231)}a_{41}a_{22}a_{33}a_{14}\\
=&(-4+3-3-1-2)x^2=-7x^2
\end{align*}
可以看出和方法1算的结果一样。
\end{jie}

\num 期中2015-2016 一5.

若$A$为4阶方阵,$A^{*}$为$A$的伴随矩阵,$|A|=\dfrac{1}{2}$,则$\left|\left(\dfrac{1}{4}A\right)^{-1}-A^{*}\right|=$\underline{\hphantom{~~~~~~~~~~}}。

\begin{jie}
$\left|\left(\dfrac{1}{4}A\right)^{-1}-A^{*}\right|=\left|\left(\dfrac{1}{4}A\right)^{-1}-|A|A^{-1}\right|=\left|4A^{-1}-\frac{1}{2}A^{-1}\right|=\left|\frac{7}{2}A^{-1}\right|=\left(\frac{7}{2}\right)^{4}|A|^{-1}=\frac{7^4}{8}$
\end{jie}

\num 期中2015-2016 一6.
设$A=
\begin{bmatrix}
  1 & 0 & 0\\
  1 & 1 & 0\\
  1 & 2 & 3
\end{bmatrix}
$,则$(A^*)^{-1}=$\underline{\hphantom{~~~~~~~~~~}}。

\begin{jie}
$A^*=|A|A^{-1}$,所以$(A^*)^{-1}=(|A|A^{-1})^{-1}=|A|^{-1}(A^{-1})^{-1}=|A|^{-1}A$

$|A|=1\times 1\times 3=3$
所以:$(A^*)^{-1}=\dfrac{1}{3}A=\begin{bmatrix}
  \frac{1}{3} & 0 & 0\\
  \frac{1}{3} & \frac{1}{3} & 0\\
  \frac{1 }{3}& \frac{2}{3} & 1
\end{bmatrix}$
\end{jie}

\num 期中2015-2016 三1.

设$A$可逆,且$A^{*}B=A^{-1}+B$,证明$B$可逆,当$A=
\begin{bmatrix}
  2 & 6 & 0 \\
  0 & 2 & 6\\
  0 & 0 & 2
\end{bmatrix}
$时,求$B$。

\begin{jie}
由题得:$A^{*}B=A^{-1}+B$,即$(A^*-E)B=A^{-1}$,两边同时左乘$A$得:$A(A^*-E)B=E$,所以$B$可逆,其逆矩阵为$A(A^*-E)=(|A|E-A)$.

由题得:$|A|=2\times2\times 2=8$
\begin{equation*}
B=(|A|E-A)^{-1}=\begin{bmatrix}
  6 & -6 & 0 \\
  0 & 6 & -6\\
  0 & 0 & 6
\end{bmatrix}^{-1}=\frac{1}{6}\begin{bmatrix}
  1 & 1 & 1 \\
  0 & 1 & 1\\
  0 & 0 & 1
\end{bmatrix}
\end{equation*}
\end{jie}

\num 期中2016-2017 一5.

若$A$为3阶方阵,$A^{*}$为$A$的伴随矩阵,$|A|=\dfrac{1}{2}$,则$\left|(3A)^{-1}-2A^{*}\right|=$\underline{\hphantom{~~~~~~~~~~}}。

\begin{jie}
$\left|(3A)^{-1}-2A^{*}\right|=\left|3^{-1}A^{-1}-2|A|A^{-1}\right|=\left|-\frac{2}{3}A^{-1}\right|=\left(-\frac{2}{3}\right)^{3}|A|^{-1}=-\frac{16}{27}$
\end{jie}

\num 期中2016-2017 二5.

若$\left(\dfrac{1}{4}A^{*}\right)^{-1}BA^{-1}=2AB+I$,且
$A=
\begin{bmatrix}
  2 & 0& 0 & 0 \\
  1 & 1 & 0& 0 \\
  0 & 0 & 2& 1\\
   0& 0 &0 &1
\end{bmatrix}
$,求$B$。

\begin{jie}
由题得:$A=
\begin{bmatrix}
  A_{11} & 0 \\
  0 & A_{22}
\end{bmatrix}
$,其中$A_{11}=\begin{bmatrix}
              2 & 0 \\
              1 & 1
            \end{bmatrix}$,
$A_{22}=
\begin{bmatrix}
  2 & 1 \\
  0 & 1
\end{bmatrix}
$,所以有:
\begin{equation*}
|A_{11}|=|A_{22}|=2\times1=2;~~|A|=|A_{11}|\cdot|A_{22}|=4;~~A_{11}^{-1}=
\begin{bmatrix}
  \frac{1}{2} & 0 \\
  -\frac{1}{2} & 1
\end{bmatrix};~~A_{22}^{-1}=
\begin{bmatrix}
  \frac{1}{2} & -\frac{1}{2} \\
   0& 1
\end{bmatrix};~~A^{-1}=\begin{bmatrix}
  A_{11}^{-1} & 0 \\
  0 & A_{22}^{-1}
\end{bmatrix}
\end{equation*}
\begin{gather*}
\left(\dfrac{1}{4}A^{*}\right)^{-1}BA^{-1}=2AB+I ~~~\Rightarrow~~~
4(|A|A^{-1})^{-1}BA^{-1}=2AB+I ~~~ \Rightarrow~~~
\frac{4}{|A|}ABA^{-1}=2AB+I\\
ABA^{-1}=2AB+I~~~\Rightarrow~~~
AB=2ABA+A~~~\Rightarrow~~~
AB(E-2A)=A~~~\Rightarrow~~~
B(E-2A)=I~~~\Rightarrow~~
B=(E-2A)^{-1}
\end{gather*}
\begin{equation*}
  B=(E-2A)^{-1}=\begin{bmatrix}
  -3 & 0& 0 & 0 \\
  -2 & -1 & 0& 0 \\
  0 & 0 & -3& -2\\
   0& 0 &0 &-1
\end{bmatrix}^{-1}=
\begin{bmatrix}
  -\frac{1}{3} & 0& 0 & 0 \\
  \frac{2}{3} & -1 & 0& 0 \\
  0 & 0 & -\frac{1}{3}& \frac{2}{3}\\
   0& 0 &0 &-1
\end{bmatrix}
\end{equation*}
\end{jie}

\num 期中2017-2018 二2.判断是否正确并说明理由。

设$A,B$为$n$阶可逆方阵,则$(AB)^{*}=B^{*}A^{*}$.

\begin{jie}
正确,理由如下:

因为$A,B$为$n$阶可逆方阵,所以$AB$可逆,所以$(AB)^{*}=|AB|(AB)^{-1}=|A||B|B^{-1}A^{-1}=|A|B^{*}A^{-1}=B^{*}A^{*}$
\end{jie}

\num 期中2018-2019 一2.

设$A,B$为3阶矩阵,且$|A|=3,|B|=2$,$A^{*}$为$A$的伴随矩阵。

(1)若交换$A$的第一行与第二行得矩阵$C$,求$|CA^{*}|$;

\begin{jie}
交换交换$A$的第一行与第二行得矩阵$C$,所以$|C|=-|A|$,所以$|CA^{*}|=|C||A^{*}|=-|A|||A|A^{-1}|-|A||A|^{3}|A|^{-1}=-|A|^3=-27$
\end{jie}

\num 期中2018-2019 一3.

已知3阶矩阵$A$的逆矩阵$
A^{-1}=
\begin{bmatrix}
  1 & 1 & 1 \\
  1 & 2 & 1 \\
  2 & 1 & 3
\end{bmatrix}
$,试求伴随矩阵$A^{*}$的逆矩阵。

\begin{jie}
由伴随矩阵的性质:$(A^{*})^{-1}=(A^{-1})^*$,所以
\begin{gather*}
A^{-1}_{11}=(-1)^{1+1}\begin{bmatrix}
 2 & 1 \\
 1 & 3
\end{bmatrix}=\textcolor[rgb]{1.00,0.00,0.00}{5}~~~A^{-1}_{21}=(-1)^{2+1}\begin{bmatrix}
 1 & 1 \\
 1 & 3
\end{bmatrix}=\textcolor[rgb]{1.00,0.00,0.00}{-2}~~~A^{-1}_{31}=(-1)^{3+1}\begin{bmatrix}
 1 & 1 \\
 2 & 1
\end{bmatrix}=\textcolor[rgb]{1.00,0.00,0.00}{-1}\\
A^{-1}_{12}=(-1)^{1+2}\begin{bmatrix}
 1 & 1 \\
 2 & 3
\end{bmatrix}=\textcolor[rgb]{1.00,0.00,0.00}{-1}~~~A^{-1}_{22}=(-1)^{2+2}\begin{bmatrix}
 1 & 1 \\
 2 & 3
\end{bmatrix}=\textcolor[rgb]{1.00,0.00,0.00}{1}~~~A^{-1}_{32}=(-1)^{3+2}\begin{bmatrix}
 1 & 1 \\
 1 & 1
\end{bmatrix}=\textcolor[rgb]{1.00,0.00,0.00}{0}\\
A^{-1}_{13}=(-1)^{1+3}\begin{bmatrix}
  1 & 2\\
 2 & 1 
\end{bmatrix}=\textcolor[rgb]{1.00,0.00,0.00}{-3}~~~A^{-1}_{23}=(-1)^{2+3}\begin{bmatrix}
 1 & 1 \\
 2 & 1
\end{bmatrix}=\textcolor[rgb]{1.00,0.00,0.00}{1}~~~A^{-1}_{33}=(-1)^{3+3}\begin{bmatrix}
 1 & 1 \\
 1 & 2
\end{bmatrix}=\textcolor[rgb]{1.00,0.00,0.00}{1}
\end{gather*}
所以$(A^{*})^{-1}=(A^{-1})^*=
\begin{bmatrix}
  5 & -2 & -1\\
  -1 & 1 & 0\\
  -3 & 1 & 1
\end{bmatrix}
$
\end{jie}

\num 期中2018-2019 二1.

若$n$阶实矩阵$Q$满足$QQ^{T}=I$,则称$Q$为正交矩阵。设$Q$为正交矩阵,则

(1)$Q$的行列式为1或-1.

(2)当$|Q|=1$且$n$为奇数时,证明$|I-Q|=0$,其中$I$是$n$阶单位矩阵;

(3)$Q$的逆矩阵$Q^{-1}$和伴随矩阵$Q^{*}$都是正交矩阵。\\

{\heiti \zihao{4} 期末试题}

\num 期末2014-2015 二.

设多项式$
f(x)=
\begin{vmatrix}
  2x & 3 & 1 & 2\\
  x & x & -2 & 1\\
  2 & 1 & x & 4\\
  x & 2 & 1 & 4x
\end{vmatrix}
$,分别求该多项式的三次项、常数项。

\begin{jie}
同第一题。$Matlab$算出的结果为:$f(x)=8x^{4}-14x^3+11x^2-53x+14$(作为参考)

取列为自然排列。分析得:行数按$2134$和$4231$排列时,对应的项为$x^3$。即
\begin{equation*}
(-1)^{\tau(2134)}a_{21}a_{12}a_{33}a_{44}+(-1)^{\tau(4231)}a_{41}a_{22}a_{33}a_{14}=(-12-2)x^{3}=-14x^3
\end{equation*}

同理,取列为自然排列。分析得行数按:$3142$、$3412$和$3421$排列时为常数项,即
\begin{equation*}
(-1)^{\tau(3142)}a_{31}a_{12}a_{43}a_{24}+(-1)^{\tau(3412)}a_{31}a_{42}a_{13}a_{24}+(-1)^{\tau(3421)}a_{31}a_{42}a_{23}a_{14}=-6+4+16=14
\end{equation*}
\end{jie}


\num 期末2014-2015 三.

设$A$的伴随矩阵
$
A^{*}=
\begin{bmatrix}
  2 & 0 & 0 & 0\\
  0 & 2 & 0 & 0\\
  1 & 0 & 2 & 0\\
  0 & -3 & 0 & 8
\end{bmatrix}
$,且$ABA^{-1}=BA^{-1}+3I$,求$B$。

\begin{jie}
由题得:$|A^*|=2\times2\times2\times8=64$
\begin{gather*}
ABA^{-1}=BA^{-1}+3I~~\Rightarrow~~AB=B+3A~~\Rightarrow~~A^*AB=A^*B+3A^*A\\
A^*A=|A|A^{-1}A=|A|I~~~|A^*|=||A|A^{-1}|=|A|^n|A|^{-1}=|A|^{n-1}=|A|^{4-1}=64~~~|A|=4
\end{gather*}
所以$4B=A^*B+3\times4~~~\Rightarrow~~~B=12(4-A^*)^{-1}$.

求逆的过程略。

最后的结果为:
\begin{equation*}
  \begin{bmatrix}
  6 & 0 & 0 & 0\\
  0 & 6 & 0 & 0\\
  3 & 0 & 6 & 0\\
  0 & 4.5 & 0 & -3
\end{bmatrix}
\end{equation*}
\end{jie}

\num 期末2016-2017 一2.

设$A$的伴随矩阵$
A^{*}=
\begin{bmatrix}
  1 & 2 & 3 & 4\\
  0 & 2 & 3 & 4\\
  0 & 0 & 2 & 3\\
  0 & 0 & 0 & 2
\end{bmatrix}
$,则$r(A^{2}-2A)=$\underline{\hphantom{~~~~~~~~~~}}。

\begin{jie}
由上一题的结论:$|A^*|=|A|^{n-1}$得:$|A^*|=2^3=|A|^{3}$.所以$|A|=2\neq0$。即$A$可逆。

所以$r(A^{2}-2A)=r(A(A-2))=r(A-2)=r(|A|(A^*)^{-1}-2)=r((A^*)^{-1}-E)=3$.(求逆和秩的过程略)
\end{jie}

\num 期末2016-2017 二2.

设$
A=
\begin{bmatrix}
  1 & 2 & 3 \\
  0 & 1 & 3\\
  0 & 0 & 1
\end{bmatrix}
$,$B$为三阶矩阵,且满足方程$A^{*}BA=I+2A^{-1}B$,求矩阵$B$。

\begin{jie}
由题得:$|A|=1$,$A^*=|A|A^{-1}=A^{-1}$.对题中方程两边同时左乘$A$得:
\begin{align*}
BA&=A+2B\\
B&=A(A-2E)^{-1}=A=
\begin{bmatrix}
  1 & 2 & 3 \\
  0 & 1 & 3\\
  0 & 0 & 1
\end{bmatrix}
\begin{bmatrix}
  -1 & 2 & 3 \\
  0 & -1 & 3\\
  0 & 0 & -1
\end{bmatrix}^{-1}=\begin{bmatrix}
  -1 & -4 & -18 \\
  0 & -1 & -6\\
  0 & 0 & -1
\end{bmatrix}
\end{align*}

(求逆的过程略。$(A-2E)^{-1}=\begin{bmatrix}
  -1 & -2 & -9 \\
  0 & -1 & -3\\
  0 & 0 & -1
\end{bmatrix}$)
\end{jie}

\num 期末2017-2018 一3.

设$
A=
\begin{bmatrix}
  2 & 0 & 0 \\
  1 & 2 & 0 \\
  1 & 2 & 2
\end{bmatrix}
$,记$A*$是$A$的伴随矩阵,则$(A^{*})^{-1}=$\underline{\hphantom{~~~~~~~~~~}}。

\begin{jie}
$(A^ {*})^{-1}=(|A|A^{-1})^{-1}=\dfrac{A}{|A|}$,由题得:$|A|=8$
\end{jie}
\num 期末2018-2019 一1.

设$A$为5阶方阵满足$|A|=2$,$A^{*}$是$A$的伴随矩阵,则$|2A^{-1}A^{*}A^{T}|=$\underline{\hphantom{~~~~~~~~~~}}。

\begin{jie}
原式=
\begin{equation*}
2^{5}|A^{-1}|\cdot|A^*|\cdot|A^T|=2^{5}\cdot|A|^{-1}\cdot|A|^{5-1}\cdot|A|=2^9=512
\end{equation*}
\end{jie}
\end{document}  