\documentclass{article}
\usepackage[space,fancyhdr,fntef]{ctexcap}
\usepackage[namelimits,sumlimits,nointlimits]{amsmath}
\usepackage[bottom=25mm,top=25mm,left=25mm,right=15mm,centering]{geometry}
\usepackage{xcolor}
\usepackage{paralist}%列表宏包
\usepackage{arydshln}%234页,虚线表格宏包
\pagestyle{fancy} \fancyhf{}
\fancyhead[OL]{~~~班序号:\hfill 学院:\hfill 学号:\hfill 姓名:王松年~~~ \thepage}
%\usepackage{parskip}
%\usepackage{indentfirst}
\usepackage{graphicx}%插图宏包,参见手册318页
\usepackage{amssymb}
\usepackage{bbm}
\begin{document}

\newcounter{num} \renewcommand{\thenum}{\arabic{num}.} \newcommand{\num}{\refstepcounter{num}\text{\thenum}}

\hphantom{~~}\hfill {\zihao{3}\heiti 第十次习题课} \hfill\hphantom{~~}

\hphantom{~~}\hfill {\zihao{4}\heiti 知识点} \hfill\hphantom{~~}


\num 在$\mathbb{R}^n$中,设向量
$\alpha=
\begin{bmatrix}
 a_1\\
  a_2\\
 \vdots\\
  a_n\\
\end{bmatrix},\beta=
\begin{bmatrix}
 b_1\\
  b_2\\
 \vdots\\
  b_n\\
\end{bmatrix}
$,实数$a_1b_1+a_2b_2+\cdots+a_nb_n=\sum\limits_{i=1}^{n}a_ib_i$称为向量$\alpha$和$\beta$的内积。记作$\alpha^T\beta$或$\alpha\cdot\beta$.

性质:

(1)$\alpha^T\beta=\beta^T\alpha$;

(2)$(k\alpha)^T\beta=k\beta^T\alpha$;

(3)$(\alpha+\beta)^T\gamma=\alpha^T\gamma+\beta^T\gamma$

(4)$\alpha^T\alpha\geq0,\alpha^T\alpha=0$,当且仅当$\alpha=0$。

\num $\forall\alpha=(a_1,\cdots,a_n)\in\mathbb{R}^n$,定义其长度
\begin{equation*}
  \|\alpha\|=\sqrt{\alpha\cdot\alpha}=\sqrt{a_1^{2}+a_2^{2}+\cdots+a_n^{2}}
\end{equation*},向量长度称为向量范数。

性质:

(1)$\|\alpha\|\geq0,\|\alpha\|=0$当且仅当$\alpha=0$。非负性

(2)$\|k\alpha\|=|k|\|\alpha\|$,$k$为实数。

(3)对任意向量$\alpha,\beta$,有$|\alpha^T\beta|\leq\|\alpha\|\|\beta\|$($Schwarz$不等式)

\num 单位向量。对任意非零向量$\alpha\in\mathbb{R}^n$,则向量$\dfrac{\alpha}{\|\alpha\|}$是一个单位向量。

\num $\alpha\beta\in\mathbb{R}^n$,若$\alpha$与$\beta$的内积为零,则称$\alpha$与$\beta$正交(垂直)。

\num 如果$\mathbb{R}^n$中的非零向量组$\alpha_{1},\cdots,\alpha_s$两两正交,$\alpha_i^T\alpha_j=0(i\neq j;i,j=1,2,\cdots,s)$,则称该向量组为正交向量组。进一步,如果$\forall i,\|\alpha_{i}\|=1$,则称为单位(规范)正交向量组。

\num $\mathbb{R}^n$中的正交向量组线性无关。

\num $\mathbb{R}^n$中的线性无关向量组$\alpha_{1},\cdots,\alpha_s$可以化为另一正交向量组$\beta_{1},\cdots,\beta_s$,并且
\begin{gather*}
  \alpha_{1}\leftrightarrow \beta_{1}\\
  \alpha_{1}\alpha_{2}\leftrightarrow \beta_{1}\beta_{2} \\
 \cdots\\
 \alpha_{1},\cdots,\alpha_s\leftrightarrow\beta_{1},\cdots,\beta_s
\end{gather*}
这一方法称为施密特正交化方法。
\begin{equation*}
  \beta_{n}=\alpha_{n}-\sum_{i=1}^{n-1}\frac{\alpha_{n}^T\beta_{i}}{\beta_{i}^T\beta_{i}}\beta_{i}~~~(n=1,2,\cdots,s)
\end{equation*}

\num 若$n$阶实矩阵$Q$,满足$Q^TQ=E$,则称$Q$为正交矩阵。

性质:

(1)若$Q$为正交矩阵,则$|Q|=1$或$|Q|=-1$。

(2)若$Q$为正交矩阵,则$Q^{-1}=Q^T$。

(3)若$P,Q$为正交矩阵,则$PQ$是正交矩阵。

(4)$Q$是正交矩阵等价于$QQ^T=E$

\num 设$Q$为$n$阶实矩阵,则$Q$为正交矩阵的充分必要条件是其列(行)向量组是规范正交向量组。

\num 实对称矩阵的特征值都是实数。

\num 实对称矩阵的对应于不同特征值的特征向量是正交的。

\num 设$A$为实对称矩阵,则存在正交矩阵$Q$,使得$Q^{-1}AQ$为对角矩阵。
\end{document}  