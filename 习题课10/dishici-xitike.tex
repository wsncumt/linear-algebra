\documentclass{article}
\usepackage[space,fancyhdr,fntef]{ctexcap}
\usepackage[namelimits,sumlimits,nointlimits]{amsmath}
\usepackage[bottom=25mm,top=25mm,left=25mm,right=15mm,centering]{geometry}
\usepackage{xcolor}
\usepackage{arydshln}%234页,虚线表格宏包
\usepackage{mathdots}%反对角省略号
\pagestyle{fancy} \fancyhf{}
\fancyhead[OL]{~~~班序号:\hfill 学院:\hfill 学号:\hfill 姓名:王松年~~~ \thepage}
%\usepackage{parskip}
%\usepackage{indentfirst}
\usepackage{graphicx}%插图宏包,参见手册318页
\begin{document}

\newcounter{num} \renewcommand{\thenum}{\arabic{num}.} \newcommand{\num}{\refstepcounter{num}\text{\thenum}}

\hphantom{~~}\hfill {\zihao{3}\heiti 第十次习题课} \hfill\hphantom{~~}

\hphantom{~~}\hfill {\zihao{4}\heiti 群文件《期中$\&$期末试题》} \hfill\hphantom{~~}

{\heiti \zihao{4} 期末试题}

\num 期末2015-2016 三2.

设3阶实对称矩阵$A$的特征值为$\lambda_{1}=-1,\lambda_{2}=\lambda_{3}=1$,对应于$\lambda_{1}$的特征向量$\alpha_{1}=(0,1,1)^{T}$。

(1)求$A$对应于特征值1的特征向量;

(2)求$A$;

(3)求$A^{2016}$。\\

\num 期末2016-2017 一4.

设$\alpha_{1}=(a,1,1)^{T},\alpha_{2}=(1,b,-1)^{T},\alpha_{3}=(1,-2,c)^{T}$是正交向量组,则$a+b+c=$\underline{\hphantom{~~~~~~~~~~}}。\\

\num 期末2016-2017 一5.

设3阶实对称矩阵$A$的特征值分别为$1,2,3$对应的特征向量分别为$\alpha_ {1}=(1,1,1)^{T},\alpha_{2}=(2,-1,-1)^{T},\alpha_{3}$,则$A$的对应于特征值3的一个特征向量$\alpha_{3}=$\underline{\hphantom{~~~~~~~~~~}}。\\


\end{document}  