\documentclass{article}
\usepackage[space,fancyhdr,fntef]{ctexcap}
\usepackage[namelimits,sumlimits,nointlimits]{amsmath}
\usepackage[bottom=25mm,top=25mm,left=25mm,right=15mm,centering]{geometry}
\usepackage{xcolor}
\usepackage{arydshln}%234页,虚线表格宏包
\usepackage{mathdots}%反对角省略号
\pagestyle{fancy} \fancyhf{}
\fancyhead[OL]{~~~班序号:\hfill 学院:\hfill 学号:\hfill 姓名:王松年~~~ \thepage}
%\usepackage{parskip}
%\usepackage{indentfirst}
\usepackage{graphicx}%插图宏包,参见手册318页
\begin{document}

\newcounter{num} \renewcommand{\thenum}{\arabic{num}.} \newcommand{\num}{\refstepcounter{num}\text{\thenum}}

\hphantom{~~}\hfill {\zihao{3}\heiti 第十次习题课} \hfill\hphantom{~~}

\hphantom{~~}\hfill {\zihao{4}\heiti 群文件《期中$\&$期末试题》} \hfill\hphantom{~~}

{\heiti \zihao{4} 考研例题--特征值}

\num 求矩阵$A=
\begin{bmatrix}
  2 & 1 & 3\\
  4& 2 & 6\\
  6 & 3 & 9
\end{bmatrix}
$的特征值与特征向量。\\

\num 已知$a\neq 0$,求矩阵
\begin{equation*}
  \begin{bmatrix}
    1 & a & a& a\\
    a & 1& a& a\\
    a& a& 1& a\\
    a& a& a& 1
  \end{bmatrix}
\end{equation*}
的特征值、特征向量。\\

\num 抽象矩阵1

设$A$是三阶矩阵,且矩阵$A$的各行元素之和均为5,则矩阵$A$必有特征向量\underline{\hphantom{~~~~~~~~~~~~~}}.\\

\num 抽象矩阵2

已知$A$是3阶矩阵,如果非齐次线性方程组$Ax=b$有通解$5b+k_1\eta_1+k_2\eta_2$,其中$\eta_1,\eta_2$是$Ax=0$的基础解系,求$A$的特征值和特征向量。\\

\num 抽象矩阵3

设$A$为3阶方阵,且$|A-2E|=|A+2E|=|A-E|=0$,则$|3A^*-2A^{-1}|=$\underline{\hphantom{~~~~~~~~~~~~~}}.\\

{\heiti \zihao{4} 考研例题--实对称矩阵}

\num 设$A$是3阶实对称矩阵,秩$r(A)=2$,若$A^2=A$,则$A$的特征值是\underline{\hphantom{~~~~~~~~~~~~~}}.\\

\num $n$阶矩阵
\begin{equation*}
A=
\begin{bmatrix}
  a & 1 & 1 & \cdots & 1\\
 1 & a & 1&\cdots & 1\\
 1 & 1 & a & \cdots &1\\
 \vdots&\vdots&\vdots&\ddots&\vdots\\
 1&1&1&\cdots&a
\end{bmatrix}
\end{equation*}
则$r(A)=$\underline{\hphantom{~~~~~~~~~~~~~}}.\\

\num 设$\alpha$为$n$维单位列向量,$E$为$n$阶单位矩阵,则

\hphantom{~}A.$E-\alpha\alpha^T$不可逆 \hfill B.$E+\alpha\alpha^T$不可逆 \hfill C.$E+2\alpha\alpha^T$不可逆 \hfill D.$E-2\alpha\alpha^T$不可逆
\hphantom{~}\\

{\heiti \zihao{4} 期末试题}

\num 期末2015-2016 三2.

设3阶实对称矩阵$A$的特征值为$\lambda_{1}=-1,\lambda_{2}=\lambda_{3}=1$,对应于$\lambda_{1}$的特征向量$\alpha_{1}=(0,1,1)^{T}$。

(1)求$A$对应于特征值1的特征向量;

(2)求$A$;

(3)求$A^{2016}$。\\

\num 期末2016-2017 一4.

设$\alpha_{1}=(a,1,1)^{T},\alpha_{2}=(1,b,-1)^{T},\alpha_{3}=(1,-2,c)^{T}$是正交向量组,则$a+b+c=$\underline{\hphantom{~~~~~~~~~~}}。\\

\num 期末2016-2017 一5.

设3阶实对称矩阵$A$的特征值分别为$1,2,3$对应的特征向量分别为$\alpha_ {1}=(1,1,1)^{T},\alpha_{2}=(2,-1,-1)^{T},\alpha_{3}$,则$A$的对应于特征值3的一个特征向量$\alpha_{3}=$\underline{\hphantom{~~~~~~~~~~}}。\\


\end{document}  