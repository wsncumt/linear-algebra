\documentclass{article}
\usepackage[space,fancyhdr,fntef]{ctexcap}
\usepackage[namelimits,sumlimits,nointlimits]{amsmath}
\usepackage[bottom=25mm,top=25mm,left=25mm,right=15mm,centering]{geometry}
\usepackage{xcolor}
\usepackage{arydshln}%234页,虚线表格宏包
\pagestyle{fancy} \fancyhf{}
\fancyhead[OL]{~~~班序号:\hfill 学院:\hfill 学号:\hfill 姓名:王松年~~~ \thepage}
%\usepackage{parskip}
%\usepackage{indentfirst}
\usepackage{graphicx}%插图宏包,参见手册318页
\usepackage{mathdots}%反对角省略号
\usepackage{extarrows}%等号上加文字
\begin{document}

\newcounter{num} \renewcommand{\thenum}{\arabic{num}.} \newcommand{\num}{\refstepcounter{num}\text{\thenum}}

\newenvironment{jie}{\kaishu\zihao{-5}\color{blue}{\noindent\em 解:}\par}{\hfill $\diamondsuit$\par}

\newenvironment{zhengming}{\kaishu\zihao{-5}\color{blue}{\noindent\em 证明:}\par}{\hfill $\diamondsuit$\par}

\hphantom{~~}\hfill {\zihao{3}\heiti 第十次习题课} \hfill\hphantom{~~}

\hphantom{~~}\hfill {\zihao{4}\heiti 群文件《期中$\&$期末试题》} \hfill\hphantom{~~}

{\heiti \zihao{4} 考研例题--特征值}

\num 求矩阵$A=
\begin{bmatrix}
  2 & 1 & 3\\
  4& 2 & 6\\
  6 & 3 & 9
\end{bmatrix}
$的特征值与特征向量。

\begin{jie}
\begin{align*}
 |\lambda E-A|=
 \begin{vmatrix}
  \lambda-2 & -1 & -3\\
  -4& \lambda-2 & -6\\
  -6 & -3 & \lambda-9
 \end{vmatrix}\xlongequal{\substack{c_{3}-3c_{2}}}
  \begin{vmatrix}
  \lambda-2 & -1 & 0\\
  -4& \lambda-2 & -3\lambda\\
  -6 & -3 & \lambda
 \end{vmatrix}\xlongequal{\substack{r_{w}-3r_{3}}}
  \begin{vmatrix}
  \lambda-2 & -1 & 0\\
  -22& \lambda-11 & 0\\
  -6 & -3 & \lambda
 \end{vmatrix}=\lambda^2(\lambda-13\lambda)=0
\end{align*}
得到矩阵$A$的特征值是$\lambda_1=13,\lambda_2=\lambda_3=0$

对$\lambda=13$:(高斯消元的步骤略,下来自己写)
\begin{equation*}
  \begin{bmatrix}
  11 & -1 & -3\\
  -4& 11 & -6\\
  -6 & -3 & 4
  \end{bmatrix}\rightarrow
  \begin{bmatrix}
  1& 7 & -5\\
  0& 3 & -2\\
  0 & 0 & 0
  \end{bmatrix}
\end{equation*}
得基础解系$\alpha_1=[1,2,3]^T$,所以属于特征值$13$的特征向量是$k_1\alpha_1,(k_1\neq 0)$

对$\lambda=0$:(高斯消元的步骤略,下来自己写)
\begin{equation*}
  \begin{bmatrix}
  -2 & -1 & -3\\
  -4& -2 & -6\\
  -6 & -3 & -9
  \end{bmatrix}\rightarrow
  \begin{bmatrix}
  2& 1 & 3\\
  0& 0 & 0\\
  0 & 0 & 0
  \end{bmatrix}
\end{equation*}
得基础解系$\alpha_2=[-1,2,0]^T$,$\alpha_3=[-3,0,2]^T$,所以属于特征值$0$的特征向量是$k_2\alpha_2+k_3\alpha_3,(k_2,k_3\text{不全为0})$。
\end{jie}

\textcolor[rgb]{1.00,0.00,0.00}{
设$A=[a_{ij}]$是三阶矩阵,则(该式不做推导,感兴趣的可以自己算一下)
\begin{equation*}|\lambda E-A|=
  \begin{bmatrix}
    \lambda-a_{11} & -a_{12} &-a_{13} \\
-a_{21} & \lambda-a_{22} &-a_{23} \\
-a_{31} & -a_{32} &\lambda-a_{33}
  \end{bmatrix}=\lambda^{3}-\sum a_{ii}\lambda^2+S_{2}\lambda-|A|
\end{equation*}
式中:$S_{2}=
\begin{vmatrix}
  a_{11} & a_{12} \\
  a_{21} & a_{22}
\end{vmatrix}+\begin{vmatrix}
  a_{11} & a_{13} \\
  a_{31} & a_{33}
\end{vmatrix}+\begin{vmatrix}
  a_{22} & a_{23} \\
  a_{32} & a_{33}
\end{vmatrix}
$。}

\textcolor[rgb]{1.00,0.00,0.00}{
若$r(A)=1$(再复习一下第二次习题课讲的这个知识点相关的例题),则$|A|=0,S_{2}=0$,代入到上式有
\begin{equation*}
|\lambda E-A|=\lambda^{3}-\sum a_{ii}\lambda^2=\lambda^2\left(\lambda-\sum a_{ii}\right)
\end{equation*}}

\textcolor[rgb]{1.00,0.00,0.00}{
做推广,对于$n$阶矩阵$A$,若$r(A)=1$,则$|\lambda E-A|=\lambda^{n-1}\left(\lambda-\sum a_{ii}\right)$
}

\num 已知$a\neq 0$,求矩阵
\begin{equation*}
  \begin{bmatrix}
    1 & a & a& a\\
    a & 1& a& a\\
    a& a& 1& a\\
    a& a& a& 1
  \end{bmatrix}
\end{equation*}
的特征值、特征向量。

\begin{jie}
方法一:(直接计算)

由特征多项式:
\begin{equation*}
  \begin{vmatrix}
\lambda E-A
  \end{vmatrix}
  =\begin{vmatrix}
     \lambda-1 & -a& -a& -a \\
     -a& \lambda-1& -a& -a\\
     -a& -a& \lambda-1& -a\\
     -a& -a& -a& \lambda-1
   \end{vmatrix}=\left[\lambda-(3a+1)\right]\left(\lambda+a-1\right)^{3}
\end{equation*}
得$A$的特征值是$3a+1,1-a$。

当$\lambda=3a+1$时,由$[(3a+1)E-A]=0$,即
\begin{align*}
\begin{bmatrix}
3a & -a& -a& -a \\
-a& 3a& -a& -a\\
-a& -a& 3a& -a\\
-a& -a& -a& 3a
\end{bmatrix}\rightarrow
\begin{bmatrix}
3 & -1& -1& -1 \\
-1& 3& -1& -1\\
-1& -1& 3& -1\\
-1& -1& -1& 3
\end{bmatrix}\rightarrow
\begin{bmatrix}
1 & -3& 1& 1 \\
1& 1& -3& 1\\
1& 1& 1& -3\\
0&0& 0& 0
\end{bmatrix}\rightarrow
\begin{bmatrix}
1 & 0& 0& -1 \\
0& 1& 0& -1\\
0& 0& 1& -1\\
0&0& 0& 0
\end{bmatrix}
\end{align*}
可得基础解系为$\alpha_{1}=(1,1,1,1)^T$,所以$\lambda=3a+1$的特征向量为$k_{1}\alpha_1,(k_1\neq 0)$。

当$\lambda=1-a$时,由$[(1-a)E-A]=0$,即
\begin{equation*}
  \begin{bmatrix}
-a & -a & -a& -a\\
-a & -a & -a& -a\\
-a & -a & -a& -a\\
-a & -a & -a& -a
  \end{bmatrix}\rightarrow
  \begin{bmatrix}
    1 & 1 & 1& 1\\
    0 & 0& 0& 0\\
    0 & 0& 0& 0\\
    0 & 0& 0& 0
  \end{bmatrix}
\end{equation*}
得基础解系$\alpha_2=(-1,1,0,0)^T,\alpha_3=(-1,0,1,0)^T\alpha_4=(-1,0,0,1)^T$,所以$\lambda=1-a$的特征向量为$k_2\alpha_2+k_3\alpha_3+k_4\alpha_4$,式中$k_2,k_3,k_4$是不全为0的任意常数。

方法二:(转换法)

由题得:
\begin{equation*}A=
  \begin{bmatrix}
    a & a& a& a \\
    a & a& a& a \\
    a & a& a& a \\
    a & a& a& a
  \end{bmatrix}+
   \begin{bmatrix}
    1-a & 0& 0& 0 \\
    0 & 1-a& 0& 0 \\
    0 & 0& 1-a& 0 \\
    0 & 0& 0& 1-a
  \end{bmatrix}=B+(1-a)E
\end{equation*}
由于$r(B)=1$,所以有
\begin{equation*}
  |\lambda E -B|=\lambda^{4-1}\left(\lambda-\sum\limits_{i=1}^{4}a_{ii}\right)=
  \lambda^{3}\left(\lambda-4a\right)
\end{equation*}
所以矩阵$B$的特征值为$0,0,0,4a$,所以由特征值的性质,$A$的特征值为$3a+1,1-a,1-a,1-a$。

下边同方法一。
\end{jie}

\num 抽象矩阵1

设$A$是三阶矩阵,且矩阵$A$的各行元素之和均为5,则矩阵$A$必有特征向量\underline{\hphantom{~~~~~~~~~~~~~}}.

\begin{jie}
由题得:
\begin{equation*}
\begin{cases}
a_{11}+a_{12}+a_{13}=5\\
a_{21}+a_{22}+a_{23}=5\\
a_{31}+a_{32}+a_{33}=5
\end{cases}~~~~~\Rightarrow~~~
\begin{bmatrix}
a_{11}&a_{12}&a_{13} \\
a_{21}&a_{22}&a_{23} \\
a_{31}&a_{32}&a_{33}
\end{bmatrix}
\begin{bmatrix}
1\\ 1\\1
\end{bmatrix}=
\begin{bmatrix}
5\\ 5\\5
\end{bmatrix}~~~\Rightarrow~~~A\begin{bmatrix}
1\\ 1\\1
\end{bmatrix}=5\begin{bmatrix}
1\\ 1\\1
\end{bmatrix}
\end{equation*}
所以矩阵$A$必有特征值5且必有特征向量$k[1,1,1]^T,(k\neq 0)$。
\end{jie}

\num 抽象矩阵2

已知$A$是3阶矩阵,如果非齐次线性方程组$Ax=b$有通解$5b+k_1\eta_1+k_2\eta_2$,其中$\eta_1,\eta_2$是$Ax=0$的基础解系,求$A$的特征值和特征向量。

\begin{jie}
\textcolor[rgb]{1.00,0.00,0.00}{非齐次线性方程组$Ax=b$的通解为$Ax=b$的特解加上$Ax=0$的通解。}

由解得结构可知$5b$是方程组$Ax=b$的一个解,即$A(5b)=b$,所以$Ab=\dfrac{1}{5}b$。即$\dfrac{1}{5}$是$A$的特征值,$k_1b,(k_1\neq 0)$是相应的特征向量。

$\eta_1,\eta_2$是$Ax=0$的基础解系,所以必有$A\eta_1=0=0\eta_1,A\eta_{2}=0=0\eta_2$,所以$\eta_1,\eta_2$是$A$关于$\lambda=0$的线性无关的特征向量,所以特征值$0$对应的特征向量为$k_2\eta_1+k_3\eta_2,(k_2,k_3\text{不全为}0)$。

综上所述,$A$的特征值为$\dfrac{1}{5},0,0$,对应的特征向量分别是$k_1b,(k_1\neq 0)$,$k_2\eta_1+k_3\eta_2,(k_2,k_3\text{不全为}0)$
\end{jie}

{\heiti \zihao{4} 考研例题--实对称矩阵}

\num 设$A$是3阶实对称矩阵,秩$r(A)=2$,若$A^2=A$,则$A$的特征值是\underline{\hphantom{~~~~~~~~~~~~~}}.

\begin{jie}
设$\lambda$是$A$的任意特征值,$\alpha$是属于$\lambda$的特征向量,即$A\alpha=\lambda\alpha,\alpha\neq 0$。所以有:
\begin{equation*}
A^2\alpha=AA\alpha=A\lambda\alpha=\lambda A\alpha=\lambda\lambda\alpha=\lambda^2\alpha
\end{equation*}
又因为$A^2=A$,所以有$A^2\alpha=A\alpha$,即$\lambda^2\alpha=\lambda\alpha$,解得$\lambda=0$或$1$。

因为$A$是实对称矩阵,所以$A\sim \Lambda$,且$\Lambda$由$A$的特征值所构成,相似矩阵具有相同的秩,所以$r(\Lambda)=r(A)=2$,所以可以推出
\begin{equation*}
\Lambda
=
\begin{bmatrix}
  1 & & \\
    & 1 &\\
    & & 0
\end{bmatrix}
\end{equation*}
所以矩阵$A$的特征值是$1,1,0$。
\end{jie}

\num $n$阶矩阵
\begin{equation*}
A=
\begin{bmatrix}
  a & 1 & 1 & \cdots & 1\\
 1 & a & 1&\cdots & 1\\
 1 & 1 & a & \cdots &1\\
 \vdots&\vdots&\vdots&\ddots&\vdots\\
 1&1&1&\cdots&a
\end{bmatrix}
\end{equation*}
则$r(A)=$\underline{\hphantom{~~~~~~~~~~~~~}}.

\begin{jie}
由第二题的方法2可快速写出$A$的特征值为$n+a-1,a-1,a-1,\cdots,a-1$.
因为$A$是实对称矩阵,所以$A\sim \Lambda$,且$\Lambda$由$A$的特征值所构成,相似矩阵具有相同的秩,所以$r(\Lambda)=r(A)$,所以
\begin{equation*}
\Lambda=
\begin{bmatrix}
  n+a-1 & && \\
   & a-1 &&\\
   &&\ddots&\\
   &&&a-1
\end{bmatrix}
\end{equation*}
这里$n$是$A$的阶数,所以不会等于$0$。所以
\begin{equation*}r(A)=
  \begin{cases}
  n,&\text{若}a\neq 1\text{且}a\neq 1-n,\\
  n-1,&\text{若}a=1-n,\\
  1,&\text{若}a=1.
  \end{cases}
\end{equation*}
\end{jie}

\num 设$\alpha$为$n$维单位列向量,$E$为$n$阶单位矩阵,则

\hphantom{~}A.$E-\alpha\alpha^T$不可逆 \hfill B.$E+\alpha\alpha^T$不可逆 \hfill C.$E+2\alpha\alpha^T$不可逆 \hfill D.$E-2\alpha\alpha^T$不可逆
\hphantom{~}

\begin{jie}
注意:单位向量指的是向量的模(长度)为1,要与$[1,1,1]$区分开来。

$\alpha\alpha^T\alpha=\alpha(\alpha^T\alpha)=1\alpha$,所以$\alpha\alpha^T$有一个特征值1.

$\alpha$为$n$维单位列向量,所以$r(\alpha\alpha^T)=1$,所以由第一题的结论,$\alpha\alpha^T$的特征值为$1,0,0,\cdots,0$。

$E$为$n$阶单位矩阵,所以$E$也为实对称矩阵(特征值为$1$),实对称矩阵相加减依然为实对称矩阵,所以上述选项中每一项均为实对称矩阵。

又由矩阵可逆则行列式一定不为0(不可逆则行列式一定为0,充要条件),矩阵的行列式等于特征值的乘积。

$A$.$c$的特征值为$1-1,1-0,1-0,\cdots,1-0$即$0,1,1,\cdots,1$,所以$|E-\alpha\alpha^T|=0\times 1\cdots 1=0$,即不可逆。

同理可以看出其他选项的行列式均不为0,即可逆。
\end{jie}

{\heiti \zihao{4} 期末试题}

\num 期末2015-2016 三2.

设3阶实对称矩阵$A$的特征值为$\lambda_{1}=-1,\lambda_{2}=\lambda_{3}=1$,对应于$\lambda_{1}$的特征向量$\alpha_{1}=(0,1,1)^{T}$。

(1)求$A$对应于特征值1的特征向量;

(2)求$A$;

(3)求$A^{2016}$。

\begin{jie}
(1)由于$A$是实对称矩阵,所以对于$A$的不同特征值的特征向量正交,所以设特征值1对应的特征向量是$\alpha=[x_1,x_2,x_3]$。所以有:
\begin{equation*}
\alpha_{1}^T\alpha=x_2+x_3=0~~~\Rightarrow~~~x_2=-x_3
\end{equation*}
分别取$
\begin{bmatrix}
x_1 \\ x_3
\end{bmatrix}=\begin{bmatrix}
1 \\ 0
\end{bmatrix},
\begin{bmatrix}
0 \\ 1
\end{bmatrix}
$得$\alpha_{2}=
\begin{bmatrix}
1 \\ 0 \\0
\end{bmatrix}
,\alpha_3=
\begin{bmatrix}
0\\ -1 \\1
\end{bmatrix}$。

$\alpha_2,\alpha_3$即为$A$对应于特征值1的特征向量。

(2)由特征值定义:$A\alpha_i=\lambda_i\alpha_i$。所以:
\begin{align*}
&A[\alpha_1,\alpha_2,\alpha_3]=[\lambda_1\alpha_1,\lambda_2\alpha_2,\lambda_3\alpha_3] ~~~\Rightarrow\\
&A=[\lambda_1\alpha_1,\lambda_2\alpha_2,\lambda_3\alpha_3] [\alpha_1,\alpha_2,\alpha_3]^{-1}=
\begin{bmatrix}
  0 & 1  & 0\\
 -1 & 0&-1\\
 -1 & 0& 1
\end{bmatrix}
\begin{bmatrix}
  0 & 1  & 0\\
 1 & 0&-1\\
 1 & 0& 1
\end{bmatrix}^{-1}=
\begin{bmatrix}
  1 & 0  & 0\\
 0 & 0&-1\\
 0 & -1& 0
\end{bmatrix}
\end{align*}
$\left(\text{式中:}[\alpha_1,\alpha_2,\alpha_3]^{-1}=
\begin{bmatrix}
  0 & \frac{1}{2} & \frac{1}{2}\\
 1 & 0&0\\
  0 & -\frac{1}{2} & \frac{1}{2}
\end{bmatrix}
\right)$

(3)由$(2)$得:$A^2=E_3$($E$表示单位矩阵。)所以$A^{2016}=(A^{2})^{1008}=E_3$。
\end{jie}

\num 期末2016-2017 一4.

设$\alpha_{1}=(a,1,1)^{T},\alpha_{2}=(1,b,-1)^{T},\alpha_{3}=(1,-2,c)^{T}$是正交向量组,则$a+b+c=$\underline{\hphantom{~~~~~~~~~~}}。

\begin{jie}
由题得:
\begin{equation*}
\begin{cases}
\alpha_1\alpha_2^T=a+b-1=0\\
\alpha_1\alpha_3^T=a-2+c=0\\
\alpha_2\alpha_3^T=1-2b-c=0
\end{cases}
~~~\Rightarrow~~~
\begin{cases}
a=1\\
b=0\\
c=1
\end{cases}~~~\Rightarrow~~~a+b+c=2
\end{equation*}
\end{jie}

\num 期末2016-2017 一5.

设3阶实对称矩阵$A$的特征值分别为$1,2,3$对应的特征向量分别为$\alpha_ {1}=(1,1,1)^{T},\alpha_{2}=(2,-1,-1)^{T},\alpha_{3}$,则$A$的对应于特征值3的一个特征向量$\alpha_{3}=$\underline{\hphantom{~~~~~~~~~~}}。

\begin{jie}
设$\alpha_3=[x_1,x_2,x_3]^T$,实对称矩阵对应于不同特征值的特征向量是正交的,所以:
\begin{equation*}
\begin{cases}
\alpha_{1}\alpha_3^T=x_1+x_2+x_3=0\\
\alpha_{2}\alpha_3^T=2x_1-x_2-x_3=0
\end{cases}~~~\Rightarrow~~~
\begin{cases}
x_1=0\\
x_2=-x_3
\end{cases}
\end{equation*}
令$x_{3}=-1$,有$\alpha_3=[0,1,-1]$
\end{jie}
\end{document}  