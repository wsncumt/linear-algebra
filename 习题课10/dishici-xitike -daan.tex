\documentclass{article}
\usepackage[space,fancyhdr,fntef]{ctexcap}
\usepackage[namelimits,sumlimits,nointlimits]{amsmath}
\usepackage[bottom=25mm,top=25mm,left=25mm,right=15mm,centering]{geometry}
\usepackage{xcolor}
\usepackage{arydshln}%234页,虚线表格宏包
\pagestyle{fancy} \fancyhf{}
\fancyhead[OL]{~~~班序号:\hfill 学院:\hfill 学号:\hfill 姓名:王松年~~~ \thepage}
%\usepackage{parskip}
%\usepackage{indentfirst}
\usepackage{graphicx}%插图宏包,参见手册318页
\usepackage{mathdots}%反对角省略号
\begin{document}

\newcounter{num} \renewcommand{\thenum}{\arabic{num}.} \newcommand{\num}{\refstepcounter{num}\text{\thenum}}

\newenvironment{jie}{\kaishu\zihao{-5}\color{blue}{\noindent\em 解:}\par}{\hfill $\diamondsuit$\par}

\newenvironment{zhengming}{\kaishu\zihao{-5}\color{blue}{\noindent\em 证明:}\par}{\hfill $\diamondsuit$\par}

\hphantom{~~}\hfill {\zihao{3}\heiti 第十次习题课} \hfill\hphantom{~~}

\hphantom{~~}\hfill {\zihao{4}\heiti 群文件《期中$\&$期末试题》} \hfill\hphantom{~~}

{\heiti \zihao{4} 期末试题}

\num 期末2015-2016 三2.

设3阶实对称矩阵$A$的特征值为$\lambda_{1}=-1,\lambda_{2}=\lambda_{3}=1$,对应于$\lambda_{1}$的特征向量$\alpha_{1}=(0,1,1)^{T}$。

(1)求$A$对应于特征值1的特征向量;

(2)求$A$;

(3)求$A^{2016}$。

\begin{jie}
(1)由于$A$是实对称矩阵,所以对于$A$的不同特征值的特征向量正交,所以设特征值1对应的特征向量是$\alpha=[x_1,x_2,x_3]$。所以有:
\begin{equation*}
\alpha_{1}^T\alpha=x_2+x_3=0~~~\Rightarrow~~~x_2=-x_3
\end{equation*}
分别取$
\begin{bmatrix}
x_1 \\ x_3
\end{bmatrix}=\begin{bmatrix}
1 \\ 0
\end{bmatrix},
\begin{bmatrix}
0 \\ 1
\end{bmatrix}
$得$\alpha_{2}=
\begin{bmatrix}
1 \\ 0 \\0
\end{bmatrix}
,\alpha_3=
\begin{bmatrix}
0\\ -1 \\1
\end{bmatrix}$。

$\alpha_2,\alpha_3$即为$A$对应于特征值1的特征向量。

(2)由特征值定义:$A\alpha_i=\lambda_i\alpha_i$。所以:
\begin{align*}
&A[\alpha_1,\alpha_2,\alpha_3]=[\lambda_1\alpha_1,\lambda_2\alpha_2,\lambda_3\alpha_3] ~~~\Rightarrow\\
&A=[\lambda_1\alpha_1,\lambda_2\alpha_2,\lambda_3\alpha_3] [\alpha_1,\alpha_2,\alpha_3]^{-1}=
\begin{bmatrix}
  0 & 1  & 0\\
 -1 & 0&-1\\
 -1 & 0& 1
\end{bmatrix}
\begin{bmatrix}
  0 & 1  & 0\\
 1 & 0&-1\\
 1 & 0& 1
\end{bmatrix}^{-1}=
\begin{bmatrix}
  1 & 0  & 0\\
 0 & 0&-1\\
 0 & -1& 0
\end{bmatrix}
\end{align*}
$\left(\text{式中:}[\alpha_1,\alpha_2,\alpha_3]^{-1}=
\begin{bmatrix}
  0 & \frac{1}{2} & \frac{1}{2}\\
 1 & 0&0\\
  0 & -\frac{1}{2} & \frac{1}{2}
\end{bmatrix}
\right)$

(3)由$(2)$得:$A^2=E_3$($E$表示单位矩阵。)所以$A^{2016}=(A^{2})^{1008}=E_3$。
\end{jie}

\num 期末2016-2017 一4.

设$\alpha_{1}=(a,1,1)^{T},\alpha_{2}=(1,b,-1)^{T},\alpha_{3}=(1,-2,c)^{T}$是正交向量组,则$a+b+c=$\underline{\hphantom{~~~~~~~~~~}}。

\begin{jie}
由题得:
\begin{equation*}
\begin{cases}
\alpha_1\alpha_2^T=a+b-1=0\\
\alpha_1\alpha_3^T=a-2+c=0\\
\alpha_2\alpha_3^T=1-2b-c=0
\end{cases}
~~~\Rightarrow~~~
\begin{cases}
a=1\\
b=0\\
c=1
\end{cases}~~~\Rightarrow~~~a+b+c=2
\end{equation*}
\end{jie}

\num 期末2016-2017 一5.

设3阶实对称矩阵$A$的特征值分别为$1,2,3$对应的特征向量分别为$\alpha_ {1}=(1,1,1)^{T},\alpha_{2}=(2,-1,-1)^{T},\alpha_{3}$,则$A$的对应于特征值3的一个特征向量$\alpha_{3}=$\underline{\hphantom{~~~~~~~~~~}}。

\begin{jie}
设$\alpha_3=[x_1,x_2,x_3]^T$,实对称矩阵对应于不同特征值的特征向量是正交的,所以:
\begin{equation*}
\begin{cases}
\alpha_{1}\alpha_3^T=x_1+x_2+x_3=0\\
\alpha_{2}\alpha_3^T=2x_1-x_2-x_3=0
\end{cases}~~~\Rightarrow~~~
\begin{cases}
x_1=0\\
x_2=-x_3
\end{cases}
\end{equation*}
令$x_{3}=-1$,有$\alpha_3=[0,1,-1]$
\end{jie}
\end{document}  