\documentclass{article}
\usepackage[space,fancyhdr,fntef]{ctexcap}
\usepackage[namelimits,sumlimits,nointlimits]{amsmath}
\usepackage[bottom=25mm,top=25mm,left=25mm,right=15mm,centering]{geometry}
\usepackage{xcolor}
\usepackage{arydshln}%234页,虚线表格宏包
\pagestyle{fancy} \fancyhf{}
\fancyhead[OL]{~~~班序号:\hfill 学院:\hfill 学号:\hfill 姓名:王松年~~~ \thepage}
%\usepackage{parskip}
%\usepackage{indentfirst}
\usepackage{graphicx}%插图宏包,参见手册318页
\usepackage{mathdots}%反对角省略号
\begin{document}

\newcounter{num} \renewcommand{\thenum}{\arabic{num}.} \newcommand{\num}{\refstepcounter{num}\text{\thenum}}

\newenvironment{jie}{\kaishu\zihao{-5}\color{blue}{\noindent\em 解:}\par}{\hfill $\diamondsuit$\par}

\newenvironment{zhengming}{\kaishu\zihao{-5}\color{blue}{\noindent\em 证明:}\par}{\hfill $\diamondsuit$\par}

\hphantom{~~}\hfill {\zihao{3}\heiti 第五次习题课} \hfill\hphantom{~~}

\hphantom{~~}\hfill {\zihao{4}\heiti 群文件《期中$\&$期末试题》} \hfill\hphantom{~~}

{\heiti \zihao{4} 期中试题}

\num 2015-2016 一1.

已知$A=
\begin{bmatrix}
  3 & 0 & 4 & 0 \\
  2 & 2 & 2 & 2 \\
  0 & -7 & 0 & 0 \\
  5 & 3 & -2 & 2 \\
\end{bmatrix}
$,则代数余子式之和$A_{41}+A_{42}+A_{43}+A_{44}=$\underline{~~~~0~~~~}。

\begin{jie}
由题得:
\begin{align*}
A_{41}+A_{42}+A_{43}+A_{44}=
\begin{vmatrix}
  3 & 0 & 4 & 0 \\
  2 & 2 & 2 & 2 \\
  0 & -7 & 0 & 0 \\
  \textcolor[rgb]{1.00,0.00,0.00}{1} & \textcolor[rgb]{1.00,0.00,0.00}{1} & \textcolor[rgb]{1.00,0.00,0.00}{1} & \textcolor[rgb]{1.00,0.00,0.00}{1} \\
\end{vmatrix}=a_{3,2}A_{32}=
(-1)^{3+2}\times (-7)\times
\begin{vmatrix}
  3 &  4 & 0 \\
  2 &  2 & 2 \\
  1 &  1 & 1 \\
\end{vmatrix}=0
\end{align*}
\end{jie}

%\num 2015-2016 一2.
%
%设$f(x)=
%\begin{vmatrix}
%  2x & x & 1 & 2\\
%  1 & x & 1 & -1\\
%  3 & 2 & x & 1\\
%  1 & 1 & 1 & x\\
%\end{vmatrix}
%$,则$x^{3}$的系数为\underline{\hphantom{~~~~~~~~~~}}。
%
%\begin{jie}
%$a_{11}$表示第一行第一列的元素,$A_{11}$表示第一行第一列元素的代数余子式。其他类似。
%\begin{align*}
%f(x)=a_{11}A_{11}+a_{12}A_{12}+a_{13}A_{13}+a_{14}A_{14}=2x
%\begin{vmatrix}
%  x & 1 & -1\\
% 2 & x & 1\\
%   1 & 1 & x\\
%\end{vmatrix}-x
%\begin{vmatrix}
%  1 &  1 & -1\\
%  3 &  x & 1\\
%  1 &  1 & x\\
%\end{vmatrix}+
%\begin{vmatrix}
%  1 & x &  -1\\
%  3 & 2 &  1\\
%  1 & 1 &  x\\
%\end{vmatrix}-2
%\begin{vmatrix}
%  1 & x & 1 \\
%  3 & 2 & x\\
%  1 & 1 & 1 \\
%\end{vmatrix}
%\end{align*}
%可以看出第一项只含$x^{4}$和$x^{2/}$第三项和第四项中不含$x^{3}$,只看第二项:
%\begin{equation*}
%-x
%\begin{vmatrix}
%  1 &  1 & -1\\
%  3 &  x & 1\\
%  1 &  1 & x\\
%\end{vmatrix}=-x((x^{2}-1)-(3x-1)+(3-x))
%\end{equation*}
%所以$x^{3}$的系数为$-1$。
%\end{jie}

\num 2015-2016 一3.

设$a,b,c$满足方程
$
\begin{vmatrix}
  1 & a & b & c\\
  a & 1 & 0 & 0\\
  b & 0 & 1 & 0\\
  c & 0 & 0 & 1\\
\end{vmatrix}=1
$,则$abc=$\underline{\hphantom{~~~~~~~~~~}}。

\begin{jie}
对行列式做如下变换:第一行减去$c$倍的第四行,第一行减去$b$倍的第三行,第一行减去$a$倍的第二行,得到如下行列式:
\begin{equation*}
\begin{vmatrix}
  1 & a & b & c\\
  a & 1 & 0 & 0\\
  b & 0 & 1 & 0\\
  c & 0 & 0 & 1\\
\end{vmatrix}=\begin{vmatrix}
  1-a^{2}-b^{2}-c^{2} & 0 & 0 & 0\\
  a & 1 & 0 & 0\\
  b & 0 & 1 & 0\\
  c & 0 & 0 & 1\\
\end{vmatrix}=1-a^{2}-b^{2}-c^{2}=1
\end{equation*}
即$a^{2}+b^{2}+c^{2}=0$,对于实数,任何数的平方都大于等于0,所以可以推出$a=0,b=0,c=0$,所以$abc=0$。
\end{jie}

\num 2015-2016 一4.

设$A=
\begin{bmatrix}
  2 & 0 \\
  1 & 4
\end{bmatrix}
$,若$B=2BA-3I$,其中$I$为单位矩阵,则$|B|=$\underline{\hphantom{~~~~~~~~~~}}。

\begin{jie}
由题得:$B=2BA-3I~~~\Rightarrow B(2A-I)=3I$,所以$|B(2A-I)|=|3I|=3^{2}=9$,$|2A-I|=
\begin{vmatrix}
  3 & 0 \\
  2 & 7
\end{vmatrix}=21
$,所以$|B(2A-I)|=|B||2A-I|=21|B|=9$,解得$|B|=\dfrac{3}{7}$
\end{jie}

\num 2015-2016 二1.

计算行列式$
\begin{vmatrix}
  5 & 0 & 4 & 2\\
  1 & -1 & 2 & 1\\
  4 & 1 & 2 & 0\\
  1 & 1 & 1 & 1\\
\end{vmatrix}
$。

\begin{jie}
(过程不唯一)

对行列式做如下变换:把第二行加到第三行,把第二行加到第四行,由行列式的性质,此时行列式的值没有改变。即
\begin{align*}
\begin{vmatrix}
  5 & 0 & 4 & 2\\
  1 & -1 & 2 & 1\\
  4 & 1 & 2 & 0\\
  1 & 1 & 1 & 1\\
\end{vmatrix}=
\begin{vmatrix}
  5 & 0 & 4 & 2\\
  1 & -1 & 2 & 1\\
  5 & 0 & 4 & 1\\
  2 & 0 & 3 & 2\\
\end{vmatrix}
=(-1)^{2+2}\times (-1)\times
\begin{vmatrix}
  5 &  4 & 2\\
  5 &  4 & 1\\
  2 &  3 & 2\\
\end{vmatrix}=-\begin{vmatrix}
  5 &  4 & 2\\
  5 &  4 & 1\\
  2 &  3 & 2\\
\end{vmatrix}=-|A|
\end{align*}
对于$|A|$:把第二行的负一倍加到第一行
\begin{equation*}
-\begin{vmatrix}
  5 &  4 & 2\\
  5 &  4 & 1\\
  2 &  3 & 2\\
\end{vmatrix}=-
\begin{vmatrix}
  0 &  0 & 1\\
  5 &  4 & 1\\
  2 &  3 & 2\\
\end{vmatrix}=-(-1)^{1+3}\times(5\times 3-2\times 4)=-7
\end{equation*}
\end{jie}


\num 2015-2016 二2.

计算行列式$
\begin{vmatrix}
  a^{2} & ab & b^{2}\\
  2a & a+b & 2b \\
  1 & 1 & 1
\end{vmatrix}
$。

\begin{jie}
对行列式做如下变换:把第二列的负一倍分别加到第一列和第三列上。得到$
\begin{vmatrix}
  a(a-b) & ab & b(b-a)\\
  a-b & a+b & b-a \\
  0 & 1 & 0
\end{vmatrix}
$

把第三列加到第一列上:
\begin{equation*}
\begin{vmatrix}
  (a-b)^{2} & ab & b(b-a)\\
  0 & a+b & b-a \\
  0 & 1 & 0
\end{vmatrix}=(-1)^{1+1}(a-b)^{2}[(a+b)0-(b-a)]=(a-b)^{3}
\end{equation*}
\end{jie}

\num 2015-2016 三2.

设$A$为$n$阶方阵,$AA^{T}=I$,$|A|<0$,证明:$|A+I|=0$。

\begin{zhengming}
由行列式的性质得$|AB|=|A||B|,|A^{T}|=|A|$,所以由题得

$|AA^{T}|=|I|$,等号左边:$|AA^{T}|=|A||A^{T}|=|A||A|=|A|^{2}$,等号右边等于1,由题得$|A|<0$,所以$|A|=-1$.

$|A+I|=|A+AA^{T}|=|A(I+A^{T})|=|A||(I+A)^{T}|=-|I+A|$,所以$|A+I|=0$。\textcolor[rgb]{1.00,0.00,0.00}{(注:两个矩阵相加的转置等于两个矩阵分别转置后相加,即$A^{T}+B^{T}=(A+B)^{T}$)}
\end{zhengming}

\num 2016-2017 一1.

设$M_{ij}$是$
\begin{vmatrix}
  0 & 4 & 0 \\
  2 & 2 & 2\\
  2 & 0 & 0
\end{vmatrix}
$的第$i$行第$j$列元素的余子式,则$M_{11}+M_{12}=$\underline{\hphantom{~~~~~~~~~~}}。

\begin{jie}
由题得:$M_ {11}+M_{12}=M_ {11}+M_{12}+0M_{13}=A_{11}-A_{12}+0A_{13}=\begin{vmatrix}
  \textcolor[rgb]{1.00,0.00,0.00}{1} & \textcolor[rgb]{1.00,0.00,0.00}{-1} & \textcolor[rgb]{1.00,0.00,0.00}{0} \\
  2 & 2 & 2\\
  2 & 0 & 0
\end{vmatrix}=2\times(-1)^{(3+1)}\times[(-1)\times 2-2\times0]=-4$
\end{jie}

\num 2016-2017 一2.

计算行列式$
\begin{vmatrix}
  1 & 1 & 1 & 1 \\
  1 & 2 & 4& 8 \\
  1 & 3 & 9& 27\\
   1 & 4 &16 &64
\end{vmatrix}
=$\underline{\hphantom{~~~~~~~~~~}}。

\begin{jie}
经观察,该行列式为四阶范德蒙行列式,且$x_{1}=1,x_{2}=2,x_{3}=3,x_{4}=4$,所以原式$=(x_{4}-x_{3})(x_{4}-x_{2})(x_{4}-x_{1})(x_{3}-x_{2})(x_{3}-x_{1})(x_{2}-x_{1})=12$
\end{jie}

\num 2016-2017 二1.

计算行列式$
\begin{vmatrix}
  1 & 2 & 3 & 4 \\
  2 & 3 & 4& 1 \\
  3 & 4 & 1& 2\\
   4 & 1 &2 &3
\end{vmatrix}
$。

\begin{jie}
步骤不唯一

把所有列加到第一列,原行列式变为
\begin{equation*}
\begin{vmatrix}
  10 & 2 & 3 & 4 \\
  10 & 3 & 4& 1 \\
  10 & 4 & 1& 2\\
   10 & 1 &2 &3
\end{vmatrix}=10\begin{vmatrix}
  1 & 2 & 3 & 4 \\
  1 & 3 & 4& 1 \\
  1 & 4 & 1& 2\\
   1 & 1 &2 &3
\end{vmatrix}
\end{equation*}
把第一行的负一倍分别加到第二行,第三行,第四行。$10
\begin{vmatrix}
  1 & 2 & 3 & 4 \\
  0 & 1 & 1& -3 \\
  0 & 2 & -2& -2\\
   0 & -1 &-1 &-1
\end{vmatrix}
$,把第四行加到第二行\begin{align*}
10
\begin{vmatrix}
  1 & 2 & 3 & 4 \\
  0 & 0 & 0& -4 \\
  0 & 2 & -2& -2\\
   0 & -1 &-1 &-1
\end{vmatrix}=10\times(-4)\times2\times(-1)\begin{vmatrix}
  1 & 2 & 3 & 4 \\
  0 & 0 & 0& 1 \\
  0 & 1 & -1& -1\\
   0 & 1 &1 &1
\end{vmatrix}=80\times1\times(-1)^{2+4}\begin{vmatrix}
  1 & 2 & 3  \\
  0 & 1 & -1\\
   0 & 1 &1
\end{vmatrix}=160
\end{align*}
\end{jie}

\num 2016-2017 二2.

设$f(x)=
\begin{vmatrix}
  x-1 & 1 & -1 & 1 \\
  -1 & x+1 & -1& 1 \\
  -1 & 1 & x-1& 1\\
   -1 & 1 &-1 &x+1
\end{vmatrix}
$,求$f(x)=0$的根。

\begin{jie}
分别把第四行的负一倍加到第一行、第二行与第三行上
\begin{equation*}
\begin{vmatrix}
  x & 0 & 0 & -x \\
  0 & x & 0& -x \\
  0 & 0 & x& -x\\
   -1 & 1 &-1 &x+1
\end{vmatrix}
\end{equation*}
把前边三列全部加到第四列上:
\begin{equation*}
\begin{vmatrix}
  x & 0 & 0 & 0 \\
  0 & x & 0& 0 \\
  0 & 0 & x& 0\\
   -1 & 1 &-1 &x
\end{vmatrix}=x^{4}
\end{equation*}
所以$f(x)=x^{4}=0$的根为$x_{1}=x_{2}=x_{3}=x_{4}=0$.
\end{jie}

%\num 期中2016-2017 三2.
%
%已知$A=(a_{ij})$是三阶的非零矩阵,设$A_{ij}$是$a_{ij}$的代数余子式,且对任意的$i,j$有$A_{ij}+a_{ij}=0$,求$A$ 的行列式。
%
%\begin{jie}
%因为$A_{ij}+a_{ij}=0$,所以可以推出$A+(A^*)^T=0$。即$(A^*)^T=-A$.两边同时取行列式:
%
%左边:$|(A^*)^T|=|A^*|=||A|A^{-1}|=|A|^n|A|^{-1}=|A|^{n-1}=|A|^2$
%
%右边:$|-A|=(-1)^3|A|=-|A|$
%
%所以$|A|^2=-|A|$,解得$|A|=-1$或$|A|=0$。
%
%又因为$(A^*)^T=-A$,所以有$r((A^*)^T)=r(-A)$,即$r(A^*)=r(A)$,所以$r(A)=n=3$。\textcolor[rgb]{1.00,0.00,0.00}{注:此步看不懂的看课本121页的例3.3.23,记住这个例题的结论。}
%
%$r(A)=3$,即满秩,满秩即(可逆$\&$行列式不为$0$),所以$|A|=-1$。
%\end{jie}

\num 2017-2018 一1.

计算行列式$
\begin{vmatrix}
  1 & 1 & 1 & 1 \\
  1 & 1 & 1& 2 \\
  1 & 1 & 3& 1\\
   1 & 4 &1 &1
\end{vmatrix}
.$

\begin{jie}
过程不唯一 。

把第一列的负一倍分别加到第二、三、四列上:
\begin{equation*}
\begin{vmatrix}
  1 & 1 & 1 & 1 \\
  1 & 1 & 1& 2 \\
  1 & 1 & 3& 1\\
   1 & 4 &1 &1
\end{vmatrix}=\begin{vmatrix}
  1 & 0 & 0 & 0 \\
  1 & 0 & 0& 1 \\
  1 & 0 & 2& 0\\
   1 & 3 &0 &0
\end{vmatrix}=1\times(-1)^{1+1}\begin{vmatrix}
 0 & 0& 1 \\
 0 & 2& 0\\
 3 &0 &0
\end{vmatrix}=(-1)^{\frac{3\times(3-1)}{2}}6=-6
\end{equation*}
\end{jie}

\num 期中2017-2018 一2.

 求方程$
\begin{vmatrix}
  1 & 2 & 1 & 1 \\
  1 & x & 2& 3 \\
  1 & 2 & x& 2\\
   0 & 0 &2 &x
\end{vmatrix}=0
$的根。

\begin{jie}
把第一行的负一倍分别加到第二、三行上:
\begin{align*}
\begin{vmatrix}
  1 & 2 & 1 & 1 \\
  1 & x & 2& 3 \\
  1 & 2 & x& 2\\
   0 & 0 &2 &x
\end{vmatrix}=\begin{vmatrix}
  1 & 2 & 1 & 1 \\
  0 & x-2 & 1& 2 \\
  0 & 0 & x-1& 1\\
   0 & 0 &2 &x
\end{vmatrix}=1\times(-1)^{1+1}\begin{vmatrix}
  x-2 & 1& 2 \\
   0 & x-1& 1\\
   0 &2 &x
\end{vmatrix}=(x-2)\times(-1)^{1+1}[(x-1)x-2\times1]=0
\end{align*}
解得:$x_{1}=x_{2}=2,x_{3}=-1$。
\end{jie}

\num 期中2017-2018 一3.

 设$\gamma_{1},\gamma_{2},\gamma_{3},\gamma_{4}$及$\beta$均为4维列向量。4阶矩阵$A=[\gamma_ {1}~\gamma_{2}~\gamma_{3}~\gamma_{4}],B=[\beta~\gamma_{2}~\gamma_{3}~\gamma_{4}]$,若$\left|A\right|=2,\left|B\right|=3$,求

(1)$\left|A+B\right|$;

(2)$\left|A^{2}+AB\right|$;

\begin{jie}
(1)$A+B=[\gamma_ {1}~\gamma_{2}~\gamma_{3}~\gamma_{4}]+[\beta~\gamma_{2}~\gamma_{3}~\gamma_{4}]=[\gamma_{1}+\beta ~~2\gamma_{2}~~2\gamma_{3}~~2\gamma_{4}]$,由行列式的性质:
\begin{align*}
|A+B|&=|\gamma_{1}+\beta ~~2\gamma_{2}~~2\gamma_{3}~~2\gamma_{4}|=|\gamma_{1}~~2\gamma_{2}~~2\gamma_{3}~~2\gamma_{4}|+|\beta ~~2\gamma_{2}~~2\gamma_{3}~~2\gamma_{4}|\\
&=2\times2\times2|\gamma_ {1}~\gamma_{2}~\gamma_{3}~\gamma_{4}|+2\times2\times2|\beta~\gamma_{2}~\gamma_{3}~\gamma_{4}|=8|A|+8|B|=8(|A|+|B|)=40
\end{align*}

(2)$\left|A^{2}+AB\right|=|A(A+B)|=|A||A+B|=2\times 40=80.$
\end{jie}

\num 期中2018-2019 一1.

计算行列式$
\begin{vmatrix}
  b^{2}+c^{2} & c^{2}+a^{2} & a^{2}+b^{2} \\
  a & b & c \\
  a^{2} & b^{2} & c^{2}
\end{vmatrix}
.$

\begin{jie}
把第三行加到第一行上:
\begin{equation*}
\begin{vmatrix}
  b^{2}+c^{2} & c^{2}+a^{2} & a^{2}+b^{2} \\
  a & b & c \\
  a^{2} & b^{2} & c^{2}
\end{vmatrix}=\begin{vmatrix}
  a^{2}+b^{2}+c^{2} & b^{2}+c^{2}+a^{2} & a^{2}+b^{2}+c^{2} \\
  a & b & c \\
  a^{2} & b^{2} & c^{2}
\end{vmatrix}=(a^{2}+b^{2}+c^{2})\begin{vmatrix}
  1 & 1 & 1 \\
  a & b & c \\
  a^{2} & b^{2} & c^{2}
\end{vmatrix}=(a^{2}+b^{2}+c^{2})|A|
\end{equation*}
可以看出$|A|$是三阶范德蒙行列式,所以原式$=(a^{2}+b^{2}+c^{2})(c-b)(c-a)(b-a)$
\end{jie}

\num 期中2018-2019 一2.

设$A,B$为3阶矩阵,且$|A|=3,|B|=2$,$A^{*}$为$A$的伴随矩阵。

(2)若$|A^{-1}+B|=2$,求$|A+B^{-1}|$.

\begin{jie}
$|A+B^{-1}|=|EA+B^{-1}E|=|B^{-1}BA+B^{-1}A^{-1}A|=|B^{-1}(BA+A^{-1}A)|=|B^{-1}|\cdot|BA+A^{-1}A|=|B^{-1}|\cdot|(B+A^{-1})A|=|B^{-1}|\cdot|(B+A^{-1})|\cdot|A|=2^{-1}\times2\times3=3$
\end{jie}

\num 期中2018-2019 一4.

设$n$阶行列式$D_{n}(n=1,2,\cdots):D_{1}=1,D_{2}=
\begin{vmatrix}
  1 & 1 \\
  1 & 1
\end{vmatrix},D_{3}=
\begin{vmatrix}
  1 & 1 & 0\\
  1 & 1 & 1\\
  0 & 1 & 1
\end{vmatrix},D_{4}=
\begin{vmatrix}
  1 & 1 & 0 & 0\\
  1 & 1 & 1 & 0\\
  0 & 1 & 1 & 1\\
  0 & 0 & 1 & 1
\end{vmatrix},\ldots\ldots,D_{n}=
\begin{vmatrix}
  1 & 1 & 0 & 0 & \cdots & 0\\
  1 & 1 & 1 & 0 & \cdots & 0\\
  0 & 1 & 1 & 1 & \cdots & 0\\
  \vdots & \vdots &\ddots & \ddots &\ddots &\vdots\\
  0 &\cdots & 0 & 1 & 1& 1\\
  0 &\cdots & 0 & 0 & 1& 1
\end{vmatrix}.
$

(1)给出$D_{n},D_{n-1},D_{n-2}$的关系;

(2)利用找到的递推关系及$D_{1}=1,D_{2}=0$,计算$D_{3},D_{4},\cdots,D_{8}$;

(3)求$D_{2018}$

\begin{jie}
(1)对$D_{n}$按第一列进行行列式展开($a_{11}$表示第一行第一列的元素,$A_{11}$表示第一行第一列元素对应的代数余子式):
\begin{align*}
D_{n}=a_{11}A_{11}+a_{21}A_{21}=1\times(-1)^{1+1}
\begin{vmatrix}
   1 & 1 & 0 & \cdots & 0\\
   1 & 1 & 1 & \cdots & 0\\
   \vdots &\ddots & \ddots &\ddots &\vdots\\
  0 &\cdots  & 1 & 1& 1\\
  0 &\cdots & 0  & 1& 1
\end{vmatrix}+1\times(-1)^(2+1)
\begin{vmatrix}
   1 & 0 & 0 & \cdots & 0\\
   1 & 1 & 1 & \cdots & 0\\
  \vdots &\ddots & \ddots &\ddots &\vdots\\
  0 &\cdots  & 1 & 1& 1\\
  0 &\cdots  & 0 & 1& 1
\end{vmatrix}=D_{n-1}-D_{n-2}
\end{align*}

(2)由(1)得:
\begin{gather*}
D_{3}=D_{3-1}-D_{3-2}=D_{2}-D_{1}=-1\\
D_{4}=D_{4-1}-D_{4-2}=D_{3}-D_{2}=-1\\
D_{5}=D_{5-1}-D_{5-2}=D_{4}-D_{3}=0\\
D_{6}=D_{6-1}-D_{6-2}=D_{5}-D_{4}=1\\
D_{7}=D_{7-1}-D_{7-2}=D_{6}-D_{5}=1\\
D_{8}=D_{8-1}-D_{8-2}=D_{7}-D_{6}=0
\end{gather*}

(3)由(2)可以看出,每6组数据为一个循环,即$D_{n}=D_{n+6}$,所以$2018\div6=336$余$2$。所以$D_{2018}=D_{2}=0$
\end{jie}

{\heiti \zihao{4} 期末试题}

\num 期末2014-2015 一1.

若已知行列式
$
\begin{vmatrix}
  1 & 3 & a \\
  5 & -1 &1\\
  3 & 2&1
\end{vmatrix}
$的代数余子式$A_{21}=1$,则$a=$\underline{\hphantom{~~~~~~~~~~}}。

\begin{jie}
$A_{21}=(-1)^{2+1}\times(3\times 1-2\times 1)=1$,解得$a=2$。
\end{jie}

\num 期末2014-2015 一3.

设3阶方阵$A=(\alpha_{1},\alpha_{2},\alpha_{3})$的行列式$|A|=3$,矩阵$B=(\alpha_{2},2\alpha_{3},-\alpha_{1})$,则行列式$|A-B|=$\underline{\hphantom{~~~~~~~~~~}}。

\begin{jie}
对$A$的第三列乘2得:$|\alpha_{1}~\alpha_{2}~2\alpha_{3}|=2|A|$,对该表达式第一列乘负一:$|-\alpha_{1}~\alpha_{2}~2\alpha_{3}|=-2|A|$,交换一二两列,$|\alpha_{2}~\alpha_ {1}~2\alpha_{3}|=2|A|$,交换二三两列,$|\alpha_ {2}~2\alpha_{3}~\alpha_ {1}|=-2|A|$,所以$|B|=-2|A|$.

\begin{align*}
|A-B|&=|\alpha_{1}-\alpha_{2}~~~\alpha_{2}-2\alpha_{3}~~~\alpha_{3}+\alpha_{1}|=|\alpha_{1}~~~\alpha_{2}-2\alpha_{3}~~~\alpha_{3}+\alpha_{1}|+|-\alpha_{2}~~~\alpha_{2}-2\alpha_{3}~~~\alpha_{3}+\alpha_{1}|\\
&=\left|\alpha_{1}~~~\alpha_{2}-2\alpha_{3}~~~\alpha_{3}\right|+|-\alpha_{2}~~-2\alpha_{3}~~~\alpha_{3}+\alpha_{1}|\\
&=\left|\alpha_{1}~~~\alpha_{2}~~~\alpha_{3}\right|+\left|\alpha_{1}~~~-2\alpha_{3}~~~\alpha_{3}\right|+|-\alpha_{2}~~-2\alpha_{3}~~~\alpha_{3}|+|-\alpha_{2}~~-2\alpha_{3}~~~\alpha_{1}|\\
&=|A|+0+0-|B|=3|A|=9
\end{align*}
\end{jie}

\num 期末2015-2016 一1.

行列式$
D=
\begin{vmatrix}
  1 & a & 0 & 0\\
  -1 & 2-a & a & 0\\
  0 & -2 & 3-a & a\\
  0 & 0 & -3 & 4-a
\end{vmatrix}=
$\underline{\hphantom{~~~~~~~~~~}}。

\begin{jie}
把第二行加到第一行上:
\begin{align*}
D=
\begin{vmatrix}
  1 & a & 0 & 0\\
  -1 & 2-a & a & 0\\
  0 & -2 & 3-a & a\\
  0 & 0 & -3 & 4-a
\end{vmatrix}=\begin{vmatrix}
  1 & a & 0 & 0\\
  0 & 2 & a & 0\\
  0 & -2 & 3-a & a\\
  0 & 0 & -3 & 4-a
\end{vmatrix}=1\times(-1)^{1+1}
\begin{vmatrix}
  2 & a & 0\\
   -2 & 3-a & a\\
   0 & -3 & 4-a
\end{vmatrix}=A
\end{align*}
把$A$的第二行加到第一行上:
\begin{align*}
A=\begin{vmatrix}
  2 & a & 0\\
   0 & 3 & a\\
   0 & -3 & 4-a
\end{vmatrix}=2\times(-1)^{1+1}
\begin{vmatrix}
 3 & a\\
 -3 & 4-a
\end{vmatrix}=24
\end{align*}
\end{jie}

\num 期末2015-2016 二1.

若行列式$D=
\begin{vmatrix}
  1 & 2 & 3 & 4 \\
  0 & 3 & 4 & 6 \\
  3 & 4 & 1 & 2 \\
  2 & 2 & 2 & 2
\end{vmatrix}
$,求$A_{11}+2A_{21}+A_{31}+2A_{41}$,其中$A_{ij}$为元素$a_{ij}$的代数余子式。

\begin{jie}
由题得:
\begin{align*}
A_{11}+2A_{21}+A_{31}+2A_{41}=\begin{vmatrix}
  1 & 2 & 3 & 4 \\
  2 & 3 & 4 & 6 \\
  1 & 4 & 1 & 2 \\
  2 & 2 & 2 & 2
\end{vmatrix}=2\begin{vmatrix}
  1 & 2 & 3 & 4 \\
  2 & 3 & 4 & 6 \\
  1 & 4 & 1 & 2 \\
  1 & 1 & 1 & 1
\end{vmatrix}
\end{align*}
把第一行的负二倍加到第二行,把第一行的负一倍分别加到第三、四行。
\begin{align*}
2\begin{vmatrix}
  1 & 2 & 3 & 4 \\
  2 & 3 & 4 & 6 \\
  1 & 4 & 1 & 2 \\
  1 & 1 & 1 & 1
\end{vmatrix}=
2\begin{vmatrix}
  1 & 2 & 3 & 4 \\
  0 & -1 & -2 & -2 \\
  0 & 2 & -2 & -2 \\
  0 & -1 & -2 & -3
\end{vmatrix}=4\begin{vmatrix}
  -1 & -2 & -2 \\
   1 & -1 & -1 \\
   -1 & -2 & -3
\end{vmatrix}
\end{align*}
把第一行的负一倍加到第三行:
\begin{align*}
4\begin{vmatrix}
  -1 & -2 & -2 \\
   1 & -1 & -1 \\
   -1 & -2 & -3
\end{vmatrix}=
4\begin{vmatrix}
  -1 & -2 & -2 \\
   1 & -1 & -1 \\
  0 & 0 & -1
\end{vmatrix}=4\times(-1)\times((-1)\times(-1)-(-2)\times 1)=-12
\end{align*}
\end{jie}

\num 期末2016-2017 一1.

行列式$
D=
\begin{vmatrix}
  1 & x & y & z\\
  x & 1 & 0 & 0\\
  y & 0 & 1 & 0\\
  z & 0 & 0 & 1
\end{vmatrix}=
$\underline{\hphantom{~~~~~~~~~~}}。

\begin{jie}
对行列式做如下变换:第一行减去$z$倍的第四行,第一行减去$y$倍的第三行,第一行减去$x$倍的第二行,得到如下行列式:
\begin{equation*}
  \begin{vmatrix}
  1-x^{2}-y^{2}-z^{2} & 0 & 0 & 0\\
  a & 1 & 0 & 0\\
  b & 0 & 1 & 0\\
  c & 0 & 0 & 1\\
\end{vmatrix}=1-x^{2}-y^{2}-z^{2}
\end{equation*}
\end{jie}

\num 期末2016-2017 二1.

若行列式$D=
\begin{vmatrix}
  1 & 2 & 3 & 4 \\
  0 & 3 & 4 & 6 \\
  0 & 4 & 1 & 2 \\
  0 & 2 & 2 & 2
\end{vmatrix}
$,求$A_{11}-2A_{21}+A_{31}-2A_{41}$,其中$A_{ij}$为元素$a_{ij}$的代数余子式。

\begin{jie}
44.过程略。
\end{jie}

\num 期末2017-2018 一1.

设$A_{ij}$是三阶行列式$
D=
\begin{vmatrix}
  2 & 2 & 2\\
  1 & 2 & 3\\
  4 & 5 & 6
\end{vmatrix}
$第$i$行第$j$列元素的代数余子式,则$A_{31}+A_{32}+A_{33}=$\underline{\hphantom{~~~~~~~~~~}}。

\begin{jie}
0.过程略。
\end{jie}

\num 期末2017-2018 二1.

计算行列式$D=
\begin{vmatrix}
  3 & 1 & -1 & 2 \\
  -5 & 1 & 3 & -4 \\
  2 & 0 & 1 & -1\\
   1 & -5 & 3 & -3
\end{vmatrix}
.$

\begin{jie}
40.过程略
\end{jie}

\num 期末2018-2019 二1.

已知$D=
\begin{vmatrix}
  1 & 1 & 1 & 1 \\
  -1 & 2 & 2 & 3 \\
  1 & 4 & 3 & 9 \\
  -1 & 8 & 5 & 27
\end{vmatrix}
$,求$A_{13}+A_{23}+A_{33}+A_{43}$,其中$A_{ij}$为元素$a_{ij}$的代数余子式。

\begin{jie}
-48.过程略。
\end{jie}

\num 期末2019-2020 二3.

记$2n$阶方阵$
A_{n}=
\begin{bmatrix}
  a_{n} & ~  & ~ & ~ & ~ & ~ & ~ & b_{n}\\
  ~ & a_{n-1}  & ~ & ~ & ~ & ~ & b_{n-1} & ~\\
  ~ & ~ & \ddots & ~ & ~ & \iddots  & ~ & ~\\
  ~ & ~ & ~ & a_{1}&b_{1} & ~ & ~ & ~\\
  ~ & ~ & ~ & c_{1}&d_{1} & ~ & ~ & ~\\
  ~ & ~ & \iddots & ~ & ~ & \ddots  & ~ & ~\\
  ~ & c_{n-1}  & ~ & ~ & ~ & ~ & d_{n-1} & ~\\
  c_{n} & ~  & ~ & ~ & ~ & ~ & ~ & d_{n}
\end{bmatrix}
$.

(1)求$|A_{1}|,|A_{2}|$

(2)求$|A_{n}|$。

\begin{jie}
(1)
\begin{align*}
|A_{1}|&=
\begin{vmatrix}
  a_{1} & b_{1} \\
  c_{1} & d_{1}
\end{vmatrix}=a_{1}d_{1}-c_{1}b_{1}\\
|A_{2}|&=\begin{vmatrix}
a_{2}& 0& 0&b_{2}\\
0&  a_{1} & b_{1}& 0 \\
0&  c_{1} & d_{1}& 0\\
c_{2}& 0& 0&d_{2}
\end{vmatrix}=a_{2}\begin{vmatrix}
  a_{1} & b_{1}& 0 \\
  c_{1} & d_{1}& 0\\
 0& 0&d_{2}
\end{vmatrix}-b_{2}
\begin{vmatrix}
0&  a_{1} & b_{1} \\
0&  c_{1} & d_{1}\\
c_{2}& 0& 0
\end{vmatrix}=a_{2}d_{2}
\begin{vmatrix}
  a_{1} & b_{1}\\
  c_{1} & d_{1}\\
\end{vmatrix}-b_{2}c_{2}\begin{vmatrix}
  a_{1} & b_{1}\\
  c_{1} & d_{1}\\
\end{vmatrix}=(a_{2}d_{2}-b_{2}c_{2})|A_{1}|\\
&=(a_{2}d_{2}-b_{2}c_{2})(a_{1}d_{1}-c_{1}b_{1})
\end{align*}

(2)用数学归纳法:

由(1)得:
\begin{gather*}
n=1:~~\left|A_ {1}\right|=a_{1}d_{1}-c_{1}b_{1}=\prod\limits_{i=0}^{1}(a_{i}d_{i}-c_{i}b_{i})\\
n=2:~~\left|A_ {2}\right|=(a_{2}d_{2}-c_{2}b_{2})\left|A_ {1}\right|=\prod\limits_{i=0}^{2}(a_{i}d_{i}-c_{i}b_{i})
\end{gather*}
则$n=k-1$时,有$|A_{n}|=\prod\limits_{i=0}^{k-1}(a_{i}d_{i}-c_{i}b_{i})$.\\
当$n=k$时,按第一列展开,得:
\begin{align*}
\left|A_ {k}\right|&=a_{11}A_{11}+a_{2k1}A_{2k1}\\
&=a_{k}\begin{bmatrix}
  a_{k-1}  & ~ & ~ & ~ & ~ & b_{k-1} & ~\\
   ~ & \ddots & ~ & ~ & \iddots  & ~ & ~\\
   ~ & ~ & a_{1}&b_{1} & ~ & ~ & ~\\
  ~ & ~ & c_{1}&d_{1} & ~ & ~ & ~\\
   ~ & \iddots & ~ & ~ & \ddots  & ~ & ~\\
   c_{k-1}  & ~ & ~ & ~ & ~ & d_{k-1} & ~\\
   0  & ~ & ~ & ~ & ~ & ~ & d_{k}
\end{bmatrix}+c_{k}(-1)^{2k+1}
\begin{bmatrix}
  0  & ~ & ~ & ~ & ~ & ~ & b_{k}\\
   a_{k-1}  & ~ & ~ & ~ & ~ & b_{k-1} & ~\\
   ~ & \ddots & ~ & ~ & \iddots  & ~ & ~\\
   ~ & ~ & a_{1}&b_{1} & ~ & ~ & ~\\
  ~ & ~ & c_{1}&d_{1} & ~ & ~ & ~\\
   ~ & \iddots & ~ & ~ & \ddots  & ~ & ~\\
   c_{k-1}  & ~ & ~ & ~ & ~ & d_{k-1} & ~
\end{bmatrix}\\
&=a_{k}d_{k}(-1)^{2k-2+1+2k-2+1}\begin{bmatrix}
  a_{k-1}  & ~ & ~ & ~ & ~ & b_{k-1} \\
   ~ & \ddots & ~ & ~ & \iddots  & ~ \\
   ~ & ~ & a_{1}&b_{1} & ~ & ~ \\
  ~ & ~ & c_{1}&d_{1} & ~ & ~ \\
   ~ & \iddots & ~ & ~ & \ddots  & ~ \\
   c_{k-1}  & ~ & ~ & ~ & ~ & d_{k-1}
\end{bmatrix}-c_{k}d_{k}(-1)^{1+2k-2+1}\begin{bmatrix}
  a_{k-1}  & ~ & ~ & ~ & ~ & b_{k-1} \\
   ~ & \ddots & ~ & ~ & \iddots  & ~ \\
   ~ & ~ & a_{1}&b_{1} & ~ & ~ \\
  ~ & ~ & c_{1}&d_{1} & ~ & ~ \\
   ~ & \iddots & ~ & ~ & \ddots  & ~ \\
   c_{k-1}  & ~ & ~ & ~ & ~ & d_{k-1}
\end{bmatrix}\\ &=(a_{k}d_{k}-c_{k}d_{k})|A_{k-1}|=\prod_{i=1}^{k}(a_{i}d_{i}-c_{i}b_{i})
\end{align*}
\end{jie}

\end{document}  