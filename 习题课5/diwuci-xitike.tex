\documentclass{article}
\usepackage[space,fancyhdr,fntef]{ctexcap}
\usepackage[namelimits,sumlimits,nointlimits]{amsmath}
\usepackage[bottom=25mm,top=25mm,left=25mm,right=15mm,centering]{geometry}
\usepackage{xcolor}
\usepackage{arydshln}%234页,虚线表格宏包
\usepackage{mathdots}%反对角省略号
\pagestyle{fancy} \fancyhf{}
\fancyhead[OL]{~~~班序号:\hfill 学院:\hfill 学号:\hfill 姓名:王松年~~~ \thepage}
%\usepackage{parskip}
%\usepackage{indentfirst}
\usepackage{graphicx}%插图宏包,参见手册318页
\begin{document}

\newcounter{num} \renewcommand{\thenum}{\arabic{num}.} \newcommand{\num}{\refstepcounter{num}\text{\thenum}}

\hphantom{~~}\hfill {\zihao{3}\heiti 第五次习题课} \hfill\hphantom{~~}

\hphantom{~~}\hfill {\zihao{4}\heiti 群文件《期中$\&$期末试题》} \hfill\hphantom{~~}

{\heiti \zihao{4} 期中试题}

\num 2015-2016 一1.

已知$A=
\begin{bmatrix}
  3 & 0 & 4 & 0 \\
  2 & 2 & 2 & 2 \\
  0 & -7 & 0 & 0 \\
  5 & 3 & -2 & 2 \\
\end{bmatrix}
$,则代数余子式之和$A_{41}+A_{42}+A_{43}+A_{44}=$\underline{\hphantom{~~~~~~~~~~}}。
\\

%\num 2015-2016 一2.
%
%设$f(x)=
%\begin{vmatrix}
%  2x & x & 1 & 2\\
%  1 & x & 1 & -1\\
%  3 & 2 & x & 1\\
%  1 & 1 & 1 & x\\
%\end{vmatrix}
%$,则$x^{3}$的系数为\underline{\hphantom{~~~~~~~~~~}}。
%\\

\num 2015-2016 一3.

设$a,b,c$满足方程
$
\begin{vmatrix}
  1 & a & b & c\\
  a & 1 & 0 & 0\\
  b & 0 & 1 & 0\\
  c & 0 & 0 & 1\\
\end{vmatrix}=1
$,则$abc=$\underline{\hphantom{~~~~~~~~~~}}。
\\

\num 2015-2016 一4.

设$A=
\begin{bmatrix}
  2 & 0 \\
  1 & 4
\end{bmatrix}
$,若$B=2BA-3I$,其中$I$为单位矩阵,则$|B|=$\underline{\hphantom{~~~~~~~~~~}}。\\

\num 2015-2016 二1.

计算行列式$
\begin{vmatrix}
  5 & 0 & 4 & 2\\
  1 & -1 & 2 & 1\\
  4 & 1 & 2 & 0\\
  1 & 1 & 1 & 1\\
\end{vmatrix}
$。\\


\num 2015-2016 二2.

计算行列式$
\begin{vmatrix}
  a^{2} & ab & b^{2}\\
  2a & a+b & 2b \\
  1 & 1 & 1
\end{vmatrix}
$。\\

\num 2015-2016 三2.

设$A$为$n$阶方阵,$AA^{T}=I$,$|A|<0$,证明:$|A+I|=0$。\\

\num 2016-2017 一1.

设$M_{ij}$是$
\begin{vmatrix}
  0 & 4 & 0 \\
  2 & 2 & 2\\
  2 & 0 & 0
\end{vmatrix}
$的第$i$行第$j$列元素的余子式,则$M_{11}+M_{12}=$\underline{\hphantom{~~~~~~~~~~}}。\\

\num 2016-2017 一2.

计算行列式$
\begin{vmatrix}
  1 & 1 & 1 & 1 \\
  1 & 2 & 4& 8 \\
  1 & 3 & 9& 27\\
   1 & 4 &16 &64
\end{vmatrix}
=$\underline{\hphantom{~~~~~~~~~~}}。\\

\num 2016-2017 二1.

计算行列式$
\begin{vmatrix}
  1 & 2 & 3 & 4 \\
  2 & 3 & 4& 1 \\
  3 & 4 & 1& 2\\
   4 & 1 &2 &3
\end{vmatrix}
$。\\

\num 2016-2017 二2.

设$f(x)=
\begin{vmatrix}
  x-1 & 1 & -1 & 1 \\
  -1 & x+1 & -1& 1 \\
  -1 & 1 & x-1& 1\\
   -1 & 1 &-1 &x+1
\end{vmatrix}
$,求$f(x)=0$的根。\\

%\num  期中2016-2017 三2.
%
%已知$A=(a_{ij})$是三阶的非零矩阵,设$A_{ij}$是$a_{ij}$的代数余子式,且对任意的$i,j$有$A_{ij}+a_{ij}=0$,求$A$ 的行列式。\\


\num 2017-2018 一1.

计算行列式$
\begin{vmatrix}
  1 & 1 & 1 & 1 \\
  1 & 1 & 1& 2 \\
  1 & 1 & 3& 1\\
   1 & 4 &1 &1
\end{vmatrix}
.$\\

\num 期中2017-2018 一2.

 求方程$
\begin{vmatrix}
  1 & 2 & 1 & 1 \\
  1 & x & 2& 3 \\
  1 & 2 & x& 2\\
   0 & 0 &2 &x
\end{vmatrix}=0
$的根。\\

\num 期中2017-2018 一3.

 设$\gamma_{1},\gamma_{2},\gamma_{3},\gamma_{4}$及$\beta$均为4维列向量。4阶矩阵$A=[\gamma_ {1}~\gamma_{2}~\gamma_{3}~\gamma_{4}],B=[\beta~\gamma_{2}~\gamma_{3}~\gamma_{4}]$,若$\left|A\right|=2,\left|B\right|=3$,求

(1)$\left|A+B\right|$;

(2)$\left|A^{2}+AB\right|$;\\

\num 期中2018-2019 一1.

计算行列式$
\begin{vmatrix}
  b^{2}+c^{2} & c^{2}+a^{2} & a^{2}+b^{2} \\
  a & b & c \\
  a^{2} & b^{2} & c^{2}
\end{vmatrix}
.$\\

\num 期中2018-2019 一2.

设$A,B$为3阶矩阵,且$|A|=3,|B|=2$,$A^{*}$为$A$的伴随矩阵。

(2)若$|A^{-1}+B|=2$,求$|A+B^{-1}|$.\\

\num 期中2018-2019 一4.

设$n$阶行列式$D_{n}(n=1,2,\cdots):D_{1}=1,D_{2}=
\begin{vmatrix}
  1 & 1 \\
  1 & 1
\end{vmatrix},D_{3}=
\begin{vmatrix}
  1 & 1 & 0\\
  1 & 1 & 1\\
  0 & 1 & 1
\end{vmatrix},D_{4}=
\begin{vmatrix}
  1 & 1 & 0 & 0\\
  1 & 1 & 1 & 0\\
  0 & 1 & 1 & 1\\
  0 & 0 & 1 & 1
\end{vmatrix},\ldots\ldots,D_{n}=
\begin{vmatrix}
  1 & 1 & 0 & 0 & \cdots & 0\\
  1 & 1 & 1 & 0 & \cdots & 0\\
  0 & 1 & 1 & 1 & \cdots & 0\\
  \vdots & \vdots &\ddots & \ddots &\ddots &\vdots\\
  0 &\cdots & 0 & 1 & 1& 1\\
  0 &\cdots & 0 & 0 & 1& 1
\end{vmatrix}.
$

(1)给出$D_{n},D_{n-1},D_{n-2}$的关系;

(2)利用找到的递推关系及$D_{1}=1,D_{2}=0$,计算$D_{3},D_{4},\cdots,D_{8}$;

(3)求$D_{2018}$\\

{\heiti \zihao{4} 期末试题}

\num 期末2014-2015 一1.

若已知行列式
$
\begin{vmatrix}
  1 & 3 & a \\
  5 & -1 &1\\
  3 & 2&1
\end{vmatrix}
$的代数余子式$A_{21}=1$,则$a=$\underline{\hphantom{~~~~~~~~~~}}。\\

\num 期末2014-2015 一3.

设3阶方阵$A=(\alpha_{1},\alpha_{2},\alpha_{3})$的行列式$|A|=3$,矩阵$B=(\alpha_{2},2\alpha_{3},-\alpha_{1})$,则行列式$|A-B|=$\underline{\hphantom{~~~~~~~~~~}}。\\

%\num 期末2014-2015 二.
%
%设多项式$
%f(x)=
%\begin{vmatrix}
%  2x & 3 & 1 & 2\\
%  x & x & -2 & 1\\
%  2 & 1 & x & 4\\
%  x & 2 & 1 & 4x
%\end{vmatrix}
%$,分别求该多项式的三次项、常数项。\\

\num 期末2015-2016 一1.

行列式$
D=
\begin{vmatrix}
  1 & a & 0 & 0\\
  -1 & 2-a & a & 0\\
  0 & -2 & 3-a & a\\
  0 & 0 & -3 & 4-a
\end{vmatrix}=
$\underline{\hphantom{~~~~~~~~~~}}。\\

\num 期末2015-2016 二1.

若行列式$D=
\begin{vmatrix}
  1 & 2 & 3 & 4 \\
  0 & 3 & 4 & 6 \\
  3 & 4 & 1 & 2 \\
  2 & 2 & 2 & 2
\end{vmatrix}
$,求$A_{11}+2A_{21}+A_{31}+2A_{41}$,其中$A_{ij}$为元素$a_{ij}$的代数余子式。\\

\num 期末2016-2017 一1.

行列式$
D=
\begin{vmatrix}
  1 & x & y & z\\
  x & 1 & 0 & 0\\
  y & 0 & 1 & 0\\
  z & 0 & 0 & 1
\end{vmatrix}=
$\underline{\hphantom{~~~~~~~~~~}}。\\

\num 期末2016-2017 二1.

若行列式$D=
\begin{vmatrix}
  1 & 2 & 3 & 4 \\
  0 & 3 & 4 & 6 \\
  0 & 4 & 1 & 2 \\
  0 & 2 & 2 & 2
\end{vmatrix}
$,求$A_{11}-2A_{21}+A_{31}-2A_{41}$,其中$A_{ij}$为元素$a_{ij}$的代数余子式。\\

\num 期末2017-2018 一1.

设$A_{ij}$是三阶行列式$
D=
\begin{vmatrix}
  2 & 2 & 2\\
  1 & 2 & 3\\
  4 & 5 & 6
\end{vmatrix}
$第$i$行第$j$列元素的代数余子式,则$A_{31}+A_{32}+A_{33}=$\underline{\hphantom{~~~~~~~~~~}}。\\

\num 期末2017-2018 二1.

计算行列式$D=
\begin{vmatrix}
  3 & 1 & -1 & 2 \\
  -5 & 1 & 3 & -4 \\
  2 & 0 & 1 & -1\\
   1 & -5 & 3 & -3
\end{vmatrix}
.$\\

\num 期末2018-2019 二1.

已知$D=
\begin{vmatrix}
  1 & 1 & 1 & 1 \\
  -1 & 2 & 2 & 3 \\
  1 & 4 & 3 & 9 \\
  -1 & 8 & 5 & 27
\end{vmatrix}
$,求$A_{13}+A_{23}+A_{33}+A_{43}$,其中$A_{ij}$为元素$a_{ij}$的代数余子式。\\

\num 期末2019-2020 二3.

记$2n$阶方阵$
A_{n}=
\begin{bmatrix}
  a_{n} & ~  & ~ & ~ & ~ & ~ & ~ & b_{n}\\
  ~ & a_{n-1}  & ~ & ~ & ~ & ~ & b_{n-1} & ~\\
  ~ & ~ & \ddots & ~ & ~ & \iddots  & ~ & ~\\
  ~ & ~ & ~ & a_{1}&b_{1} & ~ & ~ & ~\\
  ~ & ~ & ~ & c_{1}&d_{1} & ~ & ~ & ~\\
  ~ & ~ & \iddots & ~ & ~ & \ddots  & ~ & ~\\
  ~ & c_{n-1}  & ~ & ~ & ~ & ~ & d_{n-1} & ~\\
  c_{n} & ~  & ~ & ~ & ~ & ~ & ~ & d_{n}
\end{bmatrix}
$.

(1)求$|A_{1}|,|A_{2}|$

(2)求$|A_{n}|$。\\

\end{document}  