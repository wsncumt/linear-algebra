\documentclass{article}
\usepackage[space,fancyhdr,fntef]{ctexcap}
\usepackage[namelimits,sumlimits,nointlimits]{amsmath}
\usepackage[bottom=25mm,top=25mm,left=25mm,right=15mm,centering]{geometry}
\usepackage{xcolor}
\usepackage{paralist}%列表宏包
\usepackage{arydshln}%234页,虚线表格宏包
\pagestyle{fancy} \fancyhf{}
\fancyhead[OL]{~~~班序号:\hfill 学院:\hfill 学号:\hfill 姓名:王松年~~~ \thepage}
%\usepackage{parskip}
%\usepackage{indentfirst}
\usepackage{graphicx}%插图宏包,参见手册318页
\begin{document}

\newcounter{num} \renewcommand{\thenum}{\arabic{num}.} \newcommand{\num}{\refstepcounter{num}\text{\thenum}}

\hphantom{~~}\hfill {\zihao{3}\heiti 第五次习题课} \hfill\hphantom{~~}

\hphantom{~~}\hfill {\zihao{4}\heiti 知识点} \hfill\hphantom{~~}

\num 余子式、代数余子式、行列式的定义以及求法。

\num 特殊行列式:

下(上)三角矩阵是主对角线上元素的乘积。

对角阵的行列式是主对角线上元素的乘积。

反三角行列式是反对角线上元素的乘积乘$(-1)^{\frac{n(n-1)}{2}}$

\num 行列式的存在唯一性质。
\begin{asparaenum}[(1)]
\item 线性性质。
\item 交错性:如果有两行(列)相同(或成比例),则行列式为0.
\item 单位阵的行列式为1.
\item 若行列式其中一行(列)为0,则行列式为0.
\end{asparaenum}

\num 行列式的性质:
\begin{asparaenum}[(1)]
\item 方阵$A$的第$i$行乘以常数$k$,得到的矩阵为$E(i(k))A$,则$\det E(i(k))A=k\det A$。
\item 交换方阵$A$的第$i$行与第$j$行,得到的矩阵为$E(ij)A$,则$\det E(ij)A=\det E(ij)\det A=-det A$。
\item 方阵$A$的第$j$行的$l$倍加到第$i$行,得到的矩阵为$E(ij(l))A$,则$\det E(ij(l))A=\det E(ij(l))\det A=\det A$
\end{asparaenum}

\num 设$A=(a_{ij})$是$n$阶方阵,$A_{ij}$是$a_{ij}$的代数余子式,则
\begin{gather*}
  \sum_{j=1}^{n}a_{ij}A_{sj}=
  \begin{cases}
  \det A~~&i=s\\
  0~~&i\neq s
  \end{cases}\\
  \sum_{i=1}^{n}a_{ij}A_{is}=
  \begin{cases}
  \det A~~&j=s\\
  0~~&i\neq s
  \end{cases}
\end{gather*}

\num  范德蒙行列式

\num 假设$A$、$B$都是$n$阶方阵,则$|AB|=|A||B|=|B||A|=|BA|$

\num 设$A$是分块的上三角矩阵,其中对角线上都是方阵,则$A$的行列式等于对角线行列式的乘积。
\end{document}  