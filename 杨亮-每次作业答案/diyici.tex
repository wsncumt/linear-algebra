\documentclass[a4paper]{report}
\usepackage[space,fancyhdr,fntef]{ctexcap}
\usepackage{fontspec}
\fontspec{宋体}
\setmainfont{Times New Roman}
%\fontsize{50pt}{50pt}\selectfont
\renewcommand{\rmdefault}{ptm}
\usepackage[namelimits,sumlimits,nointlimits]{amsmath}
\usepackage[text={169mm,250mm},bottom=20mm,top=25mm,left=25mm,right=20mm,centering]{geometry}
\usepackage{color}
\usepackage{CJKfntef}%下划线宏包160页
\usepackage{xcolor}
\usepackage{arydshln}%234页,虚线表格宏包
\pagestyle{fancy} \fancyhf{}
\pagestyle{fancy} \fancyhf{}
\fancyhead[OL]{~~~学院:\hfill 学号:\hfill 姓名:王松年~~~ }
\fancyfoot[C]{\color{gray}\thepage}
\renewcommand{\headrule}{\color{gray}\hrule width\headwidth}
%\renewcommand{\footrulewidth}{0.4pt}%改为0pt即可去掉页脚上面的横线
%\usepackage{parskip}
%\usepackage{indentfirst}
\usepackage{graphicx}%插图宏包,参见手册318页

\usepackage[xetex,colorlinks]{hyperref}%394页  \href{网址}{文本}
\hypersetup{urlcolor=blue}
%\linebreak[2]%换行,152页
\usepackage{fancybox}%盒子宏包55页
\setcounter{secnumdepth}{4}
\CTEXoptions[contentsname={目\hspace{15pt}录}]
\CTEXsetup[beforeskip={-40pt},afterskip={20pt plus 2pt minus 2pt}]{chapter}
\usepackage{mathdots}%反对角省略号
\usepackage{extarrows}%等号上加文字
\usepackage{paralist}%列表宏包

%目录设置
\usepackage{titletoc}
\usepackage[toc]{multitoc}
\titlecontents{chapter}[4em]{\addvspace{2.3mm}\bf}{\contentslabel{4.0em}}{}{\titlerule*[5pt]{$\cdot$}\contentspage}
\titlecontents{section}[4em]{}{\contentslabel{2.5em}}{}{\titlerule*[5pt]{$\cdot$}\contentspage}
\titlecontents{subsection}[7.2em]{}{\contentslabel{3.3em}}{}{\titlerule*[5pt]{$\cdot$}\contentspage}

\begin{document}
\flushbottom%版心底部对齐
\newcounter{num}[section] \renewcommand{\thenum}{\arabic{num}.} \newcommand{\num}{\refstepcounter{num}\text{\thenum}}
\newenvironment{jie}{\kaishu\zihao{-5}\color{blue}{\noindent\em 解:}\par}{\hfill $\diamondsuit$\par}
\newenvironment{tips}{\kaishu\zihao{-6}\color{blue}{\noindent\rule[-3pt]{\textwidth}{0.5pt}\par \em \noindent {\zihao{-5} \textcolor[rgb]{1.00,0.00,0.00}{Tips}}}\par}{\\ \rule[3mm]{\textwidth}{0.5pt}\par}

%生成格式如 例1.1的带序号的示例标识
\newcounter{Emp}[chapter] \renewcommand{\theEmp}{\thechapter.\arabic{Emp}}
\newcommand{\EX}{\par {\bf 例~}\refstepcounter{Emp}{\bf\theEmp}\hspace{1em}}

\newenvironment{zhengming}{\kaishu\zihao{-5}\color{blue}{\noindent\em 证明:}\par}{\hfill $\diamondsuit$\par}

\tableofcontents
\pagenumbering{Roman}%设置目录页码
\clearpage
\pagenumbering{arabic}%设置正文页码
\setcounter{chapter}{1}
%\chapter{第一次线性代数}

\hphantom{~~}\hfill {\zihao{3}\heiti 第一次线性代数} \hfill\hphantom{~~}
\addcontentsline{toc}{section}{\protect\numberline {}第一次线性代数}

\hphantom{~~}
%\section{基本概念、阶梯型方程组}
\EX 写出一个只有零解的齐次线性方程组的例子。写出一个有非零解的齐次线性方程组的例子。

\begin{tips}
形如$a_1x_1+a_2x_2+\cdots+a_nx_n=0$的称为齐次线程方程(其中$a_i$为系数)。

齐次的含义:上式可以写为$a_1x_1+a_2x_2+\cdots+a_nx_n=0\textcolor[rgb]{1.00,0.00,0.00}{x_{n+1}}$,该式中每一项$x$的指数都为1次,所以称为齐次。

非齐次:$a_1x_1+a_2x_2+\cdots+a_nx_n=b,(b\neq 0)$,该式可以写为$a_1x_1+a_2x_2+\cdots+a_nx_n=b\textcolor[rgb]{1.00,0.00,0.00}{x_{n+1}^{0}}$,等式左边的参数都是1次,而右边是0次,所以为非齐次。

线性的含义:此处可以理解为所有的参数间仅有加减(各个参数的相加减)及数乘(即某个参数乘上一个系数)运算。而含有类似下边的项就不是线性:$x_ix_j$(参数间乘法运算,而非数乘)、$x_{i}^{n},\sqrt{x_i}$(幂运算)、$\log x_i$(对数运算)以及其他不满足上述要求的运算。

齐次线性方程组:只含有齐次线性方程。

非齐次线性方程组:含有非齐次线性方程。
\end{tips}

\begin{jie}
答案不唯一,符合题意即可。

只有零解:$
\begin{cases}
x_1+x_2 = 0\\
x_1-x_2 = 0
\end{cases}$\hphantom{~~~~~~~~~~}
非零解:$
\begin{cases}
x_1+x_2 = 0\\
2x_1+2x_2 = 0
\end{cases}$
\end{jie}

\EX 证明:如果一个线性方程组有零解,则该方程组一定是齐次线性方程组。(等价的,非齐次线性方程组一定没有零解)

\begin{zhengming}
对任意一个线程方程组:
\begin{equation*}\begin{cases}
a_{11}x_{1}+a_{12}x_{2}+\cdots+a_{1n}x_{n}=b_1\\
a_{21}x_{2}+a_{22}x_{2}+\cdots+a_{2n}x_{n}=b_2\\
\cdots\\
a_{i1}x_{1}+a_{i2}x_{2}+\cdots+a_{in}x_{n}=b_i\\
\cdots\\
a_{m1}x_{1}+a_{m2}x_{2}+\cdots+a_{mn}x_{n}=b_m \end{cases}
\end{equation*}

该线性方程组有零解,即:$x_1=x_2=\cdots=x_n=0$,把该解代入到上述方程组:
\begin{equation*}\begin{cases}
b_1=a_{11}0+a_{12}0+\cdots+a_{1n}0=0\\
b_2=a_{21}0+a_{22}0+\cdots+a_{2n}0=0\\
\cdots\\
b_i=a_{i1}0+a_{i2}0+\cdots+a_{in}0=0\\
\cdots\\
b_m=a_{m1}0+a_{m2}0+\cdots+a_{mn}0=0 \end{cases}
\end{equation*}
即该方程组为齐次线性方程组。
\end{zhengming}

式1.1.3:
$
\begin{cases}
a_{11}x_{1}+a_{12}x_{2}+\cdots+a_{1n}x_{n}=b_1\\
a_{21}x_{1}+a_{22}x_{2}+\cdots+a_{2n}x_{n}=b_2\\
\cdots\\
a_{m1}x_{1}+a_{m2}x_{2}+\cdots+a_{mn}x_{n}=b_m \end{cases}
$
式1.1.5:
$
\begin{cases}
a_{11}x_{1}+a_{12}x_{2}+\cdots+a_{1n}x_{n}=0\\
a_{21}x_{1}+a_{22}x_{2}+\cdots+a_{2n}x_{n}=0\\
\cdots\\
a_{m1}x_{1}+a_{m2}x_{2}+\cdots+a_{mn}x_{n}=0 \end{cases}
$

\EX 证明:如果
$
\begin{cases}
x_1=c_1\\
x_2=c_2\\
\cdots\\
x_n = c_n
\end{cases}
$和$
\begin{cases}
x_1=d_1\\
x_2=d_2\\
\cdots\\
x_n = d_n
\end{cases}
$都是非齐次线性方程组
1.1.3的解,则$
\begin{cases}
x_1=c_1-d_1\\
x_2=c_2-d_2\\
\cdots\\
x_n =c_n-d_n
\end{cases}$是齐次线性方程组1.1.5的解。

\begin{zhengming}
$x_i = c_i$是非齐次线性方程组的解,所以对于该方程组中的任意一个方程都有:
\begin{equation*}
a_{i1}c_{1}+a_{i2}c_{2}+\cdots+a_{in}c_{n}=b_i\tag{1}
\end{equation*}
同理:
\begin{equation*}
a_{i1}d_{1}+a_{i2}d_{2}+\cdots+a_{in}d_{n}=b_i\tag{2}
\end{equation*}
(1)式-(2)式:
\begin{align*}
&(a_{i1}c_{1}+a_{i2}c_{2}+\cdots+a_{in}c_{n})-(a_{i1}d_{1}+a_{i2}d_{2}+\cdots+a_{in}d_{n})\\
=&a_{i1}(c_{1}-d_{1})+a_{i2}(c_{2}-d_{2})+\cdots+a_{in}(c_{n}-d_{n})\\
=&b_i - b_i =0
\end{align*}

即$x_i=c_i-d_i$是上述齐次线性方程组的解。
\end{zhengming}

\EX 证明:如果$
\begin{cases}
x_1=c_1\\
x_2=c_2\\
\cdots\\
x_n = c_n
\end{cases}
$是非齐次线性方程组1.1.3的解,而
$
\begin{cases}
x_1=d_1\\
x_2=d_2\\
\cdots\\
x_n = d_n
\end{cases}
$是齐次线性方程组的解,则
$
\begin{cases}
x_1=c_1+d_1\\
x_2=c_2+d_2\\
\cdots\\
x_n =c_n+d_n
\end{cases}$是非齐次线性方程组1.1.3的解。

\begin{zhengming}
$x_i = c_i$是非齐次线性方程组的解,所以对于该方程组中的任意一个方程都有:
\begin{equation*}
a_{i1}c_{1}+a_{i2}c_{2}+\cdots+a_{in}c_{n}=b_i\tag{1}
\end{equation*}
同理:
\begin{equation*}
a_{i1}d_{1}+a_{i2}d_{2}+\cdots+a_{in}d_{n}=0\tag{2}
\end{equation*}
(1)式+(2)式:
\begin{align*}
&(a_{i1}c_{1}+a_{i2}c_{2}+\cdots+a_{in}c_{n})-(a_{i1}d_{1}+a_{i2}d_{2}+\cdots+a_{in}d_{n})\\
=&a_{i1}(c_{1}+d_{1})+a_{i2}(c_{2}+d_{2})+\cdots+a_{in}(c_{n}+d_{n})\\
=&b_i + 0 =b_i
\end{align*}

即$x_i=c_i+d_i$是非齐次线性方程组1.1.3的解。
\end{zhengming}

\EX 下述方程组中,哪些是阶梯型方程组,哪些是行简化阶梯型方程组?

(1)$
\begin{cases}
x_1+x_2-x_3=2\\
x_1\hphantom{+x_2-x_3}=3\\
\hphantom{x_1+x_2-}x_3=2
\end{cases}
$

(2)$
\begin{cases}
x_1+\hphantom{2} x_2-2x_3+\hphantom{2}x_4=1\\
\hphantom{x_1}-2x_2+\hphantom{2}x_3-2x_4=2\\
\hphantom{x_1-2x_2+2x_3-2}x_4=0\\
\hphantom{x_1-2x_2+2x_3-2}0=3
\end{cases}
$

(3)$
\begin{cases}
x_1 + x_2 - 3x_3 \hphantom{+ x_4}= 7\\
\hphantom{x_1 + x_2 - 3x_3 + }x_4= 6
\end{cases}
$

\begin{tips}
阶梯型方程组和阶梯型矩阵判断标准一致。

阶梯型矩阵(不唯一)和行最简阶梯型矩阵(唯一)\\
对于任意一个矩阵,如果满足下面三条性质,则称其为\textcolor[rgb]{1.00,0.00,0.00}{阶梯型矩阵}(注:括号里是另一种表述方式,与括号前的表述等价)
\begin{asparaenum}[(1)]
\item 每一非零行在每一零行之上;(全为0的行在不全为0的行的下边)
\item 某一行的主元素所在的列位于前一行主元素所在列的右边;(上边行的首个非0元素位于下边行首个非0元素的左边列);(在所有非0行中,每一行的第一个非零元素所在的列号严格单调递增)
\item 不全为0的行的首个非0元素下面的元素全为0
\end{asparaenum}
行最简阶梯型矩阵:若一个阶梯形矩阵满足以下性质,则称为\textcolor[rgb]{1.00,0.00,0.00}{行最简阶梯型矩阵}。
\begin{asparaenum}[(1)]
\item 每一主元都为1;(每一行第一个非零元素都是1)
\item 每一主元素是该元素所在列的唯一非0元素。
\end{asparaenum}
注:只有是行阶梯形矩阵,才能判断是不是行最简阶梯型矩阵。
\end{tips}

\begin{jie}
(1)不满足阶梯型判断的第2条。所以也不是行最简。

(2)是阶梯型方程组,行最简的两条都不满足,所以不是行最简。

(3)最简阶梯型。(满足上边的每一条)
\end{jie}


\EX 在对线性方程组做初等变换时,能否用0乘以某个方程的两边?为什么?

\begin{jie}
不能。理由:用0乘某个方程的两边,可能会改变方程组的解集。
\end{jie}

\EX 是否存在恰好有两个解的线性方程组?为什么?

\begin{jie}
不存在。任意线性方程组的解只能是三种情形之一:无解,有唯一解和有无穷多解。

假设对于任意的线性方程组$
\begin{cases}
a_{11}x_{1}+a_{12}x_{2}+\cdots+a_{1n}x_{n}=b_1\\
a_{21}x_{1}+a_{22}x_{2}+\cdots+a_{2n}x_{n}=b_2\\
\cdots\\
a_{m1}x_{1}+a_{m2}x_{2}+\cdots+a_{mn}x_{n}=b_m \end{cases}
$,$c_i$、$d_i$是该线性方程组的两个解,则对于该方程组中的任意一个方程,均有:
\begin{equation*}
  \begin{cases}
   a_{i1}c_{1}+a_{i2}c_{2}+\cdots+a_{in}c_{n}=b_i\\
   a_{i1}d_{1}+a_{i2}d_{2}+\cdots+a_{in}d_{n}=b_i
  \end{cases}
\end{equation*}
将两式相加有:
\begin{align*}
&(a_{i1}c_{1}+a_{i2}c_{2}+\cdots+a_{in}c_{n})+(a_{i1}d_{1}+a_{i2}d_{2}+\cdots+a_{in}d_{n})\\
=&a_{i1}(c_{1}+d_{1})+a_{i2}(c_{2}+d_{2})+\cdots+a_{in}(c_{n}+d_{n})\\
=&b_i + b_i =2b_i
\end{align*}
即$e_i=\dfrac{c_i+d_i}{2}$也是该方程组的解,以此类推:$\dfrac{c_i+d_i+e_i}{3}$也是该方程组的解……

所以不存在恰好有两个解的线性方程组。
\end{jie}

\EX 给定一个由$m$个方程组成的$n$元非齐次线性方程组。判断下面的说法是否正确,并简要说明理由。

(1)如果$m<n$,则该方程组一定有无穷多个解。

(2)如果$m< n$,且该方程组有解,则有可能只有一个解.

(3)如果$m=n$,则该方程组一定有唯一解.

(4)如果$m>n$则该方程组一 定没有解.

(5)如果$m> n$,则该方程组可能有且仅有一个解.

(6)如果$m> n$,则该方程组可能有无穷多个解.

\begin{jie}
(1)错误,若非齐次线性方程组中有矛盾方程,则就无解。

(2)错误,将符合此条件的线程方程组化为阶梯型方程组后,一定存在自由变量,即有无穷多解。

(3)错误,理由同(1)。

(4)错误,理由:只要该方程组不存在矛盾方程就有解。

(5)正确,理由:例如$
\begin{cases}
x_1+x_2=1\\
x_1-x_2=5\\
3x_1+3x_2=3
\end{cases}
$

(6)正确,理由:例如
$
\begin{cases}
x_1+x_2=1\\
2x_1+2x_2=2\\
3x_1+3x_2=3
\end{cases}
$
\end{jie}

\EX 给定一个由$m$个方程组成的$n$元齐次线性方程组.判断下列说法是否正确,并简要说明理由.

(1)如果$m< n$,则该方程组一定有 无穷多个解.

(2)如果$m=n$,则该方程组只有零解.

(3)如果$m>n$,则该方程组可能没有解.

(4)如果$m> n$,则该方程组可能只有零解.

(5)如果$m> n$,则该方程组可能有非零解.

\begin{jie}
(1)正确,理由:齐次方程组一定有解,将满足该条件的方程组化为阶梯型方程组后存在自由变量,即有无穷多解。

(2)错误,例如:$
\begin{cases}
x_1+x_2=0\\
2x_1+2x_2=0
\end{cases}
$

(3)错误,理由:齐次方程组一定有解。

(4)正确,例如:
$
\begin{cases}
x_1+x_2=0\\
x_1-2x_2=0\\
x_1=0
\end{cases}
$

(5)正确,例如:$
\begin{cases}
x_1+x_2=0\\
2x_1+2x_2=0\\
3x_1+3x_2=0
\end{cases}
$
\end{jie}

\EX 证明:线性方程组的初等变换一定把齐次线性方程组变为齐次线性方程组,把非齐次线性方程组变为非齐次线性方程组。

\begin{zhengming}
初等变换不改变线性方程组的解集,所以线性方程组是否有零解这个事实在初等变化下不变,即题目中结论成立。

(课本第11页的证明)
\end{zhengming}

\EX 求$a$、$b$、$c$使得
\begin{equation*}
\begin{pmatrix}
-1&a+b&0\\ c-2&-1&a-b
\end{pmatrix}=\begin{pmatrix}
-1&7&0\\ 3a-2b&-1&0
\end{pmatrix}
\end{equation*}

\begin{jie}
由题得:
\begin{equation*}
\begin{cases}
a+b=7\\
c-2=3a-2b\\
a-b=0
\end{cases}~~~\Rightarrow~~~
\begin{cases}
a=\dfrac{7}{2}\\[2pt]
b=\dfrac{7}{2}\\[2pt]
c=\dfrac{11}{2}
\end{cases}
\end{equation*}
\end{jie}

\EX 有人把线性方程组$
\begin{cases}
x_2-2x_3=-1\\
x_1+2x_2-x_4=1\\
2x_1-x_3=0
\end{cases}
$的增广矩阵写为$
\widetilde{A} =
\begin{pmatrix}
&1&-2&-1\\ 1 &2 &-1&1\\ 2&&-1&0
\end{pmatrix}
$是否正确?若不正确,请写出正确的增广矩阵和系数矩阵。

\begin{jie}
不正确。

系数矩阵:$
A =
\begin{pmatrix}
0&1&-2&0\\ 1 &2 &0&-1\\ 2&0&-1&0
\end{pmatrix}$~~~,增广矩阵$
\widetilde{A} =
\begin{pmatrix}
0&1&-2&0&-1\\ 1 &2 &0&-1&1\\ 2&0&-1&0&0
\end{pmatrix}$
\end{jie}

\EX 下述初等行变换是否正确?如果不正确,说明理由。

(1)
$
\begin{pmatrix}
0&-1&2&3\\ 1&1 &0&1
\end{pmatrix}~~\rightarrow~~
\begin{pmatrix}
1&1&2&3\\ 0&-1 &0&1
\end{pmatrix}
$.

(2)
$
\begin{pmatrix}
0&-1&2&3\\ 1&1 &0&1
\end{pmatrix}~~\rightarrow~~
\begin{pmatrix}
1&0&2&3\\ 1&1 &0&1
\end{pmatrix}
$.

(3)
$
\begin{pmatrix}
0&-1&2&3\\ 1&1 &0&1
\end{pmatrix}~~\rightarrow~~
\begin{pmatrix}
0&-1&2&3\\ 2&2 &0&1
\end{pmatrix}
$.

(4)
$
\begin{pmatrix}
0&-1&2&3\\ 1&1 &0&1
\end{pmatrix}~~\rightarrow~~
\begin{pmatrix}
0&-1&2&3\\ 0&0 &0&0
\end{pmatrix}
$.

\begin{jie}
(1)错误,交换第1,2行不正确。

(2)错误,把第二行加到第一行时不正确。

(3)错误,用2乘以第二行时不正确。

(4)错误,不能用0乘以第二行。
\end{jie}

\EX 用初等行变换把$
A=
\begin{pmatrix}
0&-1&3&2\\
2&2&1&5\\
-2&-1&-4&-7\\
1&\frac{1}{2}&2&3
\end{pmatrix}
$化为阶梯型矩阵。

\begin{jie}
\begin{align*}
A&\xrightarrow{\substack{r_{1}\leftrightarrow r_{2}}}
{
\begin{pmatrix}
2&2&1&5\\
0&-1&3&2\\
-2&-1&-4&-7\\
1&\frac{1}{2}&2&3
\end{pmatrix}
}
\xrightarrow{\substack{r_{3}+r_{1}\\ r_{4}-\frac{1}{2}r_{1}}}
{
\begin{pmatrix}
2&2&1&5\\
0&-1&3&2\\
0&1&-3&-2\\
0&-\frac{1}{2}&\frac{3}{2}&\frac{1}{2}
\end{pmatrix}
}\xrightarrow{\substack{r_{3}+r_{2}\\ r_{4}-\frac{1}{2}r_{2}}}
{
\begin{pmatrix}
2&2&1&5\\
0&-1&3&2\\
0&0&0&0\\
0&0&0&-\frac{1}{2}
\end{pmatrix}
}\\
&\xrightarrow{\substack{r_{3}\leftrightarrow r_{4}}}
{
\begin{pmatrix}
2&2&1&5\\
0&-1&3&2\\
0&0&0&-\frac{1}{2}\\
0&0&0&0
\end{pmatrix}
}\end{align*}
\end{jie}

\EX 用初等行变化把
$
A=
\begin{pmatrix}
0&1&-2&1\\
0&3&-1&2\\
2&4&6&-8\\
-3&3&4&-1
\end{pmatrix}
$化为行简化阶梯型矩阵。

\begin{jie}
\begin{align*}
A&\xrightarrow{\substack{r_{1}\leftrightarrow r_{3}}}
{
\begin{pmatrix}
2&4&6&-8\\
0&3&-1&2\\
0&1&-2&1\\
-3&3&4&-1
\end{pmatrix}
}\xrightarrow{\substack{r_{4}+\frac{3}{2} r_{1}}}
{
\begin{pmatrix}
2&4&6&-8\\
0&3&-1&2\\
0&1&-2&1\\
0&9&13&-13
\end{pmatrix}
}\xrightarrow{\substack{r_{3}-\frac{1}{3} r_{2} \\ r_{4}-3r_{2}}}
{
\begin{pmatrix}
2&4&6&-8\\
0&3&-1&2\\
0&0&-\frac{5}{3}&\frac{1}{3}\\
0&0&16&-19
\end{pmatrix}
}\\ &\xrightarrow{\substack{r_{4}-\frac{3}{5}\times r_{3}}}
{
\begin{pmatrix}
2&4&6&-8\\
0&3&-1&2\\
0&0&-\frac{5}{3}&\frac{1}{3}\\
0&0&0&-\frac{79}{5}
\end{pmatrix}
}\xrightarrow{\substack{r_{1}\times\frac{1}{2}\\ r_{2}\times\frac{1}{3}\\ r_{3}\times\left(-\frac{3}{5}\right)\\ r_{4}\times\left(-\frac{5}{79}\right)}}
{
\begin{pmatrix}
1&2&3&-4\\
0&1&-\frac{1}{3}&\frac{2}{3}\\
0&0&1&-\frac{1}{5}\\
0&0&0&1
\end{pmatrix}
}\xrightarrow{\substack{r_{3}+\frac{1}{5}r_{4}\\ r_{2}-\frac{2}{3}r_{4}\\ r_{1}+4r_{4}}}
{
\begin{pmatrix}
1&2&3&0\\
0&1&-\frac{1}{3}&0\\
0&0&1&0\\
0&0&0&1
\end{pmatrix}
}\\ &\xrightarrow{\substack{r_{2}+\frac{1}{3}r_{3}\\ r_{1}-3r_{3}}}
{
\begin{pmatrix}
1&2&0&0\\
0&1&0&0\\
0&0&1&0\\
0&0&0&1
\end{pmatrix}
}\xrightarrow{\substack{r_{1}-2r_{2}}}
{
\begin{pmatrix}
1&0&0&0\\
0&1&0&0\\
0&0&1&0\\
0&0&0&1
\end{pmatrix}
}
\end{align*}
\end{jie}

%\EX 用初等变换把$
%A=
%\begin{pmatrix}
%0&0&0&3\\
%0&0&0&1\\
%3&5&0&0\\
%1&1&0&0
%\end{pmatrix}
%$化为标准型矩阵。
%
%\begin{jie}
%\begin{align*}
%A&\xrightarrow{\substack{r_{1}\leftrightarrow r_{4}}}
%{
%\begin{pmatrix}
%1&1&0&0\\
%0&0&0&1\\
%3&5&0&0\\
%0&0&0&3
%\end{pmatrix}
%}\xrightarrow{\substack{r_{3}-3r_{1}}}
%{
%\begin{pmatrix}
%1&1&0&0\\
%0&0&0&1\\
%0&2&0&0\\
%0&0&0&3
%\end{pmatrix}
%}\xrightarrow{\substack{r_{2}\leftrightarrow r_{3}}}
%{
%\begin{pmatrix}
%1&1&0&0\\
%0&2&0&0\\
%0&0&0&1\\
%0&0&0&3
%\end{pmatrix}
%}\\ &\xrightarrow{\substack{r_{4}-3r_{3}}}
%{
%\begin{pmatrix}
%1&1&0&0\\
%0&2&0&0\\
%0&0&0&1\\
%0&0&0&0
%\end{pmatrix}
%}\xrightarrow{\substack{r_{2}\times \frac{1}{2}}}
%{
%\begin{pmatrix}
%1&1&0&0\\
%0&1&0&0\\
%0&0&0&1\\
%0&0&0&0
%\end{pmatrix}
%}\xrightarrow{\substack{r_{1}-r_{2}}}
%{
%\begin{pmatrix}
%1&0&0&0\\
%0&1&0&0\\
%0&0&0&1\\
%0&0&0&0
%\end{pmatrix}
%}\\ &\xrightarrow{\substack{c_{3}\leftrightarrow c_{4}}}
%{
%\begin{pmatrix}
%1&0&0&0\\
%0&1&0&0\\
%0&0&1&0\\
%0&0&0&0
%\end{pmatrix}
%}
%\end{align*}
%\end{jie}

\EX 设
$
A=
\begin{pmatrix}
2&-1&-4&-4\\
2&0&-2&3\\
-3&0&3&-4
\end{pmatrix}
$.用初等行变换把A变为一个阶梯型矩阵。

\begin{jie}
\begin{align*}
A\xrightarrow{\substack{r_{2}-r_{1}\\ r_{3}+\frac{3}{2}r_{1}}}
{
\begin{pmatrix}
2&-1&-4&-4\\
0&1&2&7\\
0&-\frac{3}{2}&-3&-10
\end{pmatrix}
}\xrightarrow{\substack{ r_{3}+\frac{3}{2}r_{2}}}
{
\begin{pmatrix}
2&-1&-4&-4\\
0&1&2&7\\
0&0&0&\frac{1}{2}
\end{pmatrix}
}
\end{align*}
\end{jie}

\EX 设
$
A=
\begin{pmatrix}
2&-1&-4&-4\\
2&0&-2&3\\
-3&0&3&-4
\end{pmatrix}
$.用初等行变换把A变为一个行简化阶梯型矩阵。

\begin{jie}
接上题:
\begin{align*}
\xrightarrow{\substack{ r_{1}\times \frac{1}{2}\\ r_3\times 2}}
{
\begin{pmatrix}
1&-\frac{1}{2}&-2&-2\\
0&1&2&7\\
0&0&0&1
\end{pmatrix}
}\xrightarrow{\substack{ r_{2}-7r_3\\ r_1+2r_3}}
{
\begin{pmatrix}
1&-\frac{1}{2}&-2&0\\
0&1&2&0\\
0&0&0&1
\end{pmatrix}
}\xrightarrow{\substack{ r_{1}+\frac{1}{2}r_2}}
{
\begin{pmatrix}
1&0&-1&0\\
0&1&2&0\\
0&0&0&1
\end{pmatrix}
}
\end{align*}
\end{jie}

\EX 请写出下面方程组的系数矩阵、常数项矩阵、未知量矩阵和增广矩阵。
\begin{equation*}
\begin{cases}
x_{1}+x_{2}+x_{3} = 1&\\
\hphantom{x_{1}} 2x_{2}+3x_{3} = 1&\\
\hphantom{x_{1}+x_{2}}3^2x_{3} = 3&
\end{cases}
\end{equation*}

\begin{jie}
系数矩阵:
$
\begin{pmatrix}
1&1&1\\ 0&2&3\\0&0&3^2
\end{pmatrix}
$,常数项矩阵$
\begin{pmatrix}
1\\ 1\\ 3
\end{pmatrix}
$,未知量矩阵:
$
\begin{pmatrix}
x_1\\ x_2\\ x_3
\end{pmatrix}
$,增广矩阵
$
\begin{pmatrix}
1&1&1&1\\ 0&2&3&1\\0&0&3^2&3
\end{pmatrix}
$
\end{jie}

\EX 求矩阵$A=
\begin{pmatrix}
2 &-1&4&-3&-4\\ 1&0&1&-1&-3\\ 3&1&1&0&1\\ 7&0&7&-3&3
\end{pmatrix}
$的最简阶梯型矩阵。

\begin{jie}
\begin{align*}
A&\xrightarrow{\substack{r_2-\frac{1}{2}r_1\\ r_3-\frac{3}{2}r_1\\ r_4-\frac{7}{2}r_1}}
{
\begin{pmatrix}
2 &-1&4&-3&-4\\ 0&\frac{1}{2}&-1&\frac{1}{2}&-1\\ 0&\frac{5}{2}&-5&\frac{9}{2}&7\\ 0&\frac{7}{2}&-7&\frac{15}{2}&17
\end{pmatrix}
}\xrightarrow{\substack{r_3-5r_2\\ r_4-7r_2}}
{
\begin{pmatrix}
2 &-1&4&-3&-4\\ 0&\frac{1}{2}&-1&\frac{1}{2}&-1\\ 0&0&0&2&12\\ 0&0&0&4&24
\end{pmatrix}
}\xrightarrow{\substack{r_4-2r_3}}
{
\begin{pmatrix}
2 &-1&4&-3&-4\\ 0&\frac{1}{2}&-1&\frac{1}{2}&-1\\ 0&0&0&2&12\\ 0&0&0&0&0
\end{pmatrix}
}\\ &\xrightarrow{\substack{r_1\times \frac{1}{2}\\ r_2\times 2\\ r_{3}\times\frac{1}{2}}}
{
\begin{pmatrix}
1 &-\frac{1}{2}&2&-\frac{3}{2}&-2\\ 0&1&-2&1&-2\\ 0&0&0&1&6\\ 0&0&0&0&0
\end{pmatrix}
}\xrightarrow{\substack{r_2-r_{3}\\ r_1+\frac{3}{2}r_{3}}}
{
\begin{pmatrix}
1 &-\frac{1}{2}&2&0&7\\ 0&1&-2&0&-8\\ 0&0&0&1&6\\ 0&0&0&0&0
\end{pmatrix}
}\xrightarrow{\substack{r_1+\frac{1}{2}r_{2}}}
{
\begin{pmatrix}
1 &0&1&0&3\\ 0&1&-2&0&-8\\ 0&0&0&1&6\\ 0&0&0&0&0
\end{pmatrix}
}
\end{align*}
\end{jie}

\EX 求线性方程组
$
\begin{cases}
x_1+x_2+x_3=1\\
2x_{1}+3x_{2}=2\\
3x_{1}+4x_{2}+x_{3}=3
\end{cases}
$
的解集。

\begin{jie}
由题得:增广矩阵
\begin{align*}
[A|B]=&
\left(
 \begin{array}{c:c}
\begin{matrix}
1 & 1 & 1 \\
2 & 3 & 0 \\
3 & 4 & 1
\end{matrix}&
\begin{matrix}
1  \\
 2\\
3
\end{matrix}
\end{array}
\right)\xrightarrow{\substack{r_{2}-2r_{1}\\ r_{3}-3r_{1}}}
{
\left(
 \begin{array}{c:c}
\begin{matrix}
1 & 1 & 1 \\
0 & 1 & -2 \\
0 & 1 & -2
\end{matrix}&
\begin{matrix}
1  \\
0 \\
0
\end{matrix}
\end{array}
\right)
}\xrightarrow{\substack{r_{3}-r_{2}}}
{
\left(
 \begin{array}{c:c}
\begin{matrix}
1 & 1 & 1 \\
0 & 1 & -2 \\
0 & 0 & 0
\end{matrix}&
\begin{matrix}
1  \\
0 \\
0
\end{matrix}
\end{array}
\right)
}\\ &\xrightarrow{\substack{r_{1}-r_{2}}}
{
\left(
 \begin{array}{c:c}
\begin{matrix}
1 & 0 & 3 \\
0 & 1 & -2 \\
0 & 0 & 0
\end{matrix}&
\begin{matrix}
1  \\
0 \\
0
\end{matrix}
\end{array}
\right)
}
\end{align*}
所以:$x_1=1-3x_3,x_2=2x_3$,取$x_3=k,k\in R$:
\begin{equation*}
  x=\begin{pmatrix}
     1-3k\\ 2k\\ k
    \end{pmatrix}=
\begin{pmatrix}
1\\ 0\\ 0
\end{pmatrix}
+k\begin{pmatrix}
-3\\ 2\\ 1
\end{pmatrix},k\in R
\end{equation*}
\end{jie}

\EX 线性方程组
$
\begin{cases}
x_1+5x_2+x_3+5x_4=1\\
x_1+4x_2+x_3+4x_4=0\\
x_1+3x_2+ax_3+3x_4=a
\end{cases}
$什么时候无解?什么时候有解?有解时,什么时候有唯一解,什么时候有无穷多组解?有解时,并将所有的解表示出来,其中$a$是常数。

\begin{jie}
由题得:增广矩阵
\begin{align*}
[A|B]=&
\left(
 \begin{array}{c:c}
\begin{matrix}
1 & 5 & 1 &5\\
1 & 4 & 1 &4 \\
1 & 3 & a &3
\end{matrix}&
\begin{matrix}
1  \\
0  \\
a
\end{matrix}
\end{array}
\right)\xrightarrow{\substack{r_{2}-r_{1}\\ r_{3}-r_{1}}}
{
\left(
 \begin{array}{c:c}
\begin{matrix}
1 & 5 & 1 &5\\
0 & -1 & 0 &-1 \\
0 & -2 & a-1 &-2
\end{matrix}&
\begin{matrix}
1  \\
-1  \\
a-1
\end{matrix}
\end{array}
\right)
}\xrightarrow{\substack{r_{3}-2r_{2}}}
{
\left(
 \begin{array}{c:c}
\begin{matrix}
1 & 5 & 1 &5\\
0 & -1 & 0 &-1 \\
0 & 0 & a-1 &0
\end{matrix}&
\begin{matrix}
1  \\
-1  \\
a+1
\end{matrix}
\end{array}
\right)
}
\end{align*}

讨论:

当$a-1=0$且$a+1\neq 0$即$a=1$时存在矛盾方程,此时无解。

当$a-1\neq 0$即$a\neq 1$时有解,由阶梯型矩阵可以看出$x_4$为自由变量,存在自由变量则有无穷解,所以不存在唯一解的情况。

对上述阶梯型矩阵继续高斯消元:
\begin{align*}
\xrightarrow{\substack{r_{2}\times (-1)\\ r_3\times \frac{1}{a-1}}}
{
\left(
 \begin{array}{c:c}
\begin{matrix}
1 & 5 & 1 &5\\
0 & 1 & 0 &1 \\
0 & 0 & 1 &0
\end{matrix}&
\begin{matrix}
1  \\
1  \\
\frac{a+1}{a-1}
\end{matrix}
\end{array}
\right)
}\xrightarrow{\substack{r_{1}-r_3}}
{
\left(
 \begin{array}{c:c}
\begin{matrix}
1 & 5 & 0 &5\\
0 & 1 & 0 &1 \\
0 & 0 & 1 &0
\end{matrix}&
\begin{matrix}
\frac{-2}{a-1}  \\
1  \\
\frac{a+1}{a-1}
\end{matrix}
\end{array}
\right)
}\xrightarrow{\substack{r_{1}-5r_2}}
{
\left(
 \begin{array}{c:c}
\begin{matrix}
1 & 0 & 0 &0\\
0 & 1 & 0 &1 \\
0 & 0 & 1 &0
\end{matrix}&
\begin{matrix}
\frac{-2}{a-1}-5  \\
1  \\
\frac{a+1}{a-1}
\end{matrix}
\end{array}
\right)
}
\end{align*}

所以$x_1=\dfrac{3-5a}{a-1},x_2=1-x_4,x_{3}=\frac{a+1}{a-1}$,取$x_4=k,k\in R$
\begin{equation*}
x=
\begin{pmatrix}
\dfrac{3-5a}{a-1}\\[2pt]
1-k\\
\dfrac{a+1}{a-1}\\[2pt]
k
\end{pmatrix}=\begin{pmatrix}
\dfrac{3-5a}{a-1}\\[2pt]
1\\
\dfrac{a+1}{a-1}\\[2pt]
0
\end{pmatrix}+k
\begin{pmatrix}
0\\
-1\\
0\\
1
\end{pmatrix},k\in R
\end{equation*}

综上所述:$a=1$时无解,不存在唯一解的情况,$a\neq 1$时有无穷多解,解为$\begin{pmatrix}
\dfrac{3-5a}{a-1}\\[2pt]
1\\
\dfrac{a+1}{a-1}\\[2pt]
0
\end{pmatrix}+k
\begin{pmatrix}
0\\
-1\\
0\\
1
\end{pmatrix},k\in R$
\end{jie}

\clearpage

\hphantom{~~}\hfill {\zihao{3}\heiti 第二次线性代数} \hfill\hphantom{~~}
\addcontentsline{toc}{section}{\protect\numberline {}第二次线性代数}
%\chapter{第二次线性代数}


\hphantom{~~}

\stepcounter{chapter}
\EX 计算:
\begin{equation*}
-2
\begin{pmatrix}
a&1&a\\ 1&b&-1
\end{pmatrix}+3
\begin{pmatrix}
1&a&0\\ b&1&0
\end{pmatrix}
\end{equation*}

\begin{jie}
\begin{equation*}
\text{原式}=\begin{pmatrix}
-2a&-2&-2a\\ -2&-2b&2
\end{pmatrix}+\begin{pmatrix}
3&3a&0\\ 3b&3&0
\end{pmatrix}=\begin{pmatrix}
3-2a&3a-2&-2a\\ 3b-2&3-2b&2
\end{pmatrix}
\end{equation*}
\end{jie}

\EX 计算
\begin{equation*}
\begin{pmatrix}
-1&0&1\\ 1&1&-1
\end{pmatrix}\begin{pmatrix}
1&0&0\\ 0&-1&2\\ 0&0&1
\end{pmatrix}^{T}\begin{pmatrix}
1&-1\\ 0&1\\ 2&-2
\end{pmatrix}
\end{equation*}

\begin{jie}
令$A=\begin{pmatrix}
-1&0&1\\ 1&1&-1
\end{pmatrix}$,$B=\begin{pmatrix}
1&0&0\\ 0&-1&2\\ 0&0&1
\end{pmatrix}^{T}=\begin{pmatrix}
1&0&0\\ 0&-1&0\\ 0&2&1
\end{pmatrix}$,$C=\begin{pmatrix}
1&-1\\ 0&1\\ 2&-2
\end{pmatrix}$,则
\begin{equation*}
  AB=\begin{pmatrix}
-1&2&1\\ 1&-3&-1
\end{pmatrix}~~~
ABC=\begin{pmatrix}
1&1\\ -1&-2
\end{pmatrix}
\end{equation*}
\end{jie}

\EX 判断下列说法是否正确,并说明理由。

(3)设$A\neq 0$,且$AB=AC$。则$B=C$。

(4)设$A,B$都是$n\times n$矩阵,则
\begin{equation*}
  (A+B)(A-B)=A^2-B^2
\end{equation*}

(5)如果$m\times n$矩阵$A$满足$A\beta=0$,其中$\beta$是任意的$n\times 1$矩阵,则$A=0$.

\begin{jie}
(3)错误,矩阵的乘法没有消去律。例:$A=
\begin{pmatrix}
1 &1\\ 0&0
\end{pmatrix}
,B=
\begin{pmatrix}
3 &1\\ 2&4
\end{pmatrix},C=
\begin{pmatrix}
2&4\\ 3 &1
\end{pmatrix}
$

(4)错误,矩阵的乘法没有交换律,即$AB\neq BA$,例$A=\begin{pmatrix}
1 &1\\ 0&1
\end{pmatrix}
,B=\begin{pmatrix}
1 &1\\ 1&1
\end{pmatrix}
,$

(5)正确。对任意的矩阵$\beta=(b_1,b_2,\cdots,b_{n})^{T}$,设
$
\begin{pmatrix}
a_{11}&a_{12}&\cdots&a_{1n}\\
a_{21}&a_{22}&\cdots&a_{2n}\\
\vdots&\vdots&\ddots&\vdots\\
a_{m1}&a_{m2}&\cdots&a_{mn}
\end{pmatrix}
$,由题意,对于$A$的任意一行都有$a_{i1}b_1+a_{i2}b_2+\cdots+a_{in}b_n=0$,由于$b_i$取值的任意性,所以$a_{i1} = a_{i2}=\cdots=a_{in}0$,即$A=0$。
\end{jie}

\EX 设$\alpha$为3维列向量,$\alpha^T$是$\alpha$的转置,若$\alpha\alpha^T=
\begin{pmatrix}
1&-1&1\\
-1&1&-1\\
1&-1&1
\end{pmatrix}
$,则$\alpha^T\alpha=$\underline{\hphantom{~~~~~~~~~~~~~}}.

\begin{jie}
由题,设$\alpha=(x,y,z)^T$,依题意可列:
\begin{equation*}
\alpha\alpha^T=
\begin{pmatrix}
x\\ y\\z
\end{pmatrix}\begin{pmatrix}
x& y&z
\end{pmatrix}=\begin{pmatrix}
x^2 & xy & xz\\
yx & y^2 & yz\\
zx & zy & z^2
\end{pmatrix}=\begin{pmatrix}
1&-1&1\\
-1&1&-1\\
1&-1&1
\end{pmatrix}
\end{equation*}
所以:
\begin{equation*}
\alpha\alpha^T=\begin{pmatrix}
x& y&z
\end{pmatrix}\begin{pmatrix}
x\\ y\\z
\end{pmatrix}=x^2+y^2+z^2=1+1+1=3
\end{equation*}
\end{jie}

\EX 设$A$是$n\times n$矩阵,$tr(A)$表示$A$的全部$(i,i)-$元(此定义见课本15页定义1.3.1)的和。

(1)对任意$n\times n$矩阵$A,B$,证明
\begin{align*}
&tr(A+B)=tr(A)+tr(B)\\
&tr(kA)=k tr(A)\text{$k$是任意数}\\
&tr(AB)=tr(BA)
\end{align*}

(2)设$A$是$n\times n$实矩阵且$tr(A^TA)=0$,证明:$A=0$.

\begin{zhengming}
设$A=\begin{pmatrix}
a_{11}&a_{12}&\cdots&a_{1n}\\
a_{21}&a_{22}&\cdots&a_{2n}\\
\vdots&\vdots&\ddots&\vdots\\
a_{n1}&a_{n2}&\cdots&a_{nn}
\end{pmatrix},B=\begin{pmatrix}
b_{11}&b_{12}&\cdots&b_{1n}\\
b_{21}&b_{22}&\cdots&b_{2n}\\
\vdots&\vdots&\ddots&\vdots\\
b_{n1}&b_{n2}&\cdots&b_{nn}
\end{pmatrix}$.

所以:
\begin{equation*}
A+B=
\begin{pmatrix}
a_{11}+b_{11}&a_{12}+b_{12}&\cdots&a_{1n}+b_{1n}\\
a_{21}+b_{21}&a_{22}+b_{22}&\cdots&a_{2n}+b_{2n}\\
\vdots&\vdots&\ddots&\vdots\\
a_{n1}+b_{n1}&a_{n2}+b_{n2}&\cdots&a_{nn}+b_{nn}
\end{pmatrix}
\end{equation*}
所以:
\begin{equation*}
tr(A+B)=(a_{11}+b_{11})+(a_{22}+b_{22})+\cdots+(a_{nn}+b_{nn})=(a_{11}+a_{22}+\cdots+a_{nn})+(b_{11}+b_{22}+\cdots+b_{nn})=tr(A)+tr(B)
\end{equation*}

\begin{equation*}
  kA=\begin{pmatrix}
ka_{11}&ka_{12}&\cdots&ka_{1n}\\
ka_{21}&ka_{22}&\cdots&ka_{2n}\\
\vdots&\vdots&\ddots&\vdots\\
ka_{n1}&ka_{n2}&\cdots&ka_{nn}
\end{pmatrix}
\end{equation*}
所以:
\begin{equation*}
tr(kA)=ka_{11}+ka_{22}+\cdots+ka_{nn}=k(a_{11}+a_{22}+\cdots+a_{nn})=ktr(A)
\end{equation*}

\begin{equation*}
AB=
\begin{pmatrix}
a_{11}&a_{12}&\cdots&a_{1n}\\
a_{21}&a_{22}&\cdots&a_{2n}\\
\vdots&\vdots&\ddots&\vdots\\
a_{n1}&a_{n2}&\cdots&a_{nn}
\end{pmatrix}\begin{pmatrix}
b_{11}&b_{12}&\cdots&b_{1n}\\
b_{21}&b_{22}&\cdots&b_{2n}\\
\vdots&\vdots&\ddots&\vdots\\
b_{n1}&b_{n2}&\cdots&b_{nn}
\end{pmatrix}=
\begin{pmatrix}
\sum_{i=1}^{n}(a_{1i}b_{i1})&\sum_{i=1}^{n}(a_{1i}b_{i2})&\cdots&\sum_{i=1}^{n}(a_{1i}b_{in})\\
\sum_{i=1}^{n}(a_{2i}b_{i1})&\sum_{i=1}^{n}(a_{2i}b_{i2})&\cdots&\sum_{i=1}^{n}(a_{2i}b_{in})\\
\vdots&\vdots&\ddots&\vdots\\
\sum_{i=1}^{n}(a_{ni}b_{i1})&\sum_{i=1}^{n}(a_{ni}b_{i2})&\cdots&\sum_{i=1}^{n}(a_{ni}b_{in})\\
\end{pmatrix}
\end{equation*}

\begin{equation*}
BA=\begin{pmatrix}
b_{11}&b_{12}&\cdots&b_{1n}\\
b_{21}&b_{22}&\cdots&b_{2n}\\
\vdots&\vdots&\ddots&\vdots\\
b_{n1}&b_{n2}&\cdots&b_{nn}
\end{pmatrix}
\begin{pmatrix}
a_{11}&a_{12}&\cdots&a_{1n}\\
a_{21}&a_{22}&\cdots&a_{2n}\\
\vdots&\vdots&\ddots&\vdots\\
a_{n1}&a_{n2}&\cdots&a_{nn}
\end{pmatrix}=
\begin{pmatrix}
\sum_{i=1}^{n}(b_{1i}a_{i1})&\sum_{i=1}^{n}(b_{1i}a_{i2})&\cdots&\sum_{i=1}^{n}(b_{1i}a_{in})\\
\sum_{i=1}^{n}(b_{2i}a_{i1})&\sum_{i=1}^{n}(b_{2i}a_{i2})&\cdots&\sum_{i=1}^{n}(b_{2i}a_{in})\\
\vdots&\vdots&\ddots&\vdots\\
\sum_{i=1}^{n}(b_{ni}a_{i1})&\sum_{i=1}^{n}(b_{ni}a_{i2})&\cdots&\sum_{i=1}^{n}(b_{ni}aa_{in})\\
\end{pmatrix}
\end{equation*}

可以得出:$tr(AB)$是所有$a_{ij}b_{ij}$的和,同理$tr(BA)$也是所有$a_{ij}b_{ij}$的和,即$tr(AB)=tr(BA)$.

\begin{equation*}
A^{T}T=\begin{pmatrix}
a_{11}&a_{21}&\cdots&a_{n1}\\
a_{12}&a_{22}&\cdots&a_{n2}\\
\vdots&\vdots&\ddots&\vdots\\
a_{1n}&a_{2n}&\cdots&a_{nn}
\end{pmatrix}\begin{pmatrix}
a_{11}&a_{12}&\cdots&a_{1n}\\
a_{21}&a_{22}&\cdots&a_{2n}\\
\vdots&\vdots&\ddots&\vdots\\
a_{n1}&a_{n2}&\cdots&a_{nn}
\end{pmatrix}=\begin{pmatrix}
\sum_{i=1}^{n}a_{i1}^2&\sum_{i=1}^{n}a_{i1}a_{i2}&\cdots&\sum_{i=1}^{n}a_{i1}a_{in}\\
\sum_{i=1}^{n}a_{i2}a_{i1}&\sum_{i=1}^{n}a_{i2}^2&\cdots&\sum_{i=1}^{n}a_{i2}a_{in}\\
\vdots&\vdots&\ddots&\vdots\\
\sum_{i=1}^{n}a_{in}a_{i1}&\sum_{i=1}^{n}a_{in}a_{i2}&\cdots&\sum_{i=1}^{n}a_{in}^2
\end{pmatrix}
\end{equation*}
所以:$tr(A^{T}T)$是$A$每个元素平方的和,平方一定大于等于0,而$tr(A^{T}T)=0$,则$a_{ij}^2=0$,即$a_{ij}=0$,即$A=0$
\end{zhengming}

\EX 设
$
\begin{pmatrix}
1&0&-1\\ -1&1&1\\ 0&1&-1
\end{pmatrix}
$,计算$A^2-2A+3I_{3}$。

\begin{jie}
$A^2=\begin{pmatrix}
1&-1&0\\ -2&2&1\\ -1&0&2
\end{pmatrix}$所以:
\begin{align*}
A^2-2A+3I_{3}=&\begin{pmatrix}
1&-1&0\\ -2&2&1\\ -1&0&2
\end{pmatrix}-2\begin{pmatrix}
1&0&-1\\ -1&1&1\\ 0&1&-1
\end{pmatrix}+3\begin{pmatrix}
1&0&0\\ 0&1&0\\ 0&0&1
\end{pmatrix}\\
=&\begin{pmatrix}
2&-1&2\\ 0&3&-1\\ -1&-2&7
\end{pmatrix}
\end{align*}
\end{jie}

\EX 设
$
A=
\begin{pmatrix}
0&1&0\\ 0&0&1\\ 0&0&1
\end{pmatrix}
$,求所有与$A$可交换的矩阵。

\begin{jie}
设$X=
\begin{pmatrix}
a&b&c\\ d&e&f\\ g&h&i
\end{pmatrix}
$与$A$可交换,则:$AX=XA$,所以:
\begin{align*}
&AX=
\begin{pmatrix}
0&1&0\\ 0&0&1\\ 0&0&1
\end{pmatrix}\begin{pmatrix}
a&b&d\\ c&e&f\\ g&h&i
\end{pmatrix}=\begin{pmatrix}
d&e&f\\ g&h&i\\ 0&0&0
\end{pmatrix}\\
&XA=\begin{pmatrix}
a&b&d\\ c&e&f\\ g&h&i
\end{pmatrix}\begin{pmatrix}
0&1&0\\ 0&0&1\\ 0&0&1
\end{pmatrix}=\begin{pmatrix}
0&a&b\\ 0&d&e\\ 0&g&h
\end{pmatrix}
\end{align*}
$AX=XA$,即:
\begin{equation*}
\begin{pmatrix}
d&e&f\\ g&h&i\\ 0&0&0
\end{pmatrix}=\begin{pmatrix}
0&a&b\\ 0&d&e\\ 0&g&h
\end{pmatrix}~~~~\Rightarrow~~~~
\begin{cases}
d=0~~a=e\hphantom{=0}~~b=f\\
g=0~~h=d=0~~i=e=a\\
0=0~~g=0\hphantom{=0}~~h=0
\end{cases}
\end{equation*}
所以:$X=
\begin{pmatrix}
a&b&c\\ 0&a&b\\ 0&0&a
\end{pmatrix},\text{其中}a\in R,b\in R,c\in R
$。
\end{jie}

\EX 判断下列说法是否正确,并说明理由。

(1)如果$A$满足$A^2=I_{n}$,则$A=I_{n}$或$A=-I_{n}$。

(2)设$A,B$是$n$阶方阵,则
\begin{equation*}
(A-B)(A^2+AB+B^2)=A^3-B^3
\end{equation*}

\begin{jie}
(1)错误,例:$A=
\begin{pmatrix}
0&1 \\ 1&0
\end{pmatrix}
$

(2)错误。理由$(A-B)(A^2+AB+B^2)=A^3+A^2B+AB^2-BA^{2}-BAB-B^3$,由于矩阵乘法没有交换律,所以不能继续化简,即不等于$A^3-B^3$
\end{jie}

\EX 计算$2A-3B$,其中:
\begin{equation*}
A=
\begin{pmatrix}
1&2&2&4\\
0&1&0&2\\
1&4&4&0
\end{pmatrix},B=
\begin{pmatrix}
1&1&1&1\\
2&3&0&2\\
2&3&1&2
\end{pmatrix}
\end{equation*}

\begin{jie}
\begin{equation*}
2A-3B=2\begin{pmatrix}
1&2&2&4\\
0&1&0&2\\
1&4&4&0
\end{pmatrix}-3\begin{pmatrix}
1&1&1&1\\
2&3&0&2\\
2&3&1&2
\end{pmatrix}=
\begin{pmatrix}
-1&1&1&5\\
-6&-7&0&-2\\
-4&-1&5&-6
\end{pmatrix}
\end{equation*}
\end{jie}

\EX 已知
\begin{equation*}
A=
\begin{pmatrix}
-1 & 0 &-1\\ 1 &-4 &1
\end{pmatrix},B=\begin{pmatrix}
                  2&0\\ -1&-1\\ -1&0
                \end{pmatrix}
\end{equation*}
判断运算$AB,BA$是否有意义,若有意义,算出结果。

\begin{jie}
$A$的列数等于$B$的行数,所以$AB$有意义,$AB=
\begin{pmatrix}
-1& 0\\ 5&4
\end{pmatrix}
$。

$B$的列数等于$A$的行数,所以$BA$有意义,$BA=
\begin{pmatrix}
-2&0&-2\\ 0&4&0\\ 1&0&1
\end{pmatrix}
$。
\end{jie}

\EX 设$A_{3\times 2},B_{2\times 3},C_{3\times 3}$,则以下运算
\begin{equation*}
  AC,BC,ABC,AB-BC
\end{equation*}
哪些有意义?有意义时求出得到矩阵的大小。

\begin{jie}
$A$的列数不等于$C$的行数,$AC$无意义。

$B$的列数等于$C$的行数,$BC$有意义,新矩阵的大小为$2\times 3$

$A$的列数等于$B$的行数,所以$AB$有意义,该矩阵大小为$3\times 3$,其列数等于$C$的行数,所以$ABC$有意义,矩阵大小为$3\times 3$。

$AB$大小为$3\times 3$,$BC$大小为$2\times 3$,两矩阵规模不同,无法进行加减运算,即$AB-BC$无意义。
\end{jie}

\EX 下列计算哪些是错误的:
\begin{gather*}
k\begin{pmatrix}
   1&2\\ 3&4
 \end{pmatrix}=\begin{pmatrix}
   k&2k\\ 3k&4k
 \end{pmatrix}\\
\begin{pmatrix}
   1&2\\ 3&4
 \end{pmatrix}=2\begin{pmatrix}
   1&1\\ 3&2
 \end{pmatrix}\\
 \begin{pmatrix}
   1&2\\ 3&4
 \end{pmatrix}+
 \begin{pmatrix}
   0&0
 \end{pmatrix}=\begin{pmatrix}
   1&2\\ 3&4
 \end{pmatrix}\\
  \begin{pmatrix}
   1&2\\ 3&4
 \end{pmatrix}+
 \begin{pmatrix}
   0&0
 \end{pmatrix}=\begin{pmatrix}
   0&0\\ 0&0
 \end{pmatrix}
\end{gather*}

\begin{jie}
(1)正确

(2)错误,等式右边第一列错误

(3),(4)错误,不同规模的矩阵不能相加减
\end{jie}

\EX 设$
A=
\begin{pmatrix}
0&1&3\\
0&0&2\\
0&0&0
\end{pmatrix}
$,求$A^2=AA,A^3=A(A^2)$

\begin{jie}
\begin{align*}
&A^2=AA=\begin{pmatrix}
0&1&3\\
0&0&2\\
0&0&0
\end{pmatrix}\begin{pmatrix}
0&1&3\\
0&0&2\\
0&0&0
\end{pmatrix}=\begin{pmatrix}
0&0&2\\
0&0&0\\
0&0&0
\end{pmatrix}\\
&A^3=A(A^2)=\begin{pmatrix}
0&1&3\\
0&0&2\\
0&0&0
\end{pmatrix}\begin{pmatrix}
0&0&2\\
0&0&0\\
0&0&0
\end{pmatrix}=\begin{pmatrix}
0&0&0\\
0&0&0\\
0&0&0
\end{pmatrix}
\end{align*}
\end{jie}

\begin{tips}
由上可得出如下结论,若一个$n$阶方阵$A$的\textcolor[rgb]{1.00,0.00,0.00}{主对角线}及\textcolor[rgb]{1.00,0.00,0.00}{主对角线一侧}的元素\textcolor[rgb]{1.00,0.00,0.00}{全为0},那么一定有$A^{k}=0$,其中$k\geq n$,$n$为$A$的阶数。$A$形如以下两种形式:
\begin{equation*}
\begin{pmatrix}
0&a_{12}&\cdots&a_{1n}\\
0&0&\cdots&a_{1n}\\
\vdots&\vdots&\ddots&\vdots\\
0&0&\cdots&0
\end{pmatrix},~~~
\begin{pmatrix}
0&0&\cdots&0\\
a_{21}&0&\cdots&0\\
\vdots&\vdots&\ddots&\vdots\\
a_{n1}&a_{n2}&\cdots&0
\end{pmatrix}
\end{equation*}

对于次对角线,该结论并不成立,请自行举例验证。
\end{tips}

\EX 设$\alpha,\beta$都是三维列向量,$\beta^T$表示$\beta$的转置,如果$\alpha\beta^T=
\begin{pmatrix}
1&-2&-3\\
-2&4&6\\
-3&6&9
\end{pmatrix}
$,求$\alpha^T\beta$与$(\alpha^T\beta)^2$,$(\alpha\beta^T)^2$。

\begin{jie}
设$\alpha=
\begin{pmatrix}
a_1&a_2&a_3
\end{pmatrix}^T
,\beta=
\begin{pmatrix}
b_1&b_2&b_3
\end{pmatrix}^T$,由题得:
\begin{equation*}
\alpha\beta^T=\begin{pmatrix}
a_1\\ a_2\\ a_3
\end{pmatrix}\begin{pmatrix}
b_1&b_2&b_3
\end{pmatrix}=
\begin{pmatrix}
a_1b_1 & a_1b_2 & a_1b_3\\
a_2b_1 & a_2b_2 & a_2b_3\\
a_3b_1 & a_3b_2 & a_3b_3
\end{pmatrix}=\begin{pmatrix}
1&-2&-3\\
-2&4&6\\
-3&6&9
\end{pmatrix}
\end{equation*}
所以:
\begin{equation*}
\alpha^T\beta=\begin{pmatrix}
a_1&a_2&a_3
\end{pmatrix}\begin{pmatrix}
b_1\\ b_2\\ b_3
\end{pmatrix}=a_1b_1+a_2b_2+a_3b_3=1+4+9=14
\end{equation*}
所以:$(\alpha^T\beta)^2=14^2=196$
\begin{equation*}
\beta^T\alpha=\begin{pmatrix}
b_1&b_2&b_3
\end{pmatrix}\begin{pmatrix}
a_1\\ a_2\\ a_3
\end{pmatrix}=b_1a_1+b_2a_2+b_3a_3=\alpha^T\beta
\end{equation*}
所以:
\begin{equation*}
  (\alpha\beta^T)^2=\alpha\beta^T\cdot\alpha\beta^T=\alpha(\beta^T\cdot\alpha)\beta^T=\alpha\cdot14\cdot\beta^T=14\alpha\beta^T=14\begin{pmatrix}
1&-2&-3\\
-2&4&6\\
-3&6&9
\end{pmatrix}
\end{equation*}
\end{jie}

\begin{tips}
由此题得出如下结论:对于任意方阵$A$,若该方阵能被两个列向量表出,即$A=\alpha\beta^T$,(该方阵秩为1,即$r(A)=1$,秩后边会学,请先记住此结论,秩实际上是把矩阵化为阶梯型后非零行的数目)记$A$的对角线元素为$tr(A)$,则$tr(A)=\alpha^T\beta=\beta^T\alpha$(证明见下一个题).那么:

\begin{gather*}
A^2=\alpha\beta^T\alpha\beta^T=\alpha(\beta^T\cdot\alpha)\beta^T=\alpha\cdot tr(A)\cdot\beta^T=tr(A)\alpha\beta^T=tr(A)A=[tr(A)]^{2-1}A\\
A^3=A^2A=tr(A)AA=tr(A)A^2=[tr(A)][tr(A)]A=[tr(A)]^2A=[tr(A)]^{3-1}A\\
\vdots\\
A^n=[tr(A)]^{n-1}A
\end{gather*}

以后遇到某个方阵$A$求$m$($m$为任意的正整数)次方,若发现其秩为1,即$r(A)=1$,则直接使用此处的结论。
\end{tips}

\EX 设$\alpha,\beta$均是$n$维列向量,证明$\alpha^T\beta=\beta^T\alpha$。

\begin{zhengming}
设$\alpha=
\begin{pmatrix}
a_1&a_2&\cdots&a_n
\end{pmatrix}^T
,\beta=
\begin{pmatrix}
b_1&b_2&\cdots&b_n
\end{pmatrix}^T$,则:
\begin{gather*}
\alpha^T\beta=\begin{pmatrix}
a_1&a_2&\cdots&a_n
\end{pmatrix}\begin{pmatrix}
b_1\\ b_2\\ \vdots\\ b_n
\end{pmatrix}=a_1b_1+a_2b_2+\cdots+a_nb_n=\sum_{i=1}^{n}a_{i}b_{i}\\
\beta^T\alpha=\begin{pmatrix}
b_1&b_2&\cdots&b_n
\end{pmatrix}\begin{pmatrix}
a_1\\ a_2\\ \vdots\\ a_n
\end{pmatrix}=b_1a_1+b_2a_2+\cdots+b_na_n=\sum_{i=1}^{n}b_ia_i
\end{gather*}

所以:$\alpha^T\beta=\beta^T\alpha$。
\end{zhengming}
\clearpage
\hphantom{~~}\hfill {\zihao{3}\heiti 第三次线性代数} \hfill\hphantom{~~}
\addcontentsline{toc}{section}{\protect\numberline {}第三次线性代数}
%\section{第三次线性代数}
\setcounter{Emp}{14}


\hphantom{~~}

\EX 已知方程组
$
\begin{pmatrix}
1&2&1\\
2&3&a+2\\
1&a&-2
\end{pmatrix}
\begin{pmatrix}
x_1\\ x_2\\ x_3
\end{pmatrix}=
\begin{pmatrix}
1\\ 3\\ 0
\end{pmatrix}
$无解,则$a=$\underline{\hphantom{~~~~~~~~~}}。

\begin{jie}
由题得:
\begin{align*}
[A|B]=&
\left(
 \begin{array}{c:c}
\begin{matrix}
1 & 2 & 1\\
2 & 3 & a+2 \\
1 & a & -2
\end{matrix}&
\begin{matrix}
1  \\
3 \\
0
\end{matrix}
\end{array}
\right)\xrightarrow{\substack{r_{2}-2r_{1}\\ r_{3}-r_{1}}}
{
\left(
 \begin{array}{c:c}
\begin{matrix}
1 & 2 & 1\\
0 & -1 & a \\
0 & a-2 & -3
\end{matrix}&
\begin{matrix}
1  \\
1\\
-1
\end{matrix}
\end{array}
\right)
}\xrightarrow{\substack{ r_{3}+(a-2)r_{2}}}
{
\left(
 \begin{array}{c:c}
\begin{matrix}
1 & 2 & 1\\
0 & -1 & a \\
0 & 0 & (a+1)(a-3)
\end{matrix}&
\begin{matrix}
1  \\
1\\
a-3
\end{matrix}
\end{array}
\right)
}
\end{align*}
若无解,则$(a+1)(a-3)=0$且$a-3=0$,解得$a=-1$。
\end{jie}

\EX 设$
A=
\begin{pmatrix}
1&-1&0\\
0&1&-2\\
0&0&1
\end{pmatrix},B=
\begin{pmatrix}
-1&1\\
2&0\\
1&-3
\end{pmatrix}
$,求矩阵$X$,使得$AX=B$。

\begin{jie}
由题得增广矩阵:
\begin{align*}
[A|B]=&
\left(
 \begin{array}{c:c}
\begin{matrix}
1&-1&0\\
0&1&-2\\
0&0&1
\end{matrix}&
\begin{matrix}
-1&1\\
2&0\\
1&-3
\end{matrix}
\end{array}
\right)\xrightarrow{\substack{r_{2}+2 r_{3}}}
{
\left(
 \begin{array}{c:c}
\begin{matrix}
1&-1&0\\
0&1&0\\
0&0&1
\end{matrix}&
\begin{matrix}
-1&1\\
4&-6\\
1&-3
\end{matrix}
\end{array}
\right)
}\xrightarrow{\substack{r_{1}+ r_{2}}}
{
\left(
 \begin{array}{c:c}
\begin{matrix}
1&0&0\\
0&1&0\\
0&0&1
\end{matrix}&
\begin{matrix}
3&-5\\
4&-6\\
1&-3
\end{matrix}
\end{array}
\right)
}
\end{align*}
由最简阶梯型矩阵可以看出:
\begin{equation*}
X=\begin{pmatrix}
3&-5\\
4&-6\\
1&-3
\end{pmatrix}
\end{equation*}
\end{jie}

\EX 已知$X=AX+B$,其中,$A=
\begin{pmatrix}
0&1&0\\
-1&1&1\\
-1&0&-1
\end{pmatrix},B=
\begin{pmatrix}
1&-1\\
2&0\\
5&-3
\end{pmatrix}
$,求矩阵$X$。

\begin{jie}
由$X=AX+B$得$(E-A)X+B$,其中
\begin{equation*}
E-A=\begin{pmatrix}
1&0&0\\
0&1&0\\
0&0&1
\end{pmatrix}-\begin{pmatrix}
0&1&0\\
-1&1&1\\
-1&0&-1
\end{pmatrix}=
\begin{pmatrix}
1&-1&0\\
1&0&-1\\
1&0&2
\end{pmatrix}
\end{equation*}
增广矩阵:
\begin{align*}
(E-A|B)=&
\left(
 \begin{array}{c:c}
\begin{matrix}
1&-1&0\\
1&0&-1\\
1&0&2
\end{matrix}&
\begin{matrix}
1&-1\\
2&0\\
5&-3
\end{matrix}
\end{array}
\right)\xrightarrow{\substack{r_{2}- r_{1}\\ r_3-r_1}}
{
\left(
 \begin{array}{c:c}
\begin{matrix}
1&-1&0\\
0&1&-1\\
0&1&2
\end{matrix}&
\begin{matrix}
1&-1\\
1&1\\
4&-2
\end{matrix}
\end{array}
\right)
}\xrightarrow{\substack{r_3-r_2}}
{
\left(
 \begin{array}{c:c}
\begin{matrix}
1&-1&0\\
0&1&-1\\
0&0&3
\end{matrix}&
\begin{matrix}
1&-1\\
1&1\\
3&-3
\end{matrix}
\end{array}
\right)
}\\ &\xrightarrow{\substack{r_3\div 3}}
{
\left(
 \begin{array}{c:c}
\begin{matrix}
1&-1&0\\
0&1&-1\\
0&0&1
\end{matrix}&
\begin{matrix}
1&-1\\
1&1\\
1&-1
\end{matrix}
\end{array}
\right)
}\xrightarrow{\substack{r_2+r_3}}
{
\left(
 \begin{array}{c:c}
\begin{matrix}
1&-1&0\\
0&1&0\\
0&0&1
\end{matrix}&
\begin{matrix}
1&-1\\
2&0\\
1&-1
\end{matrix}
\end{array}
\right)
}\xrightarrow{\substack{r_1+r_2}}
{
\left(
 \begin{array}{c:c}
\begin{matrix}
1&0&0\\
0&1&0\\
0&0&1
\end{matrix}&
\begin{matrix}
3&-1\\
2&0\\
1&-1
\end{matrix}
\end{array}
\right)
}
\end{align*}
由最简阶梯型矩阵可以看出:
\begin{equation*}
 X=
 \begin{pmatrix}
3&-1\\
2&0\\
1&-1
 \end{pmatrix}
\end{equation*}
\end{jie}

\EX 判断下列说法是否正确,并说明理由。

(5)如果矩阵$A,B$可以相乘,则它们的分块矩阵也能够相乘。

\begin{jie}
(5)错误。理由:只有在$A$的列分法等于$B$的行分法时才可以相乘。
\end{jie}

\EX 用分块矩阵的乘法来表示$AB$的行向量和列向量。

\begin{jie}
把$A$按每一行分块得:$A=(\alpha_1,\alpha_2,\cdots,\alpha_n)^T$,将$B$分为一块,即$B=B$。

那么,$AB$的行向量表示法为:

\begin{equation*}
  AB=
  \begin{pmatrix}
    \alpha_1\\ \alpha_2\\ \vdots\\ \alpha_n
  \end{pmatrix}B=
  \begin{pmatrix}
    \alpha_1B\\ \alpha_2B\\ \vdots\\ \alpha_nB
  \end{pmatrix}
\end{equation*}

同理,把$A$分为一块$A=A$,把$B$按每一列分块:$B=(\beta_1,\beta_2,\cdots,\beta_n)$.

那么,$AB$的列向量表示法为:
\begin{equation*}
  AB=
  A\begin{pmatrix}
    \beta_1&\beta_2&\cdots&\beta_n
  \end{pmatrix}=
  \begin{pmatrix}
    A\beta_1&A\beta_2&\cdots&A\beta_n
  \end{pmatrix}
\end{equation*}
\end{jie}

\EX 若$
A=
\begin{bmatrix}
2&0&0&0\\
1&2&0&0\\
0&0&3&1\\
0&0&0&3
\end{bmatrix}
$,求$A^{10}$。

\begin{jie}
复习一下\textcolor[rgb]{1.00,0.00,0.00}{第一次习题课}讲的\textcolor[rgb]{1.00,0.00,0.00}{二项式}定理。

复习一下《第二次线性代数》作业答案\textcolor[rgb]{1.00,0.00,0.00}{例2.13}得出的\textcolor[rgb]{1.00,0.00,0.00}{结论}。

由题得:可将$A$按如下方式进行分块。
\begin{equation*}
A=
\begin{bmatrix}
A_{11}&0\\
0&A_{22}
\end{bmatrix}
\end{equation*}
式中:$A_{11}=
\begin{bmatrix}
2&0\\
1&2
\end{bmatrix}
,A_{22}=
\begin{bmatrix}
3&1\\
0&3
\end{bmatrix}$

所以$A^n=
\begin{bmatrix}
A_{11}^n&0\\
0&A_{22}^n
\end{bmatrix}
$.

$A_{11} =
\begin{bmatrix}
2&0\\
0&2
\end{bmatrix}+
\begin{bmatrix}
0&0\\
1&0
\end{bmatrix}=2E_{2}+B
$.($E_{2}$表示二阶单位阵).

由作业的例2.13的结论有:$B^{k}=0,k\geq 2$

由二项式定理:

\begin{align*}
A_{11}^{n}&=(2E+B)^n=C_{n}^{0}B^{0}(2E)^{n}+C_{n}^{1}B^{1}(2E)^{n-1}+C_{n}^{2}\textcolor[rgb]{1.00,0.00,0.00}{B^{2}}(2E)^{n-2}+\cdots+C_{n}^{n}\textcolor[rgb]{1.00,0.00,0.00}{B^{n}}(2E)^{0}\\
&=2^nE+n2^{n-1}B=
\begin{bmatrix}
2^n&0\\
n\cdot2^{n-1}&2^n
\end{bmatrix}
\end{align*}

同理:$A_{22}=3E+
\begin{bmatrix}
0&1\\
0&0
\end{bmatrix}=3E+C
$,$C^{k}=0,k\geq2$

\begin{equation*}
A_{22}^{n}=
\begin{bmatrix}
3^n&n\cdot3^{n-1}\\
0&3^n
\end{bmatrix}
\end{equation*}
所以:
\begin{equation*}
A^n=
\begin{bmatrix}
A_{11}^n&0\\
0&A_{22}^n
\end{bmatrix}=
\begin{bmatrix}
2^n&0&0&0\\
n\cdot2^{n-1}&2^n&0&0\\
0&0&3^n&n\cdot3^{n-1}\\
0&0&0&3^n
\end{bmatrix}
\end{equation*}
代入$n=10$得:\begin{equation*}
A^10=
\begin{bmatrix}
2^{10}&0&0&0\\
10\cdot2^{9}&2^{10}&0&0\\
0&0&3^{10}&10\cdot3^{9}\\
0&0&0&3^{10}
\end{bmatrix}\end{equation*}
\end{jie}

\EX 设3阶矩阵$A=(\alpha_1,\alpha_2,\alpha_3)$,令:
\begin{equation*}
B=(\alpha_1+\alpha_2+\alpha_3,\alpha_1+2\alpha_2+3\alpha_3,\alpha_1+4\alpha_2+9\alpha_3)
\end{equation*}
求$C$使得$AC=B$.

\begin{jie}
由题得:$B$的第一列$(\alpha_1+\alpha_2+\alpha_3)=(\alpha_1,\alpha_2,\alpha_3)
\begin{pmatrix}
1\\ 1\\ 1
\end{pmatrix}=A\begin{pmatrix}
1\\ 1\\ 1
\end{pmatrix}
$,同理$\alpha_1+2\alpha_2+3\alpha_3=A(1,2,3)^T,\alpha_1+4\alpha_2+9\alpha_3=A(1,4,9)^T$.

所以$C=
\begin{pmatrix}
1&1&1\\
1&2&4\\
1&3&9
\end{pmatrix}
$
\end{jie}

\EX 设$Q$是$n$阶矩阵,如果$Q^TQ=I$,我们称$Q$是正交矩阵,设$A=(\alpha_1,\alpha_2,\alpha_3)$是3阶正交阵,证明:
\begin{equation*}
\alpha_i\alpha_j=
\begin{cases}
1,~~i=j\\
0,~~i\neq j
\end{cases}
\end{equation*}

\begin{zhengming}
由题可设:
\begin{equation*}
A=
\begin{pmatrix}
\alpha_{11}&\alpha_{12}&\alpha_{13}\\
\alpha_{21}&\alpha_{22}&\alpha_{23}\\
\alpha_{31}&\alpha_{32}&\alpha_{33}
\end{pmatrix}=
\begin{pmatrix}
\alpha_1&\alpha_2&\alpha_3
\end{pmatrix}
\end{equation*}

所以:$A^T=(\alpha_1^T,\alpha_2^T,\alpha_3^T)^T$

$A$是正交矩阵,所以有
\begin{equation*}A^TA
\begin{pmatrix}
\alpha_1^T\\ \alpha_2^T\\ \alpha_3^T
\end{pmatrix}\begin{pmatrix}
\alpha_1&\alpha_2&\alpha_3
\end{pmatrix}=
\begin{pmatrix}
\alpha_1^T\alpha_1&\alpha_1^T\alpha_2&\alpha_1^T\alpha_3\\
\alpha_2^T\alpha_1&\alpha_2^T\alpha_2&\alpha_2^T\alpha_3\\
\alpha_3^T\alpha_1&\alpha_3^T\alpha_2&\alpha_3^T\alpha_3\\
\end{pmatrix}=E_3=
\begin{pmatrix}
1&0&0\\
0&1&0\\
0&0&1
\end{pmatrix}
\end{equation*}
由上述等式可以看出:$i=j$时,$\alpha_i^T\alpha_j=1$,$i\neq j$时,$\alpha_i^T\alpha_j=0$证毕。
\end{zhengming}


\clearpage
\hphantom{~~}\hfill {\zihao{3}\heiti 第四次线性代数} \hfill\hphantom{~~}
\addcontentsline{toc}{section}{\protect\numberline {}第四次线性代数}

\hphantom{~~}

%\section{第四次线性代数}
\EX 用初等变换把$
A=
\begin{pmatrix}
0&0&0&3\\
0&0&0&1\\
3&5&0&0\\
1&1&0&0
\end{pmatrix}
$化为标准型矩阵。

\begin{jie}
\begin{align*}
A&\xrightarrow{\substack{r_{1}\leftrightarrow r_{4}}}
{
\begin{pmatrix}
1&1&0&0\\
0&0&0&1\\
3&5&0&0\\
0&0&0&3
\end{pmatrix}
}\xrightarrow{\substack{r_{3}-3r_{1}}}
{
\begin{pmatrix}
1&1&0&0\\
0&0&0&1\\
0&2&0&0\\
0&0&0&3
\end{pmatrix}
}\xrightarrow{\substack{r_{2}\leftrightarrow r_{3}}}
{
\begin{pmatrix}
1&1&0&0\\
0&2&0&0\\
0&0&0&1\\
0&0&0&3
\end{pmatrix}
}\\ &\xrightarrow{\substack{r_{4}-3r_{3}}}
{
\begin{pmatrix}
1&1&0&0\\
0&2&0&0\\
0&0&0&1\\
0&0&0&0
\end{pmatrix}
}\xrightarrow{\substack{r_{2}\times \frac{1}{2}}}
{
\begin{pmatrix}
1&1&0&0\\
0&1&0&0\\
0&0&0&1\\
0&0&0&0
\end{pmatrix}
}\xrightarrow{\substack{r_{1}-r_{2}}}
{
\begin{pmatrix}
1&0&0&0\\
0&1&0&0\\
0&0&0&1\\
0&0&0&0
\end{pmatrix}
}\\ &\xrightarrow{\substack{c_{3}\leftrightarrow c_{4}}}
{
\begin{pmatrix}
1&0&0&0\\
0&1&0&0\\
0&0&1&0\\
0&0&0&0
\end{pmatrix}
}
\end{align*}
\end{jie}

\EX 判断下列说法是否正确,并说明理由。

(1)设$A,B$都是$2\times3$矩阵且$r(A)=r(B)=2$,则$A+B$的秩可能为$2+2=4$。

(2)设$A=
\begin{pmatrix}
1&2&-1\\
0&0&0\\
2&1&1
\end{pmatrix},B=
\begin{pmatrix}
1&0\\
0&-1\\
0&0
\end{pmatrix}
$,则不存在矩阵$X$使得$AX=B$。

\begin{jie}
(1)错误。理由:因为$A,B$都是$2\times3$矩阵,所以$A+B$也是$2\times3$矩阵,根据矩阵的性质,$r(A+B)\leq \min\{2,3\}=2$,不可能为4.

(2)正确。理由:计算可得到$r(A)\neq r(A,B)$,所以$AX=B$的解不存在,即不存在矩阵$X$使得$AX=B$。
\end{jie}

\EX 设$A=
\begin{pmatrix}
-1&3&5\\
2&-4&7\\
1&-1&12
\end{pmatrix}
$.问,下面的矩阵中,哪些是与$A$相抵的?说明理由。

\begin{gather*}
B_1=
\begin{pmatrix}
-1&3&2\\ 0&1&2
\end{pmatrix},B_2=
\begin{pmatrix}
6&7&8\\
3&-1&3\\
3&8&5
\end{pmatrix}\\
B_3=
\begin{pmatrix}
1&0&0\\ 0&0&0\\ 0&0&1
\end{pmatrix},B_4=
\begin{pmatrix}
1&2&0\\
0&1&0\\
0&0&0
\end{pmatrix}
\end{gather*}

\begin{jie}
\textcolor[rgb]{1.00,0.00,0.00}{矩阵相抵:课本100页定理3.2.3}.

相抵首先要求同型(即两矩阵得行数列数相同),所以排出$B_1$。

相抵矩阵的秩相等。

\begin{align*}
&A\xrightarrow{\substack{r_{2}+2r_1\\ r_3+r_1}}
{
\begin{pmatrix}
-1&3&5\\
0&2&17\\
0&2&17
\end{pmatrix}
}\xrightarrow{\substack{r_{3}-r_2}}
{
\begin{pmatrix}
-1&3&5\\
0&2&17\\
0&0&0
\end{pmatrix}
}~~~~\Rightarrow~~~r(A)=2\\
&B_2\xrightarrow{\substack{r_{1}\leftrightarrow r_2}}
{
\begin{pmatrix}
3&-1&3\\
6&7&8\\
3&8&5
\end{pmatrix}
}\xrightarrow{\substack{r_{2}-2 r_1\\ r_3-r_1}}
{
\begin{pmatrix}
3&-1&3\\
0&9&2\\
0&9&2
\end{pmatrix}
}\xrightarrow{\substack{r_{3}-r_2}}
{
\begin{pmatrix}
3&-1&3\\
0&9&2\\
0&0&0
\end{pmatrix}
}~~~~\Rightarrow~~~r(B_2)=2
\end{align*}

由题可直接看出:$r(B_3)=r(B_4)=2$,所以$A$与$B_2,B_3,B_4$相抵。
\end{jie}

\EX 设$n$阶方阵满足$A^2-6A+5I_{n}=0$,证明:
\begin{equation*}
  r(A-5I_{n})+r(A-I_{n})=n.
\end{equation*}

\begin{tips}
矩阵秩的性质(很重要):
\begin{asparaenum}[(1)]
\item 矩阵的秩$\leq$ 矩阵的行数与列数的最小值。即:$R(A_{m\times n})\leq \min\{m,n\}$。
\item 转置不改变矩阵的秩。即$R(A)=R(A^{T})$。
\item 等价矩阵的秩相同。即$A\~{}B$,则$R(A)=R(B)$。
\item $A,B$是同型矩阵(行数和列数相同),$R(A)=R(B)$当且仅当$A\~{}B$。
\item 子矩阵的秩$\leq$矩阵的秩,矩阵的秩$\leq$所有子矩阵的秩之和。
即:$\max\{R(A),R(B)\}\leq R(A,B)\leq R(A)+R(B)$。
\item 矩阵之和的秩$\leq$矩阵的秩之和。即:$R(A+B)\leq R(A)+R(B)$。
\item 矩阵乘积的秩$\leq$乘积因子的秩之最小值。即:$R(AB)\leq \min\{R(A),R(B)\}$。
\item 若$A_{m\times n}B_{n\times l}=0$,则$R(A)+R(B)\leq n$。
\item 左乘或右乘可逆矩阵秩不变。(左乘列满秩矩阵不改变矩阵的秩,右乘行满秩矩阵不改变矩阵的秩。)
\end{asparaenum}
\hphantom{`}
\end{tips}

\begin{zhengming}
由题得:$A^2-6A+5I=(A-5I)(A-I)=0$

所以由第8条性质:$r(A-5I)+r(A-I)\leq n$。

由第6条性质:$r((A-5I)-(A-I))\leq r(A-5I)+r(-(A-I))$。

数乘不改变矩阵的秩:$r(A)=r(kA),k\leq 0$。所以
\begin{equation*}
r((A-5I)-(A-I))=r(-4I)=r(I)=n\leq r(A-5I)+r(-(A-I))=r(A-5I)+r(A-I)\leq n
\end{equation*}
$n\leq r(A-5I)+r(A-I)\leq n$所以$r(A-5I)+r(A-I)$只能等于$n$。
\end{zhengming}

\EX 判断下列说法是否正确,并说明理由。

(3) 准上三角阵
$\begin{pmatrix}
  A & B \\ 0&C
 \end{pmatrix}
$的秩等于$r(A)+r(C)$
。

\begin{jie}
错误。例如:$A=
\begin{pmatrix}
1&0\\0&0
\end{pmatrix},B=
\begin{pmatrix}
1&0\\0&1
\end{pmatrix},C=
\begin{pmatrix}
1&0\\0&0
\end{pmatrix}
$,则$r(A)=1,r(C)=1,$而准上三角阵的秩为3,不等于$r(A)+r(C)$。准三角阵的秩应该等于$\max\{r(A),r(B)\}+r(C)$。
\end{jie}

\EX 设$A$是$4\times 3$矩阵,且$A$的秩为$r(A)=2$,而$B=
\begin{pmatrix}
1&0&2\\
0&2&0\\
-1&0&3
\end{pmatrix}
$则$r(AB)=$\underline{\hphantom{~~~~~~~}}。

\begin{jie}
计算得$r(B)=3$,所以$B$是一个行满秩矩阵,因为$B$是方阵,所以也是列满秩矩阵。由例2.26中给的性质9有$r(AB)=r(A)=2$
\end{jie}

\EX 求矩阵
$
\begin{pmatrix}
0&2&0&1\\ -1&3&-1&2\\ 2&-10&2&-6\\ 1&-7&1&5
\end{pmatrix}
$的秩。

\begin{jie}
由题得:
\begin{align*}
A&\xrightarrow{\substack{r_{1}\leftrightarrow r_2}}{
\begin{pmatrix}
-1&3&-1&2\\ 0&2&0&1\\ 2&-10&2&-6\\ 1&-7&1&5
\end{pmatrix}
}\xrightarrow{\substack{r_{3}+2r_1 \\ r_4+r_1}}{
\begin{pmatrix}
-1&3&-1&2\\ 0&2&0&1\\ 0&-4&0&-2\\ 0&-4&0&7
\end{pmatrix}
}\xrightarrow{\substack{r_{3}+2r_2 \\ r_4+2r_2}}{
\begin{pmatrix}
-1&3&-1&2\\ 0&2&0&1\\ 0&0&0&0\\ 0&0&0&9
\end{pmatrix}
}\\
&\xrightarrow{\substack{r_{3}\leftrightarrow r_4}}{
\begin{pmatrix}
-1&3&-1&2\\ 0&2&0&1\\  0&0&0&9\\0&0&0&0
\end{pmatrix}
}
\end{align*}
所以$r(A)=3$.
\end{jie}

\EX 设矩阵$A$的每个$(i,j)-$元都是同一个数$a$。求$r(A)$。

\begin{jie}
(1) $a=0$时,矩阵$A$为零矩阵,$r(A)=0$.

(2)$a\neq 0$时,矩阵$A$的其他行减去第一行后剩余一行,$r(A)=1$。
\end{jie}

\EX 设线性方程组的系数矩阵为$A$,增广矩阵为$\widetilde{A}$。证明:$r(A)=r(\widetilde{A})$,或者$r(A)=r(\widetilde{A})-1$

\begin{zhengming}
$A$是$\widetilde{A}$的前$n$列构成的子矩阵,所以$r(\widetilde{A})\leq r(A)+1$;因此,如果$r(A)\neq r(\widetilde{A})$,则必然有$r(A)=r(\widetilde{A})-1$。
\end{zhengming}

\EX 设$A$是任意的$m\times n$型矩阵。任意取定$A$的$m'$行和$n'$列,设$B$是由$A$的这些行和这些列的交叉处的元按在$A$中的位置而构成的$m'\times n'$矩阵。证明:$r(B)<r(A)$。

\begin{zhengming}
任取$B$的列向量组的一个极大线性无关组$(1)$,该子组含有$r(B)$个列向量,这$r(B)$个列向量在$A$中的列向量所构成的向量组$(2)$是$(1)$的一个伸长组,从而由(1)线性无关可以得到$(2)$线性无关。即$A$的列向量中至少有$r(B)$个是线性无关的,所以$r(A)\geq r(B)$。
\end{zhengming}

\EX 设$a_{ij}$是常数。设有齐次线性方程组
\begin{equation*}
  (\uppercase\expandafter{\romannumeral1})
\begin{cases}
a_{11}x_1+a_{12}x_2+a_{13}a_3+a_{14}x_{4}=0\\
a_{21}x_1+a_{22}x_2+a_{23}a_3+a_{24}x_{4}=0\\
a_{31}x_1+a_{32}x_2+a_{33}a_3+a_{34}x_{4}=0\\
a_{41}x_1+a_{42}x_2+a_{43}a_3+a_{44}x_{4}=0
\end{cases}
\end{equation*}
和\begin{equation*}
  (\uppercase\expandafter{\romannumeral2})
\begin{cases}
a_{11}x_1+a_{21}x_2+a_{31}a_3+a_{41}x_{4}=0\\
a_{12}x_1+a_{22}x_2+a_{32}a_3+a_{42}x_{4}=0\\
a_{13}x_1+a_{23}x_2+a_{33}a_3+a_{43}x_{4}=0\\
a_{14}x_1+a_{24}x_2+a_{34}a_3+a_{44}x_{4}=0
\end{cases}
\end{equation*}
证明:(\uppercase\expandafter{\romannumeral1})有非零解$\Leftrightarrow$(\uppercase\expandafter{\romannumeral2})有非零解.

\begin{zhengming}
记\uppercase\expandafter{\romannumeral1}的系数矩阵为$A$,\uppercase\expandafter{\romannumeral2}的系数矩阵为$B$,则$A$的列向量组为$B$的行向量组,所以$r(A)=r(B)$。又因为\uppercase\expandafter{\romannumeral1}和\uppercase\expandafter{\romannumeral2}的未知个数都是4,所以有\uppercase\expandafter{\romannumeral1}有非零解$\Leftrightarrow r(A)<4\Leftrightarrow r(B)<4 \Leftrightarrow$\uppercase\expandafter{\romannumeral2}有非零解.
\end{zhengming}

\EX 判断方程组
\begin{equation*}
\begin{cases}
2x_1-x_2+4x_3-3x_4=-4\\
x_1+x_3-x_4=-3\\
3x_1+x_2+x_3=1\\
7x_1+7x_3-3x_4=3
\end{cases}
\end{equation*}
是否有解;如果有解,是唯一解还是无穷多解。

\begin{jie}
由题得增广矩阵:
\begin{align*}
\widetilde{A}&=
\begin{pmatrix}
2&-1&4&-3&-4\\
1&0&1&-1&-3\\
3&1&1&0&1\\
7&0&7&-3&3
\end{pmatrix}\xrightarrow{\substack{r_1 \Leftrightarrow r_2}}{
\begin{pmatrix}
1&0&1&-1&-3\\
2&-1&4&-3&-4\\
3&1&1&0&1\\
7&0&7&-3&3
\end{pmatrix}
}\xrightarrow{\substack{r_2-2r_1\\ r_3-3r_1\\ r_4-7r_1}}{
\begin{pmatrix}
1&0&1&-1&-3\\
0&-1&2&-1&2\\
0&1&-2&3&10\\
0&0&0&4&24
\end{pmatrix}
}\\
&\xrightarrow{\substack{r_3+r_2}}{
\begin{pmatrix}
1&0&1&-1&-3\\
0&-1&2&-1&2\\
0&0&0&2&12\\
0&0&0&4&24
\end{pmatrix}
}\xrightarrow{\substack{r_4-2r_3}}{
\begin{pmatrix}
1&0&1&-1&-3\\
0&-1&2&-1&2\\
0&0&0&2&12\\
0&0&0&0&0
\end{pmatrix}
}
\end{align*}
由阶梯型矩阵可以看出:$r(A)=r(\widetilde{A})<4$,所以原方程有无穷多解。
\end{jie}

\EX 判断方程组
\begin{equation*}
\begin{cases}
x_1+2x_2+2x_3=0\\
x_1+x_2+x_3=0\\
x_1+5x_2+5x_3=0\\
\end{cases}
\end{equation*}
是否只有零解。

\begin{jie}
由题得系数矩阵:
\begin{equation*}
A=
\begin{pmatrix}
1&2&2\\
1&1&1\\
1&5&5
\end{pmatrix}
\xrightarrow{\substack{r_2-r_1\\ r_3-r_1}}{
\begin{pmatrix}
1&2&2\\
0&-1&-1\\
0&3&3
\end{pmatrix}
}\xrightarrow{\substack{r_3+3r_2}}{
\begin{pmatrix}
1&2&2\\
0&-1&-1\\
0&0&0
\end{pmatrix}
}
\end{equation*}
由阶梯型矩阵可以看出:$r(A)=2<3$,有无穷多解。
\end{jie}

\EX 已知$\lambda$为常数,线性方程组
\begin{equation*}
\begin{cases}
x_1+x_2+x_3-x_4=2\\
3x_1+x_2-x_3+2x_4=3\\
2x_1+2\lambda x_2-2x_3+3x_4=\lambda
\end{cases}
\end{equation*}
在$\lambda$为何值时无解,有唯一解,有无穷多个解?

\begin{jie}
(1)由于方程的个数少于未知数的个数,所以不存在有唯一解的情况。

(2)由题列增广矩阵:
\begin{equation*}
\widetilde{A}=
\begin{pmatrix}
1&1&1&-1&2\\
3&1&-1&2&3\\
2&2\lambda&-2&3&\lambda
\end{pmatrix}
\xrightarrow{\substack{r_2-3r_1\\ r_3-2r_1}}{
\begin{pmatrix}
1&1&1&-1&2\\
0&-2&-4&5&-3\\
0&2\lambda-2&-4&5&\lambda-4
\end{pmatrix}
}\xrightarrow{\substack{r_3-r_2}}{
\begin{pmatrix}
1&1&1&-1&2\\
0&-2&-4&5&-3\\
0&2\lambda&0&0&\lambda-1
\end{pmatrix}
}
\end{equation*}

由阶梯型矩阵可以看出:

$\lambda = 0$时,$r(A)=2,r(\widetilde{A})=3,r(A)\neq r(\widetilde{A})$,即此时无解。

$\lambda\neq 0$时,$r(A)=r(\widetilde{A})=3<4$,此时线性方程组有无穷多解。

综上所诉:该方程组不存在唯一解的情况,$\lambda=0$时,线性方程组无解,$\lambda\neq 0$时,线性方程组有无穷解。
\end{jie}

\EX 设一个非齐次方程组和一个齐次线性方程组的系数矩阵都是矩阵$A$.

(1)证明:如果该非齐次方程组有且仅有一个解,则该齐次方程组只有零解。

(2)如果该齐次方程组有无穷多个解,那么该非齐次方程组是否也有无穷多个解?说明理由。

\begin{zhengming}
记该非齐次线性方程组为$(\uppercase\expandafter{\romannumeral1})$,未知数个数为$n$,记$(\uppercase\expandafter{\romannumeral1})$的增广矩阵为$\widetilde{A}$,记以$A$为系数矩阵的齐次线性方程组为$(\uppercase\expandafter{\romannumeral2})$.

(1)若(\uppercase\expandafter{\romannumeral1})有且仅有一个解,则$r(A)=r(\widetilde{A})=n$,从而(\uppercase\expandafter{\romannumeral2})只有零解。

(2)若(\uppercase\expandafter{\romannumeral2})有无穷多解,则$r(A)<n$,但$r(A)$不一定与$r(\widetilde{A})$相等。因此,(\uppercase\expandafter{\romannumeral1})可能无解,可能有解,且在有解时会有无穷多解。
\end{zhengming}

\EX 矩阵方程$AX = B$什么时候无解?什么时候有解?有解的时候,什么时候有唯一的一组解,什么时候有无穷多组解?

\begin{jie}
设该矩阵方程有n个未知数。

(1)若$B$为0矩阵,则一定有解。
\begin{itemize}
\item $r(A)=n$,只有零解。
\item $r(A)<n$,有无穷解
\end{itemize}

(2)若$B$不为0矩阵。
\begin{itemize}
  \item $r(A)\neq r(A,B)$,无解
  \item $r(A)= r(A,B)=n$,唯一解
  \item $r(A)= r(A,B)<n$,无穷解
\end{itemize}
\end{jie}
\clearpage
\hphantom{~~}\hfill {\zihao{3}\heiti 第五次线性代数} \hfill\hphantom{~~}
\addcontentsline{toc}{section}{\protect\numberline {}第五次线性代数}

\hphantom{~~}

\setcounter{Emp}{39}
\EX 设$
B=
\begin{pmatrix}
1&0&0\\
0&0&0\\
0&0&-1
\end{pmatrix},
P=
\begin{pmatrix}
1&0&0\\
2&-1&0\\
2&1&1
\end{pmatrix},$又已知$AP=PB$,求矩阵$A,A^5$。

\begin{jie}
由题可列:
\begin{align*}
[P|E]=&
\left(
 \begin{array}{c:c}
\begin{matrix}
1&0&0\\
2&-1&0\\
2&1&1
\end{matrix}&
\begin{matrix}
1&0&0\\
0&1&0\\
0&0&1
\end{matrix}
\end{array}
\right)\xrightarrow{\substack{r_{2}-2 r_{1}\\ r_3-2r_1}}
{
\left(
 \begin{array}{c:c}
\begin{matrix}
1&0&0\\
0&-1&0\\
0&1&1
\end{matrix}&
\begin{matrix}
1&0&0\\
-2&1&0\\
-2&0&1
\end{matrix}
\end{array}
\right)
}\xrightarrow{\substack{r_{3}+r_2}}
{
\left(
 \begin{array}{c:c}
\begin{matrix}
1&0&0\\
0&-1&0\\
0&0&1
\end{matrix}&
\begin{matrix}
1&0&0\\
-2&1&0\\
-4&1&1
\end{matrix}
\end{array}
\right)
}\\
&\xrightarrow{\substack{r_{2}\times(-1)}}
{
\left(
 \begin{array}{c:c}
\begin{matrix}
1&0&0\\
0&1&0\\
0&0&1
\end{matrix}&
\begin{matrix}
1&0&0\\
2&-1&0\\
-4&1&1
\end{matrix}
\end{array}
\right)
}
\end{align*}
所以$P$可逆,且$P^{-1}=
\begin{pmatrix}
1&0&0\\
2&-1&0\\
-4&1&1
\end{pmatrix}
$。

因为$AP=PB$,所以等式两边同时右乘$P^{-1}$得$A=PBP^{-1}$

\begin{align*}
A=PBP^{-1}=
\begin{pmatrix}
1&0&0\\
2&-1&0\\
2&1&1
\end{pmatrix}
\begin{pmatrix}
1&0&0\\
0&0&0\\
0&0&-1
\end{pmatrix}
\begin{pmatrix}
1&0&0\\
2&-1&0\\
-4&1&1
\end{pmatrix}=
\begin{pmatrix}
1&0&0\\
2&0&0\\
2&0&-1
\end{pmatrix}
\begin{pmatrix}
1&0&0\\
2&-1&0\\
-4&1&1
\end{pmatrix}=
\begin{pmatrix}
1&0&0\\
2&0&0\\
6&-1&-1
\end{pmatrix}
\end{align*}

$A^2=
(PBP^{-1})(PBP^{-1})=PB\textcolor[rgb]{1.00,0.00,0.00}{(P^{-1}P)}BP^{-1}=PB^2P^{-1}
$,同理可以推出$A^n=PB^nP^{-1}$.

因为$B$为对角阵,所以$B^5
=\begin{pmatrix}
1^5&0&0\\
0&0&0\\
0&0&(-1)^5
 \end{pmatrix}=
 \begin{pmatrix}
 1&0&0\\
0&0&0\\
0&0&-1
 \end{pmatrix}=B
$,所以$A^5=PB^5P^{-1}=PBP^{-1}=A$。
\end{jie}

\EX 设$M=
\begin{pmatrix}
A&0\\
C&B
\end{pmatrix}
$是准下三角阵,证明:$M$可逆$\Leftrightarrow A$和$B$都可逆,并求$M^{-1}$。

\begin{zhengming}
详见课本101页例3.2.23.
\end{zhengming}

\EX 判断矩阵$
A=
\begin{pmatrix}
0&2&-1\\
1&-3&2\\
1&-1&2
\end{pmatrix}
$是否可逆,如果可逆,求$A^{-1}$。

\begin{jie}
由题得:
\begin{align*}
[A|E]=&
\left(
 \begin{array}{c:c}
\begin{matrix}
0&2&-1\\
1&-3&2\\
1&-1&2
\end{matrix}&
\begin{matrix}
1&0&0\\
0&1&0\\
0&0&1
\end{matrix}
\end{array}
\right)\xrightarrow{\substack{r_{1}\Leftrightarrow r_2}}
{
\left(
 \begin{array}{c:c}
\begin{matrix}
1&-3&2\\
0&2&-1\\
1&-1&2
\end{matrix}&
\begin{matrix}
0&1&0\\
1&0&0\\
0&0&1
\end{matrix}
\end{array}
\right)
}\xrightarrow{\substack{r_{3}-r_1}}
{
\left(
 \begin{array}{c:c}
\begin{matrix}
1&-3&2\\
0&2&-1\\
0&2&0
\end{matrix}&
\begin{matrix}
0&1&0\\
1&0&0\\
0&-1&1
\end{matrix}
\end{array}
\right)
}\\
&\xrightarrow{\substack{r_{3}-r_2}}
{
\left(
 \begin{array}{c:c}
\begin{matrix}
1&-3&2\\
0&2&-1\\
0&0&1
\end{matrix}&
\begin{matrix}
0&1&0\\
1&0&0\\
-1&-1&1
\end{matrix}
\end{array}
\right)
}\xrightarrow{\substack{r_{2}+r_3\\ r_1-2r_3}}
{
\left(
 \begin{array}{c:c}
\begin{matrix}
1&-3&0\\
0&2&0\\
0&0&1
\end{matrix}&
\begin{matrix}
2&3&-2\\
0&-1&1\\
-1&-1&1
\end{matrix}
\end{array}
\right)
}\\
&\xrightarrow{\substack{r_{2}\times\frac{1}{2}}}
{
\left(
 \begin{array}{c:c}
\begin{matrix}
1&-3&0\\
0&1&0\\
0&0&1
\end{matrix}&
\begin{matrix}
2&3&-2\\
0&-\frac{1}{2}&\frac{1}{2}\\
-1&-1&1
\end{matrix}
\end{array}
\right)
}\xrightarrow{\substack{r_{1}+3r_2}}
{
\left(
 \begin{array}{c:c}
\begin{matrix}
1&0&0\\
0&1&0\\
0&0&1
\end{matrix}&
\begin{matrix}
2&\frac{3}{2}&-\frac{1}{2}\\
0&-\frac{1}{2}&\frac{1}{2}\\
-1&-1&1
\end{matrix}
\end{array}
\right)
}
\end{align*}
所以$A$可逆,且$A^{-1}=
\begin{pmatrix}
2&\frac{3}{2}&-\frac{1}{2}\\
0&-\frac{1}{2}&\frac{1}{2}\\
-1&-1&1
\end{pmatrix}
$
\end{jie}

\EX 设$n$阶方阵$A$满足$A^3=0$,证明$A-2I$可逆。

\begin{zhengming}
由立方差公式得:
\begin{align*}
(A-2I)(A^2+2A+4I)=A^3-8I=-8I
\end{align*}

所以$A-2I$可逆,且$(A-2I)^{-1}=-\dfrac{A^2+2A+4I}{8}$
\end{zhengming}

\EX 设$n$阶方阵$A$不是数量阵且满足$A^2-5A+6I_n=0$,$a$是数。证明$A-aI_n$可逆$\Leftrightarrow$ $a\neq 2$且$a\neq3$。

\begin{zhengming}
$\Rightarrow$:$A-aI_n$可逆。假设$a=2$,则$A-2I_n$可逆,由$A^2-5A+6I_n=0$得$(A-2I_n)(A-3I_{n})=0$,$A-2I_n$可逆,所以等式两边同时左乘$(A-2I_n)$得:$A-3I_{n}=0$,即$A=3I_{n}$,所以$A$是数量阵,与题目中的$A$不是数量阵矛盾,即假设不成立,所以$a\neq 2$,同理可以证明$a\neq 3$。

$\Leftarrow$:$a\neq 2$且$a\neq3$。假设$A-aI_n$不可逆,则齐次线性方程组$(A-aI_n)x=0$有非零解。任取一个非零解$\alpha$,则$(A-aI_n)\alpha=0$,即$A\alpha=a\alpha,A^2\alpha=aA\alpha=a\cdot a\alpha=a^2\alpha$,所以$(A^2-5A+6I_{n})\alpha=(a^2-5a+6)\alpha=0$。因为$\alpha$是非零解,所以$\alpha\neq 0$,所以$a^2-5a+6=0$,即$a=2$或$a=3$。与$a\neq 2$且$a\neq 3$矛盾,所以假设不成立,即$A-aI_n$可逆。
\end{zhengming}

\EX 判断下列说法是否正确,并说明理由。

(4)设$A,B$是$n$阶方阵,则$A,B$可交换$\Leftrightarrow A^{-1},B^{-1}$可交换。

\begin{jie}
(4)正确。理由:

$\Rightarrow$:$A,B$可交换,则有$AB=BA$,两边同时求逆,即$(AB)^{-1}=(BA)^{-1}$,由可逆矩阵的性质$(AB)^{-1}=\textcolor[rgb]{1.00,0.00,0.00}{B^{-1}A^{-1}}=(BA)^{-1}=\textcolor[rgb]{1.00,0.00,0.00}{A^{-1}B^{-1}}$。即$A^{-1},B^{-1}$可交换。

$\Leftarrow$:思路同$\Rightarrow$,过程略。
\end{jie}

\EX 设3阶方阵$A,B$满足$A^{-1}BA=6A+BA$,且$A
\begin{pmatrix}
\frac{1}{3}&0&0\\
0&\frac{1}{4}&0\\
0&0&\frac{1}{7}
\end{pmatrix}
$,则矩阵$B=$\underline{\hphantom{~~~~~~~~~~~}}。

\begin{jie}
由题得:$A^{-1}BA=6A+BA$,即$(A^{-1}-E)BA=6A$,同时右乘$A^{-1}$:$(A^{-1}-E)B=6E$。同时左乘$(A^{-1}-E)^{-1}$:$B=6(A^{-1}-E)^{-1}$因为$A$是对角阵(对角阵的逆为对角阵每个元素的倒数),所以
\begin{equation*}
A^{-1}
\begin{pmatrix}
3&&\\
&4&\\
&&7
\end{pmatrix}~\Rightarrow~A^{-1}-E=
\begin{pmatrix}
2&&\\
&3&\\
&&6
\end{pmatrix}~\Rightarrow~(A^{-1}-E)^{-1}=
\begin{pmatrix}
\frac{1}{2}&&\\
&\frac{1}{3}&\\
&&\frac{1}{6}
\end{pmatrix}~\Rightarrow~B=6(A^{-1}-E)^{-1}=
\begin{pmatrix}
3&&\\
&2&\\
&&1
\end{pmatrix}
\end{equation*}
\end{jie}

\EX 设$n$维向量$\alpha=(a,0,\cdots,0,a)^T,(a<0)$,$E$为$n$阶单位矩阵,矩阵$A=E-\alpha\alpha^T,B=E+\frac{1}{2}\alpha\alpha^T$,其中$A$的逆矩阵为$B$,则$a=$\underline{\hphantom{~~~~~~~~~~~}}。

\begin{jie}
由题得:
\begin{equation*}
\alpha\alpha^T=
\begin{pmatrix}
a\\ 0\\ \vdots\\ 0\\ a
\end{pmatrix}\begin{pmatrix}
a& 0&\cdots& 0&a
\end{pmatrix}=
\begin{pmatrix}
a^2&0&\cdots&0&a^2\\
0&0&\cdots&0&0\\
\vdots&\vdots&\ddots&\vdots&\vdots\\
0&0&\cdots&0&0\\
a^2&0&\cdots&0&a^2
\end{pmatrix}
\end{equation*}
所以:
\begin{gather*}
A=E-\alpha\alpha^T=
\begin{pmatrix}
1-a^2&0&\cdots&0&-a^2\\
0&1&\cdots&0&0\\
\vdots&\vdots&\ddots&\vdots&\vdots\\
0&0&\cdots&1&0\\
-a^2&0&\cdots&0&1-a^2
\end{pmatrix}\\
B=E+\frac{1}{a}\alpha\alpha^T=
\begin{pmatrix}
1+a&0&\cdots&0&a\\
0&1&\cdots&0&0\\
\vdots&\vdots&\ddots&\vdots&\vdots\\
0&0&\cdots&1&0\\
a&0&\cdots&0&1+a
\end{pmatrix}
\end{gather*}

$A,B$互为可逆矩阵:$AB=E$
\begin{align*}
AB&=
\begin{pmatrix}
1-a^2&0&\cdots&0&-a^2\\
0&1&\cdots&0&0\\
\vdots&\vdots&\ddots&\vdots&\vdots\\
0&0&\cdots&1&0\\
-a^2&0&\cdots&0&1-a^2
\end{pmatrix}
\begin{pmatrix}
1+a&0&\cdots&0&a\\
0&1&\cdots&0&0\\
\vdots&\vdots&\ddots&\vdots&\vdots\\
0&0&\cdots&1&0\\
a&0&\cdots&0&1+a
\end{pmatrix}\\ &=
\begin{pmatrix}
(1-a^2)(1+a)-a^3&0&\cdots&0&a(1-a^2)-a^{2}(1+a)\\
0&1&\cdots&0&0\\
\vdots&\vdots&\ddots&\vdots&\vdots\\
0&0&\cdots&1&0\\
a(1-a^2)-a^{2}(1+a)&0&\cdots&0&(1-a^2)(1+a)-a^3
\end{pmatrix}=
\begin{pmatrix}
1&&&&\\
&1&&&\\
&&\ddots&&\\
&&&1&\\
&&&&1
\end{pmatrix}
\end{align*}
两矩阵相等,对应位置元素相等:
\begin{equation*}
\begin{cases}
(1-a^2)(1+a)-a^3=1\\
a(1-a^2)-a^{2}(1+a)=0\\
a<0
\end{cases}~~\Rightarrow~~a=-1
\end{equation*}
综上所述:$a=-1$。
\end{jie}

\EX 设$4$阶方阵$
A=
\begin{pmatrix}
5&2&0&0\\
2&1&0&0\\
0&0&1&-2\\
0&0&1&1
\end{pmatrix}
$,则$A$的逆矩阵$A^{-1}=$\underline{\hphantom{~~~~~~~~~~~}}。

\begin{jie}
由题可将$A$按如下方式进行分块$A=
\begin{pmatrix}
A_{11}&0\\
0&A_{22}
\end{pmatrix}
$,式中$A_{11}=
\begin{pmatrix}
5&2\\
2&1
\end{pmatrix},A_{22}=
\begin{pmatrix}
1&-2\\
1&1
\end{pmatrix}
$,所以$A^{-1}=
\begin{pmatrix}
A_{11}^{-1}&0\\
0&A_{22}^{-1}
\end{pmatrix}
$

\begin{align*}
[A_{11}|E]=&
\left(
 \begin{array}{c:c}
\begin{matrix}
5&2\\
2&1
\end{matrix}&
\begin{matrix}
1&0\\
0&1
\end{matrix}
\end{array}
\right)\xrightarrow{\substack{r_{2}-\frac{2}{5}r_{1}}}
{
\left(
 \begin{array}{c:c}
\begin{matrix}
5&2\\
0&\frac{1}{5}
\end{matrix}&
\begin{matrix}
1&0\\
-\frac{2}{5}&1
\end{matrix}
\end{array}
\right)
}\xrightarrow{\substack{r_{1}-r_{2}}}
{
\left(
 \begin{array}{c:c}
\begin{matrix}
5&0\\
0&\frac{1}{5}
\end{matrix}&
\begin{matrix}
5&-10\\
-\frac{2}{5}&1
\end{matrix}
\end{array}
\right)
}\\ &\xrightarrow{\substack{r_{1}\times\left(\frac{1}{5}\right)\\ r_{2}\times 5}}
{
\left(
 \begin{array}{c:c}
\begin{matrix}
1&0\\
0&1
\end{matrix}&
\begin{matrix}
1&-2\\
-2&5
\end{matrix}
\end{array}
\right)
}~~\Rightarrow~~A_{11}^{-1}=
\begin{pmatrix}
1&-2\\
-2&5
\end{pmatrix}\\
[A_{22}|E]=&
\left(
 \begin{array}{c:c}
\begin{matrix}
1&-2\\
1&1
\end{matrix}&
\begin{matrix}
1&0\\
0&1
\end{matrix}
\end{array}
\right)\xrightarrow{\substack{r_{2}-r_{1}}}
{
\left(
 \begin{array}{c:c}
\begin{matrix}
1&-2\\
0&3
\end{matrix}&
\begin{matrix}
1&0\\
-1&1
\end{matrix}
\end{array}
\right)
}\xrightarrow{\substack{r_{2}\times\left(\frac{1}{3}\right)}}
{
\left(
 \begin{array}{c:c}
\begin{matrix}
1&-2\\
0&1
\end{matrix}&
\begin{matrix}
1&0\\
-\frac{1}{3}&\frac{1}{3}
\end{matrix}
\end{array}
\right)
}\\
&\xrightarrow{\substack{r_{1}+2r_{2}}}
{
\left(
 \begin{array}{c:c}
\begin{matrix}
1&0\\
0&1
\end{matrix}&
\begin{matrix}
\frac{1}{3}&\frac{2}{3}\\
-\frac{1}{3}&\frac{1}{3}
\end{matrix}
\end{array}
\right)
}~~\Rightarrow~~A_{22}^{-1}=
\begin{pmatrix}
\frac{1}{3}&\frac{2}{3}\\
-\frac{1}{3}&\frac{1}{3}
\end{pmatrix}
\end{align*}
所以:
$A^{-1}=
\begin{pmatrix}
1&-2&0&0\\
-2&5&0&0\\
0&0&\frac{1}{3}&\frac{2}{3}\\
0&0&-\frac{1}{3}&\frac{1}{3}
\end{pmatrix}
$。
\end{jie}

\EX 设$
A=
\begin{pmatrix}
1&0&0&0\\
-2&3&0&0\\
0&-4&5&0\\
0&0&-6&7
\end{pmatrix}
$,$E$为4阶单位矩阵,且$B=(E+A)^{-1}(E-A)$。求$(E+B)^{-1}$

\begin{jie}
由题得:$(E+A)=
\begin{pmatrix}
2&0&0&0\\
-2&4&0&0\\
0&-4&6&0\\
0&0&-6&8
\end{pmatrix},(E-A)=
\begin{pmatrix}
0&0&0&0\\
2&-2&0&0\\
0&4&-4&0\\
0&0&6&-6
\end{pmatrix}
$.

令$C=E+A$,将$C$按如下形式进行分块:
$
C=
\begin{pmatrix}
C_{11}&0\\
C_{21}&C_{22}
\end{pmatrix}
$,式中:$
C_{11}=
\begin{pmatrix}
2&0\\ -2&4
\end{pmatrix},C_{21}=
\begin{pmatrix}
0&-4\\ 0&0
\end{pmatrix},C_{22}=
\begin{pmatrix}
6&0\\ -6&8
\end{pmatrix}
$

\begin{tips}
分块矩阵的逆:

设$A=
\begin{bmatrix}
  A_{11} & A_{12} \\
  0 & A_{22}
\end{bmatrix}
$,其中$A_{11},A_{22}$是方阵,则$A$可逆当且仅当$A_{11},A_{22}$都可逆,并且
\begin{equation*}
A^{-1}=
  \begin{bmatrix}
    A_{11}^{-1} & -A_{11}^{-1}A_{12}A_{22}^{-1} \\
    0 & A_{22}^{-1}
  \end{bmatrix}\tag{\uppercase\expandafter{\romannumeral1}}
\end{equation*}

设$A=
\begin{bmatrix}
  A_{11} & 0 \\
  A_{21} & A_{22}
\end{bmatrix}
$,其中$A_{11},A_{22}$是方阵,则$A$可逆当且仅当$A_{11},A_{22}$都可逆,并且
\begin{equation*}
A^{-1}=
  \begin{bmatrix}
    A_{11}^{-1} & 0 \\
    -A_{22}^{-1}A_{21}A_{11}^{-1} & A_{22}^{-1}
  \end{bmatrix}\tag{\uppercase\expandafter{\romannumeral2}}
\end{equation*}
\hphantom{~}
\end{tips}

可以看出$C,C_{11},C_{22}$都符合(\uppercase\expandafter{\romannumeral2})式,所以:
\begin{align*}
&C_{11}^{-1}=
\begin{pmatrix}
\frac{1}{2}&0\\
-\frac{1}{4}\times(-2)\times\frac{1}{2}&\frac{1}{4}
\end{pmatrix}=\begin{pmatrix}
\frac{1}{2}&0\\
\frac{1}{4}&\frac{1}{4}
\end{pmatrix}~~~
C_{22}^{-1}=
\begin{pmatrix}
\frac{1}{6}&0\\
-\frac{1}{8}\times(-6)\times\frac{1}{6}&\frac{1}{8}
\end{pmatrix}=\begin{pmatrix}
\frac{1}{6}&0\\
\frac{1}{8}&\frac{1}{8}
\end{pmatrix}\\
& -C_{22}^{-1}\cdot C_{21}\cdot C_{11}^{-1}=
-\begin{pmatrix}
\frac{1}{6}&0\\
\frac{1}{8}&\frac{1}{8}
\end{pmatrix}\begin{pmatrix}
0&-4\\ 0&0
\end{pmatrix}\begin{pmatrix}
\frac{1}{2}&0\\
\frac{1}{4}&\frac{1}{4}
\end{pmatrix}=-
\begin{pmatrix}
0&-\frac{2}{3}\\ 0&-\frac{1}{2}
\end{pmatrix}\begin{pmatrix}
\frac{1}{2}&0\\
\frac{1}{4}&\frac{1}{4}
\end{pmatrix}=\begin{pmatrix}
\frac{1}{6}&\frac{1}{6}\\
\frac{1}{8}&\frac{1}{8}
\end{pmatrix}
\end{align*}
所以:
\begin{equation*}
  (E+A)^{-1}=C^{-1}=
  \begin{pmatrix}
   C_{11}^{-1}&0\\
   -C_{22}^{-1}\cdot C_{21}\cdot C_{11}^{-1}&C_{22}^{-1}
  \end{pmatrix}
=
\begin{pmatrix}
\frac{1}{2}&0&0&0\\
\frac{1}{4}&\frac{1}{4}&0&0\\
\frac{1}{6}&\frac{1}{6}&\frac{1}{6}&0\\
\frac{1}{8}&\frac{1}{8}&\frac{1}{8}&\frac{1}{8}
\end{pmatrix}
\end{equation*}
所以:
\begin{align*}
&B=(E+A)^{-1}(E-A)=
\begin{pmatrix}
\frac{1}{2}&0&0&0\\
\frac{1}{4}&\frac{1}{4}&0&0\\
\frac{1}{6}&\frac{1}{6}&\frac{1}{6}&0\\
\frac{1}{8}&\frac{1}{8}&\frac{1}{8}&\frac{1}{8}
\end{pmatrix}\begin{pmatrix}
0&0&0&0\\
2&-2&0&0\\
0&4&-4&0\\
0&0&6&-6
\end{pmatrix}=\begin{pmatrix}
0&0&0&0\\
\frac{1}{2}&-\frac{1}{2}&0&0\\
\frac{1}{3}&\frac{1}{3}&-\frac{2}{3}&0\\
\frac{1}{4}&\frac{1}{4}&\frac{1}{4}&-\frac{3}{4}
\end{pmatrix}\\ &
E+B=\begin{pmatrix}
1&0&0&0\\
\frac{1}{2}&\frac{1}{2}&0&0\\
\frac{1}{3}&\frac{1}{3}&\frac{1}{3}&0\\
\frac{1}{4}&\frac{1}{4}&\frac{1}{4}&\frac{1}{4}
\end{pmatrix}=D=
\begin{pmatrix}
D_{11}&0\\
D_{21}&D_{22}
\end{pmatrix}\text{式中}
D_{11}=
\begin{pmatrix}
1&0\\
\frac{1}{2}&\frac{1}{2}
\end{pmatrix},D_{21}=
\begin{pmatrix}
\frac{1}{3}&\frac{1}{3}\\
\frac{1}{4}&\frac{1}{4}
\end{pmatrix},D_{22}=
\begin{pmatrix}
\frac{1}{3}&0\\
\frac{1}{4}&\frac{1}{4}
\end{pmatrix}
\end{align*}
可以看出:$D,D_{11},D_{22}$同样符合(\uppercase\expandafter{\romannumeral2})式,所以:
\begin{align*}
&D_{11}^{-1}=
\begin{pmatrix}
1^{-1}&0\\
-\times\left(\frac{1}{2}\right)^{-1}\times\frac{1}{2}\times(1)^{-1}&\left(\frac{1}{2}\right)^{-1}
\end{pmatrix}=
\begin{pmatrix}
1&0\\
-1&2
\end{pmatrix}~
D_{22}^{-1}=
\begin{pmatrix}
\left(\frac{1}{3}\right)^{-1}&0\\
-\times\left(\frac{1}{4}\right)^{-1}\times\frac{1}{4}\times\left(\frac{1}{3}\right)^{-1}&\left(\frac{1}{4}\right)^{-1}
\end{pmatrix}=
\begin{pmatrix}
3&0\\
-3&4
\end{pmatrix}
\\ &
-D_{22}^{-1}\cdot D_{21}\cdot D_{11}^{-1}=-\begin{pmatrix}
3&0\\
-3&4
\end{pmatrix}\begin{pmatrix}
\frac{1}{3}&\frac{1}{3}\\
\frac{1}{4}&\frac{1}{4}
\end{pmatrix}\begin{pmatrix}
1&0\\
-1&2
\end{pmatrix}=
\begin{pmatrix}
0&-2\\
0&0
\end{pmatrix}
\end{align*}
所以:
$
(E+B)^{-1}=D^{-1}=\begin{pmatrix}
D_{11}^{-1}&0\\
-D_{22}^{-1}\cdot D_{21}\cdot D_{11}^{-1}&D_{22}^{-1}
\end{pmatrix}=
\begin{pmatrix}
1&0&0&0\\
-1&2&0&0\\
0&-2&3&0\\
0&0&-3&4
\end{pmatrix}
$。
\end{jie}

\EX 已知$A,B$为3阶矩阵,且满足$2A^{-1}B=B-4E$,其中$E$是3阶单位矩阵。

(1)证明:矩阵$A-2E$可逆。

(2)若$
B=
\begin{pmatrix}
1&-2&0\\
1&2&0\\
0&0&2
\end{pmatrix}
$,求矩阵$A$。

\begin{jie}
(1)
等式两边同时左乘$A$:$2B=AB-4A$,所以$AB-2B=4A$,所以$(A-2E)B=4A$,等式两边同时右乘$\dfrac{1}{4}A^{-1}$:$(A-2E)\cdot\dfrac{1}{4}(BA^{-1})=E$,所以$A-2E$可逆且$(A-2E)^{-1}=\dfrac{1}{4}(BA^{-1})$

(2)对题目中的等式两边同时右乘$\dfrac{1}{2}B^{-1}$:$A^{-1}=\dfrac{E-4B^{-1}}{2}$.

因为$B=
\begin{pmatrix}
1&-2&0\\
1&2&0\\
0&0&2
\end{pmatrix}=
\begin{pmatrix}
B_{11}&0\\
0&B_{22}
\end{pmatrix}
$,式中$B_{11}
\begin{pmatrix}
1&-2\\
1&2
\end{pmatrix},B_{22}=
\begin{pmatrix}
2
\end{pmatrix}
$,所以$B^{-1}=
\begin{pmatrix}
B_{11}^{-1}&0\\
0&B^{-1}_{22}
\end{pmatrix}
$.
\begin{align*}
[B_{11}|E]=&
\left(
 \begin{array}{c:c}
\begin{matrix}
1&-2\\
1&2
\end{matrix}&
\begin{matrix}
1&0\\
0&1
\end{matrix}
\end{array}
\right)\xrightarrow{\substack{r_{2}-r_{1}}}
{
\left(
 \begin{array}{c:c}
\begin{matrix}
1&-2\\
0&4
\end{matrix}&
\begin{matrix}
1&0\\
-1&1
\end{matrix}
\end{array}
\right)
}\xrightarrow{\substack{r_{2}\times\left(\frac{1}{4}\right)}}
{
\left(
 \begin{array}{c:c}
\begin{matrix}
1&-2\\
0&1
\end{matrix}&
\begin{matrix}
1&0\\
-\frac{1}{4}&\frac{1}{4}
\end{matrix}
\end{array}
\right)
}\\
&\xrightarrow{\substack{r_{1}+2r_2}}
{
\left(
 \begin{array}{c:c}
\begin{matrix}
1&0\\
0&1
\end{matrix}&
\begin{matrix}
\frac{1}{2}&\frac{1}{2}\\
-\frac{1}{4}&\frac{1}{4}
\end{matrix}
\end{array}
\right)
}~~~~\Rightarrow~~~B_{11}^{-1}=
\begin{pmatrix}
\frac{1}{2}&\frac{1}{2}\\
-\frac{1}{4}&\frac{1}{4}
\end{pmatrix}
\end{align*}
所以:
\begin{equation*}
B^{-1}=
\begin{pmatrix}
\frac{1}{2}&\frac{1}{2}&0\\
-\frac{1}{4}&\frac{1}{4}&0\\
0&0&\frac{1}{2}
\end{pmatrix}~~\Rightarrow~~E-4B^{-1}
=\begin{pmatrix}
-1&-2&0\\
1&0&0\\
0&0&-1
\end{pmatrix}
\end{equation*}
所以:
$A=(A^{-1})^{-1}=
\left[\frac{1}{2}(E-4B)^{-1}\right]^{-1}=2\left[(E-4B)^{-1}\right]^{-1}
$,记$E-4B^{-1}=C=
\begin{pmatrix}
C_{11}&0\\
0&C_{22}
\end{pmatrix}
$,式中$C_{11}=\begin{pmatrix}-1&-2\\
1&0\end{pmatrix},
C_{22}=
\begin{pmatrix}
-1
\end{pmatrix}
$,所以$C^{-1}=
\begin{pmatrix}
C_{11}^{-1}&0\\
0&C_{22}^{-1}
\end{pmatrix}
$.

\begin{align*}
[C_{11}|E]=&
\left(
 \begin{array}{c:c}
\begin{matrix}
-1&-2\\
1&0
\end{matrix}&
\begin{matrix}
1&0\\
0&1
\end{matrix}
\end{array}
\right)\xrightarrow{\substack{r_{2}\Leftrightarrow r_{1}}}
{
\left(
 \begin{array}{c:c}
\begin{matrix}
1&0\\
-1&-2
\end{matrix}&
\begin{matrix}
0&1\\
1&0
\end{matrix}
\end{array}
\right)
}\xrightarrow{\substack{r_{2}+ r_{1}}}
{
\left(
 \begin{array}{c:c}
\begin{matrix}
1&0\\
0&-2
\end{matrix}&
\begin{matrix}
0&1\\
1&1
\end{matrix}
\end{array}
\right)
}\\& \xrightarrow{\substack{r_{2}\times\left(-\frac{1}{2}\right)}}
{
\left(
 \begin{array}{c:c}
\begin{matrix}
1&0\\
0&1
\end{matrix}&
\begin{matrix}
0&1\\
-\frac{1}{2}&-\frac{1}{2}
\end{matrix}
\end{array}
\right)
}~~~C_{11}^{-1}=
\begin{pmatrix}
0&1\\
-\frac{1}{2}&-\frac{1}{2}
\end{pmatrix}
\end{align*}
所以:$(E-4B^{-1})^{-1}
=C^{-1}=
\begin{pmatrix}
C_{11}^{-1}&0\\
0&C_{22}^{-1}
\end{pmatrix}=
\begin{pmatrix}
0&1&0\\
-\frac{1}{2}&-\frac{1}{2}&0\\
0&0&-1
\end{pmatrix}
$,所以$A=
2(E-4B^{-1})^{-1}=
\begin{pmatrix}
0&2&0\\
-1&-1&0\\
0&0&-2
\end{pmatrix}
$。
\end{jie}

\stepcounter{chapter}
\clearpage
\hphantom{~~}\hfill {\zihao{3}\heiti 第六次线性代数} \hfill\hphantom{~~}
%\chapter{第六次线性代数}
\addcontentsline{toc}{section}{\protect\numberline {}第六次线性代数}


\hphantom{~~}

\EX 计算下列行列式。

\begin{equation*}
(1)
\begin{vmatrix}
0&2&-5\\
2&1&-3\\
2&3&5
\end{vmatrix}~~~(2)
\begin{vmatrix}
a&0&0&b\\
0&a&b&0\\
0&b&a&0\\
b&0&0&a
\end{vmatrix}
\end{equation*}

\begin{jie}
(1)
\begin{equation*}
\text{原式}\xlongequal{r_2-r_3}
\begin{vmatrix}
0&2&-5\\
0&-2&-8\\
2&3&5
\end{vmatrix}=2\times(-1)^{3+1}\times
\begin{vmatrix}
2&-5\\
-2&-8\\
\end{vmatrix}=-52
\end{equation*}

(2)按第一列展开:
\begin{align*}
\text{原式}&=a\times(-1)^{1+1}\times
\begin{vmatrix}
a&b&0\\
b&a&0\\
0&0&a
\end{vmatrix}+b\times(-1)^{4+1}
\begin{vmatrix}
0&0&b\\
a&b&0\\
b&a&0
\end{vmatrix}=a\times a\times(-1)^{3+3}
\begin{vmatrix}
a&b\\
b&a
\end{vmatrix}
-b\times b\times(-1)^{1+3}
\begin{vmatrix}
a&b\\
b&a
\end{vmatrix}\\
&=(a^2-b^2)(a^2-b^2)=(a^2-b^2)^2
\end{align*}
\end{jie}

\EX 判断下列说法是否正确,并说明理由。

(2)对任意数$a$和任意的正整数$n$,都存在一个$n$阶方阵$A$
使得$|A|=a$。

(3)如果$n$阶方阵$A$的每个$(i,j)-$元都是整数,则$|A|$一定
是整数。

(4) 设$A$是任意$n$阶方阵.则$|-A|=-|A|$。

(5)设$A$,$B$是$n$阶方阵且$A\neq B$.则$|A|\neq |B|$。

\begin{jie}
(2)正确。例如:
$
\begin{vmatrix}
 a&&&\\
 &1&&\\
 &&\ddots&\\
 &&&1
\end{vmatrix}$

(3)正确。由于每个元素都是整数,所以$|A|$的定义式中的每一项都是整数,因此它们的和也是整数。

(4)错误。由行列式的性质:$|kA|=k^{n}|A|$,所以$|-A|=(-1)^n|A|$,$n$为偶数时,才有$|-A|=-|A|$。

(5)错误。例如:$A=
\begin{pmatrix}
1&0\\
0&0
\end{pmatrix},B=
\begin{pmatrix}
0&0\\
0&1
\end{pmatrix}
$,$A\neq B$但是$|A|=|B|=0$
\end{jie}

\EX 用行列式的定义计算$n$阶行列式:
\begin{equation*}
|A|=
\begin{vmatrix}
a_{11}&a_{12}&\cdots&a_{1,n-1}&a_{1n}\\
a_{21}&a_{22}&\cdots&a_{1,n-1}&0\\
a_{31}&a_{32}&\cdots&0&0\\
\vdots&\vdots&&\vdots&\vdots\\
a_{n1}&0&\cdots&0&0\\
\end{vmatrix}
\end{equation*}

\begin{jie}
参照课本109页例3.3.7.

任取$A$的行列式的定义式(107页式3.3.3或108页命题3.3.1,这里按3.3.3式来)中的一项:
\begin{equation*}
(-1)^{\tau(i_1i_2\cdots i_{n})}A(1,i_1)A(2,i_2)\cdots A(n-1,i_{n-1}),A(n,i_n)
\end{equation*}

假设$i_n\neq 1$,则$A(n,i_n)$一定为0,则该项也一定为0,所以不妨取$i_n=1$。

由$n$元排列的定义,$i_{n-1}$只能从$2\sim n$中进行选取,假设$i_{n-1}\neq 2$,则$A(n-1,i_{n-1})$一定为0,则该项也一定为0,所以不妨取$i_{n-1}=2$。

重复上述的讨论,如果$i_1i_2\cdots i_n\neq n(n-1)\cdots 321$时,其对应的行列式中的那一项一定为0.

该排列所对应的逆序对共有$
\dfrac{n(n-1)}{2}
$(106页例3.3.2的结论)

所以:$|A|=(-1)^{\tau(n(n-1)\cdots321)}a_{1n}a_{2,n-1}\cdots a_{n1}=(-1)^{\frac{n(n-1)}{2}}a_{1n}a_{2,n-1}\cdots a_{n1}$
\end{jie}

\EX 设$A$是$n$阶反对称矩阵,即$A^T=-A$。证明:如果$|A|\neq 0$,则$n$必然是偶数。

\begin{zhengming}
有行列式的性质:转置不改变行列式即$|A^T|=|A|$。对题目中的等式两端同时求行列式:
$|A^T|=|A|=|-A|=(-1)^n|A|$。因为$|A|\neq 0$,所以$(-1)^n=1$,即$n$为偶数。
\end{zhengming}

\EX 判断下列说法是否正确,并说明理由。

(1)设$A,B$是$n$阶方阵且$|A|>0$, $|B|>0$.则$|A+B|>0$。

(2)设$A,B$是$n$阶方阵且$|A-B|=0$.则$|A|=|B|$。

(3) 设$A$是$n$阶方阵,$n>1$,$k$是不等于$1$的数.则$|kA|=
k|A|$一定不成立。

(4)设$A,B$是$n$阶方阵.由于$A,B$不一定可交换,所以,
$|AB|$与$|BA|$不一定相等。

(5)设$A$是可逆的$n$阶方阵.则由$Ax=0$得$|A||x|=0$,从
而$x=0$。

\begin{jie}
(1)错误。例如:$A=
\begin{pmatrix}
1&0\\
0&1
\end{pmatrix}
,B=
\begin{pmatrix}
-1&0\\
0&-1
\end{pmatrix}$.

(2)错误。例如:
$
A=\begin{pmatrix}
1&0\\
0&1
\end{pmatrix},B=\begin{pmatrix}
1&0\\
0&-1
\end{pmatrix}
$

(3)错误。$k=-1$且$n$为偶数时该等式成立。

(4)错误。由行列式的性质:$|AB|=|A|\cdot|B|,|BA|=|B|\cdot |A|$,而行列式是数,数的乘法具有交换律即$|A|\cdot|B|=|B|\cdot |A|$,即$|AB|=|BA|$.

(5)错误。此题是过程错误,而非结论错误。当$n>1$时,$x$为多行一列而非方阵,只有方阵才能谈论行列式,所以$|Ax|$是不存在行列式的。正确的做法是:$A$可逆所以$r(A)=n$,齐次方程,所以只有零解。
\end{jie}

\EX 设$A$是$n$阶方阵。证明:齐次线性方程组$Ax=0$有非零解$\Leftrightarrow$齐次线性方程组$A^kx=0$有非零解。其中$k$是任意的正整数。

\begin{zhengming}
\begin{align*}
Ax=0\text{有非零解}~\Leftrightarrow~&r(A)<n~\Leftrightarrow~|A|=0\\
~\Leftrightarrow~&|A|^k=0~\Leftrightarrow~|A^k|=0\\
~\Leftrightarrow~&r(A^k)<0~\Leftrightarrow~A^{k}x=0\text{有非零解}
\end{align*}
\end{zhengming}

\EX 判断下列说法是否正确,并说明理由。

(1)用初等变换把方阵$A$变为上三角阵$J$,则$|A|$等于$J$的
全部对角元的乘积。

(2)设矩阵$A$的秩为$3$.则$A$的所有$3$阶子式都不为$0$。

(3) 设非零矩阵$A$的所有$4$阶子式都为$0$,则$A$的秩一定是
$3$。

(7) 设线性方程组$Ax= \beta$的系数矩阵$A$是$n$阶方阵.如果
$|A|=0$,则该方程组没有解。

\begin{jie}
(1)错误。例如:
$A=
\begin{pmatrix}
0&1\\
1&0
\end{pmatrix}\rightarrow J=\begin{pmatrix}
1&0\\
0&1
\end{pmatrix}
$,但$|A|=-1$而$J$的对角元素的乘积为1.

(2)错误。例如:
$A=
\begin{pmatrix}
1&0&0&0\\
0&1&0&0\\
0&0&1&0\\
0&0&0&0
\end{pmatrix}
$,秩为3,但是$A$的第$2,3,4$列和$2,3,4$行对应的三阶子式为0.

(3)错误。例如:
$A=
\begin{pmatrix}
1&0&0&0\\
0&1&0&0\\
0&0&0&0\\
0&0&0&0
\end{pmatrix}
$,$A$的所有的四阶子式都为0,但$r(A)=2$。

(7)错误。若$\beta=0$即该方程组为齐次线性方程组时,一定有解。
\end{jie}

\EX 记行列式
$
\begin{vmatrix}
x-2&x-1&x-2&x-3\\
2x-2&2x-1&2x-2&2x-3\\
3x-3&3x-2&4x-5&3x-5\\
4x&4x-3&5x-7&4x-3
\end{vmatrix}
$为$f(x)$。则方程$f(x)=0$的根的个数为\underline{\hphantom{~~~~~~~~}}。

\begin{jie}
由题得:
\begin{align*}
\text{原式}\xlongequal{\substack{ c_2-c_1\\ c_3-c_1\\ c_4-c_1}}
\begin{vmatrix}
x-2&1&0&-1\\
2x-2&1&0&-1\\
3x-3&1&x-2&-2\\
4x&-3&x-7&-3
\end{vmatrix}\xlongequal{\substack{ c_4+c_2}}
\begin{vmatrix}
x-2&1&0&0\\
2x-2&1&0&0\\
3x-3&1&x-2&-2\\
4x&-3&x-7&-3
\end{vmatrix}=\begin{vmatrix}
               D_{11}&0\\
               D_{21}&D_{22}
              \end{vmatrix}
\end{align*}
式中:$D_{11}=
\begin{vmatrix}
x-2&1\\
2x-2&1\\
\end{vmatrix},D_{21}=
\begin{vmatrix}
3x-3&1\\
4x&-3
\end{vmatrix},D_{22}=
\begin{vmatrix}
x-2&-2\\
x-7&-3
\end{vmatrix}
$,下三角阵的行列式计算公式为:
\begin{equation*}
D_{11}D_{22}=
\begin{vmatrix}
x-2&1\\
2x-2&1\\
\end{vmatrix}\begin{vmatrix}
x-2&-2\\
x-7&-3
\end{vmatrix}=[(x-2)-2(x-1)][-3(x-2)+2(x-7)]=5x(x-1)
\end{equation*}
所以$f(x)$有两个根。
\end{jie}

\EX 设$f(x)=
\begin{vmatrix}
x-1 &1&-1&1\\
-1&x+1&-1&1\\
-1&1&x-1&1\\
-1&1&-1&x+1
\end{vmatrix}
$,求$f(x)=0$的根。

\begin{jie}
由题得:
\begin{align*}
f(x)&\xlongequal{r_1-r_2}
\begin{vmatrix}
x &-x&0&0\\
-1&x+1&-1&1\\
-1&1&x-1&1\\
-1&1&-1&x+1
\end{vmatrix}\\ &\xlongequal{c_2+c_1}\begin{vmatrix}
x &0&0&0\\
-1&x&-1&1\\
-1&0&x-1&1\\
-1&0&-1&x+1
\end{vmatrix}
=x\begin{vmatrix}
x&-1&1\\
0&x-1&1\\
0&-1&x+1
\end{vmatrix}
=x^2\begin{vmatrix}
x-1&1\\
-1&x+1
\end{vmatrix}=x^4=0
\end{align*}
所以$f(x)=0$的根为$x_1=x_2=x_3=x_4=0$.
\end{jie}

\EX 求$f(\lambda)=
\begin{vmatrix}
\lambda &-2&2\\
-2&\lambda-4&-4\\
2&-4&\lambda+3
\end{vmatrix}=0
$的根。

\begin{jie}
由题得:
\begin{equation*}
f(\lambda)\xlongequal{r_3-2r_1}
\begin{vmatrix}
\lambda &-2&2\\
-2&\lambda-4&-4\\
2-2\lambda&0&\lambda-1
\end{vmatrix}\xlongequal{c_1+2c_{3}}
\begin{vmatrix}
\lambda &-2&2\\
-2&\lambda-4&-4\\
0&0&\lambda-1
\end{vmatrix}=(\lambda-1)[(\lambda+4)(\lambda-4)-20]=0
\end{equation*}
解得:$\lambda_1=1,\lambda_2=6,\lambda_3=-6$.
\end{jie}

\EX 设$f(x)=
\begin{vmatrix}
2x&x&1&2\\
1&x-1&1&-1\\
3&2&x&1\\
1&1&1&x
\end{vmatrix}
$,求$f(x)$中$x^3$的系数。

\begin{jie}
把$f(x)$按第一行展开(或者其他行其他列展开)。可以看出,第一行第三列元素的代数余子式不含$x^3$,同理第一行第四列元素的代数余子式也不含该项。

所以秩考虑前两个元素:
\begin{equation*}
2x\begin{vmatrix}
x-1&1&-1\\
2&x&1\\
1&1&x
\end{vmatrix}-x\begin{vmatrix}
1&1&-1\\
3&x&1\\
1&1&x
\end{vmatrix}
\end{equation*}
对上式中的第一项按第一行展开,发现只有第一行第一列元素对应的有$x^3$项(注意不要忘了乘系数$2x$)即:$2x(x-1)(x\cdot x-1)=-2x^{3}+(\text{其他不含$x^3$的项})$。

同理对上式第二项按第一行第一列展开发现只有第一行第一列元素对应的有$x^3$项(注意不要忘了乘系数$-x$):$-x\cdot 1\cdot(x\cdot x-1)=-x^3+(\text{其他不含$x^3$的项})$。

所以$f(x)$中$x^3$的系数为$-3$。
\end{jie}

\EX 设$4$阶矩阵$A=(\alpha,\gamma_2,\gamma_3,\gamma_4),B=(\beta,\gamma_2,\gamma_3,\gamma_4)$,其中$\alpha,\beta,\gamma_2,\gamma_3,\gamma_4$均为$4$维列向量,且已知行列式$|A|=4,|B|=1$,则行列式$|A+B|=$\underline{\hphantom{~~~~~~~~~~~~}}。

\begin{jie}
由题得:$A+B=(\alpha+\beta,2\gamma_2,2\gamma_3,2\gamma_4)$。所以:(行列式的线性性质,课本111页引理3.3.3)
\begin{equation*}
|A+B|=|\alpha,2\gamma_2,2\gamma_3,2\gamma_4|+|\beta,2\gamma_2,2\gamma_3,2\gamma_4|=8|\alpha,\gamma_2,\gamma_3,\gamma_4|+8|\beta,\gamma_2,\gamma_3,\gamma_4|=8(|A|+|B|)=40
\end{equation*}
\end{jie}

\EX 计算$n$阶行列式$
D_n=
\begin{vmatrix}
5&3&&&\\
2&5&3&&\\
&2&5&\ddots&\\
&&\ddots&\ddots&3\\
&&&2&5
\end{vmatrix}
$,其中没有写出来的元素为零。

\begin{jie}由题得:$D_1=5,D_{2}=
\begin{vmatrix}
5&3\\
2&5
\end{vmatrix}=19
$
按第一行展开:
\begin{align*}
D_n=5\begin{vmatrix}
5&3&&\\
2&5&\ddots&\\
&\ddots&\ddots&3\\
&&2&5
\end{vmatrix}_{n-1}-3
\begin{vmatrix}
2&3&&\\
&5&\ddots&\\
&\ddots&\ddots&3\\
&&2&5
\end{vmatrix}_{n-1}=5D_{n-1}-3\times 2\begin{vmatrix}
5&3&\\
&\ddots&3\\
&2&5
\end{vmatrix}_{n-2}=5D_{n-1}-6D_{n-2}
\end{align*}
所以:$D_{3}=5D_{2}-6D_{1}=65$,$D_{n}-2D_{n-1}=3(D_{n-1}-2D_{n-2})$,从而
\begin{equation*}
D_{n}-2D_{n-1}=3^{n-2}(D_2-2D_1)=3^{n}
\end{equation*}
所以:
\begin{align*}
D_{n}&=3^{n}+2D_{n-1}=3^{n}+2(3^{n-1}+2D_{n-2})\\
&=3^n+2\times 3^{n-1}+2^2D_{n-2}=3^n+2\times 3^{n-1}+2^2(3^{n-2}+2D_{n-3})=\cdots=\sum_{i=0}^{n-2}2^i3^{n-i}+2^{n-1}D_{1}\\
&=\sum_{i=0}^{n-2}2^i3^{n-i}+2^{n-1}\times 3+2^n\\
&=\sum_{i=0}^{n}2^i3^{n-i}\xlongequal{\text{等比数列求和}}3^{n+1}-2^{n+1}
\end{align*}
\end{jie}

\EX 计算行列式$
D=
\begin{vmatrix}
1&1&1&1\\
1&2&0&0\\
1&0&3&0\\
1&0&0&4
\end{vmatrix}
$.

\begin{jie}
爪型行列式,求解如下(爪型行列式的求解方式一样):
\begin{align*}
D\xlongequal{r_1-\frac{1}{4}r_4}
\begin{vmatrix}
\frac{3}{4}&1&1&\textcolor[rgb]{1.00,0.00,0.00}{0}\\
1&2&0&0\\
1&0&3&0\\
1&0&0&4
\end{vmatrix}\xlongequal{r_1-\frac{1}{3}r_3}
\begin{vmatrix}
\frac{5}{12}&1&\textcolor[rgb]{1.00,0.00,0.00}{0}&\textcolor[rgb]{1.00,0.00,0.00}{0}\\
1&2&0&0\\
1&0&3&0\\
1&0&0&4
\end{vmatrix}\xlongequal{r_1-\frac{1}{2}r_2}
\begin{vmatrix}
-\frac{1}{12}&\textcolor[rgb]{1.00,0.00,0.00}{0}&\textcolor[rgb]{1.00,0.00,0.00}{0}&\textcolor[rgb]{1.00,0.00,0.00}{0}\\
1&2&0&0\\
1&0&3&0\\
1&0&0&4
\end{vmatrix}=-\frac{1}{12}\times 2\times 3\times4=-2
\end{align*}
\end{jie}

\EX 计算$n$阶行列式
$D_{n}
=\begin{vmatrix}
a_1+b&a_2&\cdots&a_{n}\\
a_1&a_2+b&\cdots&a_{n}\\
\vdots&\vdots&\ddots&\vdots\\
a_1&a_2&\cdots&a_{n}+b
 \end{vmatrix}
$。

\begin{jie}
\begin{align*}
D_{n}&\xlongequal{c_1+c_2+\cdots+c_n}
\begin{vmatrix}
a_1+a_2+\cdots+a_{n}+b&a_2&\cdots&a_{n}\\
a_1+a_2+\cdots+a_{n}+b&a_2+b&\cdots&a_{n}\\
\vdots&\vdots&\ddots&\vdots\\
a_1+a_2+\cdots+a_{n}+b&a_2&\cdots&a_{n}+b
 \end{vmatrix}\\
 &\xlongequal{\substack{r_2-r_1\\ r_3-r_1\\ \vdots\\r_n-r_1}}
 \begin{vmatrix}
a_1+a_2+\cdots+a_{n}+b&a_2&\cdots&a_{n}\\
0&b&\cdots&0\\
\vdots&\vdots&\ddots&\vdots\\
0&0&\cdots&b
 \end{vmatrix}=(a_1+a_2+\cdots+a_{n}+b)b^{n-1}\\
 &=b+b^{n-1}\sum_{i=1}^{n}a_i
\end{align*}
\end{jie}

\EX 计算$n$阶行列式$
D_{n}=
\begin{vmatrix}
1&x_1&\cdots&x_1^{n-1}\\
1&x_2&\cdots&x_2^{n-1}\\
\vdots&\vdots&\ddots&\vdots\\
1&x_n&\cdots&x_n^{n-1}\\
\end{vmatrix}
$.

\begin{jie}
范德蒙行列式,原式$=
(x_{n}-x_{n-1})(x_{n}-x_{n-2})\cdots(x_{2}-x_{1})=\prod\limits_{1\leq j<i\leq n}(x_i-x_j)
$
\end{jie}

\EX 计算$n$阶行列式$D=
\begin{vmatrix}
a_1&b_1&0&\cdots&0&0\\
0&a_2&b_2&\cdots&0&0\\
0&0&a_3&\cdots&0&0\\
\vdots&\vdots&\ddots&\ddots&\ddots&\vdots\\
0&0&0&\cdots&a_{n-1}&b_{n-1}\\
b_{n}&0&0&\cdots&0&a_n
\end{vmatrix}
$.

\begin{jie}
按第一列展开:
\begin{align*}
D&=a_{1}
\begin{vmatrix}
a_2&b_2&\cdots&0&0\\
0&a_3&\cdots&0&0\\
\vdots&\ddots&\ddots&\ddots&\vdots\\
0&0&\cdots&a_{n-1}&b_{n-1}\\
0&0&\cdots&0&a_n
\end{vmatrix}+b_{n}(-1)^{n+1}
\begin{vmatrix}
b_1&0&\cdots&0&0\\
a_2&b_2&\cdots&0&0\\
0&a_3&\cdots&0&0\\
\vdots&\ddots&\ddots&\ddots&\vdots\\
0&0&\cdots&a_{n-1}&b_{n-1}
\end{vmatrix}\\
&=a_1a_2\cdots a_{n}+(-1)^{n+1}b_1b_2\cdots b_n\\
&=\prod_{i=1}^{n}a_i+(-1)^{n+1}\prod_{i=1}^{n}b_i
\end{align*}
\end{jie}

\EX 计算行列式$
D=\begin{vmatrix}
a_1+x&a_2&a_3&a_4\\
-x&x&0&0\\
0&-x&x&0\\
0&0&-x&x
  \end{vmatrix}
$。

\begin{jie}
按第一列展开:
\begin{align*}
D=
(a_1+x)
\begin{vmatrix}
x&0&0\\
-x&x&0\\
0&-x&x
\end{vmatrix}-x\cdot(-1)^{2+1}
\begin{vmatrix}
a_2&a_3&a_4\\
-x&x&0\\
0&-x&x
\end{vmatrix}=(a_1+x)x^3-x\cdot(-1)^{2+1}
\begin{vmatrix}
a_2&a_3&a_4\\
-x&x&0\\
0&-x&x
\end{vmatrix}
\end{align*}
上式中第二项:
\begin{equation*}
\xlongequal{c_2+c_3}\begin{vmatrix}
a_2&a_3+a_4&a_4\\
-x&x&0\\
0&0&x
\end{vmatrix}=x
\begin{vmatrix}
a_2&a_3+a_4\\
-x&x
\end{vmatrix}=x(a_2x+x(a_3+a_4))
\end{equation*}
代入$D$中得:$D=(a_1+a_2+a_3+a_4)x^3+x^4=\left(x+\sum\limits_{i=1}^{4}a_i\right)x^3$
\end{jie}

\EX 计算行列式$
D=
\begin{vmatrix}
x&-1\\
&x&-1\\
&&\ddots&\ddots\\
&&&\ddots&\ddots\\
&&&&x&-1\\
a_{n}&a_{n-1}&\cdots&\cdots&a_2&a_1+x
\end{vmatrix}
$,其中没有写出来的元素为0.

\begin{jie}
\begin{align*}
D&\xlongequal{c_2+\frac{c_1}{x}}
\begin{vmatrix}
x&\textcolor[rgb]{1.00,0.00,0.00}{0}\\
&x&-1\\
&&\ddots&\ddots\\
&&&\ddots&\ddots\\
&&&&x&-1\\
a_{n}&\textcolor[rgb]{1.00,0.00,0.00}{a_{n-1}+\frac{a_n}{x}}&\cdots&\cdots&a_2&a_1+x
\end{vmatrix}\\ &\xlongequal{c_3+\frac{c_2}{x}}
\begin{vmatrix}
x&\textcolor[rgb]{1.00,0.00,0.00}{0}\\
&x&\textcolor[rgb]{1.00,0.00,0.00}{0}\\
&&\ddots&\ddots\\
&&&\ddots&\ddots\\
&&&&x&-1\\
a_{n}&\textcolor[rgb]{1.00,0.00,0.00}{a_{n-1}+\frac{a_n}{x}}&\textcolor[rgb]{1.00,0.00,0.00}{a_{n-2}+\frac{a_n}{x^2}+\frac{a_{n-1}}{x}}&\cdots&a_2&a_1+x
\end{vmatrix}
\\
\hphantom{~~~~~~~~~~~~~~~~~~}\vdots
\\
&=\begin{vmatrix}
x\\
&x\\
&&\ddots\\
&&&\ddots\\
&&&&\ddots&\\
a_{n}&\textcolor[rgb]{1.00,0.00,0.00}{a_{n-1}+\frac{a_n}{x}}&\textcolor[rgb]{1.00,0.00,0.00}{a_{n-2}+\frac{a_n}{x^2}+\frac{a_{n-1}}{x}}&\cdots&\cdots&a_1+a_2x^{-1}+a_{3}x^{-2}+\cdots+a_nx^{-(n-1)}+x
\end{vmatrix}
\\
&=x^{n-1}(a_1+a_2x^{-1}+a_{3}x^{-2}+\cdots+a_nx^{-(n-1)}+x)\\
&=a_1x^{n-1}+a_2x_{n-2}+\cdots+a_nx^{n-n}+x^{n}\\
&=x^{n}+\sum_{i=1}^{n}a_ix^{n-i}
\end{align*}
\end{jie}
\clearpage

\hphantom{~~}\hfill {\zihao{3}\heiti 第七次线性代数} \hfill\hphantom{~~}

\hphantom{~~}

%\section{第七次线性代数}
\addcontentsline{toc}{section}{\protect\numberline {}第七次线性代数}
\EX 判断下列说法是否正确,并说明理由。

(1)设$A$为$n$阶方阵。任意给定两个$n$元排列$i_1i_2\cdots i_n$和$j_1j_2\cdots j_n$,则乘积
\begin{equation*}
A(i_1,j_1)A(i_2,j_2)\cdots A(i_n,j_n)
\end{equation*}
一定出现在$n$阶行列式$|A|$的定义式中 。

\begin{jie}
(1)正确。理由:数的乘法满足交换律,而$i_1i_2\cdots i_n$是行指标的一个排列,所以:
\begin{equation*}
A(i_1,j_1)A(i_2,j_2)\cdots A(i_n,j_n)=A(1,k_1)A(2,k_2)\cdots A(n,k_n)
\end{equation*}
其中$k_1k_2\cdots k_n$是列指标的一个排列。所以结论成立,且乘积在行列式的定义式中带符号$(-1)^{\tau(k_1k_2\cdots k_n)}$。
\end{jie}

\EX 设$A$是$m$阶方阵,$B$是$n$阶方阵,且$|A|=a$,$|B|=b$,$
C=
\begin{pmatrix}
0&A\\
B&0
\end{pmatrix}
$.求$|C|$。

\begin{jie}
设$A$的列向量组为$\alpha_1,\cdots,\alpha_m$。先交换$C$的第$n+1$列与第$n$列,再交换第$n$列与第$n-1$列,$\cdots$,再交换第二列与第一列,即$|C|=
(-1)^n
\begin{vmatrix}
\alpha_1&0&\alpha_2&\cdots&a_m\\
0&B&0&\cdots&0
\end{vmatrix}
$,同理按照上述方法,把$C$的$n+2$列交换到$C$的第二列,依次类推,这样的步骤共要执行$m$次,每次都进行了$n$次交换,所以最终的结果为:
\begin{equation*}
  |C|=(-1)^{nm}
  \begin{vmatrix}
   A&0\\
   0&B
  \end{vmatrix}=(-1)^{nm}|A|\cdot|B|=(-1)^{nm}ab
\end{equation*}
\end{jie}

\EX 设三阶方阵$A,B$满足$A^2B-A-B=E$,其中$E$为三阶单位矩阵,若$
A=
\begin{pmatrix}
1&0&1\\
0&2&0\\
-2&0&1
\end{pmatrix}
$,则$|B|=$\underline{\hphantom{~~~~~~~~~~}}。

\begin{jie}
由题得:$A^2B-B=A+E~\rightarrow~(A^2-E)B=A+E~\rightarrow~(A+E)(A-E)B=A+E$。

\begin{equation*}
|A+E|=
\begin{vmatrix}
2&0&1\\
0&3&0\\
-2&0&2
\end{vmatrix}=18\neq 0
\end{equation*}
所以$A+E$可逆,等式两边同左乘$(A+E)^{-1}$:$(A-E)B=E$,所以$|(A-E)B|=|A-E|\cdot|B|=|E|=1$.所以:
\begin{equation*}|B|=
\begin{vmatrix}
0&0&1\\
0&1&0\\
-2&0&1
\end{vmatrix}^{-1}=\frac{1}{2}
\end{equation*}
\end{jie}

\EX 设$\alpha_1,\alpha_2,\alpha_3$均为三维列向量,记矩阵$A=
(\alpha_1,\alpha_2,\alpha_3)
$,$B=(\alpha_1+\alpha_2+\alpha_3,\alpha_1+2\alpha_2+3\alpha_3,\alpha_1+4\alpha_2+9\alpha_3)$,如果$|A|=1$,则$|B|=$\underline{\hphantom{~~~~~~~~~~}}。

\begin{jie}
作业\textcolor[rgb]{1.00,0.00,0.00}{第三次线性代数}例2.21。

\begin{align*}
C\xlongequal{\substack{r_2-r_1\\ r_3-r_1}}
\begin{pmatrix}
1&1&1\\
0&1&3\\
0&2&8
\end{pmatrix}=2
\end{align*}
所以$|B|=|AC|=|A||C|=2$
\end{jie}

\EX 判断下列说法是否正确,并说明理由。

(4)两个不相等的$n$阶方阵的伴随矩阵也不相等。

(5)存在$3$阶方阵,使得其伴随矩阵的秩为$2$。

(6)设线性方程组$Ax= \beta$的系数矩阵$A$是$n$阶方阵。如果
$|A|\neq 0$,则该方程组有唯一解$x=\dfrac{1}{|A|}A^*\beta$。

\begin{tips}
伴随矩阵的性质:

\begin{asparaenum}[(1)]
\item $A$可逆当且仅当$A^*$可逆.
\item 如果$A$可逆,则$A^*=|A|A^{-1}$
\item 对于$A^*$的秩有(证明见课本121页例3.3.23)
\begin{equation*}
r(A^*)
  \begin{cases}
    n,~~r(A)=n;\\
    1,~~r(A)=n-1;\\
    0,~~r(A)<n-1
  \end{cases}
\end{equation*}
\item $|A^*|=|A|^{n-1}$
\item $(kA)^*=k^{n-1}A^*$
\item 若$A$可逆,则$(A^{-1})^*=(A^*)^{-1}$
\item $(A^T)^*=(A^*)^T$
\item $(AB)^*=B^*A^*$
\item $AA^*=A^*A=|A|E$
\end{asparaenum}
\hphantom{.}
\end{tips}

\begin{jie}
(4)错误。例如:$A=
\begin{pmatrix}
1&0&0\\
0&0&0\\
0&0&0
\end{pmatrix},B=\begin{pmatrix}
0&0&0\\
0&1&0\\
0&0&0
\end{pmatrix}
$

(5)错误。理由:由性质3可看出。

(6)正确。理由:$A$的行列式不为0,则可逆,等式两边同时左乘$A^{-1}$:$x=A^{-1}\beta=\dfrac{1}{|A|}A^*\beta$。(性质2)
\end{jie}

\EX 设$A$是$3$阶方阵且$|A|=2$分别求$|(A^{-1})^*|$ 和
$|(A^*)^*|$的值.

\begin{jie}
此题说的性质指的是例3.24中给出的性质。

由性质6和性质2得:
\begin{equation*}
(A^{-1})^*=(A^*)^{-1}=(|A|A^{-1})^{-1}=\frac{A}{2}~~\Rightarrow~~|(A^{-1})^*|=\left|\frac{A}{2}\right|=\left(\frac{1}{2}\right)^3|A|=\frac{1}{4}
\end{equation*}

由性质2得:$(A^*)^*=(|A|A^{-1})^*=(2A^{-1})^*$,由性质5得:
\begin{equation*}
  (2A^{-1})^*=2^{3-1}(A^{-1})^*=4(A^{-1})^*~~~\Rightarrow~~~|(A^*)^*|=|4(A^{-1})^*|=4^3\times\frac{1}{4}=16
\end{equation*}
\end{jie}

\EX 设$A$是3阶方阵且它的伴随矩阵为:$A^*=
\begin{pmatrix}
1&-1&2\\
-1&1&-2\\
3&-3&6
\end{pmatrix}$,求齐次线性方程组$Ax=0$的通解。

\begin{jie}
此题说的性质指的是例3.24中给出的性质。

由题得:$r(A^*)=1$,所以由性质3得:$r(A)=2<3$,所以$|A|=0$。

所以$Ax=0$有无穷多解。由性质9得:$AA^*=|A|I=0$,所以$A^*$中的每一列向量均为$Ax=0$的解。任取$A^*$中的一列(以第一列为例)$\xi=(1,-1,3)^T$,则$Ax=0$的通解为$x=k\xi,k\in R$。
\end{jie}

\EX 设矩阵$A,B$满足$A^*BA=2BA-8E$,其中$E$是3阶单位矩阵,$A^*$是$A$的伴随矩阵,$
A=
\begin{pmatrix}
1&0&0\\
0&-2&0\\
0&0&1
\end{pmatrix}
$,则矩阵$B=$\underline{\hphantom{~~~~~~~~}}。

\begin{jie}
$|A|=-2,A^*=|A|A^{-1}=-2A^{-1}$。所以:
\begin{align*}
A^*BA=-2A^{-1}BA=2BA-8E~\rightarrow~2BA+2A^{-1}BA=8E~\rightarrow~(E+A^{-1})BA=4E
\end{align*}
等式两边同时左乘$(E+A^{-1})$,右乘$A^{-1}$:(下式中用到了可逆矩阵的性质:$(AB)^{-1}=B^{-1}A^{-1}$)
%\begin{equation*}
%  |(E+A)^{-1}BA|=|(E+A)^{-1}||B||A|=|E+A|^{-1}|B||A|=4
%\end{equation*}
%式中:
\begin{equation*}
B=4(E+A^{-1})^{-1}A^{-1}=4[A(E+A^{-1})]^{-1}=4(A+E)^{-1}
\end{equation*}
\begin{equation*}
A+E=
\begin{pmatrix}
2\\
&-1\\
&&2
\end{pmatrix}
\end{equation*}
对角阵的逆等于对角线上每个元素的倒数,所以:
\begin{equation*}
B=4(A+E)^{-1}=4
\begin{pmatrix}
\frac{1}{2}\\
&-1\\
&&\frac{1}{2}
\end{pmatrix}=
\begin{pmatrix}
2\\
&-4\\
&&2
\end{pmatrix}
\end{equation*}
\end{jie}

\EX 设3阶方阵$A,B$满足$(A^*)^{-1}B=ABA+2A^2$,已知$
A=
\begin{pmatrix}
1&2&0\\
2&3&0\\
1&2&3
\end{pmatrix}
$,求$B$。

\begin{jie}
由题得$|A|=3\times (1\times 3-2\times 2)=-3\neq 0$,所以$A$可逆。所以$A^*=|A|A^{-1}=-3A^{-1}$。所以:
\begin{equation*}
(A^*)^{-1}B=(-3)^{-1}AB=ABA+2A^2~\Rightarrow~AB\left(-\frac{1}{3}E-A\right)=2A^2~\Rightarrow~B=2A\left(-\frac{1}{3}E-A\right)^{-1}=-6A(3A+E)^{-1}
\end{equation*}
所以(求逆的过程略):
\begin{equation*}
3A+E=
\begin{pmatrix}
4&6&0\\
6&10&0\\
3&6&10
\end{pmatrix}~~\Rightarrow~(3A+E)^{-1}=
\begin{pmatrix}
2.5&-1.5&0\\
-1.5&1&0\\
0.15&-0.15&0.1
\end{pmatrix}~\Rightarrow~B=-6A(3A+E)^{-1}=
\begin{pmatrix}
-3&-3&0\\
-3&0&0\\
0.3&-0.3&-1.8
\end{pmatrix}
\end{equation*}
\end{jie}

\EX 设$A$是$n(n\geq2)$阶可逆矩阵,交换$A$的第$1$行与
第$2$行得矩阵$B$,$A^*$,$B^*$分别为$A$,$B$的伴随矩阵,则下列说
法正确的是( )

(A)交换$A^*$的第1列与第2列得$B^*$.

(B)交换$A^*$的第1行与第2行得$B^*$.

(C)交换$A^*$的第1列与第2列得$-B^*$.

(D)交换$A^*$的第1行与第2行得$-B^*$.

\begin{jie}
由课本83页定理3.1.1得:$B=E_{12}A$,所以$
B^*=(E_{12}A)^*=A^*E_{12}^*
$.式中:
\begin{equation*}
E_{12}=
\begin{pmatrix}
0&1&0\\
1&0&0\\
0&0&1
\end{pmatrix}\Rightarrow~|E_{12}|=-1~~\Rightarrow~E_{12}^*=|E_{12}|E_{12}^{-1}=-E_{12}^{-1}=-E_{12}
\end{equation*}
所以$B^*=A^*(-E_{12})=-A^*E_{12}$,即选项C。
\end{jie}

\EX 已知$Q=
\begin{pmatrix}
1&2&3\\
2&4&t\\
3&6&9
\end{pmatrix}
$,
$P$为3阶非零矩阵,且满足
$PQ=0$.则下列说法正确的是( )

(A) $t=6$时,$P$的秩必为1.

(B) $t=6$时,$P$的秩必为2.

(C)$t\neq 6$时,$P$的秩必为1.

(D)$t\neq6$时,$P$的秩必为2.

\begin{jie}
$P$为非零矩阵,则必有$r(P)\geq 1$。

因为$PQ=0$,所以$r(P)+r(Q)\leq n=3$。(矩阵秩的性质,第4次线性代数作业答案例2.26中给出的第8条)

(1)$t=6$时,$r(Q)=1$,代入上边两个不等式:$1\leq r(P)\leq 3-1=2$,所以$r(P)=1$或$2$。

(2)$t\neq 6$时,$r(Q)=2$,代入上边两个不等式:$1\leq r(P)\leq 3-2=1$,所以$r(P)=1$。

综上,选$C$。
\end{jie}

\EX 设$A=
\begin{pmatrix}
1&2&-2\\
4&t&3\\
3&-1&1
\end{pmatrix}
$,$B$为3阶非零矩阵,且$AB=0$。则$t=$\underline{\hphantom{~~~~~~~~}}。

\begin{jie}
由题得:$B$为非零矩阵,而$AB=0$,说明$B$的非零列向量为$Ax=0$的非零解,即$Ax=0$有非零解,即$|A|=0$。
\begin{equation*}
|A|\xlongequal{r_1+2r_3}
\begin{vmatrix}
7&0&0\\
4&t&3\\
3&-1&1
\end{vmatrix}=7(t+3)=0~~~\Rightarrow~~t=-3
\end{equation*}
\end{jie}

\stepcounter{chapter}
\clearpage
\hphantom{~~}\hfill {\zihao{3}\heiti 第八次线性代数} \hfill\hphantom{~~}
\addcontentsline{toc}{section}{\protect\numberline {}第八次线性代数}

\hphantom{~~}

%\chapter{第八次线性代数}
\EX 计算:
\begin{equation*}
\frac{1}{2}
\begin{pmatrix}
-2\\ 1\\ -3
\end{pmatrix}-3
\begin{pmatrix}
0\\ -1\\ 2
\end{pmatrix}+
\begin{pmatrix}
1\\ -1\\ -3
\end{pmatrix}
\end{equation*}

\begin{jie}
原式=
\begin{equation*}
\begin{pmatrix}
-1\\ 0.5\\ -1.5
\end{pmatrix}-
\begin{pmatrix}
0\\ -3\\ 6
\end{pmatrix}+
\begin{pmatrix}
1\\ -1\\ -3
\end{pmatrix}=\begin{pmatrix}
0\\ 2.5\\ -10.5
\end{pmatrix}
\end{equation*}
\end{jie}

\EX 写出如下矩阵的行向量组和列向量组。
\begin{equation*}
A=
\begin{pmatrix}
-1 &0&2&1\\ -3& 1&1&-2\\ 0&0&-1&2
\end{pmatrix},~~B=
\begin{pmatrix}
-1&2\\ 3&-2\\ 4&-5
\end{pmatrix}
\end{equation*}

\begin{jie}
$A$行向量组:
\begin{equation*}
\begin{pmatrix}
-1&0&2&1
\end{pmatrix},
\begin{pmatrix}
-3& 1&1&-2
\end{pmatrix},
\begin{pmatrix}
0&0&-1&2
\end{pmatrix}
\end{equation*}

$A$列向量组:
\begin{equation*}
\begin{pmatrix}
-1\\ -3\\ 0
\end{pmatrix},
\begin{pmatrix}
0\\ 1\\ 0
\end{pmatrix},
\begin{pmatrix}
2\\ 1\\ -1
\end{pmatrix},
\begin{pmatrix}
1\\ -2\\ 2
\end{pmatrix}
\end{equation*}

$B$行向量组:
\begin{equation*}
\begin{pmatrix}
-1&2
\end{pmatrix},
\begin{pmatrix}
3&-2
\end{pmatrix},
\begin{pmatrix}
4&-5
\end{pmatrix}
\end{equation*}

$B$列向量组:
\begin{equation*}
\begin{pmatrix}
-1\\ 3\\ 4
\end{pmatrix},
\begin{pmatrix}
2\\ -2\\ -5
\end{pmatrix}
\end{equation*}
\end{jie}

\EX 设$\beta$可以由$\alpha_1,\alpha_2,\alpha_3$线性表出,而每个$\alpha_i$都可以由$\gamma_1,\gamma_2$线性表出。证明$\beta$可以由$\gamma_1,\gamma_2$线性表出。

\begin{zhengming}
$\beta$可以由$\alpha_1,\alpha_2,\alpha_3$线性表出,即存在一组不全为0的数$k_i(1\leq i\leq3)$,使得:
\begin{equation*}
\beta = k_1\alpha_1+k_2\alpha_2+\k_3\alpha_3\tag{1}
\end{equation*}
同理,存在一组不全为0的数$l_{ij}(1\leq i\leq 3,1\leq j\leq 2)$使得:
\begin{equation*}
  \begin{cases}
    \alpha_1=l_{11}\gamma_{1}+l_{12}\gamma_{2}\\
    \alpha_2=l_{21}\gamma_{1}+l_{22}\gamma_{2}\\
    \alpha_3=l_{31}\gamma_{1}+l_{32}\gamma_{2}
  \end{cases}\tag{2}
\end{equation*}
把(2)代入(1)得:
\begin{equation*}
\beta=(k_1l_{11}+k_2l_{21}+k_3l_{31})\gamma_{1}+(k_1l_{12}+k_2l_{22}+k_3l_{32})\gamma_{2}
\end{equation*}
所以$\beta$可以由$\gamma_1,\gamma_2$线性表出。
\end{zhengming}

\EX 设$\alpha_1,\alpha_2,\alpha_3,\alpha_4$是3维向量,证明:3维零向量$0$由$\alpha_1,\alpha_2,\alpha_3,\alpha_4$线性表出的方式有无穷多。

\begin{zhengming}
对于线性方程组:
\begin{equation*}
\alpha_1x_1+\alpha_2x_2+\alpha_3x_3+\alpha_4x_4=0\tag{1}
\end{equation*}
$r(\alpha_1,\alpha_2,\alpha_3,\alpha_4)\leq\min\{3,4\}=3$,所以$(1)$有无穷多组解,即3维零向量$0$由$\alpha_1,\alpha_2,\alpha_3,\alpha_4$线性表出的方式有无穷多。
\end{zhengming}

\EX 设$\alpha_1,\alpha_2,\alpha_3$是3维向量,且3维零向量$0$由$\alpha_1,\alpha_2,\alpha_3$线性表出的方式是唯一的。在每个$\alpha_i$的第3个分量后任意添加两个分量,得到5维向量$\widetilde{\alpha}_{i}(1\leq i\leq 3)$.证明:5维零向量$0$由$\widetilde{\alpha}_{1},\widetilde{\alpha}_{2},\widetilde{\alpha}_{3}$线性表出的方式仍然是唯一的。

\begin{zhengming}
课本36页命题2.2.1和例题2.2.3.

考虑齐次线性方程组:
  %\uppercase\expandafter{\romannumeral1}
\begin{gather*}
\alpha_1x_1+\alpha_2x_2+\alpha_3x_3=0\tag{1}\\
\widetilde{\alpha} _{1}x_1+\widetilde{\alpha}_{2}x_2+\widetilde{\alpha}_{3}x_3=0\tag{2}
\end{gather*}
(1)中的每个方程都是(2)的方程,所以(2)的解集是(1)的解集的子集,对于任意齐次方程,其一定有零解。

由题得,线程方程组(1)只有零解,因此(2)也只有零解,即5维零向量$0$由$\widetilde{\alpha}_{1},\widetilde{\alpha}_{2},\widetilde{\alpha}_{3}$线性表出的方式仍然是唯一的。
\end{zhengming}

\EX 设$\alpha_1,\alpha_2,\alpha_3$是向量组,设
\begin{gather*}
  \beta_1=-\alpha_1+2\alpha_2-3\alpha_3 \\
  \beta_2=3\alpha_1-4\alpha_2+6\alpha_3 \\
  \beta_3=2\alpha_1-2\alpha_2+3\alpha_3
\end{gather*},判断$\beta_1,\beta_2,\beta_3$是否线性相关。

\begin{jie}
方法一:

由题得:$\beta_3=\beta_1+\beta_2$,所以$\beta_1,\beta_2,\beta_3$线性相关。

方法二:通用解法

设:$x_1\beta_1+x_2\beta_2+x_3\beta_3=0$,若该线性方程组有非零解,则线性相关,否则线性无关。
\begin{align*}
&x_1\beta_1+x_2\beta_2+x_3\beta_3=x_1(-\alpha_1+2\alpha_2-3\alpha_3)+x_2(3\alpha_1-4\alpha_2+6\alpha_3)+x_3(2\alpha_1-2\alpha_2+3\alpha_3)\\
=&(-x_1+3x_2+2x_3)\alpha_1+(2x_1-4x_2-2x_3)\alpha_2+(-3x_1+6x_2+3x_3)\alpha_3=0
\end{align*}
即解如下线性方程组:
\begin{equation*}
\begin{cases}
-x_1+3x_2+2x_3=0\\
2x_1-4x_2-2x_3=0\\
-3x_1+6x_2+3x_3=0
\end{cases}
\end{equation*}
高斯消元(步骤略)得:
\begin{equation*}
\begin{cases}
x_1+x_3=0\\
x_2+x_3=0\\
0=0
\end{cases}
\end{equation*}
有自由变量即该线性方程组有非零解,即$\beta_1,\beta_2,\beta_3$线性相关。取$x_3=k,k\neq 0$,则:$k\beta_3=k\beta_1+k\beta_2$
\end{jie}

\EX 举例说明,把两个线性无关的$m$维向量组放在一起,得到的向量组可以是线性无关的,也可以是线性相关的。

\begin{jie}
$\alpha_1=
\begin{pmatrix}
1\\ 0
\end{pmatrix}
,\alpha_2=
\begin{pmatrix}
0\\ 1
\end{pmatrix},\alpha_3=
\begin{pmatrix}
1\\ 0
\end{pmatrix},\alpha_4=
\begin{pmatrix}
1\\ 1
\end{pmatrix}$.

(1)向量组1为$(\alpha_1,\alpha_2)$,向量组2为$(\alpha_3,\alpha_4)$,显然向量组1是线性无关的,向量组2是线性无关的,把两个向量组放在一起:$(\alpha_1,\alpha_2,\alpha_3,\alpha_4)$,线性相关(向量组的维数大于向量的维数)。

(2)向量组3为$(\alpha_1)$,向量组4为$(\alpha_2)$,向量组3线性无关,向量组4线性无关,把两个向量组放到一起即向量组1,线性无关。
\end{jie}

\EX 已知向量组$\alpha_1,\alpha_2,\alpha_3,\alpha_4$线性无关,则下列说法正确的是哪些,并说明理由。

(A)向量组$\alpha_1+\alpha_2,\alpha_2+\alpha_3,\alpha_3+\alpha_4,\alpha_4+\alpha_1$线性无关。

(B)向量组$\alpha_1-\alpha_2,\alpha_2-\alpha_3,\alpha_3-\alpha_4,\alpha_4-\alpha_1$线性无关。

(C)向量组$\alpha_1+\alpha_2,\alpha_2+\alpha_3,\alpha_3+\alpha_4,\alpha_4-\alpha_1$线性无关。

(D)向量组$\alpha_1+\alpha_2,\alpha_2+\alpha_3,\alpha_3-\alpha_4,\alpha_4-\alpha_1$线性无关。

\begin{jie}

由题得:

(A)$(\alpha_1+\alpha_2)+ (\alpha_3+\alpha_4)=(\alpha_2+\alpha_3)+(\alpha_4+\alpha_1)$,即线性相关。

(B)$(\alpha_1-\alpha_2)+ (\alpha_2-\alpha_3) =-(\alpha_3-\alpha_4)-(\alpha_4-\alpha_1)$,即线性相关。

(C)设$k_1(\alpha_1+\alpha_2)+k_2(\alpha_2+\alpha_3)+k_3(\alpha_3+\alpha_4)+k_4(\alpha_4-\alpha_1)=0$,则
\begin{equation*}
  (k_1-k_4)\alpha_1+(k_1+k_2)\alpha_2+(k_2+k_3)\alpha_3+(k_3+k_4)\alpha_4=0
\end{equation*}
因为$\alpha_1,\alpha_2,\alpha_3,\alpha_4$线性无关,所以:
\begin{equation*}
\begin{cases}
k_1-k_4=0\\
k_1+k_2=0\\
k_2+k_3=0\\
k_3+k_4
\end{cases}
\end{equation*}
解得:$k_1=k_2=k_3=k_4=0$,即$\alpha_1+\alpha_2,\alpha_2+\alpha_3,\alpha_3+\alpha_4,\alpha_4-\alpha_1$线性无关。

(D)$(\alpha_1+\alpha_2)+(\alpha_4-\alpha_1)=(\alpha_2+\alpha_3)-(\alpha_3-\alpha_4)$即线性相关。
\end{jie}

\EX 设有三维列向量
\begin{gather*}
\alpha_1=
\begin{pmatrix}
1+\lambda&1&1
\end{pmatrix}^T,\alpha_2=
\begin{pmatrix}
1&1+\lambda&1
\end{pmatrix}^T\\
\alpha_3=
\begin{pmatrix}
1&1&1+\lambda
\end{pmatrix}^T,\beta=
\begin{pmatrix}
0&\lambda&\lambda^2
\end{pmatrix}^T
\end{gather*}
问$\lambda$取何值时:

(1)$\beta$可由$\alpha_1,\alpha_2,\alpha_3$线性表出,且表达式唯一。

(2)$\beta$可由$\alpha_1,\alpha_2,\alpha_3$线性表出,且表达式不唯一。

(3)$\beta$不可由$\alpha_1,\alpha_2,\alpha_3$线性表出。

\begin{jie}
记$A=\begin{pmatrix}
  \alpha_1&\alpha_2&\alpha_3
\end{pmatrix}$,则:
\begin{align*}
|A|=&\begin{vmatrix}
1+\lambda&1&1\\
1&1+\lambda&1\\
1&1&1+\lambda
\end{vmatrix}\xlongequal{c_{1}+c_{2}+c_3}
\begin{vmatrix}
3+\lambda&1&1\\
3+\lambda&1+\lambda&1\\
3+\lambda&1&1+\lambda
\end{vmatrix}=(3+\lambda)
\begin{vmatrix}
1&1&1\\
1&1+\lambda&1\\
1&1&1+\lambda
\end{vmatrix}\\
\xlongequal{\substack{r_2-r_1 \\ r_3-r_1}}&
(3+\lambda)\begin{vmatrix}
1&1&1\\
0&\lambda&0\\
0&0&\lambda
\end{vmatrix}=\lambda^2(3+\lambda)
\end{align*}

(1)当$|A|\neq 0$时,$r(A)=r(A,\beta)$,即$\beta$可由$\alpha_1,\alpha_2,\alpha_3$线性表出,且表达式唯一。此时$\lambda\neq0$且$\lambda\neq -3$

(2)当$\lambda=0$时,$r(A)=r(A,\beta)=1<3$,此时$\beta$可由$\alpha_1,\alpha_2,\alpha_3$线性表出,且表达式不唯一。

(3)当$\lambda=-3$时,$r(A)=1,r(A,\beta)=2$,$r(A)\neq r(A,\beta)$,此时$\beta$不可由$\alpha_1,\alpha_2,\alpha_3$线性表出。
\end{jie}

\EX 设$\alpha_1,\alpha_2,\alpha_3$线性相关,$\alpha_2,\alpha_3,\alpha_4$线性无关。问:

(1)$\alpha_1$能否由$\alpha_2,\alpha_3$线性表出?证明你的结论。

(2)$\alpha_4$能否由$\alpha_1,\alpha_2,\alpha_3$线性表出,证明你的结论。

\begin{zhengming}
(1)
因为$\alpha_2,\alpha_3,\alpha_4$线性无关,所以$\alpha_2,\alpha_3$线性无关,又因为$\alpha_1,\alpha_2,\alpha_3$线性相关,所以$\alpha_1$能由$\alpha_2,\alpha_3$线性表出且表出方式是唯一的。

(2)不能。

由(1)可得:$\alpha_1=k_1\alpha_2+k_2\alpha_3$。假设$\alpha_4$能由$\alpha_1,\alpha_2,\alpha_3$线性表出,则$\alpha_4=l_1\alpha_1+l_2\alpha_2+l_3\alpha_3=l_1(k_1\alpha_2+k_2\alpha_3)+l_2\alpha_2+l_3\alpha_3$,即$\alpha_4$可由$\alpha_2,\alpha_3$线性表出,即$\alpha_2,\alpha_3,\alpha_4$线性相关,与题目矛盾,所以假设不成立,即$\alpha_4$不能由$\alpha_1,\alpha_2,\alpha_3$线性表出。
\end{zhengming}

\EX 设
$\alpha_1=
\begin{pmatrix}
1&4&0&2
\end{pmatrix}^T,
\alpha_2=
\begin{pmatrix}
2&7&1&3
\end{pmatrix}^T,
\alpha_3=
\begin{pmatrix}
0&1&-1&a
\end{pmatrix}^T
$及
$
\beta=
\begin{pmatrix}
3&10&b&4
\end{pmatrix}^T
$.

(1)$a,b$为何值时,$\beta$不能表示成$\alpha_1,\alpha_2,\alpha_3$的线性组合?

(2)$a,b$为何值时,$\beta$可由$\alpha_1,\alpha_2,\alpha_3$线性表示?并写出该表示式。

\begin{jie}
记$A=(\alpha_1,\alpha_2,\alpha_3)$,设$Ax=\beta$,则:
\begin{align*}
(A|\beta)=
&
\left(
 \begin{array}{c:c}
\begin{matrix}
1 & 2 & 0\\
4 & 7 & 1 \\
0 & 1 & -1\\
2 & 3 & a
\end{matrix}&
\begin{matrix}
3  \\
10  \\
b \\
4
\end{matrix}
\end{array}
\right)\xrightarrow{\substack{r_{2}-4r_{1}\\ r_{4}-2r_{1}}}
{
\left(
 \begin{array}{c:c}
\begin{matrix}
1 & 2 & 0\\
0 & -1 & 1 \\
0 & 1 & -1\\
0 & -1 & a
\end{matrix}&
\begin{matrix}
3  \\
-2  \\
b \\
-2
\end{matrix}
\end{array}
\right)
}\xrightarrow{\substack{r_{3}-r_{2}\\ r_{4}-r_{2}}}
{
\left(
 \begin{array}{c:c}
\begin{matrix}
1 & 2 & 0\\
0 & -1 & 1 \\
0 & 0 & 0\\
0 & 0 & a-1
\end{matrix}&
\begin{matrix}
3  \\
-2  \\
b-2 \\
0
\end{matrix}
\end{array}
\right)}\\
&\xrightarrow{\substack{r_{3}\leftrightarrow r_{4}}}
{
\left(
 \begin{array}{c:c}
\begin{matrix}
1 & 2 & 0\\
0 & -1 & 1 \\
0 & 0 & a-1\\
0 & 0 & 0
\end{matrix}&
\begin{matrix}
3  \\
-2  \\
0\\
b-2
\end{matrix}
\end{array}
\right)
}
\end{align*}

(1)由阶梯矩阵可以看出,$b\neq 2$时,$r(A,\beta)\neq r(A)$,此时$\beta$不能表示成$\alpha_1,\alpha_2,\alpha_3$的线性组合。

(2)当$b=2$时,$r(A,\beta)= r(A)$,$\beta$可由$\alpha_1,\alpha_2,\alpha_3$线性表示。

$a-1=0$即$a=1$时,继续对上述阶梯矩阵进行化简:\begin{equation*}
\xrightarrow{\substack{r_{1}+2 r_{2}}}
{
\left(
 \begin{array}{c:c}
\begin{matrix}
1 & 0 & 2\\
0 & -1 & 1 \\
0 & 0 & 0\\
0 & 0 & 0
\end{matrix}&
\begin{matrix}
-1 \\
-2  \\
0\\
0
\end{matrix}
\end{array}
\right)
}\xrightarrow{\substack{r_{2}\times(-1)}}
{
\left(
 \begin{array}{c:c}
\begin{matrix}
1 & 0 & 2\\
0 & 1 & -1 \\
0 & 0 & 0\\
0 & 0 & 0
\end{matrix}&
\begin{matrix}
-1 \\
2  \\
0\\
0
\end{matrix}
\end{array}
\right)
}
\end{equation*}
由最简阶梯型矩阵可以看出:$x_1=-1-2x_3,x_2=2+x_3$,取$x_3=k$所以$\beta=(-1-2k)\alpha_{1}+(2+k)\alpha_{2}+k\alpha_3,(k\in R)$。

$a-1\neq 0$时:
\begin{equation*}
\xrightarrow{\substack{r_{2}\times(-1)\\r_{3}\times \frac{1}{a-1}}}
{
\left(
 \begin{array}{c:c}
\begin{matrix}
1 & 2 & 0\\
0 & 1 & -1 \\
0 & 0 & 1\\
0 & 0 & 0
\end{matrix}&
\begin{matrix}
3 \\
2  \\
0\\
0
\end{matrix}
\end{array}
\right)
}\xrightarrow{\substack{r_{2}+r_{3}}}
{
\left(
 \begin{array}{c:c}
\begin{matrix}
1 & 2 & 0\\
0 & 1 & 0 \\
0 & 0 & 1\\
0 & 0 & 0
\end{matrix}&
\begin{matrix}
3 \\
2  \\
0\\
0
\end{matrix}
\end{array}
\right)
}\xrightarrow{\substack{r_{1}-2r_{2}}}
{
\left(
 \begin{array}{c:c}
\begin{matrix}
1 & 0 & 0\\
0 & 1 & 0 \\
0 & 0 & 1\\
0 & 0 & 0
\end{matrix}&
\begin{matrix}
-1 \\
2  \\
0\\
0
\end{matrix}
\end{array}
\right)
}
\end{equation*}
由最简阶梯型矩阵可以看出,$x_1=-1,x_2=2,x_3=0$,所以$\beta=-\alpha_1+2\alpha_2$。
\end{jie}

\EX 设3阶矩阵
$
A=
\begin{pmatrix}
1&2&-2\\ 2&1&2\\ 3&0&4
\end{pmatrix}
$,三维列向量$\alpha=
\begin{pmatrix}
a&1&1
\end{pmatrix}^T
$。已知$A\alpha$与$\alpha$线性相关,求$a$。

\begin{jie}
由题得:
\begin{equation*}
A\alpha=\begin{pmatrix}
1&2&-2\\ 2&1&2\\ 3&0&4
\end{pmatrix}\begin{pmatrix}
a\\ 1\\ 1
\end{pmatrix}=\begin{pmatrix}
a\\ 2a+3 \\ 3a+4
\end{pmatrix}
\end{equation*}
$A\alpha$与$\alpha$线性相关,即$A\alpha$与$\alpha$对应元素成比例:
\begin{equation*}
\frac{a}{a}=\frac{1}{2a+3}=\frac{1}{3a+4}~~~~\Rightarrow~~~~a=-1
\end{equation*}
\end{jie}

\EX 设向量组
$
\begin{pmatrix}
2&1&1&1
\end{pmatrix},
\begin{pmatrix}
2&1&a&a
\end{pmatrix},
\begin{pmatrix}
3&2&1&a
\end{pmatrix},
\begin{pmatrix}
4&3&2&1
\end{pmatrix}
$线性相关,且$a\neq 1$,则$a=$\underline{\hphantom{~~~~~~~~}}。

\begin{jie}
分别记题目中的四个向量为$\alpha_1,\alpha_2,\alpha_3,\alpha_4$,记$A=
\begin{pmatrix}
\alpha_1^T&\alpha_2^T&\alpha_3^T&\alpha_4^T
\end{pmatrix}
$。

方法1:求秩。向量组线性相关,有$r(A)<4$。
\begin{equation*}
A\xrightarrow{\substack{r_{2}-\frac{1}{2}r_{1}\\ r_{3}-\frac{1}{2}r_{1}\\ r_{4}-\frac{1}{2}r_{1}}}
{
\begin{pmatrix}
2& 2&3&4\\ 0&0&\frac{1}{2}&1\\ 0&a-1&-\frac{1}{2}&0\\ 0&a-1&a-\frac{3}{2}&-1
\end{pmatrix}
}\xrightarrow{\substack{r_{4}-r_{3}}}
{
\begin{pmatrix}
2& 2&3&4\\ 0&0&\frac{1}{2}&1\\ 0&a-1&-\frac{1}{2}&0\\ 0&0&a-1&-1
\end{pmatrix}
}
\end{equation*}
要使$r(A)<4$,则第二行第四行成比例,即
\begin{equation*}
\begin{cases}
\frac{\frac{1}{2}}{a-1}=\frac{1}{-1}\\
a\neq 1
\end{cases}~~~a=0.5
\end{equation*}

方法2:$r(A)<4$,即$|A|=0$。

\begin{align*}
|A|&\xlongequal{\substack{r_1-r_4 \\ r_2-r_4\\ r_3-r_4}}
\begin{vmatrix}
0&2(1-a)&3-2a&2\\ 0&1-a&2-a&2\\ 0&0&1-a&1\\ 1&a&a&1
\end{vmatrix}
\xlongequal{\substack{r_1-2r_2}}
\begin{vmatrix}
0&0&-1&-2\\ 0&1-a&2-a&2\\ 0&0&1-a&1\\ 1&a&a&1
\end{vmatrix}=-\begin{vmatrix}
0&-1&-2\\ 1-a&2-a&2\\ 0&1-a&1
\end{vmatrix}\\ &=(1-a)\begin{vmatrix}
-1&-2\\ 1-a&1
\end{vmatrix}=(1-a)[-1+2(1-a)]=0
\end{align*}
又因为$a\neq 1$,所以$-1+2(1-a)=0$,解得$a=\dfrac{1}{2}$.
\end{jie}

\EX 判断下列说法是否正确,并简要说明理由.

(1)因为
$
\begin{pmatrix}
0\\ 0
\end{pmatrix},\begin{pmatrix}
               0\\ 4\\ -2
              \end{pmatrix}
$含有零向量,所以,线性相关.

(2)在一个线性相关的向量组中,每个向量都可以由其余的向量线性表出.

(3) 设$\alpha_1,\alpha_2,\alpha_3,\alpha_4$线性相关,则$\alpha_4,\alpha_2,\alpha_1,\alpha_3$也是线性相关的.

(4)如果一个向量组去掉它的任意一个向量后得到的向量组都是线性无关的,则该向量组是线性无关的.

\begin{jie}
(1)错误,理由:线性相关无关的前提是向量的维数一致。

(2)错误,例如对于向量组$\alpha_1=
\begin{pmatrix}
1\\ 0
\end{pmatrix}
,\alpha_2=
\begin{pmatrix}
0\\ 1
\end{pmatrix},\alpha_3=
\begin{pmatrix}
1\\ 0
\end{pmatrix}$,线性相关,但$\alpha_2$不能由其他向量线性表出。

(3)正确。理由:向量组线性相关无关与该向量组中各向量所处的位置无关。

(4)错误。例如向量组$\alpha_1=
\begin{pmatrix}
1\\ 0
\end{pmatrix}
,\alpha_2=
\begin{pmatrix}
0\\ 1
\end{pmatrix},\alpha_3=
\begin{pmatrix}
1\\ 1
\end{pmatrix}$,不难验证,去掉其中任意一个向量后向量组线性无关,而原向量组是线性相关的。
\end{jie}

\EX 判断下列说法是否正确,并简要说明理由.

(5)如果存在不全为$0$的数$k_1,\cdots,k_s$使得
\begin{equation*}
  k_1\alpha_1+\cdots+k_s\alpha_s\neq0
\end{equation*}
则$\alpha_1,\cdots,\alpha_s$线性无关.

(6)如果一个向量组的分量不成比例,则一定线性无关.

(7)向量组$
\begin{pmatrix}
a_1\\ a_2\\ a_3
\end{pmatrix},\begin{pmatrix}
b_1\\ b_2\\ b_3
\end{pmatrix},\begin{pmatrix}
c_1\\ c_2\\ c_3
\end{pmatrix},\begin{pmatrix}
d_1\\ d_2\\ d_3
\end{pmatrix}
$
有可能是线性无关的。

\begin{tips}
区分向量的维数和向量组的维数:

向量的维数:一个向量中所含元素的个数。

向量组的维数:一个向量组中所含向量的个数。
\end{tips}

\begin{jie}
(5)错误。例如对于$\alpha_1=
\begin{pmatrix}
1\\ 1
\end{pmatrix}
,\alpha_2=
\begin{pmatrix}
2\\ 2
\end{pmatrix}$,有$\alpha_1+\alpha_2\neq 0$,但 $\alpha_1,\alpha_2$线性相关。

(6)错误。例如$
\begin{pmatrix}
1\\ 0\\ 1
\end{pmatrix}
,\alpha_2=
\begin{pmatrix}
0\\ 1 \\2
\end{pmatrix},\alpha_3=
\begin{pmatrix}
1\\ 1 \\3
\end{pmatrix}$,显然这些向量不成比例,但$\alpha_3=\alpha_1+\alpha_2$,即线性相关。

(7)错误。理由:向量组的维数大于向量的维数向量组一定线性相关。
\end{jie}
\EX 判断下列说法是否正确,并简要说明理由。

(1) 设向量组(\uppercase\expandafter{\romannumeral1}) 可以由(\uppercase\expandafter{\romannumeral2})的一个子组线性表出,则(\uppercase\expandafter{\romannumeral1}) 可以由(\uppercase\expandafter{\romannumeral2}) 线性表出.

(2)设向量组(\uppercase\expandafter{\romannumeral1})可以由(\uppercase\expandafter{\romannumeral2}) 线性表出.如果(\uppercase\expandafter{\romannumeral1}) 线性相关,则(\uppercase\expandafter{\romannumeral1}) 所包含的向量个数大于(\uppercase\expandafter{\romannumeral2}) 所包含的向量个数.

(3)如果两个向量组是等价的,则它们要么都是线性相关的,要么都是线性无关的.

(4)如果一个向量组线性无关,那么它不可能与它的任意真子组等价. (真子组是除去若干个向量后得到的子组.)

(5) 如果$\alpha_1,\cdots,\alpha_n$与$\beta_1,\cdots,\beta_n$是等价的,则齐次线性方程组$x_1\alpha_1+\cdots+x_n\alpha_n=0$与
$x_1\beta_1+\cdots+x_n\beta_n$的解集相同.

\begin{jie}
(1)正确。理由:(\uppercase\expandafter{\romannumeral1}) 可以由(\uppercase\expandafter{\romannumeral2})的一个子组(\uppercase\expandafter{\romannumeral3})线性表出,而(\uppercase\expandafter{\romannumeral3})可以由(\uppercase\expandafter{\romannumeral2})线性表出,线性表出具有传递性,所以(\uppercase\expandafter{\romannumeral1}) 可以由(\uppercase\expandafter{\romannumeral2}) 线性表出。

(2)错误。理由:向量组(\uppercase\expandafter{\romannumeral1})$\alpha_1=
\begin{pmatrix}
1\\ 0\\ 1
\end{pmatrix}
,\alpha_2=
\begin{pmatrix}
2\\ 0 \\2
\end{pmatrix}$,向量组(\uppercase\expandafter{\romannumeral2})$\beta_1=
\begin{pmatrix}
1\\ 0\\ 0
\end{pmatrix}
,\beta_2=
\begin{pmatrix}
0\\ 1 \\0
\end{pmatrix},\beta_3=
\begin{pmatrix}
0\\ 0 \\1
\end{pmatrix}$,可以看出向量组(\uppercase\expandafter{\romannumeral1})可以由(\uppercase\expandafter{\romannumeral2}) 线性表出且(\uppercase\expandafter{\romannumeral1}) 线性相关,但(\uppercase\expandafter{\romannumeral1}) 所包含的向量个数小于(\uppercase\expandafter{\romannumeral2}) 所包含的向量个数.

(3)错误。理由:向量组等价,只能得到秩相等,但两向量组包含的向量个数未必相等。例如假设向量组(\uppercase\expandafter{\romannumeral1})和(\uppercase\expandafter{\romannumeral2})等价且都线性无关,此时向(\uppercase\expandafter{\romannumeral2})添加若干列零向量,此时(\uppercase\expandafter{\romannumeral1})和(\uppercase\expandafter{\romannumeral2})仍然等价,但(\uppercase\expandafter{\romannumeral2})变为了线性相关。

(4)正确。理由:反证法,假设该向量组可以与其一个真子组等价,依据等价的定义,该向量组中的向量可以由该真子组线性表出,即线性相关与向量组线性无关相矛盾,所以假设不成立,即该向量组不能与其任意真子组等价。

(5)错误。例如:取向量组(\uppercase\expandafter{\romannumeral1})$\alpha_1=
\begin{pmatrix}
1\\ 0
\end{pmatrix}
,\alpha_2=
\begin{pmatrix}
0 \\0
\end{pmatrix}$,向量组(\uppercase\expandafter{\romannumeral2})$\beta_1=
\begin{pmatrix}
0\\ 0
\end{pmatrix}
,\beta_2=
\begin{pmatrix}
1 \\0
\end{pmatrix}$则向量组(\uppercase\expandafter{\romannumeral1})与(\uppercase\expandafter{\romannumeral2})等价。但$x_1\alpha_1+x_2\alpha_2=0$解集是$x_1=(0,k),k\in R$,而$x_1\beta_1+x_2\beta_2=0$解集是$x_2=(k,0),k\in R$,显然$x_1\neq x_2$
\end{jie}
\clearpage
\hphantom{~~}\hfill {\zihao{3}\heiti 第九次线性代数} \hfill\hphantom{~~}
\addcontentsline{toc}{section}{\protect\numberline {}第九次线性代数}

\hphantom{~~}

%\section{第九次线性代数}
\EX 求向量组
\begin{equation*}
\alpha_1=
\begin{pmatrix}
1\\ -1\\ 2\\ -1
\end{pmatrix},
\alpha_2=
\begin{pmatrix}
-3\\ 2\\ -1\\ 0
\end{pmatrix},
\alpha_3=
\begin{pmatrix}
2\\ -1\\-1\\ 1
\end{pmatrix},
\alpha_4=
\begin{pmatrix}
-1\\ 1\\ -2\\ 1
\end{pmatrix}
\end{equation*}
的秩和一个极大线性无关组。

\begin{jie}
令$A=
\begin{pmatrix}
\alpha_1& \alpha_2& \alpha_3& \alpha_4
\end{pmatrix}
$则:
\begin{align*}
A\xrightarrow{\substack{r_{2}+r_1 \\ r_3-2r_1\\ r_4+r_1}}{
\begin{pmatrix}
1&-3&2&-1\\ 0&-1&1&0\\ 0&5&-5&0\\ 0&-3&3&0
\end{pmatrix}
}\xrightarrow{\substack{r_{3}+5r_2 \\ r_4-3r_2}}{
\begin{pmatrix}
1&-3&2&-1\\ 0&-1&1&0\\ 0&0&0&0\\ 0&0&0&0
\end{pmatrix}
}
\end{align*}

所以该向量组的秩为2,极大线性无关组为
$(\alpha_1,\alpha_2),(\alpha_1,\alpha_3),(\alpha_2,\alpha_3),(\alpha_2,\alpha_4),(\alpha_3,\alpha_4)$。 (依题意,任写一个即可。)
\end{jie}

\EX 判断下列说法是否正确,并简要说明理由.

(1) 如果一个向量组有且仅有一个极大无关组,则该向量组必然线性无关。

(2)在求向量组的秩时,如果该向量组含有零向量,则可以去掉零向量。

(3)在求向量组的秩时,如果该向量组含有一个可以由其余的向量线性表出的向量,则可以去掉这
个向量。

(4)在求向量组的秩时,如果该向量组含有线性相关的子组,则可以去掉线性相关的子组。

(5) 如果一个向量组含有$r$个线性无关的向量,则该向量组的秩至少是$r$。



\begin{jie}
(1)错误。例如向量组:
$\alpha_1=
\begin{pmatrix}
1\\ 0
\end{pmatrix},\alpha_2=
\begin{pmatrix}
0\\ 0
\end{pmatrix}
$只有一个极大无关组$\alpha_1$,但该向量组线性相关。

(2)正确。极大线性无关组中一定不含零向量,极大线性无关组中向量的个数为秩,所以去掉零向量对求秩无影响。

(3)正确。极大线性无关组中向量一定线性无关,所以去掉该向量对极大线性无关组无影响。

(4)错误。例如:
$\alpha_1=
\begin{pmatrix}
1\\ 0
\end{pmatrix},\alpha_2=
\begin{pmatrix}
0\\ 0
\end{pmatrix},\alpha_3=
\begin{pmatrix}
0\\ 0
\end{pmatrix}
$,若去掉线性相关的子组$\alpha_1,\alpha_2$,就只剩余$\alpha_3=0$,与原向量组的秩不相等了。

(5)正确。因为极大线性无关组包含了线性无关向量的最大个数,所以$r$小于等于该数,而极大线性无关组向量的个数即为秩,即秩大于等于$r$。
\end{jie}

\EX 判断下列说法是否正确,并简要说明理由.

(6) 如果一个向量组含有$r+1$个线性相关的向量,则该向量组的秩不超过$r$。

(7) 设$\alpha_1,\cdots,\alpha_n$是线性无关的$m$维向量.如果$n< m$,则存在$\alpha_{n+1},\cdots,\alpha_{m}$使得\\
$\alpha_1,\cdots,\alpha_n,\alpha_ {n+1},\cdots,\alpha_{m}$线性无关。

\begin{jie}
(6)错误。例如:$\alpha_1=
\begin{pmatrix}
1\\ 0
\end{pmatrix},\alpha_2=
\begin{pmatrix}
0\\ 1
\end{pmatrix},\alpha_3=
\begin{pmatrix}
0\\ 1
\end{pmatrix}
$可以看出该向量组含有两个线性相关的向量$\alpha_1,\alpha_2$,即$r+1=2,r=1$,但向量组的秩为2.

(7)正确。(\textcolor[rgb]{0.50,1.00,0.00}{ps:此题不会。})
\end{jie}
\EX 设向量组$\alpha_1=(1,4,0,2)^T,\alpha_2=(2,7,1,3)^T,\alpha_3=(0,1,-1,a)^T,\alpha_4=(3,10,b,4)^T$.已知$\alpha_1,\alpha_2,\alpha_3$是该向量组的一个极大无关组。求$a,b$的值,并把$\alpha_4$用$\alpha_1,\alpha_2,\alpha_3$线性表出。

\begin{jie}
由题得:$r(\alpha_1,\alpha_2,\alpha_3,\alpha_4)=r(\alpha_1,\alpha_2,\alpha_3)=3$.
\begin{equation*}
(\alpha_1,\alpha_2,\alpha_3,\alpha_4)
\xrightarrow{\substack{r_{2}-4r_{1}\\ r_4-2r_1}}
{
\begin{pmatrix}
1&2&0&3\\
0&-1&1&-2\\
0&1&-1&b\\
0&-1&a&-2
\end{pmatrix}
}\xrightarrow{\substack{r_{3}+r_{2}\\ r_4-r_2}}
{
\begin{pmatrix}
1&2&0&3\\
0&-1&1&-2\\
0&0&0&b-2\\
0&0&a-1&0
\end{pmatrix}
}\xrightarrow{\substack{r_{3}\leftrightarrow r_4}}
{
\begin{pmatrix}
1&2&0&3\\
0&-1&1&-2\\
0&0&a-1&0\\
0&0&0&b-2
\end{pmatrix}
}
\end{equation*}
因为$r(\alpha_1,\alpha_2,\alpha_3,\alpha_4)=r(\alpha_1,\alpha_2,\alpha_3)=3$,所以$b-2=0,a-1\neq0$.

把$a\neq1,b=2$代入上边继续高斯消元:
\begin{equation*}
\xrightarrow{\substack{r_2\times(-1)\\ r_{3}\times\frac{1}{a-1}}}
{
\begin{pmatrix}
1&2&0&3\\
0&1&-1&2\\
0&0&1&0\\
0&0&0&0
\end{pmatrix}
}\xrightarrow{\substack{r_2+ r_{3}}}
{
\begin{pmatrix}
1&2&0&3\\
0&1&0&2\\
0&0&1&0\\
0&0&0&0
\end{pmatrix}
}\xrightarrow{\substack{r_1- 2r_{2}}}
{
\begin{pmatrix}
1&0&0&-1\\
0&1&0&2\\
0&0&1&0\\
0&0&0&0
\end{pmatrix}
}
\end{equation*}
记$A=(\alpha_1,\alpha_2,\alpha_3)$,则$Ax=\alpha_4$的解为$x_1=-1,x_2=2,x_3=0$,即$\alpha_4=2\alpha_2-\alpha_1$
\end{jie}

\EX 已知向量组$\beta_1=(0,1,-1)^T,\beta_2=(a,2,1)^T,\beta_3=(b,1,0)^T$与
$\alpha_1= (1,2,-3)^T,\alpha_2=(3,0,1)^T,\alpha_3=(9,6,-7)^T$具有相同的秩,且$\beta_3$可由$\alpha_1,\alpha_2,\alpha_3$线性表出,求$a,b$的取值。

\begin{jie}
由题得:
\begin{equation*}
(\alpha_1,\alpha_2,\alpha_3,\beta_3)
\xrightarrow{\substack{r_2- 2r_{1}\\ r_3+3r_1}}
{\begin{pmatrix}
1&3&9&b\\
0&-6&-12&1-2b\\
0&10&20&3b\end{pmatrix}
}\xrightarrow{\substack{ r_3+\frac{10}{6}r_2}}
{\begin{pmatrix}
1&3&9&b\\
0&-6&-12&1-2b\\
0&0&0&3b+\frac{5}{3}(1-2b)\end{pmatrix}
}
\end{equation*}
因为$\beta_3$可由$\alpha_1,\alpha_2,\alpha_3$线性表出,所以:
$r(\alpha_1,\alpha_2,\alpha_3)=r(\alpha_1,\alpha_2,\alpha_3,\beta_3)=2$,所以$3b+\frac{5}{3}(1-2b)=0$,解得$b=5$.

\begin{equation*}
(\beta_1,\beta_2,\beta_3)
\xrightarrow{\substack{r_1\leftrightarrow r_3}}
{
\begin{pmatrix}
-1&1&0\\
1&2&1\\
0&a&5
\end{pmatrix}
}\xrightarrow{\substack{r_2+r_1}}
{
\begin{pmatrix}
-1&1&0\\
0&3&1\\
0&a&5
\end{pmatrix}
}
\end{equation*}
因为$r(\beta_1,\beta_2,\beta_3)=r(\alpha_1,\alpha_2,\alpha_3)=2$所以上述矩阵的第二第三行非零元素成比例:$
\frac{3}{a}=\frac{1}{5}
$,解得$a=15$。

综上所述:$a=15,b=5$。
\end{jie}

\EX 设向量组
$
\alpha_1=(2,2,-4,1)^T,\alpha_2=(4,2,-6,2)^T,\alpha_3=(6,3,-9,3)^T,\alpha_4=(1,1,1,1)^T
$求该向量组的秩和所有的极大线性无关组。

\begin{jie}
由题得:
\begin{equation*}
(\alpha_1,\alpha_2,\alpha_3,\alpha_4)
\xrightarrow{\substack{r_{2}-r_{1}\\ r_3+2r_1\\ r_1-\frac{1}{2}r_1}}
{
\begin{pmatrix}
2&4&6&1\\
0&-2&-3&0\\
0&2&3&3\\
0&0&0&0.5
\end{pmatrix}
}\xrightarrow{\substack{r_3+r_2}}
{
\begin{pmatrix}
2&4&6&1\\
0&-2&-3&0\\
0&0&0&3\\
0&0&0&0.5
\end{pmatrix}
}\xrightarrow{\substack{r_4-\frac{1}{6}r_3}}
{
\begin{pmatrix}
2&4&6&1\\
0&-2&-3&0\\
0&0&0&3\\
0&0&0&0
\end{pmatrix}
}
\end{equation*}
所以$(\alpha_1,\alpha_2,\alpha_3,\alpha_4)=3$,极大线性无关组为:$(\alpha_1,\alpha_2,\alpha_4)(\alpha_1,\alpha_3,\alpha_4)$.(注:$\alpha_2,\alpha_3$是线性相关的。)
\end{jie}

\EX 利用线性方程组的向量形式重新证明引理2.4.1.

引理2.4.1:设一个非齐次方程组有无穷多个解。任意取定它的一个解$\gamma_0$,则该方程组的全部解为$\gamma_0+\gamma$,其中$\gamma$为导出组的任意一个解。

\begin{zhengming}
导出组为$Ax=0$,$\gamma$为$Ax=0$的解,即$A\gamma=0$,$\gamma_0$为$Ax=\beta$的解,即$A\gamma_0=\beta$,所以$A(\gamma_0+\gamma)=A\gamma_0+A\gamma=0+\beta=\beta$,即$(\gamma_0+\gamma)$为$Ax=\beta$的解。
\end{zhengming}

\EX 已知$\beta_1,\beta_2$是非齐次线性方程组$Ax=b$的两个不同的解,$\alpha_1,\alpha_2$是对应的齐次行线方程组$Ax=0$的基础解系,$k_1,k_2$为任意常数,则方程组$Ax=b$的通解必是(\hphantom{~~~~~~~~~~~~~})。

(A)$k_1\alpha_1+k_2(\alpha_1+\alpha_2)+\dfrac{\beta_1-\beta_2}{2}$;

(B)$k_1\alpha_1+k_2(\alpha_1-\alpha_2)+\dfrac{\beta_1+\beta_2}{2}$;

(C)$k_1\alpha_1+k_2(\beta_1+\beta_2)+\dfrac{\beta_1-\beta_2}{2}$;

(D)$k_1\alpha_1+k_2(\beta_1-\beta_2)+\dfrac{\beta_1+\beta_2}{2}$;

\begin{jie}
非齐次的通解为$Ax=b$的特解+$Ax=0$的通解,同时还要满足$Ax=b$的特解与$Ax=0$的通解中的各向量组成的向量组线性无关。

把$A,B,C,D$选项代入$Ax=b$后发现$A,C$不符合要求,排除。$D$选项中,$\beta_1-\beta_2$为$Ax=0$的一个基础解系,但其可能与$\alpha_1$线性相关。(由题目并不能得出这两个向量是否线性相关,也就是有线性相关的可能性),所以排除$D$。
\end{jie}

%\EX 设矩阵$A$的每个$(i,j)-$元都是同一个数$a$。求$r(A)$。
%
%\begin{jie}
%(1) $a=0$时,矩阵$A$为零矩阵,$r(A)=0$.
%
%(2)$a\neq 0$时,矩阵$A$的其他行减去第一行后剩余一行,$r(A)=1$。
%\end{jie}

\EX 设四元线性方程组(\uppercase\expandafter{\romannumeral1})为
\begin{equation*}
\begin{cases}
x_1+x_2=0\\
x_2-x_4=0
\end{cases}
\end{equation*}
又已知某线性其次方程组(\uppercase\expandafter{\romannumeral2})的通解为:
\begin{equation*}
k_1(0,1,1,0)^T + k_2(-1,2,2,1)^T
\end{equation*}

(1)求线性方程组(\uppercase\expandafter{\romannumeral1})的基础解系。

(2)问线性方程组(\uppercase\expandafter{\romannumeral1})和(\uppercase\expandafter{\romannumeral2})是否有非零公共解?若有,则求出所有的非零公共解。若没有,则说明理由。

\begin{jie}
(1)由题得:
\begin{equation*}
\begin{pmatrix}
1&1&0&0\\
0&1&0&-1
\end{pmatrix}
\xrightarrow{\substack{r_1-r_2}}
{
\begin{pmatrix}
1&0&0&1\\
0&1&0&-1
\end{pmatrix}
}
\end{equation*}
解得$x_1=-x_4,x_2=x_4,x_3\in R$,分别取$(x_3,x_4)^T=(0,1)^T,(1,0)^T$得$\xi_1=(-1,1,0,1)^T,\xi_2=(0,0,1,0)^T$,$\xi_1,\xi_2$即为所求。

(2)假设有非零公共解,依题意可列:(式中:$l_1,l_2\in R$)
\begin{equation*}
l_1\xi_1+l_2\xi_2=k_1(0,1,1,0)^T + k_2(-1,2,2,1)^T~~\Rightarrow~~(-l_1,l_1,l_2,l_1)^T=(-k_2,k_1+2k_2,k_1+2k_2,k_2)^T
\end{equation*}
解得:$l_1=l_2=-k_1=k_2$,取$l_1=t,t\in R$,则公共解为:
\begin{equation*}
l_1\xi_1+l_2\xi_2=t(\xi_1+\xi_2)=t(-1,1,1,1)^T,t\in R
\end{equation*}
\end{jie}

\EX 已知方程组$
\begin{cases}
x_1+2x_2+3x_3=0\\
2x_1+3x_2+5x_3=0\\
x_1+x_2+ax_3=0
\end{cases}
$与
$
\begin{cases}
x_1+bx_2+cx_3=0\\
2x_1+b^2x_2+(c+1)x_3=0
\end{cases}
$同解。求$a,b,c$的值。

\begin{jie}
记第1个方程组为(\uppercase\expandafter{\romannumeral1}),第2个方程组为(\uppercase\expandafter{\romannumeral2})。分别记这两个方程组的增广矩阵为$\widetilde{A}_1,\widetilde{A}_2$.

由题得:$r(\widetilde{A}_2)\leq\min\{3,2\}=2<3$,所以(\uppercase\expandafter{\romannumeral2})有无穷多解,又因为同解,所以(\uppercase\expandafter{\romannumeral1})也有无穷多解,所以$r(\widetilde{A}_1)<3$

\begin{equation*}
\widetilde{A}_1\xrightarrow{\substack{r_2-2r_1\\ r_3-r_1}}
{\begin{pmatrix}
1&2&3\\
0&-1&-1\\
0&-1&a-3\end{pmatrix}
}\xrightarrow{\substack{r_3-r_2}}
{\begin{pmatrix}
1&2&3\\
0&-1&-1\\
0&0&a-2\end{pmatrix}
}
\end{equation*}
因为$r(\widetilde{A}_1)<3$,所以$a-2=0$,即$a=2$。

代入继续高斯消元:
\begin{equation*}
\xrightarrow{\substack{r_1+2r_2}}
{
\begin{pmatrix}
1&0&1\\
0&-1&-1\\
0&0&0
\end{pmatrix}
}\xrightarrow{\substack{r_2\times(-1)}}
{
\begin{pmatrix}
1&0&1\\
0&-1&-1\\
0&0&0
\end{pmatrix}
}
\end{equation*}
所以:$x_1=-x_3,x_2=-x_3$,取$x_3=-1$。则基础解系为$\xi_1=(1,1,-1)^T$。

(\uppercase\expandafter{\romannumeral1})、(\uppercase\expandafter{\romannumeral2})同解,所以把该基础解系代入(\uppercase\expandafter{\romannumeral2})有:
\begin{equation*}
\begin{cases}
1+b-c=0\\
2+b^2-(c+1)=0
\end{cases}~~~\Rightarrow~~~
\begin{cases}
b=0\\
c=1
\end{cases}\text{或}
\begin{cases}
b=1\\
c=2
\end{cases}
\end{equation*}

把$b=0,c=1$代入(\uppercase\expandafter{\romannumeral2})得:
\begin{equation*}
\begin{cases}
x_1+x_3=0\\
2x_1+2x_3=0
\end{cases}~~~\Rightarrow~~\begin{cases}
                             x_1=-x_3\\
                             x_2\in R
                           \end{cases}
\end{equation*}
有两个自由变量即有两个基础解系,与(\uppercase\expandafter{\romannumeral1})不同解,所以$b=0,c=1$不合题意。

把$b=1,c=2$代入(\uppercase\expandafter{\romannumeral2}):

\begin{equation*}
\begin{cases}
x_1+x_2+2x_3=0\\
2x_1+x_2+2x_3=0
\end{cases}
\end{equation*}
可以推出与(\uppercase\expandafter{\romannumeral1})同解(步骤自己写一下,这里略),综上所述:
$a=2,b=1,c=2$
\end{jie}

\EX 分别求下述两个齐次线性方程组的基础解系。
\begin{align*}
&(\uppercase\expandafter{\romannumeral1})~~~ x_1+x_2+x_3+x_4=0\\
&(\uppercase\expandafter{\romannumeral2}) ~~~
\begin{cases}
x_1-x_2+x_3-x_4=0\\
2x_1-2x_2+2x_3-3x_4=0\\
x_1-x_2+x_3-2x_4=0
\end{cases}
\end{align*}

\begin{jie}
(\uppercase\expandafter{\romannumeral1})$x_1=-x_2-x_3-x_4$
分别取$
\begin{pmatrix}
x_2&x_3&x_4
\end{pmatrix}^T
=\begin{pmatrix}
1&0&0
\end{pmatrix}^T,\begin{pmatrix}
0&1&0
\end{pmatrix}^T,\begin{pmatrix}
0&0&1
\end{pmatrix}^T$
得:
$\xi_1
=\begin{pmatrix}
-1&1&0&0
\end{pmatrix}^T,\xi_2
=\begin{pmatrix}
-1&0&1&0
\end{pmatrix}^T,\xi_3
=\begin{pmatrix}
-1&0&0&1
\end{pmatrix}^T$.

$\xi_1,\xi_2,\xi_3$即为所求。

(\uppercase\expandafter{\romannumeral2})
由题得:
\begin{equation*}
A\xrightarrow{\substack{r_2-2r_1\\ r_3-r_1}}{\begin{pmatrix}
1&-1&1&-1\\
0&0&0&-1\\
0&0&0&-1\end{pmatrix}
}\xrightarrow{\substack{ r_3-r_2}}{\begin{pmatrix}
1&-1&1&-1\\
0&0&0&-1\\
0&0&0&0\end{pmatrix}
}\xrightarrow{\substack{ r_1-r_2}}{\begin{pmatrix}
1&-1&1&0\\
0&0&0&-1\\
0&0&0&0\end{pmatrix}
}
\end{equation*}
由阶梯型矩阵可以得出:
$x_1=x_2-x_3,x_4=0$,分别取$\begin{pmatrix}x_2&x_3\end{pmatrix}^T=\begin{pmatrix}1&0\end{pmatrix}^T,\begin{pmatrix}0&1\end{pmatrix}^T$得
$
\xi_1=
\begin{pmatrix}
1&1&0&0
\end{pmatrix}^T,\xi_2=
\begin{pmatrix}
-1&0&1&0
\end{pmatrix}^T
$。

$\xi_1,\xi_2$即为所求。
\end{jie}

\EX 判断下列说法是否正确,并说明理由.

(1) 两个齐次线性方程组的解集相同$\Leftrightarrow$它们的基础解系等价。

(2) 设一个$5$元齐次线性方程组的系数矩阵的秩为$3$.则该方程组可能有$3$个线性无关的解。

(3)当一个线性方程组有无穷多个解时一定有基础解系。

\begin{jie}
(1) 正确。$\Rightarrow$,解集相同,则基础解系可以相互线性表出。

$\Leftarrow$基础解系等价,由于任意解向量都是基础解系的线性组合,所以解集相同。

(2)错误。齐次线性方程组线性无关的解的个数不超过基础解系所包含的向量的个数:5-3=2.

(3)错误。基础解系只针对齐次线性方程组来说的。
\end{jie}

\EX 判断下列说法是否正确,并说明理由.

(4)设
$
\xi_1=
\begin{pmatrix}
1\\-1 \\1
\end{pmatrix},\xi_2=
\begin{pmatrix}
1\\1 \\1
\end{pmatrix}
$是某个齐次线性方程组的一个基础解系,则$
\eta_1=
\begin{pmatrix}
1\\0 \\1
\end{pmatrix},\eta_2=
\begin{pmatrix}
0\\1 \\0
\end{pmatrix}
$也是该线性方程组的一个基础解系。

(5)向量组$
\xi_1=\begin{pmatrix}
1\\0 \\1
\end{pmatrix},\xi_2=
\begin{pmatrix}
0\\1 \\1
\end{pmatrix}
$一定是某个齐次线性方程组的某个基础解系。

\begin{jie}
(4)正确。$\eta_1=\frac{1}{2}(\xi_1+\xi_2),\eta_2=\frac{1}{2}(\xi_2+\xi_1)$,即$\xi_1,\xi_2$和$\eta_1,\eta_2$可以相互表出,所以$\xi_1,\xi_2$和$\eta_1,\eta_2$等价。即$\eta_1,\eta_2$也是该线性方程组的一个基础解系。

(5)正确。例如该向量组是$x_1+x_2-x_3=0$的基础解系。
\end{jie}

\EX 解齐次线性方程组。
\begin{equation*}
\begin{cases}
x_1-3x_2+6x_3=0\\
4x_1-x_2+x_3=0\\
3x_1+2x_2+x_3=0
\end{cases}
\end{equation*}

\begin{jie}
由题得:
\begin{equation*}
A=
\begin{pmatrix}
1&-3&6\\
4&-1&1\\
3&2&1
\end{pmatrix}
\xrightarrow{\substack{r_2-4r_1\\ r_3-3r_1}}{
\begin{pmatrix}
1&-3&6\\
0&11&-23\\
0&11&-17
\end{pmatrix}
}\xrightarrow{\substack{r_3-r_2}}{
\begin{pmatrix}
1&-3&6\\
0&11&-23\\
0&0&6
\end{pmatrix}
}
\end{equation*}
所以$r(A)=3$,该线性方程组只有零解。
\end{jie}

\EX 解齐次线性方程组
\begin{equation*}
\begin{cases}
x_1-3x_2+2x_3-x_4=0\\
-2x_1+6x_2-4x_3+3x_4=0\\
3x_1-9x_2+6x_3-4x_4=0
\end{cases}
\end{equation*}

\begin{jie}
由题得:
\begin{align*}A&=
\begin{pmatrix}
1&-3&2&-1\\
-2&6&-4&3\\
3&-9&6&-4
\end{pmatrix}
\xrightarrow{\substack{r_2+2r_1\\ r_3-3r_1}}{
\begin{pmatrix}
1&-3&2&-1\\
0&0&0&1\\
0&0&0&-1
\end{pmatrix}
}\xrightarrow{\substack{ r_3+r_2}}{
\begin{pmatrix}
1&-3&2&-1\\
0&0&0&1\\
0&0&0&0
\end{pmatrix}
}\\
&\xrightarrow{\substack{ r_1+r_2}}{
\begin{pmatrix}
1&-3&2&0\\
0&0&0&1\\
0&0&0&0
\end{pmatrix}
}
\end{align*}
由最简阶梯型矩阵得:$x_1=3x_2-2x_3,x_4=0$,分别取$
\begin{pmatrix}
x_2&x_3
\end{pmatrix}^T=\begin{pmatrix}
1&0
\end{pmatrix}^T,\begin{pmatrix}
0&1
\end{pmatrix}^T
$得$
\xi_1=
\begin{pmatrix}
3&1&0&0
\end{pmatrix}^T
,\xi_2=
\begin{pmatrix}
-2&0&1&0
\end{pmatrix}^T
$,通解为
$
x=k_1\xi_1+k_2\xi_2,k_1\in R,k_2\in R
$
\end{jie}

\EX 判断下列说法是否正确,并说明理由。

(1) 如果一个非齐次方程组的导出组有基础解系,则该非齐次线性方程组一定有无穷多个解。

(2)如果一个非齐次线性方程组有无穷多个解,则它的导出组一定有基础解系。

(3)一个非齐次线性方程组的任意解都不可能由它的导出组的任意基础解系线性表出。

(4) 设一个$5$元非齐次线性方程组的系数矩阵和增广矩阵的秩都是$3$. 则该方程组一定有$3$个线性无关的解。

(5)设一个$5$元非齐次线性方程组的系数矩阵和增广矩阵的秩都是$3$.则它的通解的任意表达式中一定含有$2$个任意常数。

\begin{jie}
记$n$为方程组中未知数的个数,$A$为系数矩阵,$\widetilde{A}$为增广矩阵。

(1)错误。导出组有基础解系,只能得出系数矩阵的秩小于未知数的个数。但对于非齐次方程组,系数矩阵的值可能会与增广矩阵的秩不相等,即可能存在无解的情况。

(2)正确。非齐次线性方程组有无穷多解,则有$r(A)=r(\widetilde{A})<n$,所以导出组有基础解系,基础解系的个数$n-r(A)$。

(3)正确。非齐次线性方程组的解是由一个特解和导出组的基础解系组成的,该特解不可能是导出组的解,即非齐次线性方程组的任意解都不可能由它的导出组的任意基础解系线性表出。

(4)正确。系数矩阵和增广矩阵的秩都是$3$,所以导出组有$5-3=2$个基础解系$\xi_1,\xi_2$。设$\eta$是该方程组的解,则$\xi_1+\eta,\xi_2+\eta$是也原方程组的解。不难验证$\eta,\xi_1+\eta,\xi_2+\eta$线性无关。(线性无关的证明方法)。

(5)正确。系数矩阵和增广矩阵的秩都是$3$,所以导出组有$5-3=2$个基础解系$\xi_1,\xi_2$。设$\mu$是该方程组的特解,则通解为$\mu+k_1\xi_1+k_2\xi_2$.由通解可以看出,有两个任意常数。
\end{jie}

\EX 解线性方程组
\begin{equation*}
\begin{cases}
x_1-2x_2+x_3-x_4=1\\
2x_1+2x_2+x_3+2x_4=-1\\
x_1+4x_2-3x_4=0
\end{cases}
\end{equation*}

\begin{jie}
由题得增广矩阵:
\begin{align*}
\widetilde{A}&=
\begin{pmatrix}
1&-2&1&-1&1\\
2&2&1&2&-1\\
1&4&0&-3&0
\end{pmatrix}
\xrightarrow{\substack{ r_2-2r_1\\ r_3-r_1}}{
\begin{pmatrix}
1&-2&1&-1&1\\
0&6&-1&4&-3\\
0&6&-1&-2&-1
\end{pmatrix}
}\xrightarrow{\substack{ r_3-r_2}}{
\begin{pmatrix}
1&-2&1&-1&1\\
0&6&-1&4&-3\\
0&0&0&-6&2
\end{pmatrix}
}\\
&\xrightarrow{\substack{ r_2\times\frac{1}{6}\\ r_3\times\left(-\frac{1}{6}\right)}}{
\begin{pmatrix}
1&-2&1&-1&1\\
0&1&-\frac{1}{6}&\frac{2}{3}&-\frac{1}{2}\\
0&0&0&1&-\frac{1}{3}
\end{pmatrix}
}\xrightarrow{\substack{ r_2-\frac{2}{3}r_3\\ r_1+r_3}}{
\begin{pmatrix}
1&-2&1&0&\frac{2}{3}\\
0&1&-\frac{1}{6}&0&-\frac{5}{18}\\
0&0&0&1&-\frac{1}{3}
\end{pmatrix}
}\xrightarrow{\substack{  r_1+2r_2}}{
\begin{pmatrix}
1&0&\frac{2}{3}&0&\frac{1}{9}\\
0&1&-\frac{1}{6}&0&-\frac{5}{18}\\
0&0&0&1&-\frac{1}{3}
\end{pmatrix}
}
\end{align*}
所以:
$x_1=\frac{1}{9}-\frac{2}{3}x_3,x_2=\frac{1}{6}x_3-\frac{5}{18},x_4=-\frac{1}{3}$,令$x_3=t,t\in R$,所以:
\begin{equation*}
x=
\begin{pmatrix}
\frac{1}{9}&-\frac{5}{18}&0&-\frac{1}{3}
\end{pmatrix}+t
\begin{pmatrix}
-\frac{2}{3}&\frac{1}{6}&1&0
\end{pmatrix},t\in R
\end{equation*}
\end{jie}

\EX 解线性方程组
\begin{equation*}
\begin{cases}
x_1-2x_2+x_3=1\\
2x_1+2x_2+x_3=-1\\
x_1+4x_2-x_3=0\\
x_1-2x_2+2x_3=-1
\end{cases}
\end{equation*}

\begin{jie}
由题得增广矩阵:
\begin{align*}
\widetilde{A}&=
\begin{pmatrix}
1&-2&1&1\\
2&2&1&-1\\
1&4&-1&0\\
1&-2&2&-1
\end{pmatrix}
\xrightarrow{\substack{r_2-2r_1\\ r_3-r_1\\ r_4-r_1}}{
\begin{pmatrix}
1&-2&1&1\\
0&6&-1&-3\\
0&6&-2&-1\\
0&0&1&-2
\end{pmatrix}
}\xrightarrow{\substack{r_3-r_2}}{
\begin{pmatrix}
1&-2&1&1\\
0&6&-1&-3\\
0&0&-1&2\\
0&0&1&-2
\end{pmatrix}
}\\
&\xrightarrow{\substack{r_4+r_3\\ r_2-r3\\ r_1+r_3}}{
\begin{pmatrix}
1&-2&0&3\\
0&6&0&-5\\
0&0&-1&2\\
0&0&0&0
\end{pmatrix}
}\xrightarrow{\substack{r_2\times\frac{1}{6}\\ r_3\times(-1)}}{
\begin{pmatrix}
1&-2&0&3\\
0&1&0&-\frac{5}{6}\\
0&0&1&-2\\
0&0&0&0
\end{pmatrix}
}\xrightarrow{\substack{r_1+2r_2}}{
\begin{pmatrix}
1&0&0&\frac{4}{3}\\
0&1&0&-\frac{5}{6}\\
0&0&1&-2\\
0&0&0&0
\end{pmatrix}
}
\end{align*}
所以:
\begin{equation*}
x=
\begin{pmatrix}
\frac{4}{3}&-\frac{5}{6}&-2
\end{pmatrix}^T
\end{equation*}
\end{jie}

\EX 解线性方程组
\begin{equation*}
\begin{cases}
x_1-x_2-x_3+x_4=1\\
2x_1-2x_2-x_3+x_4=-2\\
3x_1-3x_2-2x_3+2x_4=-1\\
5x_1-5x_2-3x_3+3x_4=-3
\end{cases}
\end{equation*}

\begin{jie}
由题得增广矩阵:
\begin{equation*}
\widetilde{A}=
\begin{pmatrix}
1&-1&-1&1&1\\
2&-2&-1&1&-2\\
3&-3&-2&2&-1\\
5&-5&-3&3&-3
\end{pmatrix}
\xrightarrow{\substack{r_2-2r_1\\ r_3-3r_1\\ r_4-5r_1}}{
\begin{pmatrix}
1&-1&-1&1&1\\
0&0&1&-1&-4\\
0&0&1&-1&-4\\
0&0&2&-2&-8
\end{pmatrix}
}
\xrightarrow{\substack{r_1+r_2\\ r_3-r_2\\ r_4-2r_2}}{
\begin{pmatrix}
1&-1&0&0&-3\\
0&0&1&-1&-4\\
0&0&0&0&0\\
0&0&0&0&0
\end{pmatrix}
}
\end{equation*}
所以$x_1=x_2-3,x_3=x_4-4$.所以:
\begin{equation*}
x=
\begin{pmatrix}
-3&0&-4&0
\end{pmatrix}^T+k_{1}
\begin{pmatrix}
1&1&0&0
\end{pmatrix}^T+k_{2}
\begin{pmatrix}
0&0&1&1
\end{pmatrix}^T,k_1,k_2\in R
\end{equation*}
\end{jie}

\EX 已知$a,b$是常数,解线性方程组.
\begin{equation*}
\begin{cases}
ax_1+x_2+x_3=4\\
ax_1+x_2+2x_3=2\\
x_1+x_2+bx_3=1
\end{cases}
\end{equation*}

\begin{jie}
由题得增广矩阵:
\begin{equation*}
\widetilde{A}=
\begin{pmatrix}
a&1&1&4\\
a&1&2&2\\
1&1&b&1
\end{pmatrix}\xrightarrow{\substack{r_2-r_1}}
{
\begin{pmatrix}
a&1&1&4\\
0&0&1&-2\\
1&1&b&1
\end{pmatrix}
}
\end{equation*}
可以得出:$x_3=-2$。把$x_3=-2$代入原方程组得:
\begin{equation*}
\begin{cases}
ax_1+x_2-2=4\\
ax_1+x_2-4=2\\
x_1+x_2-2b=1
\end{cases}~~~\Rightarrow~~~
\begin{cases}
ax_1+x_2=6\\
x_1+x_2=1+2b
\end{cases}
\end{equation*}
对新方程组列增广矩阵并进行高斯消元:
\begin{align*}
\widetilde{A}'=
\begin{pmatrix}
a&1&6\\
1&1&1+2b
\end{pmatrix}\xrightarrow{\substack{r_2\leftrightarrow r_1}}
{
\begin{pmatrix}
1&1&1+2b\\
a&1&6
\end{pmatrix}
}\xrightarrow{\substack{r_2-a r_1}}
{
\begin{pmatrix}
1&1&1+2b\\
0&1-a&6-a(1+2b)
\end{pmatrix}
}
\end{align*}
当$a-1=0$且$6-a(1+2b)\neq 0$时,$r(A')\neq r(\widetilde{A}')$,此时线性方程组无解。即$a=1,b\neq 2.5$.

当$a-1=0$且$6-a(1+2b)=0$时,$r(A')=r(\widetilde{A}')<2$,此时线性方程组有无穷解。即$a=1,b= 2.5$.代入原阶梯形矩阵后得:
$x_1=6-x_2$,取$x_2=k$,所以原方程组的解为(注意别忘了$x_3$):
\begin{equation*}
  x=
  \begin{pmatrix}
   6-k\\ k\\ -2
  \end{pmatrix}=
  \begin{pmatrix}
    6\\ 0\\ -2
  \end{pmatrix}+k\begin{pmatrix}
                  -1\\ 1\\ 0
                \end{pmatrix},k\in R
\end{equation*}

当$a-1\neq 0$时,$r(A')=r(\widetilde{A}')=2$,此时线性方程组有唯一解。把$a-1$代入上述阶梯矩阵后进行高斯消元(高斯消元步骤略),解得(同样别忘了写$x_3$)$x
=\begin{pmatrix}
\dfrac{5-2b}{a-1}\\[2pt] \dfrac{a+2ab-6}{a-1}\\[2pt] -2
 \end{pmatrix}
$.

综上所述:……
\end{jie}

\EX 设某个线性方程组的通解为
\begin{equation*}
\begin{pmatrix}
-2+3s-2t\\
1-2s+3t\\
-1+s\\
2+t
\end{pmatrix},s,t\text{是任意数}
\end{equation*}

(1)证明:该方程不是齐次线性方程组。

(2)求该方程组的系数矩阵的秩,并写出它的导出组的一个基础解系。

\begin{jie}
(1)任取通解中的一个解$x
=\begin{pmatrix}
-2+3s-2t\\
1-2s+3t\\
-1+s\\
2+t
\end{pmatrix}
$。对于齐次方程一定有零解(充要条件),所以假设$x=0$,则有
\begin{equation*}
\begin{cases}
-2+3s-2t=0\\
1-2s+3t=0\\
-1+s=0\\
2+t=0
\end{cases}~~\Rightarrow ~~s=1,t=-2
\end{equation*}
代入到$-2+3s-2t=0$得:$5=0$,产生了矛盾,所以假设不成立,即$x\neq 0$,所以该方程不是齐次线性方程组。

(2)$\begin{pmatrix}
-2+3s-2t\\
1-2s+3t\\
-1+s\\
2+t
\end{pmatrix}=
\begin{pmatrix}
-2\\ 0\\ -1\\ 2
\end{pmatrix}+s\begin{pmatrix}
3\\ -2\\ 1\\ 0
\end{pmatrix}+t\begin{pmatrix}
-2\\ 3\\ 0\\ 1
\end{pmatrix}
$

可以看出有两个自由变量,所以原方程组的秩为$r=$未知数的个数$-$自由变量的个数=$4-2=2$。

导出组的基础解系为:
$\xi_1=\begin{pmatrix}
3& -2& 1& 0
\end{pmatrix}^T,\xi_2=\begin{pmatrix}
-2& 3&0&1
\end{pmatrix}^T$。
\end{jie}

\EX 设某个$5$元非齐次线性方程组的系数矩阵秩为3,且有无穷多个解。证明:存在该方程组的3个线性无关的解$\gamma_1,\gamma_2,\gamma_3$,使得该方程组的任意解$\gamma$都可以由$\gamma_1,\gamma_2,\gamma_3$线性表出。

\begin{zhengming}
导出组基础解系的个数$=$未知数个数$-$秩$=5-3=2$。记为$\xi_1,\xi_2$。

任取原方程组的一个解$\eta$,则$\gamma_1=\eta,\gamma_2=\eta+\xi_1,\gamma_3=\eta+\xi_2$为原方程组的一组线性无关的解。

那么该方程组的任意一个解$\gamma = \eta+c_1\xi_1+c_2\xi_2=(1-c_1-c_2)\eta+c_1(\xi_1+\eta)+c_2(\xi_2+\eta)=(1-c_1-c_2)\gamma_1+c_1\gamma_2+c_2\gamma_3,c_1,c_2\in R$,即$\gamma$都可以由$\gamma_1,\gamma_2,\gamma_3$线性表出。
\end{zhengming}

\EX 设齐次线性方程组
\begin{equation*}
\begin{cases}
\lambda x_1+x_2+x_3=0\\
x_1+\lambda x_2+x_3=0\\
x_1+x_2+x_3=0
\end{cases}
\end{equation*}
只有零解,$\lambda$应满足的条件是\underline{\hphantom{~~~~~~~~~~~~~~}}。

\begin{jie}
解法一:

系数矩阵为方阵,对于齐次线性方程组只有零解即系数矩阵行列式不为0.
\begin{equation*}
\begin{vmatrix}
\lambda&1&1\\
1&\lambda&1\\
1&1&1
\end{vmatrix}\xlongequal{r_2-r_1\\ r_3-r_1}
\begin{vmatrix}
\lambda&1&1\\
1-\lambda&\lambda-1&0\\
1-\lambda&0&0
\end{vmatrix}=(1-\lambda)^2\neq 0~~~\Rightarrow~~~\lambda\neq1
\end{equation*}


解法二:
由题得:
\begin{equation*}
A=
\begin{pmatrix}
\lambda&1&1\\
1&\lambda&1\\
1&1&1
\end{pmatrix}
\xrightarrow{\substack{r_1\leftrightarrow r_3}}
{
\begin{pmatrix}
1&1&1\\
1&\lambda&1\\
\lambda&1&1
\end{pmatrix}
}\xrightarrow{\substack{r_2-r_1\\ r_3-\lambda r_1}}
{
\begin{pmatrix}
1&1&1\\
0&\lambda-1&0\\
0&1-\lambda&1-\lambda
\end{pmatrix}
}\xrightarrow{\substack{r_3+2r_2}}
{
\begin{pmatrix}
1&1&1\\
0&\lambda-1&0\\
0&0&1-\lambda
\end{pmatrix}
}
\end{equation*}
线性方程组只有零解,则$r(A)=3$,即$\lambda-1\neq 0$
\end{jie}

\EX 当$a$的值是多少时,线性方程组
\begin{equation*}
\begin{cases}
x_1+2x_2+x_3=1\\
2x_1+3x_2+(a+2)x_3=3\\
x_1+ax_2-2x_3=0
\end{cases}
\end{equation*}
无解,有唯一解,有无穷多个解。当有无穷多解时,试用一个特解及导出组的基础解系表示其全部解。

\begin{jie}
由题得系数矩阵(对其求行列式):
\begin{equation*}
|A|=
\begin{vmatrix}
1&2&1\\2 &3&a+2\\ 1&a &-2
\end{vmatrix}=3-a^2+2a
\end{equation*}

(1)当$|A|\neq 0$时,即$a\neq 3$或$a\neq -1$时,$r(A)=r(A,B)=3$,方程一定有唯一解。对其进行高斯消元(步骤略)。
\begin{equation*}
(A,B)\rightarrow
\begin{pmatrix}
1&0&0&\dfrac{a+2}{a+1}\\[3pt]
0&1&0&-\dfrac{1}{a+1}\\[3pt]
0&0&1&\frac{1}{a+1}
\end{pmatrix}
\end{equation*}
所以$a\neq 3$且$a\neq -1$时有唯一解:
\begin{equation*}
  x=\left[\dfrac{a+2}{a+1},-\dfrac{1}{a+1},\frac{1}{a+1}\right]^T
\end{equation*}

(2)$a=-1$时,代入增广矩阵进行高斯消元(步骤略):
\begin{equation*}
(A,B)\rightarrow
\begin{pmatrix}
1&2&1&1\\
0&-1&-1&1\\
0&0&0&-4
\end{pmatrix}
\end{equation*}
$r(A)\neq r(A,B)$,所以此时无解。

(3)$a=3$时,代入增广矩阵进行高斯消元(步骤略):\begin{equation*}
(A,B)\rightarrow
\begin{pmatrix}
1&0&7&3\\
0&1&-3&-1\\
0&0&0&0
\end{pmatrix}
\end{equation*}
所以:$x_1=3-7x_3,x_2=3x_3-1$.所以:
\begin{equation*}
x=[3,-1,0]^T+k[-7,3,1]^T,k\in R
\end{equation*}

综上所述:……
\end{jie}

\EX 已知线性方程组
$
\begin{cases}
x_1-x_2+2x_3+x_4=2\\
-x_1+ax_2+2x_3+x_4=0\\
2x_1+x_2+bx_3-x_4=1
\end{cases}
$有三个线性无关的解,证明系数矩阵的秩为2,进一步求$a,b$的值以及该方程组的通解。

\begin{jie}
非齐次线性方程组有3个线性无关的解,则其导出组有3-1=2个基础解系,其系数矩阵的秩=未知数的个数4-基础解系的个数2=2.

\begin{equation*}
(A,B)
\xrightarrow{\substack{r_{2}+r_{1}\\ r_3-2r_1}}
{\begin{pmatrix}
1&-1&2&1&2\\
0&a-1&4&2&2\\
0&3&b-4&-3&-3\end{pmatrix}
}
\end{equation*}

$r(A)=2$,所以
\begin{equation*}
\frac{a-1}{3}=\frac{4}{b-4}=\frac{2}{-3}~~~\Rightarrow~~~a=-1,b=-2
\end{equation*}
代入阶梯型矩阵继续高斯消元(过程略):
\begin{equation*}
\Rightarrow
\begin{pmatrix}
1&0&0&0&1\\
0&1&-2&-1&-1\\
0&0&0&0&0
\end{pmatrix}
\end{equation*}
所以$x=[1,-1,0,0]^T+k_{1}[0,2,1,0]^T+k_2[0,1,0,1]^T,k_1,k_2\in R$。
\end{jie}

\clearpage
\stepcounter{chapter}
\hphantom{~~}\hfill {\zihao{3}\heiti 第十次线性代数} \hfill\hphantom{~~}
\addcontentsline{toc}{section}{\protect\numberline {}第十次线性代数}
%\chapter{第十次线性代数}

\hphantom{~~}

\EX 设$A,B$是$3\times2$的矩阵,问$A,B$能够相似吗?说明理由。

\begin{jie}
不能,只有方阵才能讨论是否相似。
\end{jie}

\EX 设$A\sim B$,证明$3A^3+2A-5I_{n}\sim 3B^2+2B-5I_n$。

\begin{zhengming}
$A\sim B$,依定义,一定存在一个可逆矩阵$P$使得$P^{-1}AP=B$。

所以
\begin{align*}
&B^3=(P^{-1}AP)(P^{-1}AP)(P^{-1}AP)=P^{-1}A^3P\\
&P^{-1}2AP=2P^{-1}AP=2B
\end{align*}
所以$P^{-1}(3A^3+2A-5I_{n})P=3P^{-1}A^3P+2P^{-1}AP-5P^{-1}I_nP=3B^2+2B-5I_n$,即
$3A^3+2A-5I_{n}\sim 3B^2+2B-5I_n$。
\end{zhengming}

\EX 设$A\sim B$,$C\sim D$。问是否有$A+C\sim B+D$?说明理由。

\begin{jie}
否。理由:

\begin{align*}
&A=
\begin{pmatrix}
1&1\\ 0&1
\end{pmatrix},~~~B=
\begin{pmatrix}
1&0\\ 0&1
\end{pmatrix},~~~A\sim B\\
&C=
\begin{pmatrix}
1&0\\ 1&-1
\end{pmatrix},~~~D=
\begin{pmatrix}
1&0\\ 0&-1
\end{pmatrix},~~~C\sim D\\
& A+C=
\begin{pmatrix}
2&1\\ 1&0
\end{pmatrix},B+D=
\begin{pmatrix}
2&0\\
1&0
\end{pmatrix}\\
&|A+C|=-1,|B+D|=0
\end{align*}
因为相似矩阵具有相同的行列式,所以$A+C$与$B+D$不相似。
\end{jie}

\EX 设$A\sim
\begin{pmatrix}
1&-1&2\\
1&1&0\\
2&0&2
\end{pmatrix}
$,求$|A|$和$r(A)$。

\begin{jie}
相似矩阵具有相同的秩、行列式和特征值。由题得
\begin{equation*}
\begin{pmatrix}
1&-1&2\\
1&1&0\\
2&0&2
\end{pmatrix}\xrightarrow{\substack{r_{2}-r_{1}\\ r_3-2r_1}}
{
\begin{pmatrix}
1&-1&2\\
0&2&-2\\
0&2&-2
\end{pmatrix}
}\xrightarrow{\substack{ r_3-r_2}}
{
\begin{pmatrix}
1&-1&2\\
0&2&-2\\
0&0&0
\end{pmatrix}
}
\end{equation*}
所以$r(A)=2<3$,则$|A|=0$。
\end{jie}

\EX 设$A\sim B$。证明:$A$可逆$\Leftrightarrow B$可逆;且当$A$可逆时有$A^{-1}\sim B^{-1}$。

\begin{zhengming}
$A\sim B$,所以$r(A)=r(B)$,$A$可逆,则$r(A)=n=r(B)$,所以$B$可逆。反之同理。

$A$可逆时:$A\sim B$,所以存在一个可逆矩阵$P$,$A=P^{-1}BP$,两边同时求逆,$A^{-1}=(P^{-1}BP)^{-1}=P^{-1}B^{-1}(P^{-1})^{-1}=P^{-1}B^{-1}P$,即$A^{-1}\sim B^{-1}$。
\end{zhengming}

\EX 对于任意方阵$A$,称$A$的对角元的和为$A$的迹,记为$tr(A)$。证明:如果$A\sim B$,则$tr(A)=tr(B)$,并举例说明逆命题不成立。

\begin{zhengming}
由《第二次线性代数》例2.5得出:$tr(AB)=tr(BA)$.

因为$A\sim B$,依定义,则存在一个可逆矩阵$P$,使得$B=P^{-1}AP$,则$tr(B)=tr((P^{-1}A)P)=tr(P(P^{-1}A))=tr(A)$。证毕。

举例说明逆命题不成立:例如$A
\begin{pmatrix}
1&0\\0&1
\end{pmatrix},B=
\begin{pmatrix}
1&1\\
0&1
\end{pmatrix}
$,可以看出$tr(A)=tr(B)$,但根据相似的定义,单位矩阵只能跟自己相似,即逆命题不成立。
\end{zhengming}

\EX 设$A_1\sim B_1$,$A_2\sim B_2$,证明:
\begin{equation*}
\begin{pmatrix}
A_1\\
&A_2
\end{pmatrix}\sim \begin{pmatrix}
B_1\\
&B_2
\end{pmatrix}
\end{equation*}

\begin{zhengming}
$A_i\sim B_i$,则存在可逆矩阵$P_i$,使得$A_i=P_i^{-1}B_iP_i$,式中$i=1,2$。

取$
P=
\begin{pmatrix}
P_1\\
&P_2
\end{pmatrix}
$,则$P$可逆,所以:
\begin{align*}
P^{-1}\begin{pmatrix}
A_1\\
&A_2
\end{pmatrix}P&=
\begin{pmatrix}
P_1^{-1}\\
&P_2^{-1}
\end{pmatrix}\begin{pmatrix}
A_1\\
&A_2
\end{pmatrix}\begin{pmatrix}
P_1\\
&P_2
\end{pmatrix}=
\begin{pmatrix}
P_1^{-1}A_1\\
&P_2^{-1}A_2
\end{pmatrix}\begin{pmatrix}
P_1\\
&P_2
\end{pmatrix}\\&=\begin{pmatrix}
P_1^{-1}A_1P_1\\
&P_2^{-1}A_2P_2
\end{pmatrix}=\begin{pmatrix}
B_1\\
&B_2
\end{pmatrix}
\end{align*}
证毕。
\end{zhengming}

\EX 设$A=
\begin{pmatrix}
1&1\\0&a
\end{pmatrix},\begin{pmatrix}
-1&1\\b&3
\end{pmatrix}
$,设$A\sim B$。求$a,b$的值。

\begin{jie}
相似矩阵具有相同的行列式和迹,所以:
\begin{equation*}
\begin{cases}
1+a=-1+3\\
a=-3-b
\end{cases}~~~\Rightarrow~~
\begin{cases}
a=1\\ b=-4
\end{cases}
\end{equation*}
\end{jie}

\EX 设$A\sim B$,$S_1,S_2$分别是齐次线性方程组$Ax=0$和$Bx=0$的解集,证明:存在从$S_1$到$S_2$的一一映射。

\begin{zhengming}
$A\sim B$,则存在一个可逆矩阵$P$,使得$P^{-1}AP=B$。

对任意的$x\in S_1$,即$Ax=0$,令$f(x)=P^{-1}x$,则$Bf(x)=BP^{-1}x=P^{-1}PBP^{-1}x=P^{-1}Ax=0$,所以$f(x)\in S_2$。

所以我们可以得到一个映射:$f:S_1\rightarrow S_2,f(x)=P^{-1}x$。

反之,对任意的$y\in S_2$,令$g(y)=Py$,则$Ag(y)=APy=PP^{-1}APy=PBy=0$,即$g(y)\in S_1$.所以$g:S_2\rightarrow S_1:g(y)=Py$。

由于$(f\circ g)(y)=P^{-1}(Py)$,所以$(f\circ g)$是$S_2$上的恒等映射,同理,$(g\circ f)$是$S_1$上的恒等映射,综上所述,$f$是一个从$S_1$到$S_2$的一一映射。
\end{zhengming}

\EX 用矩阵相似的定义验证:$
\begin{pmatrix}
1&1\\0&1
\end{pmatrix}\sim
\begin{pmatrix}
1&0\\1&1
\end{pmatrix}
$.

\begin{zhengming}
设$A=\begin{pmatrix}
1&1\\0&1
\end{pmatrix},B=\begin{pmatrix}
1&0\\1&1
\end{pmatrix}$。

假设存在可逆矩阵$P=
\begin{pmatrix}
x_1&x_2\\
x_3&x_4
\end{pmatrix}
$,使得$P^{-1}AP=B$,即$AP=PB$:
\begin{align*}
&AP=\begin{pmatrix}
1&1\\0&1
\end{pmatrix}\begin{pmatrix}
x_1&x_2\\
x_3&x_4
\end{pmatrix}=\begin{pmatrix}
x_1+x_3&x_2+x_4\\
x_3&x_4
\end{pmatrix}\\
&PB=\begin{pmatrix}
x_1&x_2\\
x_3&x_4
\end{pmatrix}\begin{pmatrix}
1&0\\1&1
\end{pmatrix}=
\begin{pmatrix}
x_1+x_2&x_2\\
x_3+x_4&x_4
\end{pmatrix}\\
&AP=PB~~\Rightarrow~~
\begin{cases}
x_1+x_3=x_1+x_2\\
x_2+x_4=x_2\\
x_3=x_3+x_4\\
x_4=x_4
\end{cases}~~~\Rightarrow~~~
\begin{cases}
x_1\in R\\
x_2=x_3\\
x_4=0
\end{cases}
\end{align*}
取$x_1=k_1,x_3=k_2$,则:
$
P=
\begin{pmatrix}
k_1&k_2\\
k_2&0
\end{pmatrix}
$,因为$P$可逆,所以$|P|\neq 0$,即$k_2\neq0,k_1\in R$。所以$A\sim B$,证毕。
\end{zhengming}

\EX 设$A$可逆且$\lambda$为$A$的一个特征值。证明:$\lambda\neq0$且$\dfrac{1}{\lambda}$是$A^{-1}$的一个特征值。

\begin{zhengming}
$A$可逆,则$|A|\neq 0$,$r(A)=n$($n$为$A$的阶数)

设$\lambda$对应的特征值为$\alpha$,则$\alpha\neq 0$,$A\alpha =\lambda \alpha$。

假设$\lambda =0$,则$A\alpha=0$,特征向量不为0,即$\alpha\neq0$,所以$\alpha$为$Ax=0$的一个非零解,即$Ax=0$有非零解,即$r(A)< n$,这与最开始的$r(A)=n$矛盾,所以假设不成立,即$\lambda =0$。

$A\alpha=\lambda\alpha$,两边同时左乘$\dfrac{1}{\lambda}A^{-1}$,有$\dfrac{1}{\lambda}\alpha=A^{-1}\alpha$。

证毕。
\end{zhengming}

\EX 设$\xi,\eta$是$A$的分别属于$\lambda,\mu$的特征向量,且$\lambda\neq \mu$。证明:$\xi+\eta$不可能是$A$的特征向量。

\begin{zhengming}
依题意有$A\xi=\lambda\xi,A\eta=\lambda\eta$。

假设$\xi+\eta$是$A$的特征向量,其对应的特征值为$a$。则:$A(\xi+\eta)=A\xi+A\eta=\lambda\xi+\mu\eta=a(\xi+\eta)$.

即$(\lambda-a)\xi+(\mu-a)\eta=0$,因为$\xi,\eta$是属于不同特征值的不同向量,所以$\xi,\eta$线性无关,所以$\lambda-a=\mu-a=0$,即$\lambda=\mu$,与题目中$\lambda=\mu$矛盾。所以假设不成立,即$\xi+\eta$不可能是$A$的特征向量。
\end{zhengming}

\EX 设$a$是$n$阶方阵$A$的一个特征值。证明$a^3-2a+2$是$A^3-2A+2I_n$的一个特征值。

\begin{zhengming}
设$a$对应的特征向量为$\alpha$,则$\alpha\neq 0$且$A\alpha=a\alpha$。所以:
\begin{align*}
&A^2\alpha=A(A\alpha)=A(a\alpha)=a(A\alpha)=a^2\alpha\\
&A^3\alpha=A(A^2\alpha)=A(a^2\alpha)=a^2A\alpha=a^3\alpha\\
&-2A\alpha=-2a\alpha
\end{align*}
所以:$(A^3-2A+2I_n)\alpha=a^3\alpha-2a\alpha+2\alpha=(a^3-2a+2)\alpha$,所以$a^3-2a+2$是$A^3-2A+2I_n$的一个特征值。
\end{zhengming}

\EX 求$A=
\begin{pmatrix}
1&2\\0&1
\end{pmatrix}
$的全部特征值和特征向量。

\begin{jie}
由题得:
\begin{equation*}
|\lambda E-A|=
\begin{vmatrix}
\lambda-1&2\\
0&\lambda-1
\end{vmatrix}=(\lambda-1)^2=0~~~\Rightarrow~~\lambda_1=\lambda=2=1
\end{equation*}
把该特征值代入$[\lambda E-A]$高斯消元得最简阶梯型矩阵:$
\begin{pmatrix}
0&1\\0&0
\end{pmatrix},
$所以$x_2=0,x_1\in R$,取$x_1=1$得$(\lambda E-A)x=0$的基础解系为:$(1,0)^T$。所以$A$的特征值$1$对应的特征向量为$(k_1,0)^T,k_1\neq 0$.
\end{jie}

\EX 设$A=(a_{ij})$是对角元相同且至少有一个非对角元不为0的$n$阶上三角阵,证明:$A$不可对角化。

\begin{zhengming}
设$A$的对角元都同为$a$,因为$A$为上三角阵,所以$A$的特征值为对角元的值$a$,因为至少有1个非对角元不为0,所以$(i,j)-$元$\neq=0(i<j)$。所以$(\lambda E-A)x=(aE-A)x=0$的系数矩阵的秩$r(aE-A)\geq 1$。

所以解向量个数为$n-r(aE-A)\leq n-1$,即$A$的线性无关的特征向量至多有$n-1$个线性无关的特征向量,所以$A$不可以对角化。
\end{zhengming}

\EX 设$n$阶方阵$A$可对角化,证明$A^4-2A^2+5I_{n}$也可对角化,进一步假设$A$可逆,证明$A^{-1}$也可对角化。

\begin{zhengming}
设$A$可对角化为$\Lambda$,则$A\sim \Lambda$,所以$A^4-2A^2+5I_n\sim \Lambda^4-2\Lambda^2+5I_n$,而$\Lambda^4-2\Lambda^2+5I_n$是对角阵,所以$A^4-2A^2+5I_{n}$可对角化。

若$A$可逆,由相似的性质,$\Lambda$可逆,且$A^{-1}\sim \Lambda$,而$\Lambda$为对角阵,所以$A^{-1}$也可对角化。
\end{zhengming}

\EX 设$n$阶方阵$A$可对角化,证明$A$的特征多项式可以分解为$n$个一次式的乘积,举例说明逆命题不成立。

\begin{zhengming}
设$A$可对角化为$\Lambda$,则$A\sim \Lambda$,设$\Lambda$的对角元为$a_1,a_2,\cdots,a_n$。所以$\Lambda$的特征多项式为$(\lambda-a_1)(\lambda-a_2)\cdots(\lambda-a_n)=\prod\limits_{i=1}^{n}(\lambda-a_i)=0$,因为相似矩阵具有相同的特征多项式,所以$A$的特征多项式也为该式,所以$A$的特征多项式可以分解为$n$个一次式的乘积。

逆命题:$n$阶方阵$A$的特征多项式可以分解为$n$个一次式的乘积,则$A$可对角化。

例如:$A=
\begin{pmatrix}
1&1\\0&1
\end{pmatrix}
$,$|\lambda E-A|=(\lambda-1)^2=0$,即$A$的特征多项式可以分解为$2$个一次式的乘积,不难算出,$A$只有一个线性无关的特征向量,即$A$不可对角化。所以逆命题不成立。
\end{zhengming}

\EX 设方阵$A$的特征多项式为$f(\lambda)=(\lambda-1)(\lambda+1)^2$。设$A^*$是$A$的伴随矩阵,求$(A^*)^3-2A-I_3$的行列式。

\begin{jie}
由题得:$A$的特征值为$1,-1,-1$,所以$|A|=1\neq 0$,所以$A$可逆且$|A^{-1}|=1$。

由伴随矩阵的性质:$A^*=|A|A^{-1}=A^{-1}$.所以$(A^*)^3-2A-I_3=(A^{-1})^3-2A-I$,由特征值的性质,该矩阵的特征值对应为:$(\lambda_{i}^{-1})^3-2\lambda_{i}-1$(式中$\lambda_i$为$A$的第$i$个特征值。)所以其行列式:
\begin{equation*}
\prod_{i=1}^{3}\left[(\lambda_{i}^{-1})^3-2\lambda_{i}-1\right]=0
\end{equation*}
\end{jie}

\EX 设$
A=
\begin{pmatrix}
3&2&-1\\
-2&-2&2\\
3&6&-1
\end{pmatrix}
$,对任意的正整数$n$,求$A^{n}$。

\begin{jie}
由题得:$|\lambda E-A|=
\begin{vmatrix}
\lambda-3&-2&1\\
2&\lambda+2&-2\\
-3&-6&\lambda+1
\end{vmatrix}=(\lambda-2)^{2}(\lambda+4)=0~~\rightarrow~~\lambda_1=\lambda_2=,\lambda_3=4
$.

$\lambda=2$时,代入$[\lambda E-A]$后高斯消元得:$x_1=-2x_2+x_3$,分别取$[x_2,x_3]^T=[0,1]^T,[1,0]^T$有$
\alpha_1=[1,0,1]^T,\alpha_2=[-2,1,0]^T
$。

$\lambda=-4$时,代入$[\lambda E-A]$后高斯消元得:$x_1=\frac{1}{3}x_3,x_2=-\frac{2}{3}x_3$,取$x_3=3$有$\alpha_3=[1,-2,3]^T$。

令$P=[\alpha_1,\alpha_2,\alpha_3]=
\begin{pmatrix}
1&-2&1\\
0&1&-2\\
1&0&3
\end{pmatrix}
$使得$P^{-1}AP=\Lambda=
\begin{pmatrix}
2\\&2\\&&-4
\end{pmatrix}
$,所以$A=P\Lambda P^{-1}$.求得(过程略)$P^{-1}=\dfrac{1}{6}
\begin{pmatrix}
3&6&3\\
-2&2&2\\
-1&-2&1
\end{pmatrix}
$,所以由相似的性质:
\begin{align*}
A^n&=P\Lambda^nP^{-1}=
\dfrac{1}{6}\begin{pmatrix}
1&-2&1\\
0&1&-2\\
1&0&3
\end{pmatrix}\begin{pmatrix}
2^n\\&2^n\\&&(-4)^n
\end{pmatrix}\begin{pmatrix}
3&6&3\\
-2&2&2\\
-1&-2&1
\end{pmatrix}\\&=
\dfrac{1}{6}
\begin{pmatrix}
7\cdot 2^{n}-(-4)^{n}&2^{n+1}-2(-4)^{n}&-2^{n}+(-4)^{n}\\
-2^{n+1}+2(-4)^{n}&2^{n+1}+4(-4)^{n}&2^{n+1}-2(-4)^{n}\\
3\cdot 2^{n}-3(-4)^{n}&6\cdot2^{n}-6(-4)^{n}&3\cdot2^{n}+3(-4)^{n}
\end{pmatrix}
\end{align*}
\end{jie}

\EX 设$n$阶方阵$A$满足$A^2-3A+2I_n=0$。证明:$A$可对角化。

\begin{zhengming}
设$A$的特征值为$a$,其对应的特征向量为$\alpha$,则$A\alpha=a\alpha,\alpha\neq0$.所以:
\begin{equation*}
(A^2-3A+2I_n)\alpha=(a^2-3a+2)\alpha=0~~~\rightarrow~~a^2-3a+2=0~~\rightarrow~~a=1\text{或}a=2
\end{equation*}
即$a=1$或$a=2$为$A$的特征值。

又因为$(aI-A)x=0$的基础解系包含的向量的个数为$n-r(aI-A)$。

$a=1$时为$n-r(I-A)$,$a=2$时为$n-r(2I-A)$,要证明$A$可相似对角化,需证明$A$有$n$个线性无关的特征向量,即证明$(n-r(I-A))+(n-r(2I-A))=n$即可。即证明$r(I-A)+r(2I-A)=n$。(第四次线性代数例2.26)

因为$A^3-3A+2I=0$,所以$(A-I)(A-2I)=0$,所以$r(A-2I)+r(A-I)\leq n$,即$r(I-A)+r(2I-A)\leq n$。

由$I=(A-2I)+(-(A-I))$得:$n=r(I)\leq r(A-I)+r(A-2I)=r(I-A)+r(2I-A)$

综上两个不等式得$r(I-A)+r(2I-A)=n$,即$A$可对角化。
\end{zhengming}

\EX 若$4$阶矩阵$A$与$B$相似,矩阵$A$的特征值为$
\dfrac{1}{2},\dfrac{1}{3},\dfrac{1}{4},\dfrac{1}{5},
$则行列式$|B^{-1}-E|=$\underline{\hphantom{~~~~~~~}}。(E是单位阵)

\begin{jie}
相似矩阵具有相同的特征值,所以$A$与$B$的特征值一样。由特征值的性质$B^{-1}-E$的特征值为$\lambda_{i}^{-1}-1$,式中$\lambda_i$表示$B$的第$i$个特征值。由特征值和行列式的关系:
\begin{equation*}
|B^{-1}-E|=\prod_{i=1}^{4}(\lambda_{i}^{-1}-1)=24
\end{equation*}
\end{jie}

\EX 矩阵$A=
\begin{pmatrix}
0&-2&-2\\
2&2&-2\\
-2&-2&2
\end{pmatrix}
$的非零特征值是\underline{\hphantom{~~~~~~~}}。

\begin{jie}
\begin{equation*}
|\lambda E-A|=
\begin{vmatrix}
\lambda&2&2\\
-2&\lambda-2&2\\
2&2&\lambda-2
\end{vmatrix}\xlongequal{r_2+r_3}
\begin{vmatrix}
\lambda&2&2\\
0&\lambda&\lambda\\
2&2&\lambda-2
\end{vmatrix}\xlongequal{c_2-c_3}
\begin{vmatrix}
\lambda&0&2\\
0&0&\lambda\\
2&4-\lambda&\lambda-2
\end{vmatrix}=-(4-\lambda)\lambda^{2}
\end{equation*}
所以$A$的非零特征值是$4$。
\end{jie}

\EX 设矩阵$
A=
\begin{pmatrix}
-1&2&2\\
2&-1&-2\\
2&-2&-1
\end{pmatrix}
$.

(1)求矩阵$A$的特征值;

(2)利用(1)的结果,求矩阵$E+A^{-1}$的特征值,其中$E$是3阶单位矩阵。

(3)利用(1)的结果,求矩阵$E+A^{*}$的特征值,其中$A^{*}$是$A$的伴随矩阵。

\begin{jie}
(1)
\begin{align*}
|\lambda E-A|&=
\begin{vmatrix}
\lambda+1&-2&-2\\
-2&\lambda+1&2\\
-2&2&\lambda+1
\end{vmatrix}
\xlongequal{c_2-c_3}
\begin{vmatrix}
\lambda&0&-2\\
-2&\lambda-1&2\\
-2&1-\lambda&\lambda+1
\end{vmatrix}\xlongequal{r_2+r_3}
\begin{vmatrix}
\lambda&0&-2\\
-4&0&\lambda+3\\
-2&1-\lambda&\lambda+1
\end{vmatrix}\\&=-(1-\lambda)[(\lambda+3)(\lambda+1)-8]=0~~\lambda_1=\lambda_2=1,\lambda_3=-5
\end{align*}

(2)由(1)得$E+A^{-1}$的特征值为$1+\lambda_{i}^{-1}$,即$2,2,\dfrac{4}{5}$.

(3)由(1)得:$|A|=\prod\limits_{i=1}^{3}\lambda_i=-5\neq0$,所以$A$可逆,所以$A^*=|A|A^{-1}=-5A^{-1}$,所以$E+A^*=E-5A^{-1}$,特征值为$1-5\lambda_{i}^{-1}$,即$-4,-4,2$。
\end{jie}

\EX 设$
A=
\begin{pmatrix}
3&2&2\\
2&3&2\\
2&2&3
\end{pmatrix},P=
\begin{pmatrix}
0&1&0\\
1&0&1\\
0&0&1
\end{pmatrix}
,B=P^{-1}A^{*}P$,其中$A^{*}$是$A$的伴随矩阵,$E$为3阶单位矩阵。求$B+2E$的特征值与特征向量。

\begin{jie}
由题可以求出:
\begin{equation*}
P^{-1}=
\begin{pmatrix}
0&1&-1\\
1&0&0\\
0&0&1
\end{pmatrix},~~~A^*=
\begin{pmatrix}
5&-2&-2\\
-2&5&-2\\
-2&-2&5
\end{pmatrix}
\end{equation*}
所以$C=B+2E=
\begin{pmatrix}
9&0&0\\
-2&7&-4\\
-2&-2&5
\end{pmatrix}
$,其特征多项式为
\begin{equation*}
|C|=
\begin{vmatrix}
 \lambda-9&0&0\\
2& \lambda-7&4\\
2&2& \lambda-5
\end{vmatrix}=(\lambda-9)[(\lambda-7)(\lambda-5)-8]=0~~\Rightarrow~~\lambda_1=\lambda_2=9,\lambda_3=3
\end{equation*}
对于特征值$9$,$(\lambda E-C)x=0$的系数矩阵经高斯消元后得:$x_1=-x_2-2x_3$,分别取$[x_2,x_3]^T=[1,0]^T,[0,1]^T$得到基础解系:$\alpha_1=[-1,1,0]^T,\alpha_2=[-2,0,1]^T$,所以$B+2E$的属于$9$的全部特征向量为$k_1\alpha_1+k_2\alpha_2$,其中$k_1,k_2$不全为0.

同理,对于特征值3,$(\lambda E-C)x=0$的系数矩阵经高斯消元后得:$x_1=0,x_2=x_3$,取$x_3=1$得基础解系$\alpha_3=[0,1,1]^T$,即$B+2E$的属于$3$的全部特征向量为$k_3\alpha_3(k_3\neq 0)$.

综上所述:……
\end{jie}

\EX 设$\alpha=(a_1,\cdots,a_n),\beta=(b_1,\cdots,b_2)^T$都是非零向量,且满足条件$\alpha^T\beta=0$。记$n$阶矩阵$A=\alpha\beta^T$,求

(1)$A^2$;

(2)矩阵$A$的特征值和特征向量。

\begin{jie}
(1) $A^2=(\alpha\beta^T)(\alpha\beta^T)=0$(注:不难验证,$\alpha^T\beta=\beta^T\alpha$)

(2)设$\lambda$是$A$的任意特征值,则$\lambda^2$是$A^2$的特征值,由(1)得$\lambda=0$,所以$A$的全部特征值都是0.由于$\alpha,\beta$都是非零向量,所以$A=\alpha\beta^T$是非零矩阵,所以$1\leq r(A)\leq \min\{r(\alpha),r(\beta^T)\}\leq=1$,即$r(A)=1$。所以$A$的属于特征值0的线性无关的特征向量的个数为$n-1$。设$\xi$是$\beta^Tx=b_1x_1+\cdots+b_nx_n=0$的任意非零解,则$A\xi=\alpha\beta^T\xi=0\alpha=0$,即$\xi$是$A$的属于特征值$0$的特征向量。任取$\beta^Tx=0$的一个基础解系:$\xi_1,\cdots,\xi_{n-1}$,则$A$属于特征值0的全部特征向量为$c_1\xi_1+\cdots+c_{n-1}\xi_{n-1}$。其中$c_1,\cdots,c_n$不全为0.
\end{jie}

\EX 令$\alpha=(1,1,0)^T$,实对称矩阵$A=\alpha\alpha^T$。

(1)求可逆矩阵$P$使得$P^{-1}AP$是对角阵,并写出这个对角阵;

(2)求$|I-A^{2017}|$,其中$I$是三阶方阵。

\begin{jie}
由题得:$\alpha^T\alpha=1+1=2$,所以$A\alpha=\alpha\alpha^T\alpha=2\alpha$,所以$2$为$A$的一个特征值,计算得$r(A)=1$,因为$A$为实对称矩阵,所以一定可以对角化。即$A\sim \Lambda=
\begin{pmatrix}
2\\&\lambda_2\\&&\lambda_3
\end{pmatrix}
$,相似矩阵具有相同的秩,所以$r(\Lambda)=1$,所以$\lambda_2=\lambda_3=0$。

对于特征值0,代入$(\lambda E-A)x=0$的系数矩阵后高斯消元得:$x_1=-x_2,x_3\in R$,所以分别取$[x_2,x_3]^T=[1,0]^T,[0,1]^T$得基础解系$\alpha_1=[-1,1,0]^T,\alpha_2=[0,0,1]^T$。

同理对于特征值2,代入$(\lambda E-A)x=0$的系数矩阵后高斯消元得:$x_1=x_2,x_3=0$,取$x_2=1$得基础解系$\alpha_3=[1,1,0]^T$。令$
P=[\alpha_3,\alpha_1,\alpha_2]=
\begin{pmatrix}
1&-1&0\\
1&1&0\\
0&0&1
\end{pmatrix}
$,则$P^{-1}AP=\Lambda$。

(2)由特征值的性质以及特征值与行列式的关系有$
|I-A^{2017}|=\prod\limits_{i=1}^3(1-\lambda^{2017})=1-2^{2017}
$
\end{jie}

\EX 已知$
\xi=
\begin{pmatrix}
1&1&-1
\end{pmatrix}
$是$A=
\begin{pmatrix}
2&-1&2\\
5&a&3\\
-1&b&-2
\end{pmatrix}
$的一个特征向量。

(1)试确定参数$a,b$及特征向量$\xi$所对应的特征值;

(2)问$A$能否相似于对角阵?说明理由。

\begin{jie}
(1)由题设$\xi$对应的特征值为$\lambda$,则$A\xi=\lambda\xi$:
\begin{equation*}
  \begin{cases}
    2-1-2=\lambda\\
    5+a-3=\lambda\\
    -1+b-2=-\lambda
  \end{cases}~~\Rightarrow~~
\begin{cases}
\lambda=-1\\
a=3\\b=0
\end{cases}
\end{equation*}

(2)由(1)知:
\begin{equation*}
|\lambda E-A|=
\begin{vmatrix}
\lambda-2&1&-2\\
-5&\lambda+3&-3\\
1&0&\lambda+2
\end{vmatrix}=(\lambda+1)^3=0~~~\Rightarrow~~~\lambda_1=\lambda_2=\lambda_3=-1
\end{equation*}
把$-1$代入$(\lambda E-A)x=0$的系数矩阵后化简得$r(\lambda E-A)=2$,所以其对应的基础解系的个数为1,即$A$线性无关的特征向量只有1个,因此不能对角化。
\end{jie}

\EX 已知矩阵$
A=
\begin{pmatrix}
2&0&0\\
0&0&1\\
0&1&x
\end{pmatrix}
$与对角阵$
B=
\begin{pmatrix}
2&0&0\\
0&y&0\\
0&0&-1
\end{pmatrix}
$相似。

(1)求$x,y$的值;

(2)求一个满足$P^{-1}AP=B$的可逆矩阵$P$。

\begin{jie}
(1)相似矩阵具有相同的迹和行列式:
\begin{equation*}
  \begin{cases}
   2+x=2+y-1\\
   -2=-2y
  \end{cases}~~\Rightarrow~~
  \begin{cases}
   x=0\\
   y=-1
  \end{cases}
\end{equation*}

(2)$A\sim B$,所以$A$的特征值为$2,1,-1$。

对于特征值2,代入$(\lambda E-A)x=0$的系数矩阵后高斯消元得:
$x_1\in R,x_2=x_3=0$,取$x_1=1$得基础解系$\alpha_1=[1,0,0]^T$。

对于特征值1,代入$(\lambda E-A)x=0$的系数矩阵后高斯消元得:
$x_1=0,x_2=x_3$,取$x_3=1$得基础解系$\alpha_2=[0,1,1]^T$。

对于特征值-1,代入$(\lambda E-A)x=0$的系数矩阵后高斯消元得:
$x_1=0,x_2=-x_3$,取$x_3=1$得基础解系$\alpha_3=[0,-1,1]^T$。

令$
P=[\alpha_1,\alpha_2,\alpha_3]^T=
\begin{pmatrix}
1&0&0\\
0&1&-1\\
0&1&1
\end{pmatrix}
$,则$P^{-1}AP=B$。
\end{jie}

\EX 设$A=
\begin{pmatrix}
0&0&1\\
x&1&y\\
1&0&0
\end{pmatrix}
$有三个线性无关的特征向量,求$x$和$y$应满足的条件。

\begin{jie}
由题得:$|\lambda E-A|=
\begin{vmatrix}
\lambda&0&-1\\
-x&\lambda-1&-y\\
-1&0&\lambda
\end{vmatrix}=(\lambda+1)(\lambda-1)^2=0~~\Rightarrow~~\lambda_1=-1,\lambda_2=\lambda_3=1
$.

当$\lambda=1$时,$[\lambda E-A]\Rightarrow
\begin{pmatrix}
1&0&-1\\
-x&0&-y\\
0&0&0
\end{pmatrix}$当$r(\lambda E-A)=1$时,$\lambda=1$对应的特征向量有2个,此时$A$有三个线性无关的特征向量,所以$
\frac{1}{-x}=\frac{1}{-y}
$,即$x+y=0$。
\end{jie}

\EX 设矩阵$A=
\begin{pmatrix}
1&2&-3\\-1&4&-3\\1&a&5
\end{pmatrix}
$的特征方程有一个二重根,求$a$的值,并讨论$A$是否可相似对角化。

\begin{jie}
由题得:
\begin{equation*}
f(\lambda)=|\lambda E-A|=(\lambda-2)(\lambda^2-8\lambda+18+3a)
\end{equation*}

(1)$2$是二重根,则$2^2-8\times 2+18+3a=0$,即$a=-2$,此时$f(\lambda)=(\lambda-2)^2(\lambda-6)$符合要求,继续计算(验证是否可以相似对角化)$6E-A$的秩为2,所以$A$的特征值$6$的特征向量只有一个,$2E-A$的秩为1,所以$A$的特征值$2$的特征向量有两个,所以此时$A$可以对角化。

(2)$2$不是二重根,则$\lambda^2-8\lambda+18+3a=0\Rightarrow
\lambda^2-8\lambda+16=-2-3a
$,所以$-2-3a=0$,此时$f(\lambda)=(\lambda-4)^2(\lambda-2)$,符合要求。继续计算:计算得$r(2E-A)=2$,所以$A$的特征值$2$的特征向量有一个,计算得$r(4E-A)=2$,所以$A$的特征值$4$的特征向量有一个,此时$A$的线性无关的特征向量的个数为$1+1=2\leq 3$,所以$A$不可对角化。

综上所述,$a=-2$,此时$A$可以对角化,$a=-\dfrac{2}{3}$,此时$A$不可对角化。
\end{jie}

\EX 设$n$阶矩阵$
A=
\begin{pmatrix}
1&b&\cdots&b\\
b&1&\cdots&b\\
\vdots&\vdots&\ddots&\vdots\\
b&b&\cdots&1
\end{pmatrix}
$.

(1)求$A$的特征值和特征向量;

(2)求可逆矩阵$P$,使得$P^{-1}AP$为对角矩阵。

\begin{tips}
设$A=[a_{ij}]$是三阶矩阵,则(该式不做推导,感兴趣的可以自己算一下)
\begin{equation*}|\lambda E-A|=
  \begin{bmatrix}
    \lambda-a_{11} & -a_{12} &-a_{13} \\
-a_{21} & \lambda-a_{22} &-a_{23} \\
-a_{31} & -a_{32} &\lambda-a_{33}
  \end{bmatrix}=\lambda^{3}-\lambda^2\sum a_{ii}+S_{2}\lambda-|A|
\end{equation*}
式中:$S_{2}=
\begin{vmatrix}
  a_{11} & a_{12} \\
  a_{21} & a_{22}
\end{vmatrix}+\begin{vmatrix}
  a_{11} & a_{13} \\
  a_{31} & a_{33}
\end{vmatrix}+\begin{vmatrix}
  a_{22} & a_{23} \\
  a_{32} & a_{33}
\end{vmatrix}
$。

若$r(A)=1$,则$|A|=0,S_{2}=0$,代入到上式有
\begin{equation*}
|\lambda E-A|=\lambda^{3}-\sum a_{ii}\lambda^2=\lambda^2\left(\lambda-\sum a_{ii}\right)
\end{equation*}

做推广,对于$n$阶矩阵$A$,若$r(A)=1$,则$|\lambda E-A|=\lambda^{n-1}\left(\lambda-\sum a_{ii}\right)$

例题:已知$a\neq 0$,求矩阵
\begin{equation*}
  \begin{bmatrix}
    1 & a & a& a\\
    a & 1& a& a\\
    a& a& 1& a\\
    a& a& a& 1
  \end{bmatrix}
\end{equation*}
的特征值、特征向量。

\begin{jie}
方法一:(直接计算)

由特征多项式:
\begin{equation*}
  \begin{vmatrix}
\lambda E-A
  \end{vmatrix}
  =\begin{vmatrix}
     \lambda-1 & -a& -a& -a \\
     -a& \lambda-1& -a& -a\\
     -a& -a& \lambda-1& -a\\
     -a& -a& -a& \lambda-1
   \end{vmatrix}=\left[\lambda-(3a+1)\right]\left(\lambda+a-1\right)^{3}
\end{equation*}
得$A$的特征值是$3a+1,1-a$。

当$\lambda=3a+1$时,由$[(3a+1)E-A]=0$,即
\begin{align*}
\begin{bmatrix}
3a & -a& -a& -a \\
-a& 3a& -a& -a\\
-a& -a& 3a& -a\\
-a& -a& -a& 3a
\end{bmatrix}\rightarrow
\begin{bmatrix}
3 & -1& -1& -1 \\
-1& 3& -1& -1\\
-1& -1& 3& -1\\
-1& -1& -1& 3
\end{bmatrix}\rightarrow
\begin{bmatrix}
1 & -3& 1& 1 \\
1& 1& -3& 1\\
1& 1& 1& -3\\
0&0& 0& 0
\end{bmatrix}\rightarrow
\begin{bmatrix}
1 & 0& 0& -1 \\
0& 1& 0& -1\\
0& 0& 1& -1\\
0&0& 0& 0
\end{bmatrix}
\end{align*}
可得基础解系为$\alpha_{1}=(1,1,1,1)^T$,所以$\lambda=3a+1$的特征向量为$k_{1}\alpha_1,(k_1\neq 0)$。

当$\lambda=1-a$时,由$[(1-a)E-A]=0$,即
\begin{equation*}
  \begin{bmatrix}
-a & -a & -a& -a\\
-a & -a & -a& -a\\
-a & -a & -a& -a\\
-a & -a & -a& -a
  \end{bmatrix}\rightarrow
  \begin{bmatrix}
    1 & 1 & 1& 1\\
    0 & 0& 0& 0\\
    0 & 0& 0& 0\\
    0 & 0& 0& 0
  \end{bmatrix}
\end{equation*}
得基础解系$\alpha_2=(-1,1,0,0)^T,\alpha_3=(-1,0,1,0)^T\alpha_4=(-1,0,0,1)^T$,所以$\lambda=1-a$的特征向量为$k_2\alpha_2+k_3\alpha_3+k_4\alpha_4$,式中$k_2,k_3,k_4$是不全为0的任意常数。

方法二:(转换法)

由题得:
\begin{equation*}A=
  \begin{bmatrix}
    a & a& a& a \\
    a & a& a& a \\
    a & a& a& a \\
    a & a& a& a
  \end{bmatrix}+
   \begin{bmatrix}
    1-a & 0& 0& 0 \\
    0 & 1-a& 0& 0 \\
    0 & 0& 1-a& 0 \\
    0 & 0& 0& 1-a
  \end{bmatrix}=B+(1-a)E
\end{equation*}
由于$r(B)=1$,所以有
\begin{equation*}
  |\lambda E -B|=\lambda^{4-1}\left(\lambda-\sum\limits_{i=1}^{4}a_{ii}\right)=
  \lambda^{3}\left(\lambda-4a\right)
\end{equation*}
所以矩阵$B$的特征值为$0,0,0,4a$,所以由特征值的性质,$A$的特征值为$3a+1,1-a,1-a,1-a$。

下边同方法一。
\end{jie}
\hphantom{.}
\end{tips}

\begin{jie}
由题得:$A
=
(1-b)E_{n}+b
\begin{pmatrix}
1&1&\cdots&1\\
1&1&\cdots&1\\
\vdots&\vdots&\ddots&\vdots\\
1&1&\cdots&1
\end{pmatrix}=(1-b)E_{n}+bC
$。单位矩阵的特征值为$1$,$r(C)=1$由上边的结论可以得出$C$的特征值为$n,0,0,\cdots,0$,由特征值的性质得$A$的特征值为$(1-b)\times 1+b\lambda_{ci}$,即$\lambda_1=1-b+nb=1+(n-1)b,\lambda_2=\lambda_3=\cdots\lambda_n=1-b$

由上边的例题可知:

$\lambda=1-b$,代入$(\lambda E-A)$后得基础解系:$
\alpha_1=[-1,1,0,\cdots,0]^T,\alpha_2=[-1,0,1,0,\cdots,0],\cdots,\alpha_{n-1}=[-1,0,\cdots,0,1]^T
$,所以其特征向量为$k_1\alpha_1+k_2\alpha_2+\cdots+k_{n-1}\alpha_{n-1},(k_1,\cdots,k_{n-1}\text{不全为0})$.

$\lambda=1-(n-1)b$时,基础解系为$[1,1,\cdots,1]^T$,所以其对应的特征向量为$\alpha_n=k_{n}[1,1,\cdots,1]^T,k_n\neq 0$。

(2)$b=0$时,$A=E_n$,任取可逆矩阵$P$都可以使$A$对角化。

$b\neq 0$时,令$P=
\begin{pmatrix}
\alpha_1&\alpha_2&\cdots&\alpha_n
\end{pmatrix}
$,则$\Lambda=P^{-1}AP=
\begin{pmatrix}
1-b\\
&1-b\\
&&\ddots\\
&&&1-b\\
&&&&1+(n-1)b
\end{pmatrix}
$。
\end{jie}

\EX 设3阶矩阵$A$的特征值为$\lambda_1=1,\lambda_2=2,\lambda_3=3$,对应的特征向量依次为$
\xi_1=
\begin{pmatrix}
1\\1\\1
\end{pmatrix},\xi_2=
\begin{pmatrix}
1\\2\\4
\end{pmatrix},\xi_3=
\begin{pmatrix}
1\\3\\9
\end{pmatrix}
$,又向量$
\beta=
\begin{pmatrix}
1\\1\\3
\end{pmatrix}
$.

(1)将$\beta$用$\xi_1,\xi_2,\xi_3$线性表出。

(2)求$A^{n}\beta$(n为自然数).

\begin{jie}
(1)由题得:
\begin{align*}
(\xi_1,\xi_2,\xi_3,\beta)&=
\begin{pmatrix}
1&1&1&1\\
1&2&3&1\\
1&4&9&3
\end{pmatrix}\xrightarrow{\substack{r_2-r_1\\ r_{3}-r_1}}
{
\begin{pmatrix}
1&1&1&1\\
0&1&2&0\\
0&3&8&2
\end{pmatrix}
}\xrightarrow{\substack{r_{3}-3r_2}}
{
\begin{pmatrix}
1&1&1&1\\
0&1&2&0\\
0&0&2&2
\end{pmatrix}
}\\
&\xrightarrow{\substack{r_{3}\div 2}}
{
\begin{pmatrix}
1&1&1&1\\
0&1&2&0\\
0&0&1&1
\end{pmatrix}
}\xrightarrow{\substack{r_{2}-2r_3\\ r_1-r_3}}
{
\begin{pmatrix}
1&1&0&0\\
0&1&0&-2\\
0&0&1&1
\end{pmatrix}
}\xrightarrow{\substack{r_1-r_2}}
{
\begin{pmatrix}
1&0&0&2\\
0&1&0&-2\\
0&0&1&1
\end{pmatrix}
}
\end{align*}
由最简阶梯型可以看出:$\beta=2\xi_1-2\xi_2+\xi_3$。

(2)\begin{align*}
   A^n\beta=A^{n}(2\xi_1-2\xi_2+\xi_3)=2A^n\xi_1-2A^n\xi_2+A^n\xi_3
   \end{align*}
   由特征值的性质:
\begin{align*}
2A^n\xi_1-2A^n\xi_2+A^n\xi_3&=2\lambda_{1}^n\xi_1-2\lambda_{2}^n\xi_2+\lambda_{3}^n\xi_3=2\xi_1-2^{n+1}\xi_2+3^n\xi_3\\
&=(2-2^{n+1}+3^n,2-2^{n+2}+3^{n+1},2-2^{n+3}+3^{n+2})^T
\end{align*}
\end{jie}

\EX 某试验性生产线每年一月份进行熟练工与非熟练工的人数统计,然后将六分之一的熟练工支援其他生产部门,其缺额由招收新的非熟练工补齐。新、老非熟练工经过培训及实践至年终考核有五分之二成为熟练工。设第$n$年一月份统计的熟练工和非熟练工所占百分比分别为$x_n,y_n$,记成向量$
\begin{pmatrix}
x_n\\ y_n
\end{pmatrix}
$。

(1)求$
\begin{pmatrix}
x_{n+1}\\ y_{n+1}
\end{pmatrix}
$与$
\begin{pmatrix}
x_n\\ y_n
\end{pmatrix}
$的关系式并写成矩阵形式:
$
\begin{pmatrix}
x_{n+1}\\ y_{n+1}
\end{pmatrix}=A
\begin{pmatrix}
x_n\\ y_n
\end{pmatrix}
$;

(2)验证$\eta_1=
\begin{pmatrix}
4\\1
\end{pmatrix},\eta_2=
\begin{pmatrix}
-1\\1
\end{pmatrix}
$是$A$的两个线性无关的特征向量,并求出相应的特征值;

(3)当$
\begin{pmatrix}
x_{1}\\ y_{2}
\end{pmatrix}=A
\begin{pmatrix}
\dfrac{1}{2}\\[4pt] \dfrac{1}{2}
\end{pmatrix}
$,求$
\begin{pmatrix}
x_{n+1}\\ y_{n+1}
\end{pmatrix}
$。

\begin{jie}
(1)由题可设:$x_{n+1}=\dfrac{5}{6}x_n+\dfrac{2}{5}\left(\dfrac{1}{6}x_n+y_n\right),y_{n+1},y_{n+1}=\dfrac{3}{5}\left(\dfrac{1}{6}x_n+y_n\right)$,即:
\begin{equation*}
\begin{pmatrix}
x_{n+1}\\ y_{n+1}
\end{pmatrix}=\frac{1}{10}
\begin{pmatrix}
9&4\\1&6
\end{pmatrix}\begin{pmatrix}
x_{n}\\ y_{n}
\end{pmatrix}
\end{equation*}
令$
A=\frac{1}{10}
\begin{pmatrix}
9&4\\1&6
\end{pmatrix}
$,则$
\begin{pmatrix}
x_{n+1}\\ y_{n+1}
\end{pmatrix}=A
\begin{pmatrix}
x_n\\ y_n
\end{pmatrix}
$。

(2)令$B=\begin{pmatrix}
9&4\\1&6
\end{pmatrix}$
\begin{equation*}
B\eta_1=\begin{pmatrix}
9&4\\1&6
\end{pmatrix}\begin{pmatrix}
4\\1
\end{pmatrix}=10\begin{pmatrix}
4\\1
\end{pmatrix},~~~~~~B\eta_2=\begin{pmatrix}
9&4\\1&6
\end{pmatrix}\begin{pmatrix}
-1\\1
\end{pmatrix}=5\begin{pmatrix}
-1\\1
\end{pmatrix}
\end{equation*}
所以:$\eta_1,\eta_2$是$B$对应于特征值$10,5$的特征向量。

因为$A=\dfrac{1}{10}B$,所以$\eta_1,\eta_2$是对应于特征值$1,0.5$的特征向量。

(3)由(1)得$
\begin{pmatrix}
x_{n+1}\\ y_{n+1}
\end{pmatrix}=A
\begin{pmatrix}
x_n\\ y_n
\end{pmatrix}
,\begin{pmatrix}
x_n\\ y_n
\end{pmatrix}
=A\begin{pmatrix}
x_{n-1}\\ y_{n-1}
\end{pmatrix}\cdots,\begin{pmatrix}
x_{n+1}\\ y_{n+1}
\end{pmatrix}=A^n
\begin{pmatrix}
x_1\\ y_1
\end{pmatrix}$.

由(2)得$
\begin{pmatrix}
 x_1\\ x_2
\end{pmatrix}=\begin{pmatrix}
               \dfrac{1}{2}\\[4pt]\dfrac{1}{2}
              \end{pmatrix}=\dfrac{1}{5}\eta_1+\dfrac{3}{10}\eta_2
$,所以:
\begin{equation*}
\begin{pmatrix}
x_{n+1}\\ y_{n+1}
\end{pmatrix}=A^n\left(\dfrac{1}{5}\eta_1+\dfrac{3}{10}\eta_2\right)=\dfrac{1}{5}A^{n}\eta_1+\dfrac{3}{10}A^{n}\eta_2=
\begin{pmatrix}
\dfrac{4}{5}-\dfrac{3}{10\cdot 2^n}\\[4pt]\dfrac{1}{5}+\dfrac{3}{10\cdot 2^n}
\end{pmatrix}
\end{equation*}
\end{jie}

\clearpage
\hphantom{~~}\hfill {\zihao{3}\heiti 第十一次线性代数} \hfill\hphantom{~~}
\addcontentsline{toc}{section}{\protect\numberline {}第十一次线性代数}

\hphantom{~~}

%\section{第十一次线性代数}
\EX 设$\xi_1=
\begin{pmatrix}
1\\ -2\\0
\end{pmatrix},\xi_2=
\begin{pmatrix}
1\\ 0\\-1
\end{pmatrix}
$,求所有与$\xi_1,\xi_2$都正交的向量。

\begin{jie}
设该正交向量为$\xi=
(x_1,x_2,x_3)^T
$,依正交的定义:$\xi^T\xi_{1}=0,\xi^{T}\xi_{2}=0$即:
\begin{equation*}
  \begin{cases}
    x_1-2x_2=0\\ x_1-x_3=0
  \end{cases}~~\Rightarrow~~
  \begin{cases}
    x_1=x_3\\
    x_2=0.5x_3
  \end{cases}
\end{equation*}
取$x_3=2k,k\in R$,所以$\xi=(2k,k,2k)^T,k\in R$
\end{jie}

\EX 设$n$维向量$\xi$与$n$个线性无关的$n$维向量正交。证明$\xi=0$。

\begin{zhengming}
设$
\xi=
\begin{pmatrix}
x_1\\x_2\\ \vdots\\x_n
\end{pmatrix}
$,设$\alpha_1,\cdots,\alpha_n$线性无关,其中$
\alpha_i=
\begin{pmatrix}
a_{i1}\\a_{i2}\\ \vdots\\a_{in}
\end{pmatrix}
$,则$\xi$与$\alpha_1,\cdots,\alpha_n$都正交等价于$\xi$是方程组
$
\begin{cases}
a_{11}x_{1}+a_{12}x_{2}+\cdots+a_{1n}x_{n}=0\\
a_{21}x_{1}+a_{22}x_{2}+\cdots+a_{2n}x_{n}=0\\
\cdots\\
a_{n1}x_{1}+a_{n2}x_{2}+\cdots+a_{nn}x_{n}=0 \end{cases}
$的解。可以看出该齐次线性方程组的系数矩阵的行向量为$\alpha_1,\cdots,\alpha_n$,因为$\alpha_1,\cdots,\alpha_n$线性无关,所以该系数矩阵秩为$n$,所以该齐次方程组只有零解,所以$\xi=0$。
\end{zhengming}

\EX 设$\xi_1=
\begin{pmatrix}
1\\1\\0\\1
\end{pmatrix},\xi_2=
\begin{pmatrix}
1\\0\\1\\0
\end{pmatrix},\xi_3=
\begin{pmatrix}
1\\0\\0\\1
\end{pmatrix},
$求一个与$\xi_1,\xi_2,\xi_3$等价的两两正交的向量组。

\begin{jie}
$r(\xi_1,\xi_2,\xi_3)=3$,所以$\xi_1,\xi_2,\xi_3$线性无关的,使用施密特正交化方法:
\begin{align*}
&\eta_1=\xi_1\\
&\eta_2=\xi_2-
\frac{\xi_2^T\eta_1}{\eta_1^T\eta_1}\eta_1=
\begin{pmatrix}
0.5\\-0.5\\1\\0
\end{pmatrix}\\
&\eta_3=\xi_3-\sum_{i=1}^{2}\left(
\frac{\xi_3^T\eta_i}{\eta_i^T\eta_i}\eta_i\right)=
\begin{pmatrix}
\frac{1}{3}\\-\frac{1}{3}\\-\frac{1}{3}\\1
\end{pmatrix}
\end{align*}
$\eta_1,\eta_2,\eta_3$即为所求。
\end{jie}

\EX 设$A$是$n$阶方阵。

(1)设存在$n$维非零列向量$\xi$使得$A\xi=2\xi$。证明:$A$不可能是正交阵。

(2)假设$A$是正交阵。证明:$2I_{n}-A$是可逆矩阵。

\begin{zhengming}
(1)假设$A$是正交阵,则由课本146页命题4.3.2有$(A\xi)^T(A\xi)=\xi^T\xi$,又因为$A\xi=2\xi$,所以$(A\xi)^T(A\xi)= (2\xi^T)(2\xi)=4\xi^T\xi$,因为$\xi$非零,所以$(A\xi)^T(A\xi)= \xi^T\xi=4\xi^T\xi$矛盾,所以假设不成立,即$A$不是正交阵。

(2)$A$是正交矩阵,由(1)得,$2$不是$A$的特征值。所以$2I_n-A$的特征值不可能有0,由特征值和行列式的关系知,$2I_{n}-A$的行列式不会为0,即可逆。
\end{zhengming}

\EX 设$A=
\begin{pmatrix}
4&-1&-1&1\\
-1&4&1&-1\\
-1&1&4&1\\
1&-1&-1&4
\end{pmatrix}$,问$A$能否对角化?若能对角化,求一个与$A$相似的对角阵。

\begin{jie}
实对称矩阵一定可以对角化。

\begin{align*}
|\lambda E-A|=
\begin{vmatrix}
\lambda-4&1&1&-1\\
1&\lambda-4&-1&1\\
1&-1&\lambda-4&-1\\
-1&1&1&\lambda-4
\end{vmatrix}\xlongequal{\substack{r_{2}+r_{4}\\ r_{3}+r_{4}}}
\begin{vmatrix}
\lambda-4&1&1&-1\\
0&\lambda-3&0&\lambda-3\\
0&0&\lambda-3&\lambda-3\\
-1&1&1&\lambda-4
\end{vmatrix}=(\lambda-3)^3(\lambda-7)=0
\end{align*}
所以得到的对称矩阵为:$\Lambda=
\begin{pmatrix}
3\\&3\\&&3\\&&&7
\end{pmatrix}
$
\end{jie}

\EX 设$A=
\begin{pmatrix}
1&-2&-4\\
-2&4&-2\\
-4&-2&1
\end{pmatrix}
$,求正交矩阵$P$使得$P^{-1}AP$是对角阵,并写出一个这样的对角阵。

\begin{tips}
解题步骤:(以3阶为例)

(1)求矩阵$A$的特征值$\lambda_1,\lambda_2,\lambda_3;$

(2)求出矩阵的特征向量$\alpha_1,\alpha_2,\alpha_3$;

(3)把(2)特征向量改造(施密特正交化,单位化)为$\gamma_1,\gamma_2,\gamma_3$;(实对称矩阵的不同特征值对应的特征向量是正交的,如果三个特征值不同,则三个特征值对应的特征向量组成的向量组就是正交的,此时直接单位化各向量即可。)

(4)构造正交矩阵$P=[\gamma_1,\gamma_2,\gamma_3]$
\end{tips}

\begin{jie}
由题得:
\begin{align*}
|\lambda E-A|&=
\begin{vmatrix}
\lambda-1&2&4\\
2&\lambda-4&2\\
4&2&\lambda-1
\end{vmatrix}\xlongequal{r_1-r_3}
\begin{vmatrix}
\lambda-5&0&5-\lambda\\
2&\lambda-4&2\\
4&2&\lambda-1
\end{vmatrix}\xlongequal{c_3+c_1}
\begin{vmatrix}
\lambda-5&0&0\\
2&\lambda-4&4\\
4&2&\lambda-3
\end{vmatrix}\\&=(\lambda-5)[(\lambda-4)(\lambda-3)-8]~~~\Rightarrow~~~\lambda_{1}=\lambda_{2}=5,\lambda_3=-4
\end{align*}

对于$\lambda=5$,用高斯消元化简$(5E-A)x=0$的系数矩阵后得:$x_1=-\frac{1}{2}x_2-x_3$,分别取$[x_2,x_3]^T=[0,1]^T,[2,0]^T$得$\alpha_1=[-1,0,1]^T,\alpha_2=[-1,2,0]^T$,施密特正交化得:$\xi_1=\alpha_1,\xi_2=\alpha_2
-\dfrac{\alpha_2^T\xi_1}{\xi_1^T\xi}\xi_1=-\dfrac{1}{2}[1,-4,1]^T
$,单位化:$\eta_1=\dfrac{1}{\sqrt{2}}[-1,0,1]^T,\eta_{2}=-\dfrac{\sqrt{2}}{6}[1,-4,1]^T$。

对于$\lambda=-4$,用高斯消元化简$(5E-A)x=0$的系数矩阵后得:$x_1=x_3,x_2=0.5x_3$,取$x_3=2$得$\alpha_3=[2,1,2]^T$,对于实对称矩阵,不同的特征值对应的特征向量是正交的,即$\alpha_3$与$\eta_1,\eta_2$一定是正交的,只需对$\alpha_3$单位化:$\eta_3=\dfrac{1}{3}[2,1,2]^T$。

令$P=
(\eta_1,\eta_2,\eta_3)
$,则$P^{-1}AP=
\begin{pmatrix}
5\\&5\\&&-4
\end{pmatrix}
$.
\end{jie}

\EX 设$A$是$n$阶实对称矩阵且$A^2=0$。证明$A=0$。

\begin{zhengming}
$A$是实对称矩阵,所以可设:
\begin{equation*}
A=
\begin{pmatrix}
a_{11}&a_{12}&\cdots&a_{1n}\\
a_{12}&a_{22}&\cdots&a_{2n}\\
\vdots&\vdots&\ddots&\vdots\\
a_{1n}&a_{2n}&\cdots&a_{nn}
\end{pmatrix}
\end{equation*}
因为$A^2=0$,我们考察$A^2$的对角线上元素:$A^2(1,1)=\sum\limits_{i=1}^{n}a_{1i}^2=0$,任何数的平方只能大于等于0,所以得出$a_{1i}=0,i=1,\cdots,n$,同理也可以推出$A$的其他行的每一个元素为0.证毕。
\end{zhengming}

\EX 设$A$是$n$阶实对称矩阵,且它的特征值都是正数。证明:存在$n$阶实对称阵$B$使得$A=B^2$。

\begin{zhengming}
$A$是$n$阶实对称矩阵,所以存在正交阵$P$使得:
\begin{equation*}
A=P^{-1}
\begin{pmatrix}
\lambda_1\\
&\lambda_2\\
&&\ddots\\
&&&\lambda_n
\end{pmatrix}P
\end{equation*}
式中$\lambda_i$是$A$的特征值,又因为$A$的特征值都是正数,所以可设$B=
P^{-1}
\begin{pmatrix}
\sqrt{\lambda_1}\\
&\sqrt{\lambda_2}\\
&&\ddots\\
&&&\sqrt{\lambda_n}
\end{pmatrix}P
$,$P$是正交矩阵,所以有$P^T=P^{-1}$,所以:$(P^{-1})^T=P$,对$B$取转置(注意转置的性质:$(AB)^T=B^TA^T$):
\begin{align*}
B^T&=P^T
\begin{pmatrix}
\sqrt{\lambda_1}\\
&\sqrt{\lambda_2}\\
&&\ddots\\
&&&\sqrt{\lambda_n}
\end{pmatrix}^T(P^{-1})^T\\
&=P^{-1}
\begin{pmatrix}
\sqrt{\lambda_1}\\
&\sqrt{\lambda_2}\\
&&\ddots\\
&&&\sqrt{\lambda_n}
\end{pmatrix}P=B
\end{align*}

即$B$是实对称矩阵,且:
\begin{equation*}
B^2=P^{-1}
\begin{pmatrix}
\sqrt{\lambda_1}\\
&\sqrt{\lambda_2}\\
&&\ddots\\
&&&\sqrt{\lambda_n}
\end{pmatrix}^2P=A
\end{equation*}
所以存在$n$阶实对称阵$B$使得$A=B^2$。
\end{zhengming}

\EX 设$A,B$是$n$阶实对称阵,满足:对任意$n$维向量$\xi$都有$\xi^TA\xi=\xi^TB\xi$。证明:$A=B$。

\begin{zhengming}
设$C=A-B$,要证$A=B$,只需证$C=0$即可,$A,B$是$n$阶实对称阵,实对称矩阵加减后仍是实对称矩阵,$C$也是实对称矩阵。且对于任意的$n$维列向量$\eta$有:
\begin{equation*}
\eta^TC\eta=\eta^T(A-B)\eta=\eta^TA\eta-\eta^TB\eta=0\tag{*}
\end{equation*}
由于$C$是实对称矩阵,所以存在正交矩阵$P$使得:
\begin{equation*}
P^{-1}CP=D
=\begin{pmatrix}
\lambda_1\\
&\lambda_2\\
&&\ddots\\
&&&\lambda_n
\end{pmatrix}
\end{equation*}
其中$\lambda_i$是$A$的特征值。对任意的$n$维列向量$\xi$,由$P^{-1}=P^T$和式(*)得:令$\eta=P\xi$
\begin{equation*}
  0=(P\xi)^TC(P\xi)=\xi^T(P^TCP)\xi=\xi^T(P^TCP)\xi=\xi^{T}(P^{-1}CP)\xi
\end{equation*}
对于任意$\xi=
\begin{pmatrix}
x_1\\x_2\\ \vdots\\x_n
\end{pmatrix}
$,都有$0=\xi^T(P^{-1}CP)\xi=\xi^TD\xi=\lambda_1x_1^2+\cdots+\lambda_nx_n^2$

对于每个$i$,取$x_i=1$,其余的$x_j$为0,得$\lambda_i=0$,即$D=0$,从而$C=PDP^{-1}=P0P^{-1}=0$。
\end{zhengming}

\EX 设$P,Q$是$n$阶正交矩阵,且$|P|=1,|Q|=-1$。证明:$|P+Q|=0$。

\begin{zhengming}
$P,Q$是$n$阶正交矩阵,所以$P^{-1}=P^T,Q^{-1}=Q^{T}$。所以:(注:转置不改变行列式,转置的性质:$(A+B)^T=A^T+B^T$)
\begin{align*}
|P+Q|&=|P(I_n+P^{-1}Q)|=|P(Q^{-1}Q+P^{-1}Q)|=|P(Q^{-1}+P^{-1})Q|\\
&=|P(Q^T+P^T)Q|=|P(Q+P)^TQ|=|P||(P+Q)^T||Q|=-|P+Q|
\end{align*}
所以:$|P+Q|=0$。
\end{zhengming}

\EX 已知实对称矩阵$A$的秩为2,且
\begin{equation*}
A\begin{pmatrix}
1&1\\
0&0\\
-1&1
 \end{pmatrix}=
 \begin{pmatrix}
-1&1\\
0&0\\
1&1
 \end{pmatrix}
\end{equation*}

(1)求矩阵$A$的所有特征值与特征向量;

(2)求矩阵$A$。

\begin{jie}
(1)令$\xi_1=(1,0,-1)^T,\xi_2=(1,0,1)^T$,由题目中的等式得$A(\xi_1,\xi_2)=(-\xi_1,\xi_2)$,所以$A\xi_1=-\xi_1,A\xi_2=\xi_2$,所以$A$的其中两个特征值为$-1,1$,又因为$r(A)=2$,即$|A|=0$,所以$A$的另一个特征值为0.

实对称矩阵不同特征值对应的特征向量相交,所以设特征值$0$的特征向量为$\xi_3=[\xi_1,\xi_2,\xi_3]^T$,所以:
\begin{equation*}
\begin{cases}
\xi_1^T\xi_3=0\\
\xi_2^T\xi_3=0
\end{cases}~~\Rightarrow
\begin{cases}
x_1=x_3=0\\
x_2\in R
\end{cases}~~~\Rightarrow~~\xi_3=[0,1,0]^T
\end{equation*}
综上所述:$A$的特征值为1对应的特征向量为$k_1[1,0,1]^T,k_1\neq 0$,$A$的特征值为-1对应的特征向量为$k_2[1,0,-1]^T,k_2\neq 0$,$A$的特征值为0对应的特征向量为$k_3[0,1,0]^T,k_3\neq 0$.

(2)取$P=(\xi_1,\xi_2,\xi_3)$,则$P^{-1}AP=\Lambda=
\begin{pmatrix}
-1\\&1\\&&0
\end{pmatrix}
$,计算得:$
P^{-1}=
\begin{pmatrix}
0.5&0&-0.5\\0.5&0&0.5\\0&1&0
\end{pmatrix}
$,所以:
\begin{equation*}
A=P\Lambda P^{-1}=
\begin{pmatrix}
0&0&1\\0&0&0\\1&0&0
\end{pmatrix}
\end{equation*}
\end{jie}

\EX 设矩阵$A
\begin{pmatrix}
1&1&a\\
1&a&1\\
a&1&1
\end{pmatrix},
\beta=
\begin{pmatrix}
1\\1\\-2
\end{pmatrix}
$。已知线性方程组$Ax=\beta$有解但不唯一,试求:

(1) $a$的值;

(2)正交矩阵$Q$,使$Q^TAQ$为对角矩阵。

\begin{jie}
(1)由题得$|A|=-(a+2)(a-1)^2$因为$Ax=\beta$解不唯一,所以$r(A)<3,|A|=0,$即$a=-2$或$a=1$。

$a=-2$时$r(A)=r(A,\beta)=2$,$a=1$时,$r(A)\neq r(A,\beta)$。所以$a=-2$。

(2)
\begin{align*}
|\lambda E-A|&=
\begin{vmatrix}
\lambda -1&-1&2\\
-1&\lambda+2&-1\\
2&-1&\lambda -1
\end{vmatrix}
\xlongequal{c_1+c_2+c_3}
\begin{vmatrix}
\lambda&-1&2\\
\lambda&\lambda+2&-1\\
\lambda&-1&\lambda -1
\end{vmatrix}\xlongequal{\substack{r_2-r_1\\ r_3-r_1}}
\begin{vmatrix}
\lambda&-1&2\\
0&\lambda+3&-3\\
0&0&\lambda -3
\end{vmatrix}\\&=\lambda(\lambda-3)(\lambda+3)~~\Rightarrow~~\lambda_1=0,\lambda_2=3,\lambda_3=-3
\end{align*}

对于$\lambda=0$,用高斯消元化简$(0E-A)x=0$的系数矩阵后得:$x_1=x_3,x_2=x_3$,所以$\alpha_1=[1,1,1]^T$,单位化:$\eta_1=\dfrac{1}{\sqrt{3}}[1,1,1]^T$。

对于$\lambda=3$,用高斯消元化简$(3E-A)x=0$的系数矩阵后得:$x_1=-x_3,x_2=0$,所以$\alpha_2=[-1,0,1]^T$,单位化:$\eta_2=\dfrac{1}{\sqrt{2}}[-1,0,1]^T$。

对于$\lambda=-3$,用高斯消元化简$(-3E-A)x=0$的系数矩阵后得:$x_1=x_3,x_2=-2x_3$,所以$\alpha_3=[1,-2,1]^T$,单位化:$\eta_3=\dfrac{1}{\sqrt{6}}[1,-2,1]^T$。

所以:$Q=[\eta_1,\eta_2,\eta_3]$,此时$Q^TAQ=
\begin{pmatrix}
0\\&3\\&&-3
\end{pmatrix}
$.
\end{jie}

\stepcounter{chapter}
\clearpage
\hphantom{~~}\hfill {\zihao{3}\heiti 第十二次线性代数} \hfill\hphantom{~~}
\addcontentsline{toc}{section}{\protect\numberline {}第十二次线性代数}

\hphantom{~~}

%\chapter{第十二次线性代数}
\EX 设$A$与$B$合同。证明:

(1)$r(A)=r(B)$。

(2)$|A|$与$|B|$同为正,或同为负,或同为0.

(3)设$A$可逆,则$B$可逆,且$A^{-1}$与$B^{-1}$合同。

\begin{zhengming}
$A$与$B$合同,所以存在可逆矩阵$P$使得$A=P^TBP$。

(1)由于$P$可逆,并且左乘或右乘可逆矩阵不改变矩阵的秩,所以$r(A)=r(B)$。

(2)对$A=P^TBP$两边同时求行列式:$|A|=|P^TBP|=|P^T||B||P|=|P|^2|B|$,因为$P$可逆,所以$|P|\neq 0$,所以$|P|^2>0$,所以$|A|$与$|B|$同号或同为0.

(3)$A$可逆,则由(1)得$r(A)=n=r(B)$,即$B$可逆。

对$A=P^TBP$两边同时求逆:$A^{-1}=(P^TBP)^{-1}=P^{-1}B^{-1}(P^{T})^{-1}=((P^{-1})^T)^TB^{-1}(P^{-1})^{T}$.

因为$P$可逆,所以$P^{-1}$可逆,所以$(P^{-1})^T$可逆,记$Q=(P^{-1})^T$,则$A^{-1}=Q^TB^{-1}Q$
即$A^{-1}$与$B^{-1}$合同。
\end{zhengming}

\EX 设$A$与$B$合同,$C$与$D$合同。证明:$
\begin{pmatrix}
A&0\\
0&C
\end{pmatrix}
$与$\begin{pmatrix}
B&0\\
0&D
\end{pmatrix}
$合同。

\begin{zhengming}
$A$与$B$合同,则存在可逆矩阵$P$使得$A=P^TBP$,同理存在可逆矩阵$Q$使得$C=Q^TDQ$。则存在可逆矩阵$
\begin{pmatrix}
P&0\\
0&Q
\end{pmatrix}
$使得:
\begin{align*}
&\begin{pmatrix}
P&0\\
0&Q
\end{pmatrix}^T\begin{pmatrix}
B&0\\
0&D
\end{pmatrix}\begin{pmatrix}
P&0\\
0&Q
\end{pmatrix}\\=&
\begin{pmatrix}
P^T&0\\
0&Q^T
\end{pmatrix}\begin{pmatrix}
B&0\\
0&D
\end{pmatrix}\begin{pmatrix}
P&0\\
0&Q
\end{pmatrix}\\=&
\begin{pmatrix}
P^TBP&0\\
0&Q^TDQ
\end{pmatrix}=\begin{pmatrix}
A&0\\
0&C
\end{pmatrix}
\end{align*}
所以$
\begin{pmatrix}
A&0\\
0&C
\end{pmatrix}
$与$\begin{pmatrix}
B&0\\
0&D
\end{pmatrix}
$合同。
\end{zhengming}

\EX 设$f(x_1,x_2,x_3)=(x_1-2x_3)^2-3x_2^{2}$。通过换元:$
\begin{cases}
y_1=x_1-2x_3\\y_2=x_2
\end{cases}$可以得到$f(x_1,x_2,x_3)=y_1^2-3y_2^2$。问:这里的换元是否是非退化线性替换?若不是,请写出正确的非退化线性替换。

\begin{jie}
不是。由$
\begin{cases}
y_1=x_1-2x_3\\
y_2=x_2\\
y_3=x_3
\end{cases}
$得:$
\begin{cases}
x_1=y_1+2y_3\\x_2=y_2\\x_3=y_3
\end{cases}$,所以所作的非退化线性替换应为$x=cy$,其中$C=
\begin{pmatrix}
1&0&2\\0&1&0\\0&0&1
\end{pmatrix}
$是可逆矩阵。
\end{jie}

\EX 举例说明:当$A$与$B$合同时,不一定有$A\sim B$;当$A\sim B$时不一定有$A$与$B$合同。

\begin{jie}
(1)$A$与$B$合同时,不一定有$A\sim B$:

由课本157页定理5.1.1,得:
$
A=\begin{pmatrix}
   1\\&1
  \end{pmatrix},B=\begin{pmatrix}
                   1\\&2
                  \end{pmatrix}
$,由相似矩阵定义,单位矩阵只与自身相似,即$A$与$B$相似。

(2)当$A\sim B$时不一定有$A$与$B$合同.

$
A=\begin{pmatrix}
   1\\&-3
  \end{pmatrix},B=\begin{pmatrix}
   1&0\\2&-3
  \end{pmatrix}
$,$B$的特征值为$1,-3$互不相同,一定可以对角化,因此有$A\sim B$。由课本156页例5.1.5(4)得出,与$A$合同的只能是对称阵,而$B$非对称,即不能合同。
\end{jie}

\EX 设$A$与$B$都是$n$阶实对称矩阵。证明:如果$A\sim B$则$A$与$B$合同。

\begin{zhengming}
因为$A$与$B$相似,所以他们的特征值完全相同,所以他们的正负惯性指数相等,又因为$A$和$B$都是实对称矩阵,所以$A$与$B$合同。
\end{zhengming}


\EX 设二次型$f(x_1,x_2,x_3)$的矩阵$A$可逆。证明:

(1)如果$d_1y_1^2+d_2y_2^2+d_3y_3^2$是$f(x_1,x_2,x_3)$的一个标准型,则:$
\dfrac{1}{d_1}y_1^2+\dfrac{1}{d_2}y_2^2+\dfrac{1}{d_3}y_3^2
$是二次型$g(x_1,x_2,x_3)=x^TA^{-1}x$的一个标准型。

(2)二次型$h(x_1,x_2,x_3)=y_1^2+y_2^2+y_3^2$.

\begin{zhengming}
由$A$可逆可知,$A$的特征值不为0.设$A$的正惯性指数为$p$,则其负惯性指数为$3-p$.

如果$\lambda$是$A$的一个特征值,由特征值的性质$\frac{1}{\lambda}$是$A^{-1}$的一个特征值。$\lambda$与$\frac{1}{\lambda}$同号。所以$A$与$A^{-1}$的正惯性指数都为$p$,负惯性指数都为$3-p$。

(1)若$d_1y_1^2+d_2y_2^2+d_3y_3^2$是$f(x_1,x_2,x_3)$的一个标准型,则$d_1,d_2,d_3$中有$p$个正数,$3-p$个负数.从而$\frac{1}{d_1},\frac{1} {d_2},\frac{1} {d_3}$中有$p$个正数,$3-p$个负数,即由惯性定理,对角阵
$
\begin{pmatrix}
\frac{1} {d_1}\\&\frac{1} {d_2}\\&&\frac{1} {d_3}
\end{pmatrix}
$与$A^{-1}$合同,所以结论成立。

(2)$A$的特征值不为0,所以$A^2$的特征值大于0,从而$A^2$的正惯性指数为$3$,所以结论成立。
\end{zhengming}

\EX 设3元实二次型$g(x_1,x_2,x_3)$的矩阵是$A$。方阵$B$的特征多项式为$f(\lambda)=(\lambda-1)(\lambda+2)^2$,且$B$与$A$合同。证明:对任意的$x
=\begin{pmatrix}
x_1\\ x_2\\x_3
 \end{pmatrix}
$都有:
$-2(x_1^2+x_2^2+x_3^2)\leq x^TBx\leq(x_1^2+x_2^2+x_3^2)$。

\begin{zhengming}
由于$A$是实二次型矩阵,所以$A$是实对称矩阵,又因为$A$与$B$合同,所以$B$也是实对称矩阵,因此$B$可以对角化,由题可得$B$的特征值为$1,-2,-2$,所以存在正交阵$P$使得
\begin{equation*}
P^{-1}BP=P^TBP=
\begin{pmatrix}
1\\&-2\\&&-2
\end{pmatrix}
\end{equation*}
对于任意的$x=
\begin{pmatrix}
x_1\\x_2\\x_3
\end{pmatrix}
$,令$
y=\begin{pmatrix}
y_1\\y_2\\y_3
  \end{pmatrix}=P^{-1}x
$,则:
\begin{align*}
x^TBx=&(Py)^TB(Py)=y^T(P^TBP)y\\
=&\begin{pmatrix}
y_1&y_2&y_3
  \end{pmatrix}\begin{pmatrix}
1\\&-2\\&&-2
\end{pmatrix}\begin{pmatrix}
y_1\\ y_2\\ y_3
  \end{pmatrix}\\=&
  y_1^2-2y_2^2-2y_3^2
\end{align*}

由此式可得:\begin{equation*}-2(y_1^2+y_2^2+y_3^2)\leq x^TBx\leq(y_1^2+y_2^2+y_3^2)\tag{*}\end{equation*}
因为$P$是正交矩阵,所以$x^Tx=x_1^2+x_2^2+x_3^2=(Py)^{T}(Py)=y^TP^TPy=y^Ty=y_1^2+y_2^2+y_3^2$。代入$(*)$式即可得证。
\end{zhengming}

\EX 用配方法求二次型
\begin{equation*}
f(x_1,x_2,x_3)=x_2^2+x_3^2-2x_1x_2+4x_2x_3
\end{equation*}的标准型。

\begin{jie}
\textcolor[rgb]{1.00,0.00,0.00}{配方法:课本161页。}

标准型不唯一,所以答案不唯一。但正负惯性指数是不变的。

\begin{align*}
f&=x_2^2+x_3^2-2x_1x_2+4x_2x_3=(x_3+2x_2)^2-4x_2^2+x_2^2-2x_1x_2\\
&=(x_3+2x_2)^2-3x_2^2-2x_1x_2=(x_3+2x_2)^2-3(x_2^2+\frac{2}{3}x_1x_2)\\
&=(x_3+2x_2)^2-3(x_2+\frac{1}{3}x_1)^2+\frac{1}{3}x_1^2
\end{align*}
所以$f(x_1,x_2,x_3)$的一个标准型为$(x_3+2x_2)^2-3(x_2+\frac{1}{3}x_1)^2+\frac{1}{3}x_1^2$
\end{jie}

\EX 设$
A=
\begin{pmatrix}
-1&1&-3\\
1&-2&-2\\
-3&-2&2
\end{pmatrix}
$.用合同变换求可逆矩阵$C$使得$C^TAC$为对角阵,并求$A$的正惯性指数和负惯性指数。

\begin{jie}
\textcolor[rgb]{1.00,0.00,0.00}{合同变换:课本162页。}
\begin{align*}
&\left( \begin{array}{c}
A\\
\hdashline[2pt/3pt]
E_3
\end{array}
\right)=
\left(
 \begin{array}{ccc}
-1&1&-3\\
1&-2&-2\\
-3&-2&2\\
\hdashline[2pt/3pt]
1 &0&0 \\
0 &1&0 \\
0&0&1\\
\end{array}
\right)
\xrightarrow{\substack{ r_2+r_1}}{
\left(
 \begin{array}{ccc}
-1&1&-3\\
0&-1&-5\\
-3&-2&2\\
\hdashline[2pt/3pt]
1 &0&0 \\
0 &1&0 \\
0&0&1\\
\end{array}
\right)
}\xrightarrow{\substack{ c_2+c_1}}{
\left(
 \begin{array}{ccc}
-1&0&-3\\
0&-1&-5\\
-3&-5&2\\
\hdashline[2pt/3pt]
1 &1&0 \\
0 &1&0 \\
0&0&1\\
\end{array}
\right)
}\\&\xrightarrow{\substack{ r_3-3r_1}}{
\left(
 \begin{array}{ccc}
-1&0&-3\\
0&-1&-5\\
0&-5&11\\
\hdashline[2pt/3pt]
1 &1&0 \\
0 &1&0 \\
0&0&1\\
\end{array}
\right)
}\xrightarrow{\substack{ c_3-3c_1}}{
\left(
 \begin{array}{ccc}
-1&0&0\\
0&-1&-5\\
0&-5&11\\
\hdashline[2pt/3pt]
1 &1&-3 \\
0 &1&0 \\
0&0&1\\
\end{array}
\right)
}\xrightarrow{\substack{ r_3-5r_2}}{
\left(
 \begin{array}{ccc}
-1&0&0\\
0&-1&-5\\
0&0&36\\
\hdashline[2pt/3pt]
1 &1&-3 \\
0 &1&0 \\
0&0&1\\
\end{array}
\right)
}\\&
\xrightarrow{\substack{ c_3-5c_2}}{
\left(
 \begin{array}{ccc}
-1&0&0\\
0&-1&0\\
0&0&36\\
\hdashline[2pt/3pt]
1 &1&-8 \\
0 &1&-5 \\
0&0&1\\
\end{array}
\right)
}=\left( \begin{array}{c}
D\\
\hdashline[2pt/3pt]
C
\end{array}
\right)
\end{align*}
所以:$
C=\begin{pmatrix}
1 &1&-8 \\
0 &1&-5 \\
0&0&1
  \end{pmatrix}
$,则$C^TAC=
\begin{pmatrix}
-1\\&-1\\&&36
\end{pmatrix}
$,正惯性指数为1,负惯性指数为2.
\end{jie}

\EX 用正交变换求二次型:
\begin{equation*}
f(x_1,x_2,x_3)=2x_1^2+5x_2^2+2x_3^2-4x_1x_2-8x_1x_3-4x_2x_3
\end{equation*}的标准型。并求$f(x_1,x_2,x_3)$的正惯性指数和负惯性指数。

\begin{tips}
正交变换化二次型为标准型的步骤:(以3阶为例)

(1)写出二次型矩阵$A$;

(2)求矩阵$A$的特征值$\lambda_1,\lambda_2,\lambda_3;$

(3)求出矩阵的特征向量$\alpha_1,\alpha_2,\alpha_3$;

(4)把(3)特征向量改造(施密特正交化,单位化)为$\gamma_1,\gamma_2,\gamma_3$;(实对称矩阵的不同特征值对应的特征向量是正交的,如果三个特征值不同,则三个特征值对应的特征向量组成的向量组就是正交的,此时直接单位化各向量即可。)

(5)构造正交矩阵$Q=[\gamma_1,\gamma_2,\gamma_3]$,经坐标变换$x=Qy$得:
\begin{equation*}
  x^TAx=y^T\Lambda y=\lambda_1y_1^2+\lambda_2y_2^2+\lambda_3y_3^2
\end{equation*}

应当注意:$\Lambda$中特征值的顺序应与$Q$中对应的向量顺序一致。
\end{tips}

\begin{jie}
\textcolor[rgb]{1.00,0.00,0.00}{正交变换:课本163页。}

由题得:$f$的矩阵$
A=\begin{pmatrix}
    2&-2&-4\\
    -2&5&-2\\
    -4&-2&2
  \end{pmatrix}
$,所以:
\begin{equation*}
|\lambda E-A|\xlongequal{c_1+c_2+c_3}
\begin{vmatrix}
\lambda+4&2&4\\
\lambda-1&\lambda-5&2\\
\lambda+4&2&\lambda-2
\end{vmatrix}\xlongequal{r_3-r_1}
\begin{vmatrix}
\lambda+4&2&4\\
\lambda-1&\lambda-5&2\\
0&0&\lambda-6
\end{vmatrix}=(\lambda-6)[(\lambda+4)(\lambda-5)-2(\lambda-1)]=0
\end{equation*}
所以$\lambda_1=\lambda_2=6,\lambda=-3$。

所以$f$的正惯性指数为2,负惯性指数为1.


对于$\lambda=6$,用高斯消元化简$(6E-A)x=0$的系数矩阵后得:$x_1=-\frac{1}{2}x_2-x_3$,即$\alpha_1=[1,-2,0]^T,\alpha_2=[1,0,-1]$。施密特正交化后单位化:(步骤略)$\eta_1=\frac{1}{\sqrt{5}}[1,-2,0]^T,\eta_2=\frac{\sqrt{5}}{5}[-\frac{4}{5},-\frac{3}{5},1]^T$。

对于$\lambda=-3$,用高斯消元化简$(6E-A)x=0$的系数矩阵后得:$
x_1=x_3,x_2=\frac{1}{2}x_3
$,即$\alpha_3=[2,1,2]^T$单位化:$\eta_{3}=\frac{1}{3}[2,1,2]$。

令$Q=[\eta_1,\eta_2,\eta_3]$,在正交变换$x=Qy$下得到一个标准型为$f=6y_1^{2}+6y_2^{2}-3y_3^{2}$。
\end{jie}

\EX 二次型$f(x_1,x_2,x_3)=(x_1+x_2)^2+(x_2-x_3)^2+(x_3+x_1)^2$的秩为\underline{\hphantom{~~~~~~~~~~}}。

\begin{jie}
由题得:$f(x_1,x_2,x_3)=2x_1^2+2x_2^2+2x_3^2+2x_1x_2-2x_2x_3+2x_1x_3$。所以:
\begin{equation*}
A=
\begin{pmatrix}
2&1&1\\
1&2&-1\\
1&-1&2
\end{pmatrix}\rightarrow
\begin{pmatrix}
1&2&-1\\
0&1&-1\\
0&0&0
\end{pmatrix}
\end{equation*}
所以$r(A)=2$。
\end{jie}

\EX 已知实二次型$f(x_1,x_2,x_3)=a(x_1^2+x_2^2+x_3^2)+4x_1x_2+4x_2x_3+4x_1x_3$经正交变换$x=Py$可化成标准型$f=6y_1^2$,则$a=$\underline{\hphantom{~~~~~~~~~~}}。

\begin{jie}
由题得:$A=
\begin{pmatrix}
a&2&2\\
2&a&2\\
2&2&a
\end{pmatrix},P^TAP=P^{-1}AP
\begin{pmatrix}
  6\\&0\\&&0
\end{pmatrix}=\Lambda
$,相似矩阵迹相同:$tr(A)=tr(\Lambda)$即$3a=6,a=2$。
\end{jie}

\EX 已知二次型$f(x_1,x_2,x_3)=4x_2^2-3x_3^2+4x_1x_2+8x_2x_3-4x_1x_3$.

(1)写出二次型$f$的矩阵表达式;

(2)用正交变换把二次型$f$化为标准型,并写出相应的正交矩阵。

\begin{jie}
(1)
由题得:$
A=
\begin{pmatrix}
0&2&-2\\
2&4&4\\
-2&4&-3
\end{pmatrix}x=(x_1,x_2,x_3)^T
$,所以$f=x^TAx$。

(2)
\begin{align*}
|\lambda E-A|&\xlongequal{r_2+r_3}
\begin{vmatrix}
\lambda&-2&2\\
0&\lambda-8&\lambda-1\\
2&-4&\lambda+3
\end{vmatrix}\xlongequal{c_2+c_3}
\begin{vmatrix}
\lambda&0&2\\
0&2\lambda-9&\lambda-1\\
2&\lambda-1&\lambda+3
\end{vmatrix}\\&=\lambda[(2\lambda-9)(\lambda+3)-(\lambda-1)^2]-4(2\lambda-9)\\
&=\lambda^3-\lambda^2-36\lambda+36=(\lambda^2-36)(\lambda-1)
\end{align*}
解得$\lambda_1=6,\lambda_2=-6,\lambda_3=1$。

对于$\lambda=6$,用高斯消元化简$(6E-A)x=0$的系数矩阵后得:$x_1=\frac{1}{2}x_3,x_2=\frac{5}{2}x_3$,即$\alpha_1=[1,5,2]^T$,单位化:$\eta_1=\frac{1}{\sqrt{30}}
[1,5,2]^T
$。

对于$\lambda=-6$,用高斯消元化简$(-6E-A)x=0$的系数矩阵后得:$x_1=\frac{1}{2}x_3,x_2=-\frac{1}{2}x_3$,即$\alpha_2=[1,-1,2]^T$,单位化:$\eta_2=\frac{1}{\sqrt{6}}
[1,-1,2]^T
$。

对于$\lambda=1$,用高斯消元化简$(E-A)x=0$的系数矩阵后得:$x_1=-2x_3,x_2=0$,即$\alpha_3=[-2,0,1]^T$,单位化:$\eta_2=\frac{1}{\sqrt{5}}
[-2,0,1]^T
$。

令$Q=
\begin{pmatrix}
\frac{1}{\sqrt{30}}&\frac{1}{\sqrt{6}}&-\frac{2}{\sqrt{5}}\\
\frac{5}{\sqrt{30}}&-\frac{1}{\sqrt{6}}&0\\
\frac{2}{\sqrt{30}}&\frac{2}{\sqrt{6}}&\frac{1}{\sqrt{5}}
\end{pmatrix}
$,则$Q$为一个正交矩阵,令$x=Qy$,标准型为$f=6y_1^2-6y_2^2+y_3^2$。
\end{jie}

\EX 设矩阵$
A=
\begin{pmatrix}
0&1&0&0\\
1&0&0&0\\
0&0&y&1\\
0&0&1&2
\end{pmatrix}
$.

(1)已知$A$的一个特征值为$3$,试求$y$;

(2)求可逆矩阵$P$,使得$(AP)^T(AP)$为对角阵。

\begin{jie}
(1)由题得:$|3E-A|=
\begin{vmatrix}
3&-1&0&0\\
-1&3&0&0\\
0&0&3-y&-1\\
0&0&-1&1
\end{vmatrix}=8(3-y-1)=0
$,解得$y=2$。

(2)不能使用第一问的结论。

$(AP)^T(AP)=P^TA^TAP=P^TBP$,式中$B=A^TA$。
\begin{align*}
A^TA=\begin{pmatrix}
0&1&0&0\\
1&0&0&0\\
0&0&y&1\\
0&0&1&2
\end{pmatrix}\begin{pmatrix}
0&1&0&0\\
1&0&0&0\\
0&0&y&1\\
0&0&1&2
\end{pmatrix}=\begin{pmatrix}
0&1&0&0\\
1&0&0&0\\
0&0&y^2+1&y+2\\
0&0&y+2&5
\end{pmatrix}=
\begin{pmatrix}
I_2\\&B_{22}
\end{pmatrix}
\end{align*}
式中$I_2$为二阶单位矩阵,$B_{22}=
\begin{pmatrix}
y^2+1&y+2\\
y+2&5
\end{pmatrix}
$,对$B_{22}$做合同变换:
\begin{align*}
&\left( \begin{array}{c}
B_{22}\\
\hdashline[2pt/3pt]
E_2
\end{array}
\right)=
\left(
 \begin{array}{ccc}
y^2+1&y+2\\
y+2&5\\
\hdashline[2pt/3pt]
1 &0 \\
0 &1
\end{array}
\right)
\xrightarrow{\substack{ r_2\leftrightarrow r_1}}{
\left(
 \begin{array}{ccc}
y+2&5\\
y^2+1&y+2\\
\hdashline[2pt/3pt]
1 &0 \\
0 &1
\end{array}
\right)
}\xrightarrow{\substack{ c_2\leftrightarrow c_1}}{
\left(
 \begin{array}{ccc}
5&y+2\\
y+2&y^2+1\\
\hdashline[2pt/3pt]
0 &1\\
1 &0
\end{array}
\right)
}\xrightarrow{\substack{r_2-\frac{y+2}{5}r_1}}{
\left(
 \begin{array}{ccc}
5&y+2\\
0&\frac{(2y-1)^2}{5}\\
\hdashline[2pt/3pt]
0 &1\\
1 &0
\end{array}
\right)
}\\
&\xrightarrow{\substack{c_2-\frac{y+2}{5}c_1}}{
\left(
 \begin{array}{ccc}
5&0\\
0&\frac{(2y-1)^2}{5}\\
\hdashline[2pt/3pt]
0 &1\\
1 &-\frac{y+2}{5}
\end{array}
\right)
}
\end{align*}
所以令$
P=
\begin{pmatrix}
I_2\\
&P_1
\end{pmatrix}
$,其中$P_1=
\begin{pmatrix}
5&0\\
0&\frac{(2y-1)^2}{5}
\end{pmatrix}
$,则$(AP)^T(AP)$为对角阵$
\begin{pmatrix}
1\\&1\\&&5\\&&&\frac{(2y-1)^2}{5}
\end{pmatrix}$。
\end{jie}

\EX 设二次型
\begin{equation*}
f(x_1,x_2,x_3)=x_1^2+x_2^2+x_3^2+2ax_1x_2+2bx_2x_3+2x_1x_3
\end{equation*}
经正交变换$x=Py$化为$f=y_2^2+2y_3^2$,其中$x=(x_1,x_2,x_3)^T$和$y=(y_1,y_2,y_3)^T$是三维列向量,$P$是3阶正交矩阵。试求$a,b$的取值。

\begin{jie}
由题得:
$
A=
\begin{pmatrix}
1&a&1\\
a&1&b\\
1&b&1
\end{pmatrix},\Lambda
=\begin{pmatrix}
0\&1\\&&2
 \end{pmatrix}
$且$A$与$\Lambda$相似。

相似矩阵具有相同的秩和行列式和特征值:$|A|=|\Lambda|$,$tr(A)=tr(\Lambda),\lambda_A=\lambda_{\Lambda}$。

$|\lambda E-A|=(\lambda-1)^3-2ab-(\lambda-1)-(a^2+b^2)(\lambda-1)=0$.把$\lambda=0,1,2$分别代入该等式得:
\begin{equation*}
\begin{cases}
a^2+b^2-2ab=0\\
2ab=0\\
a^2+b^2+2ab=0
\end{cases}~~~\rightarrow~~~
\begin{cases}
a=0\\
b=0
\end{cases}
\end{equation*}
\end{jie}

\EX 设$A$为$n$阶实对称矩阵,$r(A)=n,A_{ij}$是$A=(a_{ij})$中的元素$a_{ij}$的代数余子式$(i,j=1,2,\cdots,n)$。二次型$
f(x_1,x_2,\cdots,x_n)=\sum\limits_{i=1}^{n}\sum\limits_{j=1}^{n}\dfrac{A_{ij}}{|A|}x_ix_j.
$

(1)记$x=(x_1,x_2,\cdots,x_n)^T$,把$f(x_1,x_2,\cdots,x_n)$写出矩阵形式,并证明二次型$f(x)$的矩阵为$A^{-1}$。

(2)二次型$g(x)=x^TAx$与$f(x)$的规范型是否相同?并说明理由。

\begin{zhengming}
(1)设$B=
\frac{1}{|A|}A^*=A^{-1}
$,其中$A^*$是$A$的伴随矩阵,则$f(x_1,\cdots,x_n)=x^TBx$.为证明这就是$f$的矩阵形式,只需验证$B$是对称矩阵。

因为$A$是对称矩阵,所以有$(A^T)^{-1}=A^{-1}=(A^{-1})^T$,所以$A^{-1}$是对称阵,所以二次型$f(x)$的矩阵为$A^{-1}$。

(2)$g$的矩阵是$A$,其特征值为$\lambda$,因为$r(A)=n$,所以$|A|\neq0$,所以$\lambda\neq0$。由(1)知,$A$的矩阵是$A^{-1}$,所以其特征值为$\frac{1}{\lambda}$,所以$A$与$A^{-1}$的特征值相同,即它们的正负惯性指数相同,所以规范型一致。
\end{zhengming}

\EX 二次型$f(x_1,x_2,x_3)=(x_1-2x_2+3x_3)^2$是否正定?说明理由。

\begin{jie}
若该二次型正定,则它对应的矩阵各阶顺序主子式均大于0.由题得该二次型矩阵:
\begin{align*}
f(x_1,x_2,x_3)=x_1^2+4x_2^2+9x_3^2+4x_1x_2-12x_2x_3-6x_1x_3~~\rightarrow~~A=
\begin{pmatrix}
1&2&-3\\
2&4&-6\\
-3&-6&9
\end{pmatrix}
\end{align*}
显然$A$的各行成比例,由行列式的性质:$|A|=0$,所以该二次型不是正定的。
\end{jie}

\EX 设二次型$f(x_1,x_2,x_3)$是正定的。证明:对任意正数$a$,二次型$af(x_1,x_2,x_3)$也是正定的。

\begin{zhengming}
对于任意的$x_1,x_2,x_3$,由于$f$正定,且$a>0$,所以$af\geq 0$.

若$af=0$,由于$a>0$,所以$f=0$。由166页定义5.2.1得$x_1=x_2=x_3=0$.综上所述,$af(x_1,x_2,x_3)$是正定的。
\end{zhengming}

\EX 判断二次型
\begin{equation*}
f(x_1,x_2,x_3)=x_1^2+3x_2^2+6x_3^2-2x_1x_2-2x_1x_3
\end{equation*}
是否正定。

\begin{jie}
由题得:$A=
\begin{pmatrix}
1&-1&-1\\
-1&3&0\\
-1&0&6
\end{pmatrix}
$,则
\begin{equation*}
\triangle_1=1>0,~~\triangle_2=\begin{vmatrix}
                               1&-1\\
-1&3
                              \end{vmatrix}
=4>0,~~\triangle_3=|A|=21>0
\end{equation*}
所以是正定的。
\end{jie}

\EX 设$A$是任意实对称矩阵。证明:$A^2+2A+2I_n$是正定阵。

\begin{zhengming}
$A$为实对称矩阵,所以$A^2+2A+2I_n$也为实对称矩阵,设$\lambda_1,\cdots,\lambda_n$为$A$的特征值,则$\lambda_i^2+\lambda_i+2$为$A^2+2A+2I_n$的任意特征值。

对于任意$\lambda$均有$\lambda^2+\lambda+2=\left(\lambda+\dfrac{1}{2}\right)^2+\dfrac{7}{4}\geq \dfrac{7}{4}>0$。

由课本169页推论5.2.2知:$A^2+2A+2I_n$是正定阵。
\end{zhengming}

\EX 设$A$是正定阵。证明:$A$可逆,且$A^{-1}$也是正定阵。

\begin{zhengming}
由课本169页推论5.2.2知:$A$正定,则其全部特征值为正数,所以$A^{-1}$的全部特征值也为正数。正定阵一定是对称阵,即$A$对称,所以$A^{-1}$也为对称阵,综上所述$A^{-1}$是正定阵。
\end{zhengming}

\EX 设$A$是$n$阶正定阵。证明:对于任意非负数$a$都有:$A+aI_n$可逆。

\begin{zhengming}
$A$是$n$阶正定阵,设其任意特征值为$\lambda$,所以$\lambda>0$。因为$a$非负,所以$A+aI_{n}$的特征值$\lambda+a>0$,所以$|A+a_I|=\prod\limits_{i=1}^{n}>0$,即$|A+aI_n|\neq0$,即$A+aI_n$可逆。
\end{zhengming}

\EX 设二次型$f(x_1,x_2,x_3,x_4)$是正定的且其矩阵为$A=(a_{ij})$,设$B$是由$A$的第二、三、四行和第二、三、四列所构成的子矩阵。证明$B$也是正定的。

\begin{zhengming}
由题知$B$是三阶实对称矩阵。其次,令$g(x_2,x_3,x_4)=f(0,x_2,x_3,x_4)$,则$g$是关于$x_2,x_3,x_4$的三元实二次型,且$g$的矩阵就是$B$。

首先$g(x_2,x_3,x_4)=f(0,x_2,x_3,x_4)\geq 0$.

其次,若$g(x_2,x_3,x_4)=0$,则$0=g(x_2,x_3,x_4)=f(0,x_2,x_3,x_4)$,从而由$f$正定得:$x_2=x_3=x_4=0$。综上所述,$g$是正定的三元实二次型,所以$B$是正定矩阵。
\end{zhengming}

\EX 如果二次型$f(x_1,x_2,\cdots,x_n)$的值域是$[0,+\infty)$,则称$f(x_1,x_2,\cdots,x_n)$是半正定的。设二次型$f(x_1,x_2,x_3)$是半正定的,且其矩阵是$A$。

(1)证明:$|A|\geq 0$;

(2)设$B$是正定的三阶方阵。证明:$|A+B|> 0$。

\begin{zhengming}
(1)设$A$的特征值为$\lambda_1,\lambda_2,\lambda_3$,假设至少有一个特征值是负数,不妨设$\lambda_1<0$。

取正交变换:$x=Py$使得:
\begin{equation*}
f(x_1,x_2,x_3)\xlongequal{x=Py}\lambda_1y_1^2+\lambda_2y_2^2+\lambda_3y_3^2
\end{equation*}
取$y_0=(1,0,0)^T$,则$x_0=Py_0\neq0$且$f$在$x_0$的函数值为$\lambda_1$,但$\lambda_1<0$,与$f$半正定矛盾。所以$\lambda_i\geq0,1\leq i\leq3$,所以$|A|\geq0$。

(2)由于$B$是正定阵,所以存在可逆矩阵$Q$使得$Q^TBQ=D_1$和$Q^TAQ=D_2$都是对角阵。

由于$B$是正定的,所以$D_1$的对角元都是正数,$A$是半正定的,所以$D_2$的对角元非负,所以$D_1+D_2$的对角元都是正数,所以$|Q^T(B+A)Q|=|Q|^2|B+A|=|Q|^2|D_1+D_2|$,因为$Q$可逆,所以$|Q|\neq0$,即$|Q|^2>0$,所以$|Q|^2|B+A|=|Q|^2|D_1+D_2|>0$,所以$|A+B|>0$。
\end{zhengming}

\EX 如果二次型$f(x_1,\cdots,x_n)$的值域是$(-\infty,0]$,且只有当$x_1=\cdots=x_n=0$时才有$f(x_1,\cdots,x_n)=0$则称$f(x_1,x_2,\cdots,x_n)$是负定的。设$f(x_1,x_2,x_3)=x^TAx$,其中,$A$是3阶实对称矩阵,$\triangle_1,\triangle_2,\triangle_3$是$A$的顺序主子式。证明:$f(x_1,x_2,x_3)$是负定的$\Leftrightarrow~~\triangle_1<0,\triangle_2>0,\triangle_3<0$。

\begin{zhengming}
设$g(x_1,x_2,x_3)=-f(x_1,x_2,x_3)$,则$g$的矩阵是:$
-A=
\begin{pmatrix}
-a_{11}&-a_{12}&-a_{13}\\
-a_{21}&-a_{22}&-a_{23}\\
-a_{31}&-a_{32}&-a_{33}\\
\end{pmatrix}
$,所以:
\begin{align*}
&f\text{是负定的}\Leftrightarrow g\text{是正定的}\\
\Leftrightarrow&0<-a_{11}=-\triangle_1,0<
\begin{vmatrix}
-a_{11}&-a_{12}\\
-a_{21}&-a_{22}
\end{vmatrix}=\triangle_2,0<\begin{vmatrix}
-a_{11}&-a_{12}&-a_{13}\\
-a_{21}&-a_{22}&-a_{23}\\
-a_{31}&-a_{32}&-a_{33}\\
\end{vmatrix}=-\triangle_3\\
\Leftrightarrow&\triangle_1<0,\triangle_2>0,\triangle_3<0
\end{align*}
\end{zhengming}

\EX 求实数$a$的取值范围,使得二次型
\begin{equation*}
f(x_1,x_2,x_3)=ax_1^2+2x_2^2+3x_3^2+2x_1x_2+2ax_1x_3+2x_2x_3
\end{equation*}是正定的。

\begin{jie}
由题得:
$
A=
\begin{pmatrix}
a&1&a\\
1&2&1\\
a&1&3
\end{pmatrix}
$,要使其正定,则各阶顺序主子式均大于0:
\begin{equation*}
\begin{cases}
\triangle_1=a>0\\
\triangle_2=\begin{vmatrix}
              a&1\\
1&2
            \end{vmatrix}=2a-1>0\\
\triangle_3=
\begin{vmatrix}
a&1&a\\
1&2&1\\
a&1&3
\end{vmatrix}=(1-2a)(3-a)>0
\end{cases}~~\Rightarrow~~
\frac{1}{2}<a<3
\end{equation*}
\end{jie}

\EX 已知二次型$f=x_1^2+4x_2^2+4x_3^2+2\lambda x_1x_2-2x_1x_3+4x_2x_3$,$f$为正定二次型,求$\lambda$取值范围。

\begin{jie}
由题得:$
A=
\begin{pmatrix}
1&\lambda&-1\\
\lambda&4&2\\
-1&2&4
\end{pmatrix}
$,要使其正定,则各阶顺序主子式均大于0:
\begin{equation*}
\begin{cases}
\triangle_1=1>0\\
\triangle_2=\begin{vmatrix}
1&\lambda\\
\lambda&4
            \end{vmatrix}=4-\lambda^2>0\\
\triangle_3=
\begin{vmatrix}
1&\lambda&-1\\
\lambda&4&2\\
-1&2&4
\end{vmatrix}=3(4-\lambda^2)-(\lambda+2)^2>0
\end{cases}~~\Rightarrow~~
-2<\lambda<1
\end{equation*}
\end{jie}

\EX 设$A$为$m\times n$实矩阵,$E$为$n$阶单位矩阵。已知矩阵$B
=\lambda E+A^TA
$.试证:当$\lambda>0$时,矩阵$B$为正定矩阵。

\begin{zhengming}
对于任意实数$x_i$,令$x=(x_1,\cdots,x_n)^T$,则:
\begin{align*}
x^TBx=x^T(\lambda E+A^TA)x=\lambda x^Tx+x^TA^TAx=\lambda x^Tx+(Ax)^T(Ax)\geq \lambda x^Tx\geq0
\end{align*}
如果$x^TBx=0$,则由$\lambda x^tx\geq0$和$(Ax)^T(Ax)\geq0$可知$\lambda x^Tx=0$,但$\lambda>0$,所以必然有$x=0$。

综上所述,矩阵$B$为正定矩阵。
\end{zhengming}

\EX 设有$n$元实二次型$f(x_1,x_2,\cdots,x_n)=(x_1+a_1x_2)^2+(x_2+a_2x_3)^2+\cdots+(x_{n-1}+a_{n-1}x_n)^2+(x_{n}+a_{n}x_1)^2$,其中$a_i(i=1,2,\cdots,n)$为实数。试问:当$a_1,a_2,\cdots,a_n$满足何种条件时,二次型为正定二次型。

\begin{zhengming}
对任意的$x=(x_1,\cdots,x_n)^T$,$x_i$为实数,都有$f(x_1,x_2,\cdots,x_n)\geq 0$,所以$f$正定等价于齐次线性方程组
\begin{equation*}
  \begin{cases}
x_1+a_1x_2=0\\
x_2+a_2x_3=0\\
\vdots\\
x_{n-1}+a_{n-1}x_n=0\\
x_{n}+a_{n}x_1=0
  \end{cases}
\end{equation*}只有零解,等价于系数矩阵
\begin{equation*}
A=
\begin{pmatrix}
1&a_1&0&\cdots&0&0\\
0&1&a_2&\cdots&0&0\\
0&0&1&\cdots&0&0\\
\vdots&\vdots&\vdots&\ddots&\vdots&\vdots\\
0&0&0&\cdots&1&a_{n-1}\\
a_n&0&0&\cdots&0&1
\end{pmatrix}
\end{equation*}
的行列式非0.按第一列展开可得$|A|=1+(-1)^{n+1}a_1a_2\cdots a_n$,所以原二次型正定当且仅当$1+(-1)^{n+1}a_1a_2\cdots a_n\neq 0$。
\end{zhengming}
\end{document} 