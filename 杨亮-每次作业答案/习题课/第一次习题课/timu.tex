\documentclass[a4paper]{report}
\usepackage[space,fancyhdr,fntef]{ctexcap}
\usepackage{fontspec}
\fontspec{宋体}
\setmainfont{Times New Roman}
%\fontsize{50pt}{50pt}\selectfont
\renewcommand{\rmdefault}{ptm}
\usepackage[namelimits,sumlimits,nointlimits]{amsmath}
\usepackage[text={169mm,250mm},bottom=20mm,top=25mm,left=25mm,right=20mm,centering]{geometry}
\usepackage{color}
\usepackage{CJKfntef}%下划线宏包160页
\usepackage{xcolor}
\usepackage{arydshln}%234页,虚线表格宏包
\pagestyle{fancy} \fancyhf{}
\pagestyle{fancy} \fancyhf{}
\fancyhead[OL]{~~~学院:\hfill 学号:\hfill 姓名:王松年~~~ }
\fancyfoot[C]{\color{gray}\thepage}
\renewcommand{\headrule}{\color{gray}\hrule width\headwidth}
%\renewcommand{\footrulewidth}{0.4pt}%改为0pt即可去掉页脚上面的横线
%\usepackage{parskip}
%\usepackage{indentfirst}
\usepackage{graphicx}%插图宏包,参见手册318页

\usepackage[xetex,colorlinks]{hyperref}%394页  \href{网址}{文本}
\hypersetup{urlcolor=blue}
%\linebreak[2]%换行,152页
\usepackage{fancybox}%盒子宏包55页
\setcounter{secnumdepth}{4}
\CTEXoptions[contentsname={目\hspace{15pt}录}]
\CTEXsetup[beforeskip={-40pt},afterskip={20pt plus 2pt minus 2pt}]{chapter}
\usepackage{mathdots}%反对角省略号
\usepackage{extarrows}%等号上加文字
\usepackage{paralist}%列表宏包

%目录设置
\usepackage{titletoc}
\usepackage[toc]{multitoc}
\titlecontents{chapter}[4em]{\addvspace{2.3mm}\bf}{\contentslabel{4.0em}}{}{\titlerule*[5pt]{$\cdot$}\contentspage}
\titlecontents{section}[4em]{}{\contentslabel{2.5em}}{}{\titlerule*[5pt]{$\cdot$}\contentspage}
\titlecontents{subsection}[7.2em]{}{\contentslabel{3.3em}}{}{\titlerule*[5pt]{$\cdot$}\contentspage}

\begin{document}
\flushbottom%版心底部对齐
\newcounter{num}[section] \renewcommand{\thenum}{\arabic{num}.} \newcommand{\num}{\refstepcounter{num}\text{\thenum}}
\newenvironment{jie}{\kaishu\zihao{-5}\color{blue}{\noindent\em 解:}\par}{\hfill $\diamondsuit$\par}
\newenvironment{tips}{\kaishu\zihao{-6}\color{blue}{\noindent\rule[-3pt]{\textwidth}{0.5pt}\par \em \noindent {\zihao{-5} \textcolor[rgb]{1.00,0.00,0.00}{Tips}}}\par}{\\ \rule[3mm]{\textwidth}{0.5pt}\par}

%生成格式如 例1.1的带序号的示例标识
\newcounter{Emp}[chapter] \renewcommand{\theEmp}{\thechapter.\arabic{Emp}}
\newcommand{\EX}{\par {\bf 例~}\refstepcounter{Emp}{\bf\theEmp}\hspace{1em}}

\newenvironment{zhengming}{\kaishu\zihao{-5}\color{blue}{\noindent\em 证明:}\par}{\hfill $\diamondsuit$\par}

%\tableofcontents
%\pagenumbering{Roman}%设置目录页码
%\clearpage
\pagenumbering{arabic}%设置正文页码

\hphantom{~~}\hfill {\zihao{4}\heiti 补充} \hfill\hphantom{~~}

\num 二项式定理(记住这个结论,后边会用到):
%
\begin{equation*}(a+b)^{n}=\sum_{i=0}^{n}\left(C_{n}^{i}a^{i}b^{n-i}\right)\end{equation*}

\hphantom{~~}\hfill {\zihao{4}\heiti 期中试题} \hfill\hphantom{~~}

\num ~~16-17学年(一.3)

设方程组$
\begin{cases}
 2x_{1}-x_{2}+x_{3}=0\\
 x_{1}+kx_{2}-x_{3}=0\\
 kx_{1}+x_{2}+x_{3}=0
\end{cases}
$有非零解,则$k=$
\underline{\hphantom{~~~~-1或4~~~~}}。

\begin{jie}
%该方程组的系数矩阵为一个三阶方阵,由解得存在唯一性定理,如果$Ax=0$有非零解,则$|A|=0$。所以:
%\begin{equation*}
%|A|=
%\begin{vmatrix}
%2 & -1 & 1 \\
%1 & k &-1\\
%k & 1 & 1
%\end{vmatrix}=(1+k)(4-k)=0~~~\Rightarrow k_{1}=-1 ~k_{2}=4
%\end{equation*}
%(上述行列式第一行减去第三行,第二行加上第三行后按第三列展开)
由题得,系数矩阵:
\begin{equation*}
A=
\begin{pmatrix}
2&-1&1\\ 1&k&-1\\ k&1&1
\end{pmatrix}\xrightarrow{\substack{r_{2}-\frac{1}{2}r_{1}\\ r_{3}-\frac{k}{2}r_{1}}}
{
\begin{pmatrix}
2&-1&1\\ 0&k+\frac{1}{2}&-\frac{3}{2}\\ 0&1+\frac{k}{2}&1-\frac{k}{2}
\end{pmatrix}
}
\end{equation*}
要使齐次方程组有非零解,则需存在自由变量。对于齐次方程,方程的个数小于未知数个数则有非零解,此时只需让第二个和第三个方程中对应的非零系数成比例即可(这样便可以消去其中一个方程),即
\begin{equation*}
\frac{k+1/2}{1+k/2}=\frac{-3/2}{1-k/2}~~~\Rightarrow~~~k=4\text{或}k=-1
\end{equation*}
\end{jie}

\hphantom{~~}\hfill {\zihao{4}\heiti 期末试题} \hfill\hphantom{~~}


\num ~~14-15学年(四)

$\lambda$为何值时,方程组$
\begin{cases}
 2x_{1}+\lambda x_{2}-x_{3}=1\\
 \lambda x_{1}-x_{2}+x_{3}=2\\
 4x_{1}+5 x_{2}-5x_{3}=-1
\end{cases}
$有无穷多组解?并在有无穷多解时,写出方程组的通解。

\begin{jie}
%记$A=
%\begin{bmatrix}
% 2&\lambda &-1\\
% \lambda &-1&1\\
% 4& 5 &-5
%\end{bmatrix}
%,B=\begin{bmatrix}
%     1 \\ 2\\ -1
%   \end{bmatrix}$,
%   \begin{equation*}
%   |A|=
%   \begin{vmatrix}
%    2&\lambda &-1\\
% \lambda &-1&1\\
% 4& 5 &-5
%   \end{vmatrix}
%   \xlongequal{c_{3}+c_{2}}
%  \begin{vmatrix}
%    2&\lambda &\lambda-1\\
% \lambda &-1&0\\
% 4& 5 &0
%   \end{vmatrix}  =(\lambda-1)(5\lambda+4)
%   \end{equation*}
%   可以看出$\lambda\neq1$且$\lambda\neq-\dfrac{4}{5}$时即$|A|\neq0$时,方程有唯一解。
%
%   $\lambda=1$时:
%   \begin{align*}
% [A|B]=
% \left[
% \begin{array}{c:c}
%\begin{matrix}
%2 & 1 & -1 \\
%  1 & -1 & 1 \\
%  4 & 5 & -5
%\end{matrix}&
%\begin{matrix}
%1  \\
% 2\\
%-1
%\end{matrix}
%\end{array}
%\right]
%\xrightarrow{\substack{r_{2}-\frac{1}{2}r_{1}\\ r_{3}-2r_{1}}}
%{
% \left[
% \begin{array}{c:c}
%\begin{matrix}
%2 & 1 & -1 \\
% 0 & -\frac{3}{2} & \frac{3}{2} \\
%  0 & 3 & -3
%\end{matrix}&
%\begin{matrix}
%1  \\
% \frac{3}{2}\\
%-3
%\end{matrix}
%\end{array}
%\right]
%}
%\xrightarrow{r_{3}+2r_{2}}
%{
% \left[
% \begin{array}{c:c}
%\begin{matrix}
%2 & 1 & -1 \\
% 0 & -\frac{3}{2} & \frac{3}{2} \\
%  0 & 0 & 0
%\end{matrix}&
%\begin{matrix}
%1  \\
% \frac{3}{2}\\
%0
%\end{matrix}
%\end{array}
%\right]
%}
%\xrightarrow{\text{中间略}}
%{
% \left[
% \begin{array}{c:c}
%\begin{matrix}
%1 & 0 & 0 \\
% 0 & 1 & -1 \\
%  0 & 0 & 0
%\end{matrix}&
%\begin{matrix}
%1  \\
%-1\\
%0
%\end{matrix}
%\end{array}
%\right]
%}
%   \end{align*}
%所以$\lambda=1$时有无穷多解,$x_{1}=1,x_{2}=x_{3}-1$.
%
%所以通解为$x=[1,-1,0]^T+k[0,1,1]^T,(k\in R)$
%
%$\lambda=-\dfrac{4}{5}$时,$r(A)\neq r(A,B)$,此时无解。
由题得增广矩阵:
\begin{equation*}
(A|b)=\left(
 \begin{array}{c:c}
\begin{matrix}
2 & \lambda & -1 \\
\lambda & -1 & 1 \\
4 & 5 & -5
\end{matrix}&
\begin{matrix}
1  \\
2\\
-1
\end{matrix}
\end{array}
\right)
\xrightarrow{\substack{r_{2}-\frac{1}{2}r_{1}\\ r_{3}-2r_{1}}}
{
 \left(
 \begin{array}{c:c}
\begin{matrix}
2 & \lambda & -1 \\
0 & -1-\frac{\lambda^2}{2} & 1+\frac{\lambda}{2} \\
0 & 5-2\lambda & -3
\end{matrix}&
\begin{matrix}
1  \\
2-\frac{\lambda}{2}\\
-3
\end{matrix}
\end{array}
\right)
}
\end{equation*}

第二行中:$\lambda^2\geq 0$,所以可以推出$-1-\frac{\lambda^2}{2}\leq-1$,即$-1-\frac{\lambda^2}{2}\neq 0$,继续高斯消元:
\begin{equation*}
\xrightarrow{\substack{r_{3}+r_2\frac{5-2\lambda}{1+\frac{\lambda^2}{2}}}}
{
 \left(
 \begin{array}{c:c}
\begin{matrix}
2 & \lambda & -1 \\
0 & -1-\frac{\lambda^2}{2} & 1+\frac{\lambda}{2} \\
0 & 0 & \frac{-5\lambda^2+\lambda+4}{\lambda^2+2}
\end{matrix}&
\begin{matrix}
1  \\
2-\frac{\lambda}{2}\\
\frac{-\lambda^2-13\lambda+14}{\lambda^2+2}
\end{matrix}
\end{array}
\right)
}
\end{equation*}

要使方程组有无穷多解,则需存在自由变量,而$-1-\frac{\lambda^2}{2}\neq 0$,即$x_1,x_2$都不为自由变量,所以$x_3$为自由变量,所以:
\begin{equation*}
\frac{-5\lambda^2+\lambda+4}{\lambda^2+2}=0~~~\text{且}~~~\frac{-\lambda^2-13\lambda+14}{\lambda^2+2}=0~~~\Rightarrow~~~\lambda =1
\end{equation*}
计算结果代入上述阶梯型矩阵:
\begin{equation*}
 \left(
 \begin{array}{c:c}
\begin{matrix}
2 & 1 & -1 \\
0 & -\frac{3}{2} & \frac{3}{2} \\
0 & 0 & 0
\end{matrix}&
\begin{matrix}
1  \\
\frac{3}{2}\\
0
\end{matrix}
\end{array}
\right)\xrightarrow{\substack{r_{1}\times\frac{1}{2}\\ r_2\times\left(-\frac{2}{3}\right)}}
{
 \left(
 \begin{array}{c:c}
\begin{matrix}
1 & \frac{1}{2} & -\frac{1}{2} \\
0 & 1 & -1 \\
0 & 0 & 0
\end{matrix}&
\begin{matrix}
\frac{1 }{2} \\
-1\\
0
\end{matrix}
\end{array}
\right)
}\xrightarrow{\substack{r_{1}-\frac{1}{2}r_2}}
{
 \left(
 \begin{array}{c:c}
\begin{matrix}
1 & 0 & 0 \\
0 & 1 & -1 \\
0 & 0 & 0
\end{matrix}&
\begin{matrix}
1 \\
-1\\
0
\end{matrix}
\end{array}
\right)
}
\end{equation*}
由最简阶梯型矩阵可以得出:
$x_1=1,x_2=x_3-1$,令$x_3=k,k\in R$则
\begin{equation*}
x=
\begin{pmatrix}
1\\ k-1\\ k
\end{pmatrix}=
\begin{pmatrix}
1\\ -1\\ 0
\end{pmatrix}+k\begin{pmatrix}
0\\ 1\\ 1
\end{pmatrix},~~k\in R
\end{equation*}
\end{jie}

\num ~~15-16学年(三.1)

当$k$为何值时,线性方程组
$
\begin{cases}
kx_{1}+x_{2}+x_{3}=k-3\\
x_{1}+kx_{2}+x_{3}=-2\\
x_{1}+x_{2}+kx_{3}=-2
\end{cases}
$
有唯一解,无解和有无穷多解?当方程组有无穷多解时求出所有解。

\begin{jie}

增广矩阵
\begin{align*}
&\left[
\begin{array}{c:c}
\begin{matrix}
  k & 1 & 1 \\
  1 & k & 1 \\
  1 & 1 & k \\
\end{matrix}
&
\begin{matrix}
  k-3 \\
  -2 \\
  -2 \\
\end{matrix}
\end{array}
\right]
\xrightarrow{r_{1}\leftrightarrow r_{2}}
{
\left[
\begin{array}{c:c}
\begin{matrix}
  1 & k & 1 \\
  k & 1 & 1 \\
  1 & 1 & k \\
\end{matrix}
&
\begin{matrix}
  -2 \\
  k-3 \\
  -2 \\
\end{matrix}
\end{array}
\right]
}
\xrightarrow{\substack{r_{2}-kr_{1} \\ r_{3}-r_{1}}}
{
\left[
\begin{array}{c:c}
\begin{matrix}
  1 & k & 1 \\
  0 & 1-k^{2} & 1-k \\
  0 & 1-k & k-1 \\
\end{matrix}
&
\begin{matrix}
  -2 \\
  3(k-1) \\
  0 \\
\end{matrix}
\end{array}
\right]
}\\
\xrightarrow{r_{2}\leftrightarrow r_{3}} &
{
\left[
\begin{array}{c:c}
\begin{matrix}
  1 & k & 1 \\
  0 & 1-k & k-1 \\
  0 & 1-k^{2} & 1-k \\
\end{matrix}
&
\begin{matrix}
  -2 \\
  0 \\
  3(k-1) \\
\end{matrix}
\end{array}
\right]
}
\xrightarrow{r_{3}-(1+k)r_{2}}
{
\left[
\begin{array}{c:c}
\begin{matrix}
  1 & k & 1 \\
  0 & 1-k & k-1 \\
  0 & 0 & (1-k)(k+2) \\
\end{matrix}
&
\begin{matrix}
  -2 \\
  0 \\
  3(k-1) \\
\end{matrix}
\end{array}
\right]
}
\end{align*}
讨论:

(1)解不存在:即存在矛盾方程(增广矩阵主元列在最右列)。即对于$r_{3}$
\begin{equation*}
  \begin{cases}
    (1-k)(k+2)=0\\
    3(k-1)\neq 0
  \end{cases}~~~
  \Rightarrow~~~k=-2
\end{equation*}

(2)存在唯一解:主元列三个元素都不为0.即
\begin{equation*}
  \begin{cases}
    1\neq 0\\
    1-k\neq 0 \\
    (1-k)(k+2)\neq 0
  \end{cases}
  ~~~\Rightarrow~~~k\neq 1\text{且}k\neq -2
\end{equation*}
$k\neq 1\text{且}k\neq -2$,继续对阶梯矩阵进行初等行变换
\begin{align*}
 \xrightarrow{\substack{r_{2}\times\frac{1}{1-k}\\r_{3}\times\frac{1}{(k+2)(1-k)} }}
{
\left[
\begin{array}{c:c}
\begin{matrix}
  1 & k & 1 \\
  0 & 1 & -1 \\
  0 & 0 & 1 \\
\end{matrix}
&
\begin{matrix}
  -2 \\
  0 \\
  \frac{3}{(k+2)} \\
\end{matrix}
\end{array}
\right]
}
\xrightarrow{\substack{r_{2}+r_{3}\\r_{1}-r_{3} }}
{
\left[
\begin{array}{c:c}
\begin{matrix}
  1 & k & 0 \\
  0 & 1 & 0 \\
  0 & 0 & 1 \\
\end{matrix}
&
\begin{matrix}
  -2-\frac{3}{k+2} \\
  \frac{3}{k+2} \\
  \frac{3}{k+2} \\
\end{matrix}
\end{array}
\right]
}
\xrightarrow{r_{1}-kr_{2}}
{
\left[
\begin{array}{c:c}
\begin{matrix}
  1 & 0 & 0 \\
  0 & 1 & 0 \\
  0 & 0 & 1 \\
\end{matrix}
&
\begin{matrix}
  -\frac{5k+1}{k+2} \\
  \frac{3}{k+2} \\
  \frac{3}{k+2} \\
\end{matrix}
\end{array}
\right]
}
\end{align*}
所以方程组存在唯一解时:$k\neq 1$且$k\neq -2$,解为
\begin{equation*}
\mathbf{x}=
\begin{bmatrix}
 -\frac{5k+1}{k+2} \\
  \frac{3}{k+2} \\
  \frac{3}{k+2}
\end{bmatrix}
~,~~~k\neq1\text{且}k\neq -2
\end{equation*}

(3)存在无穷解:至少存在一个自由变量。由阶梯矩阵可以看出
\begin{equation*}
  \begin{cases}
    (k-1)(k+2)=0\\
    3(k-1)=0
  \end{cases}
  ~~~\Rightarrow~~~k=1
\end{equation*}
把$k=1$代入阶梯矩阵:
\begin{equation*}
\left[
  \begin{array}{c:c}
    \begin{matrix}
      1 & 1 & 1 \\
      0 & 0 & 0 \\
      0 & 0 & 0 \\
    \end{matrix}
     &
     \begin{matrix}
      -2 \\
      0 \\
      0\\
    \end{matrix}
  \end{array}
\right]
~~~\Rightarrow~~~
x=
\begin{bmatrix}
  -2-c_{1}-c_{2} \\
  c_{1} \\
  c_{2}
\end{bmatrix}
\end{equation*}
所以$x$的解为$x=
\begin{bmatrix}
  -2 \\
  0\\
   0
\end{bmatrix}+
\begin{bmatrix}
  -1 \\
  1\\
   0
\end{bmatrix}c_{1}+
\begin{bmatrix}
  -1 \\
  0\\
   1
\end{bmatrix}c_{1}
~~~c_{1},c_{2}\in R$。
\end{jie}

\num ~~17-18学年(二.3)

求线性方程组
$
\begin{cases}
 2x_{1}-x_{2}+4x_{3}-3x_{4}=-4\\
 x_{1}+x_{3}-x_{4}=-3\\
 3x_{1}+x_{2}+x_{3}=1\\
 7x_{1}+7x_{3}-3x_{4}=3\\
\end{cases}
$
的通解。

\begin{jie}
增广矩阵
\begin{align*}
&\left[
\begin{array}{c:c}
 \begin{matrix}
   2 & -1 & 4 &-3 \\
   1 & 0 & 1 &-1 \\
   3&1&1&0\\
   7 & 0 & 7 &-3\\
 \end{matrix}
 &
  \begin{matrix}
   -4 \\
   -3\\
   1\\
   3\\
 \end{matrix}
\end{array}
\right]
\xrightarrow{\substack{r_{2}-\frac{1}{2}r_{1} \\ r_{3}-\frac{3}{2}r_{1} \\ r_{4}-\frac{7}{2}r_{1}}}
{
\left[
\begin{array}{c:c}
 \begin{matrix}
   2 & -1 & 4 &-3 \\
   0 & \frac{1}{2} & -1 &\frac{1}{2} \\
   0& \frac{5}{2} & -5 &\frac{9}{2} \\
   0 & \frac{7}{2} & -7 &\frac{15}{2}\\
 \end{matrix}
 &
  \begin{matrix}
   -4 \\
   -1\\
   7\\
   17\\
 \end{matrix}
\end{array}
\right]
}
\xrightarrow{\substack{r_{3}-5r_{2} \\ r_{4}-7r_{2}}}
{
\left[
\begin{array}{c:c}
 \begin{matrix}
   2 & -1 & 4 &-3 \\
   0 & \frac{1}{2} & -1 &\frac{1}{2} \\
   0&0&&0&2\\
   0 &0& 0 &4\\
 \end{matrix}
 &
  \begin{matrix}
   -4 \\
   -1\\
   12\\
   24\\
 \end{matrix}
\end{array}
\right]
}\\
\xrightarrow{r_4-2r_3}
&{
\left[
\begin{array}{c:c}
 \begin{matrix}
   2 & -1 & 4 &-3 \\
   0 & \frac{1}{2} & -1 &\frac{1}{2} \\
   0&0&0&2\\
   0 &0& 0 &0\\
 \end{matrix}
 &
  \begin{matrix}
   -4 \\
   -1\\
   12\\
   0\\
 \end{matrix}
\end{array}
\right]
}
\xrightarrow{\substack{r_{1}\times\frac{1}{2} \\ r_{2}\times 2 \\ r_{3}\times \frac{1}{2}}}
{
\left[
\begin{array}{c:c}
 \begin{matrix}
   1 & -\frac{1}{2} & 2 &-\frac{3}{2} \\
   0 & 1 & -2 &1 \\
   0 &0& 0 &1\\
    0 &0& 0 &0\\
 \end{matrix}
 &
  \begin{matrix}
   -2 \\
   -2\\
   6\\ 0
 \end{matrix}
\end{array}
\right]
}
\xrightarrow{\substack{r_{2}-r_{3}\\r_{1}+\frac{3}{2} r_{3}}}
{
\left[
\begin{array}{c:c}
 \begin{matrix}
   1 & -\frac{1}{2} & 2 &0 \\
   0 & 1 & -2 &0 \\
   0 &0& 0 &1\\  0 &0& 0 &0\\
 \end{matrix}
 &
  \begin{matrix}
   7 \\
   -8\\
   6\\ 0
 \end{matrix}
\end{array}
\right]
}\\
\xrightarrow{r_{1}+\frac{1}{2} r_{2}}&
{
\left[
\begin{array}{c:c}
 \begin{matrix}
   1 & 0 & 1 &0 \\
   0 & 1 & -2 &0 \\
   0 &0& 0 &1\\  0 &0& 0 &0
 \end{matrix}
 &
  \begin{matrix}
   3 \\
   -8\\
   6\\ 0
 \end{matrix}
\end{array}
\right]
}
\end{align*}
由最简阶梯型矩阵可以看出:
\begin{equation*}
  x_{1}=3-x_{3}~~x_{2}=2x_{3}-8~~x_{3}=x_{3}~~x_{4}=6
\end{equation*}
令$x_{3}=C,C\in R$,则
\begin{equation*}
x=
 \begin{bmatrix}
   3-C \\
   2C-8 \\
   C\\
   6
 \end{bmatrix}
 =
  \begin{bmatrix}
   3 \\
   -8 \\
   0\\
   6
 \end{bmatrix}
 +
  \begin{bmatrix}
   -1 \\
   2 \\
   1\\
   0
 \end{bmatrix}C
 ~,~C\in R
\end{equation*}
\end{jie}

\num ~~18-19学年(三.1)

设
$
\begin{cases}
\lambda x_{1}+x_{2}+x_{3}=\lambda-2\\
x_{1}+\lambda x_{2} +x_{3}=2\\
x_{1}+ x_{2} +\lambda x_{3}=2
\end{cases}
$
,$\lambda$为何值时,该方程组无解、唯一解、无穷解?并且在有唯一解时求出解;有无穷多解时,求出全部解并用向量表示。

\begin{jie}

增广矩阵
\begin{align*}
&\left[
\begin{array}{c:c}
\begin{matrix}
\lambda & 1 & 1 \\
1& \lambda & 1 \\
1 & 1&\lambda  \\
\end{matrix}
&
\begin{matrix}
\lambda-2 \\
2 \\
2 \\
\end{matrix}
\end{array}
\right]
\xrightarrow{r_{1}\Leftrightarrow r_{2}}
{
\left[
\begin{array}{c:c}
\begin{matrix}
1& \lambda & 1 \\
\lambda & 1 & 1 \\
1 & 1&\lambda  \\
\end{matrix}
&
\begin{matrix}
2 \\
\lambda-2 \\
2 \\
\end{matrix}
\end{array}
\right]
}
\xrightarrow{\substack{r_{2}-\lambda r_{1}\\ r_{3}- r_{1} }}
{
\left[
\begin{array}{c:c}
\begin{matrix}
1& \lambda & 1 \\
0 & 1-\lambda^{2} & 1-\lambda \\
0 & 1-\lambda &\lambda-1  \\
\end{matrix}
&
\begin{matrix}
2 \\
-\lambda-2 \\
0 \\
\end{matrix}
\end{array}
\right]
}\\
\xrightarrow{r_{2}\Leftrightarrow r_{3}}&
{
\left[
\begin{array}{c:c}
\begin{matrix}
1& \lambda & 1 \\
0 & 1-\lambda &\lambda-1  \\
0 & 1-\lambda^{2} & 1-\lambda \\
\end{matrix}
&
\begin{matrix}
2 \\
0 \\
-\lambda-2 \\
\end{matrix}
\end{array}
\right]
}
\xrightarrow{r_{3}-(\lambda+1) r_{2}}
{
\left[
\begin{array}{c:c}
\begin{matrix}
1& \lambda & 1 \\
0 & 1-\lambda &\lambda-1  \\
0 & 0 & (\lambda-1)(-2-\lambda) \\
\end{matrix}
&
\begin{matrix}
2 \\
0 \\
-\lambda-2 \\
\end{matrix}
\end{array}
\right]
}
\end{align*}
讨论:

(1)解不存在:即存在矛盾方程(增广矩阵主元列在最右列)。即对于$r_{3}$
\begin{equation*}
  \begin{cases}
    (\lambda-1)(-2-\lambda)=0\\
    -\lambda-2\neq 0
  \end{cases}~~~
  \Rightarrow~~~\lambda=1
\end{equation*}

(2)存在唯一解:主元列三个元素都不为0.即
\begin{equation*}
  \begin{cases}
    1\neq 0\\
    1-\lambda\neq 0 \\
    (\lambda-1)(-2-\lambda)\neq 0
  \end{cases}
  ~~~\Rightarrow~~~\lambda\neq 1\text{且}\lambda\neq -2
\end{equation*}
$\lambda\neq 1\text{且}\lambda\neq -2$,继续对阶梯矩阵进行初等行变换
\begin{equation*}
\xrightarrow{\substack{ r_{2}\times \frac{1}{1-\lambda}\\  r_{3}\times \frac{1}{(\lambda-1)(-2-\lambda)}}}
{
\left[
\begin{array}{c:c}
\begin{matrix}
1& \lambda & 1 \\
0 & 1 & -1  \\
0 & 0 & 1 \\
\end{matrix}
&
\begin{matrix}
2 \\
0 \\
\frac{1}{\lambda-1} \\
\end{matrix}
\end{array}
\right]
}
\xrightarrow{\substack{ r_{2}+r_{3}\\  r_{1}-r_{3}}}
{
\left[
\begin{array}{c:c}
\begin{matrix}
1& \lambda & 0 \\
0& 1 & 0  \\
0 & 0 & 1 \\
\end{matrix}
&
\begin{matrix}
\frac{2\lambda-3}{\lambda-1} \\
\frac{1}{\lambda-1} \\
\frac{1}{\lambda-1} \\
\end{matrix}
\end{array}
\right]
}
\xrightarrow{ r_{1}-\lambda r_{2}}
{
\left[
\begin{array}{c:c}
\begin{matrix}
1& 0 & 0 \\
0 & 1 & 0  \\
0 & 0 & 1 \\
\end{matrix}
&
\begin{matrix}
\frac{\lambda-3}{\lambda-1} \\
\frac{1}{\lambda-1} \\
\frac{1}{\lambda-1} \\
\end{matrix}
\end{array}
\right]
}
\end{equation*}
所以方程组存在唯一解时:$\lambda\neq 1$且$\lambda\neq -2$,解为
\begin{equation*}
\mathbf{x}=
\begin{bmatrix}
\frac{\lambda-3}{\lambda-1} \\
\frac{1}{\lambda-1} \\
\frac{1}{\lambda-1}
\end{bmatrix}
~,~~~\lambda\neq1\text{且}\lambda\neq -2
\end{equation*}

(3)存在无穷解:至少存在一个自由变量。由阶梯矩阵可以看出
\begin{equation*}
  \begin{cases}
    (\lambda-1)(-2-\lambda)=0\\
    -2-\lambda=0
  \end{cases}
  ~~~\Rightarrow~~~\lambda=-2
\end{equation*}
把$\lambda=-2$代入阶梯矩阵:
\begin{equation*}
\left[
\begin{array}{c:c}
\begin{matrix}
1& -2 & 1 \\
0 & 3 & -3  \\
0 & 0 & 0 \\
\end{matrix}
&
\begin{matrix}
2 \\
0 \\
0 \\
\end{matrix}
\end{array}
\right]
\xrightarrow{ r_{2}\times\frac{1}{3}}
{\left[
\begin{array}{c:c}
\begin{matrix}
1& -2 & 1 \\
0 & 1 & -1  \\
0 & 0 & 0 \\
\end{matrix}
&
\begin{matrix}
2 \\
0 \\
0 \\
\end{matrix}
\end{array}
\right]
}
\xrightarrow{ r_{1}+2r_{2}}
{\left[
\begin{array}{c:c}
\begin{matrix}
1& 0 & -1 \\
0 & 1 & -1  \\
0 & 0 & 0 \\
\end{matrix}
&
\begin{matrix}
2 \\
0 \\
0 \\
\end{matrix}
\end{array}
\right]
}
\end{equation*}
由最简阶梯型矩阵可以看出:
\begin{equation*}
  x_{1}=2+x_{3}~~x_{2}=x_{3}~~x_{3}=x_{3}
\end{equation*}
令$x_{3}=C,C\in R$,则
\begin{equation*}
x=
 \begin{bmatrix}
   2+C \\
   C \\
   C
 \end{bmatrix}
 =
  \begin{bmatrix}
   2 \\
   0 \\
   0
 \end{bmatrix}
 +
  \begin{bmatrix}
   1 \\
   1 \\
   1
 \end{bmatrix}C
 ~,~C\in R
\end{equation*}
\end{jie}

\num ~~19-20学年(一.4)

若线性方程组
$
\begin{cases}
x_{1}+x_{2}=-a_{1}\\
x_{2}+x_{3}=a_{2}\\
x_{3}+x_{4}=-a_{3}\\
x_{4}+x_{1}=a_{4}
\end{cases}
$
有解,$a_{1},a_{2},a_{3},a_{4}$应满足的条件是\underline{\hphantom{~~\textcolor[rgb]{1.00,0.00,0.00}{$a_{1}+a_{2}+a_{3}+a_{4}=0$}~~}}。

%\begin{jie}
%增广矩阵
%\begin{align*}
%&\left[
%\begin{array}{c:c}
%\begin{matrix}
%1 & 1 & 0 & 0 \\
%0 & 1 & 1 & 0 \\
%0 & 0 & 1 & 1 \\
%1 & 0 & 0 & 1 \\
%\end{matrix}
%&
%\begin{matrix}
%-a_{1} \\
%a_{2} \\
%-a_{3} \\
%a_{4} \\
%\end{matrix}
%\end{array}
%\right]
%\xrightarrow{r_{4}-r_{1}}
%{
%\left[
%\begin{array}{c:c}
%\begin{matrix}
%1 & 1 & 0 & 0 \\
%0 & 1 & 1 & 0 \\
%0 & 0 & 1 & 1 \\
%0 & -1 & 0 & 1 \\
%\end{matrix}
%&
%\begin{matrix}
%-a_{1} \\
%a_{2} \\
%-a_{3} \\
%a_{1}+a_{4} \\
%\end{matrix}
%\end{array}
%\right]
%}
%\xrightarrow{r_{4}+r_{2}}
%{
%\left[
%\begin{array}{c:c}
%\begin{matrix}
%1 & 1 & 0 & 0 \\
%0 & 1 & 1 & 0 \\
%0 & 0 & 1 & 1 \\
%0 & 0 & 1 & 1 \\
%\end{matrix}
%&
%\begin{matrix}
%-a_{1} \\
%a_{2} \\
%-a_{3} \\
%a_{1}+a_{2}+a_{4} \\
%\end{matrix}
%\end{array}
%\right]
%}\\
%\xrightarrow{r_{4}-r_{3}}&
%{
%\left[
%\begin{array}{c:c}
%\begin{matrix}
%1 & 1 & 0 & 0 \\
%0 & 1 & 1 & 0 \\
%0 & 0 & 1 & 1 \\
%0 & 0 & 0 & 0 \\
%\end{matrix}
%&
%\begin{matrix}
%-a_{1} \\
%a_{2} \\
%-a_{3} \\
%a_{1}+a_{2}+a_{3}+a_{4} \\
%\end{matrix}
%\end{array}
%\right]
%}
%\end{align*}
%若方程有解:$a_{1}+a_{2}-a_{3}+a_{4}=0$
%\end{jie}


\num ~~19-20学年(二.2)

求线性方程组$
\begin{cases}
x_{1}+3x_{2}+2x_{3}+3x_{4}=0\\
2x_{1}+4x_{2}+x_{3}+3x_{4}=0\\
2x_{1}+4x_{2}+4x_{4}=0\\
\end{cases}
$的一个基础解系。

%\begin{jie}
%注:基础解系的定义后边会学,此处主要还是学会熟练使用高斯消元法。
%
%由题得:增广矩阵
%\begin{align*}
%A&=
%\begin{bmatrix}
%1 & 3 & 2 &3\\
%2 & 4 & 1 & 3\\
%2 & 4 & 0 & 4
%\end{bmatrix}
%\xrightarrow{\substack{r_{2}-2r_1\\ r_3-2r_1}}
%{
%\begin{bmatrix}
%1 & 3 & 2 &3\\
%0 & -2 & -3 & -3\\
%0 & -2 & -4 & -2
%\end{bmatrix}
%}
%\xrightarrow{\substack{r_{3}-r_2}}
%{
%\begin{bmatrix}
%1 & 3 & 2 &3\\
%0 & -2 & -3 & -3\\
%0 & 0 & -1 & 1
%\end{bmatrix}
%}
%\xrightarrow{\substack{r_{1}+2r_3 \\ r_2-3r_3}}
%{
%\begin{bmatrix}
%1 & 3 & 0 & 5\\
%0 & -2 & 0 & -6\\
%0 & 0 & -1 & 1
%\end{bmatrix}
%}\\
%&
%\xrightarrow{\substack{r_{2}\times\left(-\frac{1}{2}\right) \\ r_3\times\left(-1\right)}}
%{
%\begin{bmatrix}
%1 & 3 & 0 & 5\\
%0 & 1 & 0 & 3\\
%0 & 0 & 1 & -1
%\end{bmatrix}
%}\xrightarrow{\substack{r_{1}-3r_2}}
%{
%\begin{bmatrix}
%1 & 0 & 0 & -4\\
%0 & 1 & 0 & 3\\
%0 & 0 & 1 & -1
%\end{bmatrix}
%}
%\end{align*}
%所以$x_{1}=4x_4,x_2=-3x_4,x_3=x_4$,令$x_4=1$,得基础解系:$\xi=[4,-3,1,1]^T$。
%\end{jie}

\num ~~20-21学年(三.2)

判断线性方程组
$
\begin{cases}
x_1+2x_2+3x_3+2x_4=1\\
x_1+2x_2+4x_3+5x_4=2\\
2x_1+4x_2+ax_3+x_4=1\\
-x_1-2x_2-3x_3+7x_4=8\\
\end{cases}
$何时无解?何时有解?并在有无穷多组解时求出其通解。

%\begin{jie}
%由题可列增广矩阵并进行高斯消元:
%\begin{align*}
%(A|b)=
%\begin{bmatrix}
%1 & 2 & 3 & 2 &1 \\
%1 & 2 & 4 & 5 &2 \\
%2 & 4 & a & 1 &1 \\
%-1 & -2 & -3 & 7 &8 \\
%\end{bmatrix}&
%\xrightarrow{\substack{ r_2-r_1\\ r_3-2r_1 \\ r_4+r_1}}
%{
%\begin{bmatrix}
%1 & 2 & 3 & 2 &1 \\
%0 & 0 & 1 & 3 &1 \\
%0 & 0 & a-6 & -3 &1 \\
%0 & 0 & 0& 9 &9
%\end{bmatrix}
%}\xrightarrow{\substack{ r_3-(a-6)r_2 \\ r_4\div 9}}
%{
%\begin{bmatrix}
%1 & 2 & 3 & 2 &1 \\
%0 & 0 & 1 & 3 &1 \\
%0 & 0& 0 & 15-3a & 5-a \\
%0 & 0 & 0& 1 &1
%\end{bmatrix}
%}\\
%&
%\xrightarrow{\substack{ r_3-(15-3a)r_4}}
%{
%\begin{bmatrix}
%1 & 2 & 3 & 2 &1 \\
%0 & 0 & 1 & 3 &1 \\
%0 & 0& 0 & 0 & 2a-10 \\
%0 & 0 & 0& 1 &1
%\end{bmatrix}
%}\xrightarrow{\substack{ r_3\leftrightarrow r_4}}
%{
%\begin{bmatrix}
%1 & 2 & 3 & 2 &1 \\
%0 & 0 & 1 & 3 &1 \\
% 0 & 0 & 0& 1 &1\\
%0 & 0& 0 & 0 & 2a-10
%\end{bmatrix}
%}
%\end{align*}
%
%由阶梯型矩阵可以看出:
%
%(1)无解:
%当$2a-10\neq 0~~~\Rightarrow~~~\textcolor[rgb]{1.00,0.00,0.00}{a\neq 5}$时,存在矛盾方程,则该\textcolor[rgb]{1.00,0.00,0.00}{线性方程组无解}
%
%(2)当$2a-10=0~~~\Rightarrow~~~\textcolor[rgb]{1.00,0.00,0.00}{a=5}$时,该\textcolor[rgb]{1.00,0.00,0.00}{线性方程组有解},此时$x_2$为自由变量,所以有无穷多组解。
%
%把$a=5$代入上式继续化简:
%\begin{align*}
%\xrightarrow{\substack{ r_2-3r_3 \\ r_1-2r_3}}
%{
%\begin{bmatrix}
%1 & 2 & 3 & 0 &-1 \\
%0 & 0 & 1 & 0 &-2 \\
% 0 & 0 & 0& 1 &1\\
%0 & 0& 0 & 0 & 0
%\end{bmatrix}
%}\xrightarrow{\substack{ r_1-3r_2}}
%{
%\begin{bmatrix}
%1 & 2 & 0 & 0 &5 \\
%0 & 0 & 1 & 0 &-2 \\
% 0 & 0 & 0& 1 &1\\
%0 & 0& 0 & 0 & 0
%\end{bmatrix}
%}
%\end{align*}
%由最简阶梯型矩阵可以看出:$x_1=5-2x_2,x_3=-2,x_4=1$。
%
%令$x_2=k,k\in R$,则:
%\begin{equation*}
%x=
%\begin{pmatrix}
%5-2k \\ k \\ -2\\ 1
%\end{pmatrix}=\begin{pmatrix}
%5 \\ 0 \\ -2\\ 1
%\end{pmatrix}+\begin{pmatrix}
%-2 \\ 1\\ 0\\ 0
%\end{pmatrix},~~k\in R
%\end{equation*}
%\end{jie}
\end{document} 