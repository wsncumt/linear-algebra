\documentclass{article}
\usepackage[space,fancyhdr,fntef]{ctexcap}
\usepackage[namelimits,sumlimits,nointlimits]{amsmath}
\usepackage[bottom=25mm,top=25mm,left=25mm,right=15mm,centering]{geometry}
\usepackage{xcolor}
\usepackage{paralist}%列表宏包
\usepackage{arydshln}%234页,虚线表格宏包
\pagestyle{fancy} \fancyhf{}
\fancyhead[OL]{学院:\hfill 学号:\hfill 姓名:王松年}
\fancyfoot[C]{\color{gray}\thepage}
\renewcommand{\headrule}{\color{gray}\hrule width\headwidth}
%\usepackage{parskip}
%\usepackage{indentfirst}
\usepackage{graphicx}%插图宏包,参见手册318页
\begin{document}
\newenvironment{tips}{\kaishu\zihao{-6}\color{blue}{\noindent\rule[-3pt]{\textwidth}{0.5pt}\par \em \noindent {\zihao{-5} \textcolor[rgb]{1.00,0.00,0.00}{Tips}}}\par}{\\ \rule[3mm]{\textwidth}{0.5pt}\par}
\newcounter{num} \renewcommand{\thenum}{\arabic{num}.} \newcommand{\num}{\refstepcounter{num}\text{\thenum}}

\hphantom{~~}\hfill {\zihao{3}\heiti 第一次习题课} \hfill\hphantom{~~}

\hphantom{~~}\hfill {\zihao{4}\heiti 知识点} \hfill\hphantom{~~}

1.矩阵:矩阵的定义。

2.矩阵的初等行变换。
\begin{asparaenum}[(1)]
\item 倍加行:把其中一行的常数倍加到另一行;
\item 交换行:交换其中的两行;
\item 倍乘行:将其中1行乘非零常数。
\end{asparaenum}

3.阶梯型矩阵(不唯一)和行最简阶梯型矩阵(唯一)\\
对于任意一个矩阵,如果满足下面三条性质,则称其为\textcolor[rgb]{1.00,0.00,0.00}{阶梯型矩阵}(注:括号里是另一种表述方式,与括号前的表述等价)
\begin{asparaenum}[(1)]
\item 每一非零行在每一零行之上;(全为0的行在不全为0的行的下边)
\item 某一行的主元素所在的列位于前一行主元素所在列的右边;(上边行的首个非0元素位于下边行首个非0元素的左边列);(在所有非0行中,每一行的第一个非零元素所在的列号严格单调递增)
\item 不全为0的行的首个非0元素下面的元素全为0
\end{asparaenum}
行最简阶梯型矩阵:若一个阶梯形矩阵满足以下性质,则称为\textcolor[rgb]{1.00,0.00,0.00}{行最简阶梯型矩阵}。(\textcolor[rgb]{0.00,1.00,0.25}{注:如无特殊说明,最简阶梯型,行最简,最简行,最简矩阵通常都是同一概念})
\begin{asparaenum}[(1)]
\item 每一主元都为1;(每一行第一个非零元素都是1)
\item 每一主元素是该元素所在列的唯一非0元素。
\end{asparaenum}
注:只有是行阶梯形矩阵,才能判断是不是行最简阶梯型矩阵。

4.\textcolor[rgb]{1.00,0.00,0.00}{高斯消元法}(6大步骤):前4步用于产生阶梯型矩阵,第5,6步产生最简阶梯型矩阵。
\begin{asparaenum}[(1)]
\item 从左边不全为零的列开始,该列称为主元列,主元素位置位于该列的最顶端;
\item 若(1)中主元素位置上的元素包含下边两种情况:a)0;b)含有未知参数。\\
    则需通过矩阵的初等行变换(交换行)把非零元素换到第一行;例如:
    \begin{align*}
      &\text{主元素位置元素为0:}
      \begin{bmatrix}
        0 & 3 & 1 \\
        5 & 7 & 2 \\
        13 & 0 & 2
      \end{bmatrix}
      \text{交换$r_{1},r_{2}$,或交换$r_{1},r_{3}$}\\
      &\text{主元素位置元素含有未知参数(1):}
      \begin{bmatrix}
        k-1 & 3 & 1 \\
        5 & 7 & 2 \\
        13 & 0 & 2
      \end{bmatrix}
      \text{交换$r_{1},r_{2}$,或交换$r_{1},r_{3}$}\\
      &\text{主元素位置元素含有未知参数(2):}
      \begin{bmatrix}
        k^{2}-1 & 3 & 1 \\
        k-1 & 7 & 2
      \end{bmatrix}
      \text{交换$r_{1},r_{2}$}
    \end{align*}
\item 通过初等行变换(倍加行),将(2)中主元素下边的元素全部变为0.
\item 暂时不管包含主元位置的行以及它上边的各行,对剩余的子矩阵重复步骤(1)\~{}(3),直到剩余的矩阵为0或者没有子矩阵需要进行处理为止;
\item 若某个主元素不是1,则通过倍乘行变换将该元素变为1
\item  从最右边的主元素开始,把每个主元素所在列的上方的元素变为0(倍加行)。
\end{asparaenum}

5.\textcolor[rgb]{1.00,0.00,0.00}{存在唯一性定理}:

%\begin{tips}
%形如$a_1x_1+a_2x_2+\cdots+a_nx_n=0$的称为齐次线程方程(其中$a_i$为系数)。
%
%齐次的含义:上式可以写为$a_1x_1+a_2x_2+\cdots+a_nx_n=0\textcolor[rgb]{1.00,0.00,0.00}{x_{n+1}}$,该式中每一项$x$的指数都为1次,所以称为齐次。
%
%非齐次:$a_1x_1+a_2x_2+\cdots+a_nx_n=b,(b\neq 0)$,该式可以写为$a_1x_1+a_2x_2+\cdots+a_nx_n=b\textcolor[rgb]{1.00,0.00,0.00}{x_{n+1}^{0}}$,等式左边的参数都是1次,而右边是0次,所以为非齐次。
%
%线性的含义:此处可以理解为所有的参数间仅有加减(各个参数的相加减)及数乘(即某个参数乘上一个系数)运算。而含有类似下边的项就不是线性:$x_ix_j$(参数间乘法运算,而非数乘)、$x_{i}^{n},\sqrt{x_i}$(幂运算)、$\log x_i$(对数运算)以及其他不满足上述要求的运算。
%
%齐次线性方程组:只含有齐次线性方程。
%
%非齐次线性方程组:含有非齐次线性方程。
%\end{tips}

对于齐次线性方程组:一定存在解。
\begin{asparaenum}[(1)]
\item 只有0解:没有自由变量。
\item 有非零解:此时一定有无穷多组解。存在至少一个自由变量。
\end{asparaenum}

对于非齐次线性方程组:
\begin{asparaenum}[(1)]
\item 解不存在:将增广矩阵化为阶梯型矩阵后,最右列是主元列。即存在形如
    \begin{equation*}
      \begin{bmatrix}
        0& 0& \cdots&b
      \end{bmatrix}
    \end{equation*}
的行。
\item 唯一解:不存在自由变量;

\item 无穷解:至少有一个自由变量。
\end{asparaenum}

6.\textcolor[rgb]{1.00,0.00,0.00}{解线性方程组的步骤}
\begin{asparaenum}[(1)]
\item 将方程标准化;
\item 写出增广矩阵;
\item 执行高斯消元法1到4步得到阶梯型矩阵;
\item 判断解的存在性:若不存在,直接结束。若存在,执行高斯消元法第5,6步得到最简阶梯型矩阵。
\item  写出上一步求出的解。
\end{asparaenum}

%7.特殊矩阵:对角矩阵(\textcolor[rgb]{1.00,0.00,0.00}{单位矩阵}、数量矩阵)、三角矩阵(上三、下三)、对称矩阵、反对称矩阵、\textcolor[rgb]{1.00,0.00,0.00}{向量}(区分向量的维数和向量空间的维数两个概念)
%
%8.矩阵的运算:
%\begin{asparaenum}[(1)]
%\item 加减、数乘。运算律同实数运算律。注意行数列数相同矩阵才能加减。
%\item 乘法:AB=C。只有A的列数等于B的行数才能乘,C的大小为A行B列。注意矩阵的乘法没有交换律和消去律。即$A\cdot B\neq B \cdot A$,如果$A\cdot B=A\cdot C$,推不出$B=C$。
%\end{asparaenum}
7.排列数组合数。
\end{document}  