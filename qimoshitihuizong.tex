\documentclass{article}
\usepackage[space,fancyhdr,fntef]{ctexcap}
\usepackage[namelimits,sumlimits,nointlimits]{amsmath}
\usepackage[bottom=20mm,top=25mm,left=20mm,right=15mm,centering]{geometry}
\usepackage{xcolor}
\usepackage{arydshln}%234页,虚线表格宏包
\pagestyle{fancy} \fancyhf{}
\fancyhead[OL]{~~~班序号:\hfill 学院:\hfill 学号:\hfill 姓名:王松年~~~ \thepage}
%\usepackage{parskip}
%\usepackage{indentfirst}
\usepackage{graphicx}%插图宏包,参见手册318页
\usepackage{mathdots}%反对角省略号
\usepackage{extarrows}%等号上加文字
\usepackage{fancybox}%盒子宏包55页
\usepackage{cancel}
\begin{document}

\newcounter{num} \renewcommand{\thenum}{\arabic{num}.} \newcommand{\num}{\refstepcounter{num}\text{\thenum}}

\newenvironment{jie}{\kaishu\zihao{-5}\color{blue}{\noindent\em 解:}\par}{\hfill $\diamondsuit$\par}

\newenvironment{zhengming}{\kaishu\zihao{-5}\color{blue}{\noindent\em 证明:}\par}{\hfill $\diamondsuit$\par}

{\zihao{3}\heiti 写在前面:}

\num 答案里说的课本指的是由四川大学数学学院编写的《线性代数》(中国人民大学出版社)。

\num 仅供内部参考使用,请勿将此文档上传到百度文库以及其同类网站上。

\num 由个人整理,如果发现错误或有更好的解题方法请发送邮件至$2905816868@qq.com$.

\hphantom{~~}\hfill {\zihao{4}\heiti 习题课上总结的一些题} \hfill\hphantom{~~}

一.求矩阵的$m$次方。(这里之所以写$m$是为了与矩阵的阶数$n$区别开来,以防混淆)

1.数学归纳法。即依次求出$A,A^2,A^3,\cdots$。一般不用,但也有特殊情况(例如17-18年半期测试的一大题的第四小题,当然这个题有两种解法,任选其一;15-16年的期末试题三大题的第二小题的最后一问)。

2.对角矩阵的$m$次方。(显然不会单独出题)

设$A
=
\begin{bmatrix}
  a_{11} & & &\\
    & a_{22}&&\\
    &&\ddots&\\
    &&&a_{nn}
\end{bmatrix}
$,则
$
A^m=
\begin{bmatrix}
  a_{11}^{m} & & &\\
    & a_{22}^{m}&&\\
    &&\ddots&\\
    &&&a_{nn}^{m}
\end{bmatrix}
$.

3.如果一个n阶方阵A\textcolor[rgb]{1.00,0.00,0.00}{主对角线}上以及\textcolor[rgb]{1.00,0.00,0.00}{主对角线}的一侧元素全为0,那么必有$A^{k}=0$,其中$k\geq n$。即A是下边的几种形状之一:(一定要注意是针对主对角线,副对角线该结论不成立,第二次习题课讲这里时讲错了,后边已更正)
\begin{equation*}
\begin{bmatrix}
  0 &\textcolor[rgb]{0.00,0.50,1.00}{ a_{12}}&\textcolor[rgb]{0.00,0.50,1.00}{a_{13}}&\textcolor[rgb]{0.00,0.50,1.00}{\cdots}&\textcolor[rgb]{0.00,0.50,1.00}{a_{1n}} \\
  0& 0&\textcolor[rgb]{0.00,0.50,1.00}{a_{23}}&\textcolor[rgb]{0.00,0.50,1.00}{\cdots}&\textcolor[rgb]{0.00,0.50,1.00}{a_{2n}}\\
   0&0&0&\textcolor[rgb]{0.00,0.50,1.00}{\cdots}&\textcolor[rgb]{0.00,0.50,1.00}{a_{3n}}\\
  \vdots&\vdots&\vdots&\ddots&\textcolor[rgb]{0.00,0.50,1.00}{\vdots}\\
  0&0&0&\cdots&0
\end{bmatrix}
~~
\begin{bmatrix}
  0 & 0&0&\cdots&0 \\
  \textcolor[rgb]{0.00,0.50,1.00}{a_{21}}& 0&0&\cdots&0\\
   \textcolor[rgb]{0.00,0.50,1.00}{a_{31}}&\textcolor[rgb]{0.00,0.50,1.00}{a_{32}}&0&\cdots&0\\
  \textcolor[rgb]{0.00,0.50,1.00}{\vdots}&\textcolor[rgb]{0.00,0.50,1.00}{\vdots}&\textcolor[rgb]{0.00,0.50,1.00}{\vdots}&\ddots&\vdots\\
  \textcolor[rgb]{0.00,0.50,1.00}{a_{n1}}&\textcolor[rgb]{0.00,0.50,1.00}{a_{n2}}&\textcolor[rgb]{0.00,0.50,1.00}{a_{n3}}&\cdots&0
\end{bmatrix}
\end{equation*}

例如:若$
  A=
  \begin{bmatrix}
   0&0&0\\
   2&0&0\\
   1&3&0
  \end{bmatrix}
  $,则$A^{2}=\underline{~~~~~  \begin{bmatrix}
   0&0&0\\
   0&0&0\\
   6&0&0
  \end{bmatrix}~~~~},A^{3}=\underline{~~~~~~0~~~~~}$.

\begin{jie}
由矩阵的乘法:
\begin{equation*}
 A^{2}= \begin{bmatrix}
   0&0&0\\
   2&0&0\\
   1&3&0
  \end{bmatrix}  \begin{bmatrix}
   0&0&0\\
   2&0&0\\
   1&3&0
  \end{bmatrix}=
    \begin{bmatrix}
   0&0&0\\
   0&0&0\\
   6&0&0
  \end{bmatrix}
\end{equation*}
\begin{equation*}
 A^{3}=A^{2}A=    \begin{bmatrix}
   0&0&0\\
   0&0&0\\
   6&0&0
  \end{bmatrix}  \begin{bmatrix}
   0&0&0\\
   2&0&0\\
   1&3&0
  \end{bmatrix}=
    \begin{bmatrix}
   0&0&0\\
   0&0&0\\
   0&0&0
  \end{bmatrix}
\end{equation*}
\end{jie}

4.一般来说,上边的2,3不会单独出题,因为太过简单,都是组合起来出题。

\shadowbox{\parbox{170mm}{
\textcolor[rgb]{1.00,0.00,0.00}{二项式定理}:$(a+b)^{n}=\sum\limits_{i=0}^{n}C_{n}^{i}a^i\cdot b^{n-i}$,式中:$C_{n}^i$为\textcolor[rgb]{1.00,0.00,0.00}{组合数},$C_{n}^i=\dfrac{n!}{i!(n-i)!}$。
}}

例如:若$
  A=
  \begin{bmatrix}
   1 & 2 & 3\\
   0& 1&4\\
   0& 0&1
  \end{bmatrix}
  $,则$A^{m}=\underline{~~~~~~~~~~~~~}.$

\begin{jie}

由题得:
$
A=
\begin{bmatrix}
  1&2&3\\
  0&1&4\\
  0&0&1\\
\end{bmatrix}=
\begin{bmatrix}
1&0&0\\
  0&1&0\\
  0&0&1\\
\end{bmatrix}+
\begin{bmatrix}
  0&2&3\\
  0&0&4\\
  0&0&0\\
\end{bmatrix}=I+B
$\\

由3的结论可知:$B^{k}=0,k\geq3$。计算得:$B^{2}=\begin{bmatrix}
                                        0 & 0&8 \\
                                       0 & 0&0 \\  0 & 0&0 \\
                                      \end{bmatrix}$所以
\begin{align*}
A^{m}&=(I+B)^{m}=C_{m}^{0}I^{m}B^{0}+C_{m}^{1}I^{m-1}B^{1}+C_{m}^{2}I^{m-2}B^{2}+C_{m}^{3}I^{m-3}\textcolor[rgb]{1.00,0.00,0.00}{B^{3}}+\cdots+C_{m}^{0}I^{0}\textcolor[rgb]{1.00,0.00,0.00}{B^{m}}\\
&=I^{m}B^{0}+mI^{m-1}B^{1}+\frac{m(m-1)}{2}I^{m-2}B^{2}=I+mB+\frac{m(m-1)}{2}B^{2}\\
&=\begin{bmatrix}
1&0&0\\
  0&1&0\\
  0&0&1\\
\end{bmatrix}+m\begin{bmatrix}
  0&2&3\\
  0&0&4\\
  0&0&0\\
\end{bmatrix}+\frac{m(m-1)}{2}
\begin{bmatrix}
                                        0 & 0&8 \\
                                       0 & 0&0 \\  0 & 0&0 \\
                                      \end{bmatrix}\\
&=
\begin{bmatrix}
  1&2m&4m^{2}-m\\
  0&1&4m\\
  0&0&1\\
\end{bmatrix}
\end{align*}
\end{jie}

5.如果一个$n$阶方阵$A$的秩$r(A)=1$,那么$A$一定可以写成一个列向量与一个行向量的乘积,即(我们以3阶的方阵为例,最后把结果推广到$n$阶):

设$3$阶方阵$A$的秩$r(A)=1$,设$\alpha=\begin{bmatrix}
          x_{1} & x_{2} & x_{3}
        \end{bmatrix}^{T},\beta=\begin{bmatrix}
          y_{1} & y_{2} & y_{3}
        \end{bmatrix}^{T}$,则

        $A=\alpha\beta^T=\begin{bmatrix}
          x_{1}y_{1} & x_{1}y_{2} & x_{1}y_{3}\\
          x_{2}y_{1} & x_{2}y_{2} & x_{2}y_{3}\\
          x_{3}y_{1} & x_{3}y_{2} & x_{3}y_{3}
        \end{bmatrix}$(至于这里的$x$和$y$的具体值,我们并不关心。)

        $l=\alpha^T\beta=\beta^T\alpha=x_{1}y_{1}+x_{2}y_{2}+x_{3}y_{3}$.可以看出$l$是矩阵$A$的对角线元素之和(又称$A$的迹)。

        $A^2=(\alpha\beta^T)^2=\alpha(\beta^T\alpha)\beta^T=l\alpha\beta^T=lA,A^3=AA^2=AlA=lA^2=llA=l^2A\cdots, \Rightarrow A^m=l^{m-1}A$。\\


例如:(对2,3,4,5综合运用)

设$A=
  \begin{bmatrix}
 3&1&0&0\\
 0&3&0&0\\
 0&0&3&9\\
 0&0&1&3
  \end{bmatrix}
  $,则$A^{m}=\underline{~~~~~~~~~~~~~}.$

\begin{jie}
经观察,我们可将矩阵按下列方式进行分块:
\begin{equation*}
A=
\begin{bmatrix}
3&1&0&0\\
 0&3&0&0\\
 0&0&3&9\\
 0&0&1&3
\end{bmatrix}
=
\begin{bmatrix}
  A_{11} & \mathbf{0} \\
  \mathbf{0} & A_{22}
\end{bmatrix}~~~
A_{11}=
\begin{bmatrix}
  3 & 1 \\
  0 & 3
\end{bmatrix}
~A_{22}=
\begin{bmatrix}
  3 & 9 \\
  1 & 3
\end{bmatrix}
\end{equation*}
$A$分块后是对角矩阵,所以:$A^{m}=\begin{bmatrix}
  A_{11}^{m} & \mathbf{0} \\
  \mathbf{0} & A_{22}^{m}
\end{bmatrix}$(注:2的结论)\\
对于$A_{11}$:
\begin{align*}A_{11}&=
\begin{bmatrix}
  3 & 1 \\
  0 & 3
\end{bmatrix}=
\begin{bmatrix}
  3 & 0 \\
  0 & 3
\end{bmatrix}+
\begin{bmatrix}
  0 & 1 \\
  0 & 0
\end{bmatrix}=3I+B\\
&\text{和4一样了,此时$B^{k}=0,k\geq2$}\\
A^{m}_{11}&=(3I+B)^{m}=C_{m}^{0}B^{0}(3I)^{m}+C_{m}^{1}B^{1}(3I)^{m-1}+C_{m}^{2}\textcolor[rgb]{1.00,0.00,0.00}{B^{2}}(3I)^{m-2}+\cdots+C_{m}^{m}\textcolor[rgb]{1.00,0.00,0.00}{B^{m}}(3I)^{0}\\
&=3^{m}I+3^{m-1}mB\\
&=\begin{bmatrix}
  3^{m} & m3^{m-1} \\
  0 & 3^{m}
\end{bmatrix}
\end{align*}


对于$A_{22}$,经过高斯消元法变换($r_{1}-3r_{2}$)后:$
\begin{bmatrix}
  3 & 9 \\
  0 & 0
\end{bmatrix}
$,可以看出$r(A_{22})=1$,符合本条的描述,所以
\begin{align*}
 l&=3+3=6\\
 A_{22}^{m}&=l^{m-1}A=6^{m-1}
 \begin{bmatrix}
   3 & 9 \\
   1 & 3
 \end{bmatrix}
\end{align*}
所以:
\begin{equation*}
  A^{m}=
\begin{bmatrix}
  A_{11}^{m} & \mathbf{0} \\
  \mathbf{0} & A_{22}^{m}
\end{bmatrix}=
\begin{bmatrix}
  3^{m} & m\cdot3^{m-1}&0&0 \\
  0 & 3^{m}&0&0\\
  0&0&3\cdot6^{m-1}&9\cdot6^{m-1}\\
  0&0&6^{m-1}&3\cdot6^{m-1}
\end{bmatrix}
\end{equation*}
\end{jie}

6.用相似矩阵的性质来做,即若$A\sim B$,则$A^m\sim B^m$(例如17-18年半期测试的一大题的第四小题)。

但通常有一些矩阵隐含了此属性,也可以用此方法来做,要注意辨别。(例如:\textcolor[rgb]{1.00,0.00,0.00}{课本的140页第9题,答案在224页})\\

二、特殊矩阵的特征值求法

1.对角阵,上下三角阵的行列式均为对角线的元素。(由这些特殊行列式的算法很容易看出来)

2.设$A=[a_{ij}]$是三阶矩阵,则(下式不做推导,感兴趣的可以自己算一下)
\begin{equation*}|\lambda E-A|=
  \begin{bmatrix}
    \lambda-a_{11} & -a_{12} &-a_{13} \\
-a_{21} & \lambda-a_{22} &-a_{23} \\
-a_{31} & -a_{32} &\lambda-a_{33}
  \end{bmatrix}=\lambda^{3}-\sum a_{ii}\lambda^2+S_{2}\lambda-|A|
\end{equation*}
式中:$S_{2}=
\begin{vmatrix}
  a_{11} & a_{12} \\
  a_{21} & a_{22}
\end{vmatrix}+\begin{vmatrix}
  a_{11} & a_{13} \\
  a_{31} & a_{33}
\end{vmatrix}+\begin{vmatrix}
  a_{22} & a_{23} \\
  a_{32} & a_{33}
\end{vmatrix}
$。

若$r(A)=1$(再复习一下一的第五点),则$|A|=0,S_{2}=0$,代入到上式有
\begin{equation*}
|\lambda E-A|=\lambda^{3}-\sum a_{ii}\lambda^2=\lambda^2\left(\lambda-\sum a_{ii}\right)
\end{equation*}

做推广,对于$n$阶矩阵$A$,若$r(A)=1$,则$|\lambda E-A|=\lambda^{n-1}\left(\lambda-\sum a_{ii}\right)$

例如:

已知$a\neq 0$,求矩阵
\begin{equation*}
  \begin{bmatrix}
    1 & a & a& a\\
    a & 1& a& a\\
    a& a& 1& a\\
    a& a& a& 1
  \end{bmatrix}
\end{equation*}
的特征值、特征向量。

\begin{jie}
方法一:(直接计算)

由特征多项式:
\begin{equation*}
  \begin{vmatrix}
\lambda E-A
  \end{vmatrix}
  =\begin{vmatrix}
     \lambda-1 & -a& -a& -a \\
     -a& \lambda-1& -a& -a\\
     -a& -a& \lambda-1& -a\\
     -a& -a& -a& \lambda-1
   \end{vmatrix}=\left[\lambda-(3a+1)\right]\left(\lambda+a-1\right)^{3}
\end{equation*}
得$A$的特征值是$3a+1,1-a$。

当$\lambda=3a+1$时,由$[(3a+1)E-A]=0$,即
\begin{align*}
\begin{bmatrix}
3a & -a& -a& -a \\
-a& 3a& -a& -a\\
-a& -a& 3a& -a\\
-a& -a& -a& 3a
\end{bmatrix}\rightarrow
\begin{bmatrix}
3 & -1& -1& -1 \\
-1& 3& -1& -1\\
-1& -1& 3& -1\\
-1& -1& -1& 3
\end{bmatrix}\rightarrow
\begin{bmatrix}
1 & -3& 1& 1 \\
1& 1& -3& 1\\
1& 1& 1& -3\\
0&0& 0& 0
\end{bmatrix}\rightarrow
\begin{bmatrix}
1 & 0& 0& -1 \\
0& 1& 0& -1\\
0& 0& 1& -1\\
0&0& 0& 0
\end{bmatrix}
\end{align*}
可得基础解系为$\alpha_{1}=(1,1,1,1)^T$,所以$\lambda=3a+1$的特征向量为$k_{1}\alpha_1,(k_1\neq 0)$。

当$\lambda=1-a$时,由$[(1-a)E-A]=0$,即
\begin{equation*}
  \begin{bmatrix}
-a & -a & -a& -a\\
-a & -a & -a& -a\\
-a & -a & -a& -a\\
-a & -a & -a& -a
  \end{bmatrix}\rightarrow
  \begin{bmatrix}
    1 & 1 & 1& 1\\
    0 & 0& 0& 0\\
    0 & 0& 0& 0\\
    0 & 0& 0& 0
  \end{bmatrix}
\end{equation*}
得基础解系$\alpha_2=(-1,1,0,0)^T,\alpha_3=(-1,0,1,0)^T\alpha_4=(-1,0,0,1)^T$,所以$\lambda=1-a$的特征向量为$k_2\alpha_2+k_3\alpha_3+k_4\alpha_4$,式中$k_2,k_3,k_4$是不全为0的任意常数。

方法二:(转换法)\textcolor[rgb]{1.00,0.00,0.00}{注:请注意观察下述这种方法适用的特点:其中一个矩阵必须特征值全相等才能用前边的矩阵快速求解}

由题得:
\begin{equation*}A=
  \begin{bmatrix}
    a & a& a& a \\
    a & a& a& a \\
    a & a& a& a \\
    a & a& a& a
  \end{bmatrix}+
   \begin{bmatrix}
    1-a & 0& 0& 0 \\
    0 & 1-a& 0& 0 \\
    0 & 0& 1-a& 0 \\
    0 & 0& 0& 1-a
  \end{bmatrix}=B+(1-a)E
\end{equation*}
由于$r(B)=1$,所以有
\begin{equation*}
  |\lambda E -B|=\lambda^{4-1}\left(\lambda-\sum\limits_{i=1}^{4}a_{ii}\right)=
  \lambda^{3}\left(\lambda-4a\right)
\end{equation*}
所以矩阵$B$的特征值为$0,0,0,4a$,所以由特征值的性质,$A$的特征值为$3a+1,1-a,1-a,1-a$。

下边同方法一。
\end{jie}

$n$阶矩阵
\begin{equation*}
A=
\begin{bmatrix}
  a & 1 & 1 & \cdots & 1\\
 1 & a & 1&\cdots & 1\\
 1 & 1 & a & \cdots &1\\
 \vdots&\vdots&\vdots&\ddots&\vdots\\
 1&1&1&\cdots&a
\end{bmatrix}
\end{equation*}
则$r(A)=$\underline{\hphantom{~~~~~~~~~~~~~}}.

\begin{jie}
由上题可快速写出$A$的特征值为$n+a-1,a-1,a-1,\cdots,a-1$.
因为$A$是实对称矩阵,所以$A\sim \Lambda$,且$\Lambda$由$A$的特征值所构成,相似矩阵具有相同的秩,所以$r(\Lambda)=r(A)$,所以
\begin{equation*}
\Lambda=
\begin{bmatrix}
  n+a-1 & && \\
   & a-1 &&\\
   &&\ddots&\\
   &&&a-1
\end{bmatrix}
\end{equation*}
这里$n$是$A$的阶数,所以不会等于$0$。所以
\begin{equation*}r(A)=
  \begin{cases}
  n,&\text{若}a\neq 1\text{且}a\neq 1-n,\\
  n-1,&\text{若}a=1-n,\\
  1,&\text{若}a=1.
  \end{cases}
\end{equation*}
\end{jie}

设$\alpha$为$n$维单位列向量,$E$为$n$阶单位矩阵,则

\hphantom{~}A.$E-\alpha\alpha^T$不可逆 \hfill B.$E+\alpha\alpha^T$不可逆 \hfill C.$E+2\alpha\alpha^T$不可逆 \hfill D.$E-2\alpha\alpha^T$不可逆
\hphantom{~}

\begin{jie}
注意:单位向量指的是向量的模(长度)为1,要与$[1,1,1]$区分开来。

$\alpha\alpha^T\alpha=\alpha(\alpha^T\alpha)=1\alpha$,所以$\alpha\alpha^T$有一个特征值1.

$\alpha$为$n$维单位列向量,所以$r(\alpha\alpha^T)=1$,所以由第一题的结论,$\alpha\alpha^T$的特征值为$1,0,0,\cdots,0$。

$E$为$n$阶单位矩阵,所以$E$也为实对称矩阵(特征值为$1$),实对称矩阵相加减依然为实对称矩阵,所以上述选项中每一项均为实对称矩阵。

又由矩阵可逆则行列式一定不为0(不可逆则行列式一定为0,充要条件),矩阵的行列式等于特征值的乘积。

$A$.$c$的特征值为$1-1,1-0,1-0,\cdots,1-0$即$0,1,1,\cdots,1$,所以$|E-\alpha\alpha^T|=0\times 1\cdots 1=0$,即不可逆。

同理可以看出其他选项的行列式均不为0,即可逆。
\end{jie}

2.抽象矩阵特征值和特征向量的求法

设$A$是三阶矩阵,且矩阵$A$的各行元素之和均为5,则矩阵$A$必有特征向量\underline{\hphantom{~~~~~~~~~~~~~}}.

\begin{jie}
由题得:
\begin{equation*}
\begin{cases}
a_{11}+a_{12}+a_{13}=5\\
a_{21}+a_{22}+a_{23}=5\\
a_{31}+a_{32}+a_{33}=5
\end{cases}~~~~~\Rightarrow~~~
\begin{bmatrix}
a_{11}&a_{12}&a_{13} \\
a_{21}&a_{22}&a_{23} \\
a_{31}&a_{32}&a_{33}
\end{bmatrix}
\begin{bmatrix}
1\\ 1\\1
\end{bmatrix}=
\begin{bmatrix}
5\\ 5\\5
\end{bmatrix}~~~\Rightarrow~~~A\begin{bmatrix}
1\\ 1\\1
\end{bmatrix}=5\begin{bmatrix}
1\\ 1\\1
\end{bmatrix}
\end{equation*}
所以矩阵$A$必有特征值5且必有特征向量$k[1,1,1]^T,(k\neq 0)$。
\end{jie}

已知$A$是3阶矩阵,如果非齐次线性方程组$Ax=b$有通解$5b+k_1\eta_1+k_2\eta_2$,其中$\eta_1,\eta_2$是$Ax=0$的基础解系,求$A$的特征值和特征向量。

\begin{jie}
\textcolor[rgb]{1.00,0.00,0.00}{非齐次线性方程组$Ax=b$的通解为$Ax=b$的特解加上$Ax=0$的通解。}

由解得结构可知$5b$是方程组$Ax=b$的一个解,即$A(5b)=b$,所以$Ab=\dfrac{1}{5}b$。即$\dfrac{1}{5}$是$A$的特征值,$k_1b,(k_1\neq 0)$是相应的特征向量。

$\eta_1,\eta_2$是$Ax=0$的基础解系,所以必有$A\eta_1=0=0\eta_1,A\eta_{2}=0=0\eta_2$,所以$\eta_1,\eta_2$是$A$关于$\lambda=0$的线性无关的特征向量,所以特征值$0$对应的特征向量为$k_2\eta_1+k_3\eta_2,(k_2,k_3\text{不全为}0)$。

综上所述,$A$的特征值为$\dfrac{1}{5},0,0$,对应的特征向量分别是$k_1b,(k_1\neq 0)$,$k_2\eta_1+k_3\eta_2,(k_2,k_3\text{不全为}0)$
\end{jie}
\newpage
\hphantom{~~}\hfill {\zihao{4}\heiti 2014-2015年第一学期} \hfill\hphantom{~~}

一、填空题

1.若已知行列式
$
\begin{vmatrix}
  1 & 3 & a \\
  5 & -1 &1\\
  3 & 2&1
\end{vmatrix}
$的代数余子式$A_{21}=1$,则$a=$\underline{\hphantom{~~~~~~~~~~}}。

\begin{jie}
$A_{21}=(-1)^{2+1}\times(3\times 1-2\times a)=1$,解得$a=2$。
\end{jie}

2.设
$
A=
\begin{bmatrix}
  1 & 2 & -2 \\
  2 & 5 &0\\
  3 & t&4
\end{bmatrix}
$,$B$为3阶非零矩阵且$AB=0$,则$t=$\underline{\hphantom{~~~~~~~~~~}}。

\begin{jie}
$B$为3阶非零矩阵且$AB=0$即$B$的非零列向量为$Ax=0$的解,即$Ax=0$有非零解,即$|A|=0$,把$|A|$按第三列展开。
\begin{equation*}
  |A|=
  \begin{vmatrix}
  1 & 2 & -2 \\
  2 & 5 &0\\
  3 & t&4
  \end{vmatrix}=-2\times(-1)^{1+3}
  \begin{vmatrix}
  2 & 5 \\
  3 & t
  \end{vmatrix}+4
  \begin{vmatrix}
  1 & 2  \\
  2 & 5
  \end{vmatrix}=-2(2t-15)+4=0~~~\Rightarrow~~~t=\frac{17}{2}
\end{equation*}
\end{jie}

3.设3阶方阵$A=(\alpha_{1},\alpha_{2},\alpha_{3})$的行列式$|A|=3$,矩阵$B=(\alpha_{2},2\alpha_{3},-\alpha_{1})$,则行列式$|A-B|=$\underline{\hphantom{~~~~~~~~~~}}。

\begin{jie}
对$A$的第三列乘2得:$|\alpha_{1}~\alpha_{2}~2\alpha_{3}|=2|A|$,对该表达式第一列乘负一:$|-\alpha_{1}~\alpha_{2}~2\alpha_{3}|=-2|A|$,交换一二两列,$|\alpha_{2}~\alpha_ {1}~2\alpha_{3}|=2|A|$,交换二三两列,$|\alpha_ {2}~2\alpha_{3}~\alpha_ {1}|=-2|A|$,所以$|B|=-2|A|$.

\begin{align*}
|A-B|&=|\alpha_{1}-\alpha_{2}~~~\alpha_{2}-2\alpha_{3}~~~\alpha_{3}+\alpha_{1}|=|\alpha_{1}~~~\alpha_{2}-2\alpha_{3}~~~\alpha_{3}+\alpha_{1}|+|-\alpha_{2}~~~\alpha_{2}-2\alpha_{3}~~~\alpha_{3}+\alpha_{1}|\\
&=\left|\alpha_{1}~~~\alpha_{2}-2\alpha_{3}~~~\alpha_{3}\right|+|-\alpha_{2}~~-2\alpha_{3}~~~\alpha_{3}+\alpha_{1}|\\
&=\left|\alpha_{1}~~~\alpha_{2}~~~\alpha_{3}\right|+\left|\alpha_{1}~~~-2\alpha_{3}~~~\alpha_{3}\right|+|-\alpha_{2}~~-2\alpha_{3}~~~\alpha_{3}|+|-\alpha_{2}~~-2\alpha_{3}~~~\alpha_{1}|\\
&=|A|+0+0-|B|=3|A|=9
\end{align*}
\end{jie}

4.已知3阶矩阵$A$的特征值为$-1,3,2$,$A^{*}$是$A$的伴随矩阵,则矩阵$A^{3}+2A^{*}$主对角线元素之和为\underline{\hphantom{~~~~~~~~~~}}。

\begin{jie}
由题得:$|A|=\prod\limits_{i=1}^3\lambda_i=-1\times3\times2=-6$。所以$A^*$的特征值为:$\dfrac{|A|}{\lambda_i}$。由特征值的性质:

$A^ {3}+2A^{*}$的特征值为$\lambda_i^3+2\dfrac{|A|}{\lambda_i}$.所以
$A^ {3}+2A^{*}$主对角线元素之和为

\begin{equation*}
trace(A^ {3}+2A^{*})=\sum_{i=1}^{3}\left(\lambda_i^3+2\dfrac{|A|}{\lambda_i}\right)=36
\end{equation*}
\end{jie}

5.已知实二次型$f(x_{1},x_{2},x_{3})=a(x_{1}^{2}+x_{2}^{2}+x_{3}^{2})+4x_{1}x_{2}+4x_{1}x_{3}+4x_{2}x_{2}$经正交变换$x=py$可化为标准形:$f=6y^{2}$,则$a=$\underline{\hphantom{~~~~~~~~~~}}。

\begin{jie}
任意二次型$x^TAx$经过正交变换化为标准型时,标准型中平方项的系数即为二次型矩阵$A$的特征值,即$6,0,0$是$A$的特征值,而$A$的对角线元素是$a,a,a$,由特征值性质$trace(A)=a+a+a=\sum_{i=1}^{3}\lambda=6$,所以$a=2$。
\end{jie}

6.设$(1,1,1)^{T}$是矩阵$
\begin{bmatrix}
  1 & 2 & 3 \\
  0 & a & 2\\
  2 & 2 & b
\end{bmatrix}
$的一个特征向量,则$a-b=$\underline{\hphantom{~~~~~~~~~~}}。

\begin{jie}
由特征向量的定义有
\begin{equation*}
A
\begin{bmatrix}
1\\ 1\\ 1
\end{bmatrix}=\lambda
\begin{bmatrix}
1\\ 1\\ 1
\end{bmatrix}~~~\Rightarrow~~~
\begin{bmatrix}
6\\ a+2\\ b+4
\end{bmatrix}=\begin{bmatrix}
\lambda\\ \lambda\\ \lambda
\end{bmatrix}~~~\Rightarrow~~~
\begin{cases}
\lambda=6\\
a=4\\
b=2
\end{cases}~~~\Rightarrow~~a-b=2
\end{equation*}
\end{jie}

二.设多项式$
f(x)=
\begin{vmatrix}
  2x & 3 & 1 & 2\\
  x & x & -2 & 1\\
  2 & 1 & x & 4\\
  x & 2 & 1 & 4x
\end{vmatrix}
$,分别求该多项式的三次项、常数项。

\begin{jie}
$Matlab$算出的结果为:$f(x)=8x^{4}-14x^3+11x^2-53x+14$(作为参考)

两种方法:

\textcolor[rgb]{1.00,0.00,0.00}{方法一}:逆序数定义行列式法:

取列为自然排列。分析得:行数按$2134$和$4231$排列时,对应的项为$x^3$。即
\begin{equation*}
(-1)^{\tau(2134)}a_{21}a_{12}a_{33}a_{44}+(-1)^{\tau(4231)}a_{41}a_{22}a_{33}a_{14}=(-12-2)x^{3}=-14x^3
\end{equation*}

同理,取列为自然排列。分析得行数按:$3142$、$3412$和$3421$排列时为常数项,即
\begin{equation*}
(-1)^{\tau(3142)}a_{31}a_{12}a_{43}a_{24}+(-1)^{\tau(3412)}a_{31}a_{42}a_{13}a_{24}+(-1)^{\tau(3421)}a_{31}a_{42}a_{23}a_{14}=-6+4+16=14
\end{equation*}

\textcolor[rgb]{1.00,0.00,0.00}{方法二}:拉普拉斯展开式定义行列式

按第一列展开(当然也可以按任意一行或一列展开):
\begin{equation*}
\text{原式}=2x
\begin{vmatrix}
  x & -2 & 1 \\
  1 & x & 4\\
  2 & 1 & 4x
\end{vmatrix}
-x
\begin{vmatrix}
  3 & 1 & 2 \\
  1 & x & 4\\
  2 & 1 & 4x
\end{vmatrix}
+2
\begin{vmatrix}
3 & 1 & 2 \\
  x & -2 & 1 \\
  2 & 1 & 4x
\end{vmatrix}
-x
\begin{vmatrix}
3 & 1 & 2 \\
  x & -2 & 1 \\
  1 & x & 4
\end{vmatrix}
\end{equation*}
分别从上述四项中找三次项、常数项。

第一项:按第一行展开
\begin{equation*}
2x
\begin{vmatrix}
  x & -2 & 1 \\
  1 & x & 4\\
  2 & 1 & 4x
\end{vmatrix}=2x[x(4x^2-4)+2(4x-8)+(1-2x)]
\end{equation*}
显然没有三次项也没有常数项,就没必要接着算了。

第二项:按第一行展开
\begin{equation*}
-x
\begin{vmatrix}
  3 & 1 & 2 \\
  1 & x & 4\\
  2 & 1 & 4x
\end{vmatrix}
\end{equation*}
可以看出只有第一行第一列展开对应的那项中含有三次项,而第一行第二列、第一行第三列的既不构成三次项也不构成常数项,所以没必要计算。
\begin{equation*}
\text{原式}=-x\cdot 3\begin{vmatrix}
 x & 4\\
 1 & 4x
\end{vmatrix}=-3x\cdot x\cdot 4x = -12x^3
\end{equation*}

第三项:该行列式中总共只有两个$x$,所以一定不可能构成三次项。按第一行展开:
\begin{equation*}
  2
\begin{vmatrix}
3 & 1 & 2 \\
  x & -2 & 1 \\
  2 & 1 & 4x
\end{vmatrix}=2[3(\underline{~~~~}-1)-(\underline{~~~~}-2)+2(\underline{~~~~}+4)]=2(-3+2+8)=14
\end{equation*}
(上式中画下划线的部分是含有$x$项的,不构成常数项,没必要计算)

第四项:有个系数$x$,所以不可能构成常数项,按第一行展开,只有第一行第三列的位置处构成三次项
\begin{equation*}
-x\cdot 2\begin{vmatrix}
           x & -2 \\
           1 & x
         \end{vmatrix}=-2x^3
\end{equation*}

所以综上所述,三次项为$-12x^3-2x^3=-14x^3$,常数项为14
\end{jie}

三.设$A$的伴随矩阵
$
A^{*}=
\begin{bmatrix}
  2 & 0 & 0 & 0\\
  0 & 2 & 0 & 0\\
  1 & 0 & 2 & 0\\
  0 & -3 & 0 & 8
\end{bmatrix}
$,且$ABA^{-1}=BA^{-1}+3I$,求$B$。

\begin{jie}
由题得:$|A^*|=2\times2\times2\times8=64$
\begin{gather*}
ABA^{-1}=BA^{-1}+3I~~\Rightarrow~~AB=B+3A~~\Rightarrow~~A^*AB=A^*B+3A^*A\\
A^*A=|A|A^{-1}A=|A|I~~~|A^*|=||A|A^{-1}|=|A|^n|A|^{-1}=|A|^{n-1}=|A|^{4-1}=64~~~|A|=4
\end{gather*}
所以$4B=A^*B+3\times4~~~\Rightarrow~~~B=12(4-A^*)^{-1}$.

求逆的过程略。

最后的结果为:
\begin{equation*}
  B=\begin{bmatrix}
  6 & 0 & 0 & 0\\
  0 & 6 & 0 & 0\\
  3 & 0 & 6 & 0\\
  0 & 4.5 & 0 & -3
\end{bmatrix}
\end{equation*}
\end{jie}

四.$\lambda$为何值时,方程组$
\begin{cases}
 2x_{1}+\lambda x_{2}-x_{3}=1\\
 \lambda x_{1}-x_{2}+x_{3}=2\\
 4x_{1}+5 x_{2}-5x_{3}=-1
\end{cases}
$有无穷多组解?并在有无穷多解时,写出方程组的通解。

\begin{jie}
记$A=
\begin{bmatrix}
 2&\lambda &-1\\
 \lambda &-1&1\\
 4& 5 &-5
\end{bmatrix}
,B=\begin{bmatrix}
     1 \\ 2\\ -1
   \end{bmatrix}$,
   \begin{equation*}
   |A|=
   \begin{vmatrix}
    2&\lambda &-1\\
 \lambda &-1&1\\
 4& 5 &-5
   \end{vmatrix}
   \xlongequal{c_{3}+c_{2}}
  \begin{vmatrix}
    2&\lambda &\lambda-1\\
 \lambda &-1&0\\
 4& 5 &0
   \end{vmatrix}  =(\lambda-1)(5\lambda+4)
   \end{equation*}
   可以看出$\lambda\neq1$且$\lambda\neq-\dfrac{4}{5}$时即$|A|\neq0$时,方程有唯一解。

   $\lambda=1$时:
   \begin{align*}
 [A|B]=
 \left[
 \begin{array}{c:c}
\begin{matrix}
2 & 1 & -1 \\
  1 & -1 & 1 \\
  4 & 5 & -5
\end{matrix}&
\begin{matrix}
1  \\
 2\\
-1
\end{matrix}
\end{array}
\right]
\xrightarrow{\substack{r_{2}-\frac{1}{2}r_{1}\\ r_{3}-2r_{1}}}
{
 \left[
 \begin{array}{c:c}
\begin{matrix}
2 & 1 & -1 \\
 0 & -\frac{3}{2} & \frac{3}{2} \\
  0 & 3 & -3
\end{matrix}&
\begin{matrix}
1  \\
 \frac{3}{2}\\
-3
\end{matrix}
\end{array}
\right]
}
\xrightarrow{r_{3}+2r_{2}}
{
 \left[
 \begin{array}{c:c}
\begin{matrix}
2 & 1 & -1 \\
 0 & -\frac{3}{2} & \frac{3}{2} \\
  0 & 0 & 0
\end{matrix}&
\begin{matrix}
1  \\
 \frac{3}{2}\\
0
\end{matrix}
\end{array}
\right]
}
\xrightarrow{\text{中间略}}
{
 \left[
 \begin{array}{c:c}
\begin{matrix}
1 & 0 & 0 \\
 0 & 1 & -1 \\
  0 & 0 & 0
\end{matrix}&
\begin{matrix}
1  \\
-1\\
0
\end{matrix}
\end{array}
\right]
}
   \end{align*}
所以$\lambda=1$时有无穷多解,$x_{1}=1,x_{2}=x_{3}-1$.

所以通解为$x=[1,-1,0]^T+k[0,1,1]^T,(k\in R)$

$\lambda=-\dfrac{4}{5}$时,$r(A)\neq r(A,B)$,此时无解。
\end{jie}

五.设$
\alpha_{1}=
\begin{bmatrix}
1\\ 1 \\ 2\\ 3
\end{bmatrix},
\alpha_{2}=
\begin{bmatrix}
1\\ -1 \\ 1\\ 1
\end{bmatrix},
\alpha_{3}=
\begin{bmatrix}
1\\ 3 \\ 3\\ 5
\end{bmatrix},
\alpha_{4}=
\begin{bmatrix}
4\\ -2 \\ 5\\ 6
\end{bmatrix}
$.

(1)求向量组$\alpha_{1},\alpha_{2},\alpha_{3},\alpha_{4}$的秩与一个最大线性无关组;

(2)将其余向量用极大线性无关组线性表示。

\begin{jie}
\begin{align*}
(\alpha_{1},\alpha_{2},\alpha_{3},\alpha_{4})&=
\begin{bmatrix}
  1 & 1 & 1 & 4 \\
  1 & -1 & 3 & -2\\
  2 & 1 & 3 & 5\\
  3 & 1 & 5 & 6
\end{bmatrix}
\xrightarrow{\substack{r_{2}-r_{1}\\ r_{3}-2r_{1}\\ r_4-3r
-1}}
{
\begin{bmatrix}
  1 & 1 & 1 & 4 \\
  0 & -2 & 2 & -6\\
  0 & -1 & 1 & -3\\
    0 & -2 & 2 & -6
\end{bmatrix}
}
\xrightarrow{\substack{r_{3}-\frac{1}{2}r_{2}\\ r_{4}-r_{2}}}
{
\begin{bmatrix}
  1 & 1 & 1 & 4 \\
  0 & -2 & 2 & -6\\
  0 & 0 & 0 & 0\\
 0 & 0 & 0 & 0
\end{bmatrix}
}
\xrightarrow{\substack{r_{2}\times\frac{1}{2}}}
{
\begin{bmatrix}
  1 & 1 & 1 & 4 \\
  0 & 1 & -1 & 3\\
  0 & 0 & 0 & 0\\
 0 & 0 & 0 & 0
\end{bmatrix}
}\\
\xrightarrow{\substack{r_{1}-r_{2}}}&
{
\begin{bmatrix}
  1 & 0 & 2 & 1 \\
  0 & 1 & -1 & 3\\
  0 & 0 & 0 & 0\\
 0 & 0 & 0 & 0
\end{bmatrix}
}
\end{align*}

(1)$r(\alpha_{1},\alpha_{2},\alpha_{3},\alpha_{4})=2$,极大线性无关组为:$(\alpha_{1},\alpha_{2}),(\alpha_{1},\alpha_{3}),(\alpha_{1},\alpha_{4}),(\alpha_{2},\alpha_{3}),(\alpha_{2},\alpha_{4}),(\alpha_{3},\alpha_{4})$.\textcolor[rgb]{1.00,0.00,0.00}{(任写一个即可。)}

(2)取$(\alpha_ {1},\alpha_{2})$,由最简阶梯型矩阵可以看出:
\begin{equation*}
\begin{cases}
 \alpha_3=2\alpha_1-\alpha_2\\
  \alpha_4=\alpha_1+3\alpha_2
\end{cases}
\end{equation*}
\end{jie}

六.设实二次型
\begin{equation*}
  f(x_{1},x_{2},x_{3})=X^{T}AX=ax_{1}^{2}+2x_{2}^{2}-2x_{3}^{2}+2bx_{1}x_{3}~~(b>0)
\end{equation*}
的矩阵$A$的特征值之和为$1$,特征值之积为-12。

(1)求$a,b$的值;

(2)利用正交变换将二次型$f$化为标准型,并写出所用正交变换。

\begin{jie}
由题得:$A=
\begin{bmatrix}
  a & 0 & b\\
  0 & 2 & 0\\
  b & 0 & -2
\end{bmatrix},|A|=2(-2a-b^2)
$

(1)由特征值的性质有:
\begin{equation*}
\begin{cases}
trace(A)=a+2-2=1\\
\prod\limits_{i=1}^{3}\lambda_i=-12=|A|=2(-2a-b^2)\\
b>0
\end{cases}~~~\Rightarrow~~~
\begin{cases}
a=1\\
b=2
\end{cases}
\end{equation*}

(2)
\begin{equation*}
  |\lambda E-A|=
  \begin{vmatrix}
  \lambda-1& 0 & -2\\
  0 & \lambda-2 & 0\\
  -2 & 0 & \lambda+2
  \end{vmatrix}=(\lambda-2)[(\lambda-1)(\lambda+2)-4]=0~~~~\Rightarrow~~~~\lambda_1=\lambda_2=2,\lambda_3=-3
\end{equation*}

$\lambda_1=\lambda_2=2$时:
\begin{equation*}
  [\lambda E-A]=
  \begin{vmatrix}
  1& 0 & -2\\
  0 & 0 & 0\\
  -2 & 0 & 4
  \end{vmatrix}~~~\Rightarrow~~~
  \begin{cases}
   x_1=2x_3\\
   x_2\in R
  \end{cases}
\end{equation*}
分别取$[x_2,x_3]^T=[1,0]^T$和$[0,1]^T$得:$\alpha_1=[0,1,0]^T,\alpha_{2}=[2,0,1]^T$,可以看出$\alpha_1$与$\alpha_2$正交。

$\lambda_3=-3$时:
\begin{equation*}
  [\lambda E-A]=
  \begin{vmatrix}
  -4& 0 & -2\\
  0 & -5 & 0\\
  -2 & 0 & -1
  \end{vmatrix}~~~\Rightarrow~~~
  \begin{cases}
   x_1=-\dfrac{1}{2}x_3\\
   x_2=0
  \end{cases}
\end{equation*}
取$x_3=-2$得:$\alpha_3=[1,0,-2]^T$。因为对称矩阵对应于不同特征值的特征向量正交,所以$[\alpha_1,\alpha_2,\alpha_3]$为正交向量组。

单位化:
\begin{equation*}
\begin{cases}
\gamma_1=\dfrac{\alpha_1}{\|\alpha_1\|}=[0,1,0]^T\\
\gamma_2=\dfrac{\alpha_2}{\|\alpha_2\|}=\left[\dfrac{2}{\sqrt{5}},0,\dfrac{1}{\sqrt{5}}\right]^T\\[2mm]
\gamma_3=\dfrac{\alpha_3}{\|\alpha_3\|}=\left[\dfrac{1}{\sqrt{5}},0,-\dfrac{2}{\sqrt{5}}\right]^T
\end{cases}~~~\Rightarrow~~~Q=
\begin{bmatrix}
  0 & \dfrac{2}{\sqrt{5}}& \dfrac{1}{\sqrt{5}} \\
  1 & 0& 0\\
  0&  \dfrac{1}{\sqrt{5}}& -\dfrac{2}{\sqrt{5}}
\end{bmatrix}
\end{equation*}
所以$f$可经正交变换$x=Qy$化为标准型:
\begin{equation*}
  f=2y_{1}^2+2y_{2}^2-3y_{3}^2
\end{equation*}
\end{jie}

七.设$A$为$n$阶矩阵,且$A^{2}-A-2I=0$。

(1)证明:$r(A-2I)+r(A+I)=n$.

(2)证明:矩阵$A+2I$可逆,并求$(A+2I)^{-1}$。

\begin{zhengming}
(1)由题得:$A^{2}-A-2I=(A-2I)(A+I)=0$。

所以由矩阵秩的性质有:$r(A+2I)+r(A+I)\leq n$。

$r((A-2I)-(A+I))=r(-3I)\leq r(A-2I)+r(-(A+I))=r(A-2I)+r((A+I))$,即$n\leq r(A-2I)+r((A+I))$

所以$r(A-2I)+r(A+I)=n$.

(2) 由题得:$A^{2}-A-2I=0$,所以
\begin{gather*}
  A^{2}\textcolor[rgb]{1.00,0.00,0.00}{+2A-2A}-A-2I=0 \\
  A(A+2I)-3A\textcolor[rgb]{1.00,0.00,0.00}{-6I+6I}-2I=0 \\
  A(A+2I)-3(A+2I)=(A-3I)(A+2I)=-4I\\
  -\frac{1}{4}(A-3I)(A+2I)=I
\end{gather*}
所以$A+2I$可逆,$(A+2I)^{-1}=-\dfrac{1}{4}(A-3I)$
\end{zhengming}

2.设$X_{0}$是线性方程组$Ax=b~(b\neq0)$的一个解,$X_{1},X_{2}$是导出组$Ax=0$的一个基础解系。令$\xi_{0}=X_{0},\xi_{1}=X_{0}+X_{1},\xi_{2}=X_{0}+X_{2}$,证明:$\xi_{0},\xi_{1},\xi_{2}$线性无关。

\begin{zhengming}
设
\begin{equation*}
k_{0}\xi_0+k_1\xi_1+k_2\xi_2=0\tag{$1$}
\end{equation*}
要证明$\xi_{0},\xi_{1},\xi_{2}$线性无关,根据定义,只需证明$k_1=k_2=k_3=0$。

由题得:$AX_0=b,AX_1=AX_2=0$,因为$X_1,X_2$为$AX=0$的基础解系,所以$X_1,X_2$线性无关。

把题中条件代入(1)式:
\begin{equation*}
k_1X_0+k_2X_0+k_2X_1+k_3X_0+k_3X_2= (k_1+k_2+k_3)X_0+k_2X_1+k_3X_2=0\tag{$2$}
\end{equation*}
(2)式两边同时左乘矩阵$A$:
\begin{equation*}
(k_1+k_2+k_3)AX_0+k_2AX_1+k_3AX_2=(k_1+k_2+k_3)b=0
\end{equation*}
因为$b\neq0$,所以$k_1+k_2+k_3=0$.代入(2)式得:$k_2X_1+k_3X_2=0$,因为$X_1,X_2$线性无关,所以$k_2=k_3=0$。所以$k_1=0-k_2-k_3=0$。

所以$\xi_{0},\xi_{1},\xi_{2}$线性无关。
\end{zhengming}

八.设3阶方阵$A$的特征值-1,1对应的特征向量分别为$\alpha_{1},\alpha_{2}$,向量$\alpha_{3}$满足$A\alpha_{3}=\alpha_{2}+\alpha_{3}$.

(1)证明:$\alpha_{1},\alpha_{2},\alpha_{3}$线性无关;

(2)设$P=[\alpha_{1},\alpha_{2},\alpha_{3}]$,求$P^{-1}AP$。

\begin{jie}
由题得:$A\alpha_ {1}=-\alpha_ {1},A\alpha_{2}=\alpha_{2}$。

(1)设
\begin{equation*}
k_1\alpha_{1}+k_2\alpha_{2}+k_3\alpha_{3}=0\tag{$1$}
\end{equation*}
要证明$\alpha_{1},\alpha_{2},\alpha_{3}$线性无关,只需证明$k_1=k_2=k_3=0$,(1)式两边同左乘$A$:
\begin{equation*}
k_1A\alpha_{1}+k_2A\alpha_{2}+k_3A\alpha_{3}=-k_1\alpha_{1}+k_2\alpha_{2}+k_3(\alpha_{2}+\alpha_{3})=-k_1\alpha_{1}+(k_2+k_3)\alpha_{2}+k_3\alpha_{3}=0\tag{$2$}
\end{equation*}
(1)式减(2)式:$-2k_1\alpha_{1}-k_3\alpha_{2}=0$,因为$\alpha_1$和$\alpha_2$是分属于不同的特征值的特征向量,所以$\alpha_1$和$\alpha_2$线性无关。即$
\begin{cases}
-2k_1=0\\
-k_3=0
\end{cases}~\Rightarrow k_1=k_3=0
$,代入到$(1)$式:$k_2\alpha_{2}=0$,又因为特征向量不为0,所以$k_{2}=0$。

综上所述:$k_{1}=k_2=k_3=0$,所以$\alpha_{1},\alpha_{2},\alpha_{3}$线性无关。

(2)由题得:
\begin{equation*}
AP=[A\alpha_1,A\alpha_2,A\alpha_3]=[-\alpha_1,\alpha_2,\alpha_2+\alpha_3]=[\alpha_1,\alpha_2,\alpha_3]
\begin{bmatrix}
  -1 & 0 & 0 \\
  0 & 1 & 1\\
  0 & 0 & 1
\end{bmatrix}=P\Lambda
\end{equation*}
所以$\Lambda=P^-1AP=
\begin{bmatrix}
  -1 & 0 & 0 \\
  0 & 1 & 1\\
  0 & 0 & 1
\end{bmatrix}
$。
\end{jie}
\newpage

\hphantom{~~}\hfill {\zihao{4}\heiti 2015-2016年第一学期} \hfill\hphantom{~~}

一、填空题

1.行列式$
D=
\begin{vmatrix}
  1 & a & 0 & 0\\
  -1 & 2-a & a & 0\\
  0 & -2 & 3-a & a\\
  0 & 0 & -3 & 4-a
\end{vmatrix}=
$\underline{\hphantom{~~~~~~~~~~}}。

\begin{jie}
把第二行加到第一行上:
\begin{align*}
D=
\begin{vmatrix}
  1 & a & 0 & 0\\
  -1 & 2-a & a & 0\\
  0 & -2 & 3-a & a\\
  0 & 0 & -3 & 4-a
\end{vmatrix}=\begin{vmatrix}
  1 & a & 0 & 0\\
  0 & 2 & a & 0\\
  0 & -2 & 3-a & a\\
  0 & 0 & -3 & 4-a
\end{vmatrix}=1\times(-1)^{1+1}
\begin{vmatrix}
  2 & a & 0\\
   -2 & 3-a & a\\
   0 & -3 & 4-a
\end{vmatrix}=A
\end{align*}
把$A$的第二行加到第一行上:
\begin{align*}
A=\begin{vmatrix}
  2 & a & 0\\
   0 & 3 & a\\
   0 & -3 & 4-a
\end{vmatrix}=2\times(-1)^{1+1}
\begin{vmatrix}
 3 & a\\
 -3 & 4-a
\end{vmatrix}=24
\end{align*}
\end{jie}

2. 设$
A=\begin{bmatrix}
    1 & 1 & 1 & 1\\
    0 & 2 & 2 & 2\\
    0 & 0 & 3 & 3\\
    0 & 0 & 0 & 4
  \end{bmatrix}
$,求$A^{2}-2A$的秩$r(A^{2}-2A)$。

\begin{jie}
由题得:$A^{2}-2A=A(A-2E)$,可以看出$A$是满秩方阵,即$A$可逆。\textcolor[rgb]{1.00,0.00,0.00}{满秩方阵一定可逆},所以
$r(A^{2}-2A)=r(A(A-2E))=r(A-2E)$。
\begin{align*}
A-2E=
\begin{bmatrix}
-1 & 1 & 1 & 1\\
0 & 0 & 2 & 2\\
0 & 0 & 1 & 3\\
0 & 0 & 0 & 2
\end{bmatrix}
\xrightarrow{r_{2}\times\frac{1}{2}}
{
\begin{bmatrix}
-1 & 1 & 1 & 1\\
0 & 0 & 1 & 1\\
0 & 0 & 1 & 3\\
0 & 0 & 0 & 2
\end{bmatrix}
}\xrightarrow{r_{3}-r_{2}}
{
\begin{bmatrix}
-1 & 1 & 1 & 1\\
0 & 0 & 1 & 1\\
0 & 0 & 0 & 2\\
0 & 0 & 0 & 2
\end{bmatrix}
}\xrightarrow{r_{4}-r_{3}}
{
\begin{bmatrix}
-1 & 1 & 1 & 1\\
0 & 0 & 1 & 1\\
0 & 0 & 0 & 2\\
0 & 0 & 0 & 0
\end{bmatrix}
}
\end{align*}
所以r(A-2E)=3,所以$r(A^{2}-2A)=3$
\end{jie}

3.设$\alpha_{1},\alpha_{2},\alpha_{3}$是非齐次线性方程组$Ax=b$的解,若$\sum\limits_{i=1}^{3}c_{i}\alpha_{i}$也是$Ax=b$的解,则$\sum\limits_{i=1}^{3}c_{i}=$\underline{\hphantom{~~~~~~~~~~}}。

\begin{jie}
由题得:$A\alpha_{1}=b,A\alpha_{2}=b,A\alpha_{3}=b$,$A\sum\limits_{i=1}^{3}c_{i}\alpha_{i}=A(c_{1}\alpha_{1}+c_{2}\alpha_{2}+c_{3}\alpha_{3})=b$.

所以$A\alpha_{1}+A\alpha_{2}+A\alpha_{3}=3b$,即$A\left(\dfrac{1}{3}\alpha_{1}+\dfrac{1}{3}\alpha_{2}+\dfrac{1}{3}\alpha_{3}\right)=b=A(c_{1}\alpha_{1}+c_{2}\alpha_{2}+c_{3}\alpha_{3})$,所以$\sum\limits_ {i=1}^{3}c_{i}=\dfrac{1}{3}+\dfrac{1}{3}+\dfrac{1}{3}=1$。
\end{jie}

4.已知矩阵$
A=
\begin{bmatrix}
  3 & 2 & -1 \\
  a & -2 & 2\\
  3 & b & -1
\end{bmatrix}
$,若$\alpha=(1,-2,3)^{T}$是其特征向量,则$a+b=$\underline{\hphantom{~~~~~~~~~~}}。

\begin{jie}
设$\alpha$对应的特征值为$\lambda$。由特征向量的定义:
\begin{equation*}
A\alpha=\lambda\alpha~~~\Rightarrow~~~
\begin{bmatrix}
-4 \\ a+10 \\ -2b
\end{bmatrix}=
\begin{bmatrix}
\lambda \\ -2\lambda \\ 3\lambda
\end{bmatrix}~~~\Rightarrow ~~~
\begin{cases}
\lambda=-4\\
a=-2\\
b=6
\end{cases}~~~\Rightarrow~~~a+b=4
\end{equation*}
\end{jie}

5.任意3维实列向量都可以由向量组$\alpha_{1}=(1,0,1)^{T},\alpha_{2}=(1,-2,3)^{T}\alpha_{3}=(t,1,2)^{T}$线性表示,则$t$应满足条件\underline{\hphantom{~~~~~~~~~~}}。

\begin{jie}
任意3维实列向量都可以由向量组$\alpha_{1},\alpha_{2},\alpha_{3}$线性表示,则$e_{1}=[1,0,0]^T,e_2=[0,1,0]^T,e_3=[0,0,1]^T$也可由$\alpha_ {1},\alpha_{2},\alpha_{3}$线性表示,而$e_1,e_2,e_3$可以表示任意三维实列向量,即向量组$\alpha_ {1},\alpha_{2},\alpha_{3}$和$e_1,e_2,e_3$可以相互线性表示,所以$r(\alpha_ {1},\alpha_{2},\alpha_{3})=r(\alpha_ {1},\alpha_{2},\alpha_{3},e_1,e_2,e_3)=r(e_1,e_2,e_3)=3$.所以$|\alpha_ {1},\alpha_{2},\alpha_{3}|=2t-6\neq0$,即$t\neq3$。
\end{jie}

6.若矩阵$
A=
\begin{bmatrix}
  1 & 1 & 2 \\
  1 & 2 & 3\\
  2 & 3 & \lambda
\end{bmatrix}
$正定,则$\lambda$满足的条件为\underline{\hphantom{~~~~~~~~~~}}。

\begin{jie}
由题得:$A$为对称矩阵,如果$A$正定,则$|A|>0$,所以$|A|=\lambda-5>0~~\Rightarrow~~\lambda>5$.
\end{jie}

二、计算题

1.若行列式$D=
\begin{vmatrix}
  1 & 2 & 3 & 4 \\
  0 & 3 & 4 & 6 \\
  3 & 4 & 1 & 2 \\
  2 & 2 & 2 & 2
\end{vmatrix}
$,求$A_{11}+2A_{21}+A_{31}+2A_{41}$,其中$A_{ij}$为元素$a_{ij}$的代数余子式。

\begin{jie}
由题得:
\begin{align*}
A_{11}+2A_{21}+A_{31}+2A_{41}=\begin{vmatrix}
  1 & 2 & 3 & 4 \\
  2 & 3 & 4 & 6 \\
  1 & 4 & 1 & 2 \\
  2 & 2 & 2 & 2
\end{vmatrix}=2\begin{vmatrix}
  1 & 2 & 3 & 4 \\
  2 & 3 & 4 & 6 \\
  1 & 4 & 1 & 2 \\
  1 & 1 & 1 & 1
\end{vmatrix}
\end{align*}
把第一行的负二倍加到第二行,把第一行的负一倍分别加到第三、四行。
\begin{align*}
2\begin{vmatrix}
  1 & 2 & 3 & 4 \\
  2 & 3 & 4 & 6 \\
  1 & 4 & 1 & 2 \\
  1 & 1 & 1 & 1
\end{vmatrix}=
2\begin{vmatrix}
  1 & 2 & 3 & 4 \\
  0 & -1 & -2 & -2 \\
  0 & 2 & -2 & -2 \\
  0 & -1 & -2 & -3
\end{vmatrix}=4\begin{vmatrix}
  -1 & -2 & -2 \\
   1 & -1 & -1 \\
   -1 & -2 & -3
\end{vmatrix}
\end{align*}
把第一行的负一倍加到第三行:
\begin{align*}
4\begin{vmatrix}
  -1 & -2 & -2 \\
   1 & -1 & -1 \\
   -1 & -2 & -3
\end{vmatrix}=
4\begin{vmatrix}
  -1 & -2 & -2 \\
   1 & -1 & -1 \\
  0 & 0 & -1
\end{vmatrix}=4\times(-1)\times((-1)\times(-1)-(-2)\times 1)=-12
\end{align*}
\end{jie}

2.

已知矩阵$X$满足方程$X
\begin{bmatrix}
  1 & 0 & -2 \\
  0 & 1 & 2 \\
  -1 & 0 & 3
\end{bmatrix}
=
\begin{bmatrix}
  -1 & 2 & 0 \\
  3 & 0 & 5
\end{bmatrix}$,求矩阵$X$。

\begin{jie}
由题得:
\begin{align*}
\left[
\begin{array}{c:c}
\begin{matrix}
 1 & 0 & -2 \\
  0 & 1 & 2 \\
  -1 & 0 & 3
\end{matrix} &
\begin{matrix}
  1 & 0 & 0\\
  0 & 1 & 0\\
  0 & 0 & 1
\end{matrix}
\end{array}
\right]\xrightarrow{\substack{r_{3}+r_{1}}}
{\left[
\begin{array}{c:c}
\begin{matrix}
 1 & 0 & -2 \\
  0 & 1 & 2 \\
  0 & 0 & 1
\end{matrix} &
\begin{matrix}
  1 & 0 & 0\\
  0 & 1 & 0\\
  1 & 0 & 1
\end{matrix}
\end{array}
\right]
}\xrightarrow{\substack{r_{2}-2r_{3}\\ r_{1}+2r_{3}}}
{\left[
\begin{array}{c:c}
\begin{matrix}
 1 & 0 & 0 \\
  0 & 1 & 0 \\
  0 & 0 & 1
\end{matrix} &
\begin{matrix}
  3 & 0 & 2\\
  -2 & 1 & -2\\
  1 & 0 & 1
\end{matrix}
\end{array}
\right]
}
\end{align*}
所以$X=\begin{bmatrix}
  -1 & 2 & 0 \\
  3 & 0 & 5
\end{bmatrix}\begin{bmatrix}
  3 & 0 & 2\\
  -2 & 1 & -2\\
  1 & 0 & 1
\end{bmatrix}=\begin{bmatrix}
  -7 & 2 & -6 \\
  14 & 0 & 11
\end{bmatrix}$
\end{jie}

3.设向量组$\alpha_{1}=(1,-1,2,4),\alpha_{2}=(0,3,1,2),\alpha_{3}=(3,0,7,14),\alpha_{4}=(1,-1,2,0),\alpha_{5}=(2,1,5,6)$,求向量组的秩、极大线性无关组,并将其余向量由极大无关组线性表示出。

\begin{jie}
\begin{align*}
(\alpha_1,\alpha_2,\alpha_3,\alpha_4,\alpha_5)&=
\begin{bmatrix}
  1 & 0 & 3 &1 &2\\
  -1 & 3 & 0&-1&1\\
  2 & 1 & 7&2&5\\
  4 & 2 &14&0&6
\end{bmatrix}
\xrightarrow{\substack{r_{2}+r_{1}\\ r_3-2r_1 \\ r_4-4r_1}}
{
\begin{bmatrix}
  1 & 0 & 3 &1&2\\
  0 & 3 & 3&0&3\\
  0 & 1 & 1&0&1\\
  0 & 2 &2&-4&-2
\end{bmatrix}
}
\xrightarrow{\substack{ r_3-\frac{1}{3}r_2 \\ r_4-\frac{2}{3}r_2}}
{
\begin{bmatrix}
  1 & 0 & 3 &1&2\\
  0 & 3 & 3&0&3\\
  0 & 0 & 0&0&0\\
  0 & 0 &0&-4&-4
\end{bmatrix}
}\\
&
\xrightarrow{\substack{ r_2\times\frac{1}{3} \\ r_4\times\left(-\frac{1}{4}\right)}}
{
\begin{bmatrix}
  1 & 0 & 3 &1&2\\
  0 & 1 & 1&0&1\\
  0 & 0 & 0&0&0\\
  0 & 0 &0&1&1
\end{bmatrix}
}
\xrightarrow{\substack{ r_1-r_4}}
{
\begin{bmatrix}
  1 & 0 & 3 &0&1\\
  0 & 1 & 1&0&1\\
  0 & 0 & 0&0&0\\
  0 & 0 &0&1&1
\end{bmatrix}
}
\xrightarrow{\substack{ r_3\leftrightarrow r_4}}
{
\begin{bmatrix}
  1 & 0 & 3 &0&1\\
  0 & 1 & 1&0&1\\
  0 & 0 & 0&1& 0\\
  0 & 0 &0&0& 0
\end{bmatrix}
}
\end{align*}
所以该向量组的秩为$3$,极大线性无关组为$(\alpha_1,\alpha_2,\alpha_4),(\alpha_1,\alpha_3,\alpha_4),(\alpha_1,\alpha_4,\alpha_5),(\alpha_2,\alpha_3,\alpha_4),(\alpha_2,\alpha_4,\alpha_5)$.

由$(\alpha_1,\alpha_2,\alpha_4)$表示$\alpha_3,\alpha_5$:由最简阶梯型矩阵可以看出
\begin{equation*}
\begin{cases}
\alpha_3=3\alpha_1+\alpha_2+0\alpha_4=3\alpha_1+\alpha_2\\ \alpha_5=\alpha_1+\alpha_2+0\alpha_4=\alpha_1+\alpha_2
\end{cases}
\end{equation*}

\end{jie}

三、解答题

1.当$k$为何值时,线性方程组
$
\begin{cases}
kx_{1}+x_{2}+x_{3}=k-3\\
x_{1}+kx_{2}+x_{3}=-2\\
x_{1}+x_{2}+kx_{3}=-2
\end{cases}
$
有唯一解,无解和有无穷多解?当方程组有无穷多解时求出所有解。

\begin{jie}
\textcolor[rgb]{1.00,0.00,0.00}{(系数矩阵是方阵,也可以用行列式来做这个题。具体看14-15年期末试题的第四题,推荐这种方法)}

增广矩阵
\begin{align*}
&\left[
\begin{array}{c:c}
\begin{matrix}
  k & 1 & 1 \\
  1 & k & 1 \\
  1 & 1 & k \\
\end{matrix}
&
\begin{matrix}
  k-3 \\
  -2 \\
  -2 \\
\end{matrix}
\end{array}
\right]
\xrightarrow{r_{1}\leftrightarrow r_{2}}
{
\left[
\begin{array}{c:c}
\begin{matrix}
  1 & k & 1 \\
  k & 1 & 1 \\
  1 & 1 & k \\
\end{matrix}
&
\begin{matrix}
  -2 \\
  k-3 \\
  -2 \\
\end{matrix}
\end{array}
\right]
}
\xrightarrow{\substack{r_{2}-kr_{1} \\ r_{3}-r_{1}}}
{
\left[
\begin{array}{c:c}
\begin{matrix}
  1 & k & 1 \\
  0 & 1-k^{2} & 1-k \\
  0 & 1-k & k-1 \\
\end{matrix}
&
\begin{matrix}
  -2 \\
  3(k-1) \\
  0 \\
\end{matrix}
\end{array}
\right]
}\\
\xrightarrow{r_{2}\leftrightarrow r_{3}} &
{
\left[
\begin{array}{c:c}
\begin{matrix}
  1 & k & 1 \\
  0 & 1-k & k-1 \\
  0 & 1-k^{2} & 1-k \\
\end{matrix}
&
\begin{matrix}
  -2 \\
  0 \\
  3(k-1) \\
\end{matrix}
\end{array}
\right]
}
\xrightarrow{r_{3}-(1+k)r_{2}}
{
\left[
\begin{array}{c:c}
\begin{matrix}
  1 & k & 1 \\
  0 & 1-k & k-1 \\
  0 & 0 & (1-k)(k+2) \\
\end{matrix}
&
\begin{matrix}
  -2 \\
  0 \\
  3(k-1) \\
\end{matrix}
\end{array}
\right]
}
\end{align*}
讨论:

(1)解不存在:即存在矛盾方程(增广矩阵主元列在最右列)。即对于$r_{3}$
\begin{equation*}
  \begin{cases}
    (1-k)(k+2)=0\\
    3(k-1)\neq 0
  \end{cases}~~~
  \Rightarrow~~~k=-2
\end{equation*}

(2)存在唯一解:主元列三个元素都不为0.即
\begin{equation*}
  \begin{cases}
    1\neq 0\\
    1-k\neq 0 \\
    (1-k)(k+2)\neq 0
  \end{cases}
  ~~~\Rightarrow~~~k\neq 1\text{且}k\neq -2
\end{equation*}
$k\neq 1\text{且}k\neq -2$,继续对阶梯矩阵进行初等行变换
\begin{align*}
 \xrightarrow{\substack{r_{2}\times\frac{1}{1-k}\\r_{3}\times\frac{1}{(k+2)(1-k)} }}
{
\left[
\begin{array}{c:c}
\begin{matrix}
  1 & k & 1 \\
  0 & 1 & -1 \\
  0 & 0 & 1 \\
\end{matrix}
&
\begin{matrix}
  -2 \\
  0 \\
  \frac{3}{(k+2)} \\
\end{matrix}
\end{array}
\right]
}
\xrightarrow{\substack{r_{2}+r_{3}\\r_{1}-r_{3} }}
{
\left[
\begin{array}{c:c}
\begin{matrix}
  1 & k & 0 \\
  0 & 1 & 0 \\
  0 & 0 & 1 \\
\end{matrix}
&
\begin{matrix}
  -2-\frac{3}{k+2} \\
  \frac{3}{k+2} \\
  \frac{3}{k+2} \\
\end{matrix}
\end{array}
\right]
}
\xrightarrow{r_{1}-kr_{2}}
{
\left[
\begin{array}{c:c}
\begin{matrix}
  1 & 0 & 0 \\
  0 & 1 & 0 \\
  0 & 0 & 1 \\
\end{matrix}
&
\begin{matrix}
  -\frac{5k+1}{k+2} \\
  \frac{3}{k+2} \\
  \frac{3}{k+2} \\
\end{matrix}
\end{array}
\right]
}
\end{align*}
所以方程组存在唯一解时:$k\neq 1$且$k\neq -2$,解为
\begin{equation*}
\mathbf{x}=
\begin{bmatrix}
 -\frac{5k+1}{k+2} \\
  \frac{3}{k+2} \\
  \frac{3}{k+2}
\end{bmatrix}
~,~~~k\neq1\text{且}k\neq -2
\end{equation*}

(3)存在无穷解:至少存在一个自由变量。由阶梯矩阵可以看出
\begin{equation*}
  \begin{cases}
    (k-1)(k+2)=0\\
    3(k-1)=0
  \end{cases}
  ~~~\Rightarrow~~~k=1
\end{equation*}
把$k=1$代入阶梯矩阵:
\begin{equation*}
\left[
  \begin{array}{c:c}
    \begin{matrix}
      1 & 1 & 1 \\
      0 & 0 & 0 \\
      0 & 0 & 0 \\
    \end{matrix}
     &
     \begin{matrix}
      -2 \\
      0 \\
      0\\
    \end{matrix}
  \end{array}
\right]
~~~\Rightarrow~~~
x=
\begin{bmatrix}
  -2-c_{1}-c_{2} \\
  c_{1} \\
  c_{2}
\end{bmatrix}
\end{equation*}
所以$x$的解为$x=
\begin{bmatrix}
  -2 \\
  0\\
   0
\end{bmatrix}+
\begin{bmatrix}
  -1 \\
  1\\
   0
\end{bmatrix}c_{1}+
\begin{bmatrix}
  -1 \\
  0\\
   1
\end{bmatrix}c_{1}
~~~c_{1},c_{2}\in R$。
\end{jie}

2.设3阶实对称矩阵$A$的特征值为$\lambda_{1}=-1,\lambda_{2}=\lambda_{3}=1$,对应于$\lambda_{1}$的特征向量$\alpha_{1}=(0,1,1)^{T}$。

(1)求$A$对应于特征值1的特征向量;

(2)求$A$;

(3)求$A^{2016}$。

\begin{jie}
(1)由于$A$是实对称矩阵,所以对于$A$的不同特征值的特征向量正交,所以设特征值1对应的特征向量是$\alpha=[x_1,x_2,x_3]$。所以有:
\begin{equation*}
\alpha_{1}^T\alpha=x_2+x_3=0~~~\Rightarrow~~~x_2=-x_3
\end{equation*}
分别取$
\begin{bmatrix}
x_1 \\ x_3
\end{bmatrix}=\begin{bmatrix}
1 \\ 0
\end{bmatrix},
\begin{bmatrix}
0 \\ 1
\end{bmatrix}
$得$\alpha_{2}=
\begin{bmatrix}
1 \\ 0 \\0
\end{bmatrix}
,\alpha_3=
\begin{bmatrix}
0\\ -1 \\1
\end{bmatrix}$。

$\alpha_2,\alpha_3$即为$A$对应于特征值1的特征向量。

(2)由特征值定义:$A\alpha_i=\lambda_i\alpha_i$。所以:
\begin{align*}
&A[\alpha_1,\alpha_2,\alpha_3]=[\lambda_1\alpha_1,\lambda_2\alpha_2,\lambda_3\alpha_3] ~~~\Rightarrow\\
&A=[\lambda_1\alpha_1,\lambda_2\alpha_2,\lambda_3\alpha_3] [\alpha_1,\alpha_2,\alpha_3]^{-1}=
\begin{bmatrix}
  0 & 1  & 0\\
 -1 & 0&-1\\
 -1 & 0& 1
\end{bmatrix}
\begin{bmatrix}
  0 & 1  & 0\\
 1 & 0&-1\\
 1 & 0& 1
\end{bmatrix}^{-1}=
\begin{bmatrix}
  1 & 0  & 0\\
 0 & 0&-1\\
 0 & -1& 0
\end{bmatrix}
\end{align*}
$\left(\text{式中:}[\alpha_1,\alpha_2,\alpha_3]^{-1}=
\begin{bmatrix}
  0 & \frac{1}{2} & \frac{1}{2}\\
 1 & 0&0\\
  0 & -\frac{1}{2} & \frac{1}{2}
\end{bmatrix}
\right)$

(3)由$(2)$得:$A^2=E_3$($E$表示单位矩阵。)所以$A^{2016}=(A^{2})^{1008}=E_3$。
\end{jie}

3.设$
A=
\begin{bmatrix}
  1 & 0 & 1 \\
  0 & 1 & 1 \\
  -1 & 0 & a \\
  0 & a & -1
\end{bmatrix},A^{T}
$为矩阵$A$的转置,已知$r(A)=2$,且二次型$f(x)=x^{T}A^{T}Ax$.

(1)求$a$;

(2)写出二次型$f(x)$的矩阵$B=A^{T}A$;

(3)求正交变换$x=Qy$将二次型$f(x)$化为标准型,并写出所用的正交变换。

\begin{jie}
(1)由题得:
\begin{equation*}
A
\xrightarrow{\substack{r_{3}+r_{1} \\r_4-ar_2}}
{
\begin{bmatrix}
  1 & 0 & 1 \\
  0 & 1 & 1 \\
  0 & 0 & a+1 \\
  0 & 0 & -1-a
\end{bmatrix}
}
\end{equation*}
$r(A)=2$,所以$1+a=0$且$-1-a=0$解得:$a=-1$。

(2)
\begin{equation*}
B=A^TA=
\begin{bmatrix}
1 & 0 &-1 & 0\\
0 & 1 & 0 &-1\\
1 & 1 & -1 &-1
\end{bmatrix}
\begin{bmatrix}
  1 & 0 & 1 \\
  0 & 1 & 1 \\
  -1 & 0 & -1 \\
  0 & -1 & -1
\end{bmatrix}=\begin{bmatrix}
                2 & 0&2 \\
                0 & 2&2\\
                2& 2&4
              \end{bmatrix}
\end{equation*}

(3)
\begin{equation*}
|\lambda E-B|=
\begin{vmatrix}
\lambda-2 & 0 &-2\\
0&\lambda-2& -2\\
-2&-2&\lambda-4
\end{vmatrix}=\lambda(\lambda-2)(\lambda-6)~~~\Rightarrow~~~\lambda_1=0,\lambda_2=2,\lambda_3=3
\end{equation*}

$\lambda_1=0$时:
\begin{equation*}
  [\lambda E-A]=
  \begin{vmatrix}
-2 & 0 &-2\\
0&-2& -2\\
-2&-2&-4
  \end{vmatrix}~~~\Rightarrow~~~
  \begin{cases}
   x_1=-x_3\\
   x_2=-x_3
  \end{cases}
\end{equation*}
取$x_3=1$得:$\alpha_1=[1,1,-1]^T$.

$\lambda_2=2$时:
\begin{equation*}
  [\lambda E-A]=
  \begin{vmatrix}
0 & 0 &-2\\
0&0& -2\\
-2&-2&-2
  \end{vmatrix}~~~\Rightarrow~~~
  \begin{cases}
   x_1=-x_2\\
   x_3=0
  \end{cases}
\end{equation*}
取$x_2=1$得:$\alpha_2=[1,-1,0]^T$.

$\lambda_3=6$时:
\begin{equation*}
  [\lambda E-A]=
  \begin{vmatrix}
4 & 0 &-2\\
0&4& -2\\
-2&-2&2
  \end{vmatrix}~~~\Rightarrow~~~
  \begin{cases}
   x_1=\frac{1}{2}x_3\\
   x_2=\frac{1}{2}x_3
  \end{cases}
\end{equation*}
取$x_3=2$得:$\alpha_3=[1,1,2]^T$.因为对称矩阵对应于不同特征值的特征向量正交,所以$[\alpha_1,\alpha_2,\alpha_3]$为正交向量组。

单位化:
\begin{equation*}
\begin{cases}
\gamma_1=\dfrac{\alpha_1}{\|\alpha_1\|}=\left[\dfrac{1}{\sqrt{3}},\dfrac{1}{\sqrt{3}},-\dfrac{1}{\sqrt{3}}\right]^T\\[2mm]
\gamma_2=\dfrac{\alpha_2}{\|\alpha_2\|}=\left[\dfrac{1}{\sqrt{2}},-\dfrac{1}{\sqrt{2}},0\right]^T\\[2mm]
\gamma_3=\dfrac{\alpha_3}{\|\alpha_3\|}=\left[\dfrac{1}{\sqrt{6}},\dfrac{1}{\sqrt{6}},\dfrac{2}{\sqrt{6}}\right]^T
\end{cases}~~~\Rightarrow~~~Q=
\begin{bmatrix}
\dfrac{1}{\sqrt{3}}&\dfrac{1}{\sqrt{2}}&\dfrac{1}{\sqrt{6}}\\[2mm]
\dfrac{1}{\sqrt{3}}&-\dfrac{1}{\sqrt{2}}&\dfrac{1}{\sqrt{6}}\\[2mm]
-\dfrac{1}{\sqrt{3}}&0&\dfrac{2}{\sqrt{6}}
\end{bmatrix}
\end{equation*}
所以$f$可经正交变换$x=Qy$化为标准型:
\begin{equation*}
  f=2y_{2}^2+6y_{3}^2
\end{equation*}
\end{jie}

四、证明题

1. 设$A$为$n$阶实对称矩阵,且满足$A^{2}-3A+2E=0$,其中$E$为单位矩阵,试证:

(1)$A+2E$可逆;

(2)$A$为正定矩阵。

\begin{zhengming}
(1)由题得:$A^{2}-3A+2E=0$,所以:
\begin{gather*}
  A^{2}\textcolor[rgb]{1.00,0.00,0.00}{+2A-2A}-3A+2E=0 \\
  A(A+2E)-5A\textcolor[rgb]{1.00,0.00,0.00}{-10E+10E}+2E=0 \\
  A(A+2E)-5(A+2E)=(A-5E)(A+2E)=-12E\\
  -\frac{1}{12}(A-5E)(A+2E)=E
\end{gather*}
所以$A+2E$可逆,$(A+2E)^{-1}=-\dfrac{1}{12}(A-5E)$。

(2) 对于$n$阶实对称矩阵,如果$A$为正定矩阵,则$A$的全部特征值大于0.
设$A$的特征值为$\lambda$。由题得:
\begin{equation*}
\lambda^2-3\lambda+2=0~~~\Rightarrow~~~\lambda_1=1>0,\lambda_2=2>0
\end{equation*}
所以$A$为正定矩阵。
\end{zhengming}

2.设向量组$\alpha_{1},\alpha_{2},\alpha_{3}$线性无关,向量$\beta$可由$\alpha_{1},\alpha_{2},\alpha_{3}$线性表示,向量$\gamma$不能由$\alpha_{1},\alpha_{2},\alpha_{3}$线性表示,证明向量组$\alpha_{1},\alpha_{2},\alpha_{3},\beta+\alpha$线性无关。

\begin{zhengming}
向量$\beta$可由$\alpha_{1},\alpha_{2},\alpha_{3}$线性表示,即存在一组不全为0的$k_{i}(1\leq i\leq3)$,使得
\begin{equation*}
 \beta=k_ {1}\alpha_{1}+k_{2}\alpha_{2}+k_{3}\alpha_{3}\tag{$1$}
\end{equation*}

反证:假设向量组$\alpha_{1},\alpha_{2},\alpha_{3},\beta+\alpha$线性相关。则存在一组不全为0的$l_{i}(1\leq i\leq3)$和$l$,使得
\begin{equation*}
l_ {1}\alpha_{1}+l_{2}\alpha_{2}+l_{3}\alpha_{3}+l(\beta+\gamma)=0\tag{$2$}
\end{equation*}

若$l=0$,则$l_ {1}\alpha_{1}+l_{2}\alpha_{2}+l_{3}\alpha_{3}+l(\beta+\gamma)=l_ {1}\alpha_{1}+l_{2}\alpha_{2}+l_{3}\alpha_{3}=0$,此时$\alpha_{1},\alpha_{2},\alpha_{3}$线性相关,与题中的条件矛盾,所以$l\neq0$。所以$(1)$式可变形为
\begin{equation*}
\beta+\gamma=-\frac{1}{l}(l_ {1}\alpha_{1}+l_{2}\alpha_{2}+l_{3}\alpha_{3})
\end{equation*}
代入$(1)$式:
\begin{equation*}
  \gamma=-\frac{1}{l}(l_ {1}\alpha_{1}+l_{2}\alpha_{2}+l_{3}\alpha_{3})-k_ {1}\alpha_{1}+k_{2}\alpha_{2}+k_{3}\alpha_{3}
\end{equation*}

可以看出此时$\gamma$可以由$\alpha_{1},\alpha_{2},\alpha_{3}$线性表示,与题目矛盾,所以假设错误,即向量组$\alpha_{1},\alpha_{2},\alpha_{3},\beta+\alpha$线性无关。
\end{zhengming}

\newpage
\hphantom{~~}\hfill {\zihao{4}\heiti 2016-2017年第一学期} \hfill\hphantom{~~}

一、填空题

1.行列式$
D=
\begin{vmatrix}
  1 & x & y & z\\
  x & 1 & 0 & 0\\
  y & 0 & 1 & 0\\
  z & 0 & 0 & 1
\end{vmatrix}=
$\underline{\hphantom{~~~~~~~~~~}}。

\begin{jie}
(注意爪型行列式的做法)
对行列式做如下变换:第一行减去$z$倍的第四行,第一行减去$y$倍的第三行,第一行减去$x$倍的第二行,得到如下行列式:
\begin{equation*}
  \begin{vmatrix}
  1-x^{2}-y^{2}-z^{2} & 0 & 0 & 0\\
  a & 1 & 0 & 0\\
  b & 0 & 1 & 0\\
  c & 0 & 0 & 1\\
\end{vmatrix}=1-x^{2}-y^{2}-z^{2}
\end{equation*}
\end{jie}

2. 设$A$的伴随矩阵$
A^{*}=
\begin{bmatrix}
  1 & 2 & 3 & 4\\
  0 & 2 & 3 & 4\\
  0 & 0 & 2 & 3\\
  0 & 0 & 0 & 2
\end{bmatrix}
$,则$r(A^{2}-2A)=$\underline{\hphantom{~~~~~~~~~~}}。

\begin{jie}
$|A^*|=|A|^{n-1}$得:$|A^*|=2^3=|A|^{3}$.所以$|A|=2\neq0$。即$A$可逆。

所以$r(A^{2}-2A)=r(A(A-2E))=r(A-2E)=r(|A|(A^*)^{-1}-2E)=r((A^*)^{-1}-E)=3$.(求逆和秩的过程略)
\end{jie}

3. 已知线性方程组
$
\begin{cases}
 x_{1}+2x_{2}+x_{3}=2\\
 ax_{1}-x_{2}-2x_{3}=-3
\end{cases}
$与线性方程$ax_{2}+x_{3}=1$有公共的解,则$a$的取值范围为\underline{\hphantom{~~~~~~~~~~}}。

\begin{jie}
\textcolor[rgb]{1.00,0.00,0.00}{注意同解与公共解的区别:}

\textcolor[rgb]{1.00,0.00,0.00}{对于方程组(1)和方程组(2),如果$\alpha$既是方程组(1)的解,也是方程组$(2)$的解,则称$\alpha$是方程组(1)和方程组$(2)$的公共解。}

\textcolor[rgb]{1.00,0.00,0.00}{对于方程组(1)和方程组(2),如果$\alpha$是方程组(1)的解,则$\alpha$一定是方程组$(2)$的解,反之如果$\alpha$是方程组(2)的解,则$\alpha$一定是方程组$(1)$的解,则称,方程组(1)和方程组$(2)$同解。}

 关于公共解,通常有如下解法:(假设方程组(1)有两组不同的基础解系$\xi_1,\xi_2$,方程组(2)有两组不同的基础解系$\eta_1,\eta_2$)

 方法1:分别求出方程组(1)和(2)的通解:即(1)得到通解为:$k_1\xi_1+k_2\xi_2$,(2)的通解为$l_1\eta_1=l_2\eta$。令两个方程组的通解相等,找出对应的$k$和$l$的关系。

 方法2:求出其中一个通解,代入到另外一个方程组中,找出相应的系数所满足的关系式进一步求出公共解。

 方法3:联立两个方程组得到一个新的方程组(3),求出(3)的解即为公共解。

 显然,此题用方法3更合适:联立两个方程组,得到增广矩阵:
 \begin{equation*}[A|B]=
\begin{bmatrix}
  1 & 2 & 1 & 2 \\
  a & -1 & -2 &-3\\
  0 & a & 1 &1
\end{bmatrix}
 \end{equation*}
 讨论:
 $a=0$时:显然$r(A)=r(A|B)$,即线性方程组有解,即这两个线性方程组有公共解。

 $a\neq 0$时,对增广矩阵进一步进行高斯消元:
 \begin{align*}
\xrightarrow{\substack{r_{2}-a r_{1}}}
{
\begin{bmatrix}
  1 & 2 & 1 & 2 \\
  0 & -1-2a & -2-a &-3-2a\\
  0 & a & 1 &1
\end{bmatrix}
}\xrightarrow{\substack{r_{2}\Leftrightarrow r_{3}}}
{
\begin{bmatrix}
  1 & 2 & 1 & 2 \\
  0 & a & 1 &1\\
    0 & -1-2a & -2-a &-3-2a
\end{bmatrix}
}\xrightarrow{\substack{r_{3}+\frac{1+2a}{a} r_{3}}}
{
\begin{bmatrix}
  1 & 2 & 1 & 2 \\
  0 & a & 1 &1\\
    0 & 0 & -\frac{a^2-1}{a} &-\frac{2a^2+a-1}{a}
\end{bmatrix}
}
 \end{align*}
 当$-\frac{a^2-1}{a}=0$且$-\frac{2a^2+a-1}{a}\neq0$时,无解,此时没有公共解,解得$a=1$.所以$a=1$时无公共解,所以$a\neq1$。
\end{jie}

4. 设$\alpha_{1}=(a,1,1)^{T},\alpha_{2}=(1,b,-1)^{T},\alpha_{3}=(1,-2,c)^{T}$是正交向量组,则$a+b+c=$\underline{\hphantom{~~~~~~~~~~}}。

\begin{jie}
由题得:
\begin{equation*}
\begin{cases}
\alpha_1\alpha_2^T=a+b-1=0\\
\alpha_1\alpha_3^T=a-2+c=0\\
\alpha_2\alpha_3^T=1-2b-c=0
\end{cases}
~~~\Rightarrow~~~
\begin{cases}
a=1\\
b=0\\
c=1
\end{cases}~~~\Rightarrow~~~a+b+c=2
\end{equation*}
\end{jie}

5. 设3阶实对称矩阵$A$的特征值分别为$1,2,3$对应的特征向量分别为$\alpha_ {1}=(1,1,1)^{T},\alpha_{2}=(2,-1,-1)^{T},\alpha_{3}$,则$A$的对应于特征值3的一个特征向量$\alpha_{3}=$\underline{\hphantom{~~~~~~~~~~}}。

\begin{jie}
设$\alpha_3=[x_1,x_2,x_3]^T$,实对称矩阵对应于不同特征值的特征向量是正交的,所以:
\begin{equation*}
\begin{cases}
\alpha_{1}\alpha_3^T=x_1+x_2+x_3=0\\
\alpha_{2}\alpha_3^T=2x_1-x_2-x_3=0
\end{cases}~~~\Rightarrow~~~
\begin{cases}
x_1=0\\
x_2=-x_3
\end{cases}
\end{equation*}
令$x_{3}=-1$,有$\alpha_3=[0,1,-1]$
\end{jie}

6. 设
$
B=
\begin{bmatrix}
  1 & 2 & 4 \\
  0 & 2 & 6\\
  0 & 0 & \lambda
\end{bmatrix}
$,已知二次型$f(x)=x^{T}Bx$是正定的,则$\lambda$的取值范围为\underline{\hphantom{~~~~~~~~~~}}。

\begin{jie}
由题得:
\begin{equation*}
f(x)=x^{T}Bx=[x_1,x_2,x_3]\begin{bmatrix}
  1 & 2 & 4 \\
  0 & 2 & 6\\
  0 & 0 & \lambda
\end{bmatrix}
\begin{bmatrix}
x_1\\ x_2\\ x_3
\end{bmatrix}=\begin{bmatrix}
x_1&  2x_1+2x_2& 4x_1+6x_2+\lambda x_3
\end{bmatrix}\begin{bmatrix}
x_1\\ x_2\\ x_3
\end{bmatrix}=x_1^2+2x_2^2+\lambda x_3^2+2x_1x_2+4x_1x_3+6x_2x_3
\end{equation*}
所以其对应的二次型矩阵为:
$A=\begin{bmatrix}
  1 & 1 & 2 \\
  1 & 2 & 3\\
  2 & 3 & \lambda
\end{bmatrix}$,$A$为对称矩阵,如果$A$正定,则$|A|>0$,所以$|A|=\lambda-5>0~~\Rightarrow~~\lambda>5$.
\end{jie}

二、计算题

1. 若行列式$D=
\begin{vmatrix}
  1 & 2 & 3 & 4 \\
  0 & 3 & 4 & 6 \\
  0 & 4 & 1 & 2 \\
  0 & 2 & 2 & 2
\end{vmatrix}
$,求$A_{11}-2A_{21}+A_{31}-2A_{41}$,其中$A_{ij}$为元素$a_{ij}$的代数余子式。

\begin{jie}
\begin{align*}
A_{11}-2A_{21}+A_{31}-2A_{41}&=
\begin{vmatrix}
  \textcolor[rgb]{1.00,0.00,0.00}{1} & 2 & 3 & 4 \\
  \textcolor[rgb]{1.00,0.00,0.00}{-2} & 3 & 4 & 6 \\
  \textcolor[rgb]{1.00,0.00,0.00}{1} & 4 & 1 & 2 \\
  \textcolor[rgb]{1.00,0.00,0.00}{-2}  & 2 & 2 & 2
\end{vmatrix}=2
\begin{vmatrix}
  1 & 2 & 3 & 4 \\
  -2 & 3 & 4 & 6 \\
  1 & 4 & 1 & 2 \\
  -1 & 1 & 1 & 1
\end{vmatrix}\xlongequal{\substack{r_2+2r_1\\ r_3-r_1\\ r_4+r_1}}4\begin{vmatrix}
 7 & 10& 14 \\
1 & -1 & -1 \\
 3& 4 & 5
\end{vmatrix}\xlongequal{\substack{r_1-7r_2\\ r_3-3r_2}}4\begin{vmatrix}
 0 & 17& 21 \\
1 & -1 & -1 \\
 0& 7 & 8
\end{vmatrix}\\
& =4\times1\times(-1)^{1+2}(17\times8-7\times21)=44
\end{align*}
\end{jie}
%\xlongequal{def}
2. 设$
A=
\begin{bmatrix}
  1 & 2 & 3 \\
  0 & 1 & 3\\
  0 & 0 & 1
\end{bmatrix}
$,$B$为三阶矩阵,且满足方程$A^{*}BA=I+2A^{-1}B$,求矩阵$B$。

\begin{jie}
由题得:$|A|=1$,$A^*=|A|A^{-1}=A^{-1}$.对题中方程两边同时左乘$A$得:
\begin{align*}
BA&=A+2B\\
B&=A(A-2E)^{-1}=A=
\begin{bmatrix}
  1 & 2 & 3 \\
  0 & 1 & 3\\
  0 & 0 & 1
\end{bmatrix}
\begin{bmatrix}
  -1 & 2 & 3 \\
  0 & -1 & 3\\
  0 & 0 & -1
\end{bmatrix}^{-1}=\begin{bmatrix}
  -1 & -4 & -18 \\
  0 & -1 & -6\\
  0 & 0 & -1
\end{bmatrix}
\end{align*}

(求逆的过程略。$(A-2E)^{-1}=\begin{bmatrix}
  -1 & -2 & -9 \\
  0 & -1 & -3\\
  0 & 0 & -1
\end{bmatrix}$)
\end{jie}

3. 设向量组$\alpha_{1}=(3,1,4,3)^{T},\alpha_{2}=(1,1,2,1)^{T},\alpha_{3}=(0,1,1,0)^{T},\alpha_{4}=(2,2,4,2)^{T}$,求向量组的所有的极大线性无关组。

\begin{jie}
\begin{align*}
&(\alpha_1,\alpha_2,\alpha_3,\alpha_4)=
\begin{bmatrix}
   3 &1 & 0 &2\\
  1 & 1 & 1&2\\
  4 & 2 & 1&4\\
   3 &1 & 0 &2
\end{bmatrix}
\xrightarrow{\substack{r_{2}\leftrightarrow r_{1}}}
{
\begin{bmatrix}
  1 & 1 & 1&2 \\
 3 &1 & 0 &2 \\
  4 & 2 & 1&4\\
   3 &1 & 0 &2
\end{bmatrix}
}
\xrightarrow{\substack{r_{4}- r_{2}}}
{
\begin{bmatrix}
  1 & 1 & 1&2 \\
 3 &1 & 0 &2 \\
  4 & 2 & 1&4\\
   0 &0 & 0 &0
\end{bmatrix}
}
\xrightarrow{\substack{r_{2}-3 r_{1}\\ r_3-4r_1}}
{
\begin{bmatrix}
  1 & 1 & 1&2 \\
 0 &-2 & -3 &-4 \\
 0 &-2 & -3 &-4\\
   0 &0 & 0 &0
\end{bmatrix}
}\\
&\xrightarrow{\substack{r_{3}-r_{2}}}
{
\begin{bmatrix}
  1 & 1 & 1&2 \\
 0 &-2 & -3 &-4 \\
0 &0 & 0 &0\\
   0 &0 & 0 &0
\end{bmatrix}
}
\end{align*}
所以该向量组的秩为$2$,极大线性无关组为$(\alpha_1,\alpha_2),(\alpha_1,\alpha_3),(\alpha_1,\alpha_4),(\alpha_2,\alpha_3),(\alpha_3,\alpha_4)$.
\end{jie}

三、解答题

1. 令$\alpha=(1,1,0)^{T}$,实对称矩阵$A=\alpha\alpha^{T}$.

(1)把矩阵$A$相似对角化;

(2)求$|6I-A^{2017}|$.

\begin{jie}
由题得:$A=\alpha\alpha_{T}=(1,1,0)^{T}(1,1,0)=
\begin{bmatrix}
  1 & 1 & 0 \\
  1 & 1 & 0 \\
   0 & 0 & 0 \\
\end{bmatrix}
$。所以
\begin{equation*}
|\lambda E-A|=
\begin{vmatrix}
  \lambda-1 & -1 & 0 \\
  -1 & \lambda-1 & 0 \\
   0 & 0 & \lambda
\end{vmatrix}=\lambda[(\lambda-1)^2-1]=0~~~\Rightarrow~~~\lambda_{1}=2,\lambda_{2}=\lambda_3=0
\end{equation*}

$\lambda_{1}=2$时:
\begin{align*}
[\lambda E-A]=
\begin{bmatrix}
  1 & -1 & 0 \\
  -1 & 1 & 0 \\
   0 & 0 & 2
\end{bmatrix}
\xrightarrow{\text{高斯消元,步骤略}}&
{
\begin{bmatrix}
  1 & -1 & 0 \\
  0 & 0 & 1 \\
   0 & 0 & 0
\end{bmatrix}
}~~~\Rightarrow~~~\alpha_{1}=
\begin{bmatrix}
  1  \\
   1 \\
    0
\end{bmatrix}
\end{align*}
可以看出,$\lambda_1=2$时:代数重数等于几何重数。

$\lambda_{2}=\lambda_3=0$时:
\begin{align*}
[\lambda E-A]=
\begin{bmatrix}
  -1 & -1 & 0 \\
  -1 & -1 & 0 \\
   0 & 0 & 0
\end{bmatrix}
\xrightarrow{\text{高斯消元,步骤略}}&
{
\begin{bmatrix}
  1 & 1 & 0 \\
  0 & 0 & 0 \\
   0 & 0 & 0
\end{bmatrix}
}~~~\Rightarrow~~~\alpha_{2}=
\begin{bmatrix}
  1  \\
   -1 \\
    0
\end{bmatrix},
\alpha_{2}=
\begin{bmatrix}
0  \\
0 \\
1
\end{bmatrix}
\end{align*}
可以看出,$\lambda_2=\lambda_2=0$时:代数重数等于几何重数。

所以$A$可以相似对角化:即存在可逆矩阵$P$,使得$P^{-1}AP=\Lambda=
\begin{bmatrix}
  0 &  &   \\
   &  0&   \\
   &  &  2
\end{bmatrix}
$,其中$P=
\begin{bmatrix}
  1 &  0&  1 \\
  -1 &  0&  1 \\
  0 &  1&  0
\end{bmatrix}
$

(2)由特征值的性质:$6I-A^{2017}$的特征值为:$6-\lambda_{i}^{2017}$,所以
\begin{equation*}
|6I-A^{2017}|=\prod_{i=1}^{3}(6-\lambda_ {i}^{2017})=36\times(6-2^{2017})
\end{equation*}
\end{jie}

2. 已知实对称矩阵$A=
\begin{bmatrix}
  a & -1 & 4 \\
  -1 & 3 & b\\
  4 & b & 0
\end{bmatrix}
$与
$B=
\begin{bmatrix}
  2 & ~ & ~ \\
  ~ & -4 & ~\\
  ~ & ~ & 5
\end{bmatrix}
$相似。

(1)求矩阵$A$;

(2)求正交线性变换$x=Qy$,把二次型$f(x)=x^{T}Ax$化为标准型.

\begin{jie}
对于对角矩阵,其特征值为对角线上的元素。因为$A$与$B$相似,所以$A$与$B$有相同的特征值。

(1)由特征值的性质
\begin{equation*}
\begin{cases}
trace(A)=a+3+0=\sum\limits_{i=1}^{3}\lambda_i=2-4+5=3\\
|A|=(-1)\times(-1)^{1+2}
\begin{vmatrix}
  -1 & b\\
  4 & 0
\end{vmatrix}+4
\begin{vmatrix}
  -1 & 3\\
  4  & b
\end{vmatrix}=-8b-48=\prod\limits_{i=1}^{3}\lambda_i=2\times(-4)\times5=-40
\end{cases}~~\Rightarrow~~\begin{cases}a=0\\ b=-1\end{cases}
\end{equation*}

(2)

$\lambda_1=2$时:
\begin{equation*}
[\lambda E-A]=
\begin{bmatrix}
  2 & 1 & -4\\
  1 & -1 &1\\
  -4 & 1 & 2
\end{bmatrix}~~~\Rightarrow~~~
\begin{cases}
 x_1=x_3\\
 x_2=2x_3
\end{cases}
\end{equation*}
取$x_3=1$得$\alpha_1=[1,2,1]^T$。

$\lambda_2=-4$时:
\begin{equation*}
[\lambda E-A]=
\begin{bmatrix}
  -4 & 1 & -4\\
  1 & -7 &1\\
  -4 & 1 & -4
\end{bmatrix}~~~\Rightarrow~~~
\begin{cases}
 x_1=-x_3\\
 x_2=0
\end{cases}
\end{equation*}
取$x_3=-1$得$\alpha_2=[1,0,-1]^T$。

$\lambda_3=5$时:
\begin{equation*}
[\lambda E-A]=
\begin{bmatrix}
  5 & 1 & -4\\
  1 & 2 &1\\
  -4 & 1 & 5
\end{bmatrix}~~~\Rightarrow~~~
\begin{cases}
 x_1=x_3\\
 x_2=-x_3
\end{cases}
\end{equation*}
取$x_3=1$得$\alpha_2=[1,-1,1]^T$。因为对称矩阵对应于不同特征值的特征向量正交,所以$[\alpha_1,\alpha_2,\alpha_3]$为正交向量组。

单位化:
\begin{equation*}
\begin{cases}
\gamma_1=\dfrac{\alpha_1}{\|\alpha_1\|}=\left[\dfrac{1}{\sqrt{6}},\dfrac{2}{\sqrt{6}},\dfrac{1}{\sqrt{6}}\right]^T\\[2mm]
\gamma_2=\dfrac{\alpha_2}{\|\alpha_2\|}=\left[\dfrac{1}{\sqrt{2}},0,-\dfrac{1}{\sqrt{2}}\right]^T\\[2mm]
\gamma_3=\dfrac{\alpha_3}{\|\alpha_3\|}=\left[\dfrac{1}{\sqrt{3}},-\dfrac{1}{\sqrt{3}},\dfrac{1}{\sqrt{3}}\right]^T
\end{cases}~~~\Rightarrow~~~Q=
\begin{bmatrix}
\dfrac{1}{\sqrt{6}}&\dfrac{1}{\sqrt{2}}&\dfrac{1}{\sqrt{3}}\\[2mm]
\dfrac{2}{\sqrt{6}}&0&-\dfrac{1}{\sqrt{3}}\\[2mm]
\dfrac{1}{\sqrt{6}}&-\dfrac{1}{\sqrt{2}}&\dfrac{1}{\sqrt{3}}
\end{bmatrix}
\end{equation*}
所以$f$可经正交变换$x=Qy$化为标准型:
\begin{equation*}
  f=2y_{1}^2-4y_{2}^2+5y_{3}^2
\end{equation*}
\end{jie}

3.在对观测数据拟合的时候经常遇到线性方程组$Ax=b$是矛盾方程的情形,是没有解的。此时我们转而解$A^{T}Ax=A^{T}b$,我们称$A^{T}Ax=A^{T}b$是原线性方程组的正规方程组。称正规方程组的解为原方程组的最小二乘解。设
$
A=
\begin{bmatrix}
  1 & 1 & 0\\
  1 & 1 & 0\\
  1 & 0 & 1\\
  1 & 1 & 1
\end{bmatrix},b=
\begin{bmatrix}
1 \\ 3 \\ 8 \\ 2
\end{bmatrix}
$.

(1)证明$Ax=b$无解;

(2)求$Ax=b$的最小二乘解。

\begin{jie}
(1)
由题得:
\begin{equation*}
[A|B]\xrightarrow{\substack{r_2-r_1\\r_3-r_1 \\ r_4-r_1}}
{
\begin{bmatrix}
  1 & 1 & 0 & 1\\
  0 & 0 & 0 &2 \\
  0 & -1 & 1 & 7\\
  0 & 0& 1 &1
\end{bmatrix}
}
\end{equation*}
可以看出$r(A)=3\neq r(A|B)=4$,所以$Ax=b$无解。

(2)
\begin{gather*}
  A^TA=\begin{bmatrix}
  1 & 1 & 1& 1\\
  1 & 1 & 0& 1\\
  0 & 0 & 1& 1
\end{bmatrix}\begin{bmatrix}
  1 & 1 & 0\\
  1 & 1 & 0\\
  1 & 0 & 1\\
  1 & 1 & 1
\end{bmatrix}=
\begin{bmatrix}
  4 & 3 & 2\\
  3 & 3 & 1\\
  2 & 1 & 2
\end{bmatrix}\\
A^Tb=\begin{bmatrix}
  1 & 1 & 1& 1\\
  1 & 1 & 0& 1\\
  0 & 0 & 1& 1
\end{bmatrix}\begin{bmatrix}
1 \\ 3 \\ 8 \\ 2
\end{bmatrix}=
\begin{bmatrix}
14 \\ 6 \\ 10
\end{bmatrix}
\end{gather*}
高斯消元步骤略(考试必须写上)。
最后解得:$x_1=8,x_2=-6,x_3=0$
\end{jie}


四、证明题

1. 已知$\alpha_{1},\alpha_{2},\alpha_{3}$是线性无关的向量组,若$\alpha_{1},\alpha_{2},\alpha_{3},\beta$线性相关,证明$\beta$可以由$\alpha_{1},\alpha_{2},\alpha_{3}$线性表示并且表示方法唯一。

\begin{zhengming}
$\alpha_{1},\alpha_{2},\alpha_{3},\beta$线性相关,则存在一组不全为0的$l_{i}(1\leq i\leq3)$和$l$,使得
\begin{equation*}
l_ {1}\alpha_{1}+l_{2}\alpha_{2}+l_{3}\alpha_{3}+l\beta=0\tag{$1$}
\end{equation*}
若$l=0$,则$l_ {1}\alpha_{1}+l_{2}\alpha_{2}+l_{3}\alpha_{3}+l\beta=l_ {1}\alpha_{1}+l_{2}\alpha_{2}+l_{3}\alpha_{3}=0$,此时$\alpha_{1},\alpha_{2},\alpha_{3}$线性相关,与题中的条件矛盾,所以$l\neq0$。所以$(1)$式可变形为
\begin{equation*}
\beta=-\frac{1}{l}(l_ {1}\alpha_{1}+l_{2}\alpha_{2}+l_{3}\alpha_{3})
\end{equation*}
即$\beta$可以由$\alpha_{1},\alpha_{2},\alpha_{3}$线性表示。

$\beta$可以由$\alpha_{1},\alpha_{2},\alpha_{3}$线性表示,不妨设任意两组不全为0的数$m_{i},n_i,(1\leq i\leq3)$,使得
\begin{gather*}
\beta=m_{1}\alpha_{1}+m_{2}\alpha_{2}+m_{3}\alpha_{3}\tag{$2$}\\
\beta=n_{1}\alpha_{1}+n_{2}\alpha_{2}+n_{3}\alpha_{3}\tag{$3$}
\end{gather*}
$(2)$式减$(3)$式:$0=(m_{1}-n{1})\alpha_{1}+(m_{2}-n{2})\alpha_{2}+(m_{3}-n{3})\alpha_{3}$,因为$\alpha_{1},\alpha_{2},\alpha_{3}$线性无关,所以有
$m_{1}-n{1}=0,m_{2}-n{2}=0,m_{3}-n{3}=0$,即$m_{1}=n{1},m_{2}=n{2},m_{3}=n{3}$,由于$m_{i}$和$n_{i}$的任意性,所以可证得表示方法唯一。
\end{zhengming}

2. 已知$A,B$是同阶实对称矩阵。

(1)证明如果$A\~{}B$,则$A\simeq B$,也就是相似一定合同;

(2)举例说明反过来不成立。

\begin{zhengming}
(1)因为$A\~{}B$,所以$A,B$具有相同的特征值,记为$\lambda_i,(1\leq i\leq n)$。对于实对称矩阵$A$存在正交矩阵$Q$,使得$Q^{-1}AQ_1$为对角矩阵。即存在正交矩阵$Q_{1}$,使得$Q_1^{-1}AQ_1=\Lambda=
\begin{bmatrix}
  \lambda_1 & & \\
   & \ddots &\\
   &&\lambda_n
\end{bmatrix}
$,对于正交矩阵$Q_1$,有$Q_{1}^{-1}=Q_1^T$,即$Q_1^{T}AQ_1=\Lambda$,所以$A$合同于$\Lambda$,同理$B$合同于$\Lambda$,所以$A$合同于$B$。

(2)反过来描述:$A,B$是同阶实对称矩阵,$A\simeq B$,则$A\~{}B$。

由惯性定理(\textcolor[rgb]{1.00,0.00,0.00}{157页})知:
如果:$A=
\begin{bmatrix}
  1 &   \\
    & 2
\end{bmatrix},B=\begin{bmatrix}
  1 &   \\
    & 3
\end{bmatrix}
$,$A$,$B$为对角阵,且$A\simeq B$,但$A$和$B$的特征值不同,即$A$与$B$不相似。
\end{zhengming}
\newpage
\hphantom{~~}\hfill {\zihao{4}\heiti 2017-2018年第一学期} \hfill\hphantom{~~}

一、填空题

1. 设$A_{ij}$是三阶行列式$
D=
\begin{vmatrix}
  2 & 2 & 2\\
  1 & 2 & 3\\
  4 & 5 & 6
\end{vmatrix}
$第$i$行第$j$列元素的代数余子式,则$A_{31}+A_{32}+A_{33}=$\underline{\hphantom{~~~~~~~~~~}}。

\begin{jie}
\begin{align*}
A_ {31}+A_{32}+A_{33}=\begin{vmatrix}
  2 & 2 & 2\\
  1 & 2 & 3\\
  1 & 1 & 1
\end{vmatrix}=0
\end{align*}
\end{jie}

2. 设$
A=
\begin{bmatrix}
  1 & 1 & 1 \\
  0 & 1 & 1 \\
  2 & 3 & 3
\end{bmatrix},
B=\begin{bmatrix}
    1\\ 2\\ 0
  \end{bmatrix}
  \begin{bmatrix}
  1 & 2&3
  \end{bmatrix}
$,则$r(A+AB)=$\underline{\hphantom{~~~~~~~~~~}}。

\begin{jie}
由题得:$|A|=0$(第一行减第二行然后按第一行展开)
\begin{equation*}
B=\begin{bmatrix}
    1\\ 2\\ 0
  \end{bmatrix}
  \begin{bmatrix}
  1 & 2&3
  \end{bmatrix}
  =\begin{bmatrix}
     1 & 2 & 3\\
2 & 4 & 6\\
0 & 0 & 0
   \end{bmatrix}
\end{equation*}
\begin{align*}
&C=A+AB=
\begin{bmatrix}
  1 & 1 & 1 \\
  0 & 1 & 1 \\
  2 & 3 & 3
\end{bmatrix}+\begin{bmatrix}
  1 & 1 & 1 \\
  0 & 1 & 1 \\
  2 & 3 & 3
\end{bmatrix}\begin{bmatrix}
     1 & 2 & 3\\
2 & 4 & 6\\
0 & 0 & 0
   \end{bmatrix}=\begin{bmatrix}
  1 & 1 & 1 \\
  0 & 1 & 1 \\
  2 & 3 & 3
\end{bmatrix}+\begin{bmatrix}
  3 & 6 & 9 \\
  2 & 4 & 6 \\
  8 & 16 & 24
\end{bmatrix}=
\begin{bmatrix}
  4 & 7 & 10 \\
  2 & 5 & 7 \\
  10 & 19 & 27
\end{bmatrix}
\\
&C\xrightarrow{r_{1}\leftrightarrow r_{2}}
{
\begin{bmatrix}
  2 & 5 & 7\\
  4 & 7 & 10 \\
  10 & 19 & 27
\end{bmatrix}
}\xrightarrow{\substack{r_{2}-2 r_{1}\\r_{3}-5 r_{1}}}
{
\begin{bmatrix}
  2 & 5 & 7\\
  0& -3 & -4 \\
  0 & -6 & -8
\end{bmatrix}
}\xrightarrow{\substack{r_{3}-2 r_{2}}}
{
    \begin{bmatrix}
  2 & 5 & 7\\
  0& -3 & -4 \\
  0 & 0 & 0
\end{bmatrix}
}
\end{align*}
%\begin{equation*}
%A\xrightarrow{r_{3}-2r_{1}}
%{
%\begin{bmatrix}
%  1 & 1 & 1 \\
%  0 & 1 & 1 \\
%  0 & 1 & 0
%\end{bmatrix}
%}\xrightarrow{r_{3}-r_{2}}
%{
%\begin{bmatrix}
%  1 & 1 & 1 \\
%  0 & 1 & 1 \\
%  0 & 0 & -1
%\end{bmatrix}
%}
%\end{equation*}
%$r(A)=3$,满秩。所以$r(A+AB)=r(A(E+B))=r(E+B)$
%\begin{align*}
%B=&\begin{bmatrix}
%    1\\ 2\\ 0
%  \end{bmatrix}
%  \begin{bmatrix}
%  1 & 2&3
%  \end{bmatrix}
%  =\begin{bmatrix}
%     1 & 2 & 3\\
%2 & 4 & 6\\
%0 & 0 & 0
%   \end{bmatrix}
%  \\
%E+B=&
%\begin{bmatrix}
%2 & 2 & 3\\
%2 & 5 & 6\\
%0 & 0 & 1
%\end{bmatrix}\xrightarrow{r_{2}-r_{1}}
%{
%\begin{bmatrix}
%2 & 2 & 3\\
%0 & 3 & 3\\
%0 & 0 & 1
%\end{bmatrix}
%}
%\end{align*}
%所以$r(E+B)=3$,
所以$r(A+AB)=2$
\end{jie}

3. 设$
A=
\begin{bmatrix}
  2 & 0 & 0 \\
  1 & 2 & 0 \\
  1 & 2 & 2
\end{bmatrix}
$,记$A^*$是$A$的伴随矩阵,则$(A^{*})^{-1}=$\underline{\hphantom{~~~~~~~~~~}}。

\begin{jie}
$(A^ {*})^{-1}=(|A|A^{-1})^{-1}=\dfrac{A}{|A|}$,由题得:$|A|=8$
\end{jie}

4. 已知3阶方阵$A$的秩为2,设$\alpha_ {1}=(2,2,0)^{T},\alpha_{2}=(3,3,1)^{T}$是非齐次线性方程组$Ax=b$的解,则导出$Ax=0$的基础解系为\underline{\hphantom{~~~~~~~~~~}}。

\begin{jie}
因为$\alpha_ {1},\alpha_{2}$是非齐次线性方程组$Ax=b$的解,所以$A\alpha_1=b,A\alpha_2=b$,且$\alpha_ {1},\alpha_{2}$不相等,所以$\alpha_1-\alpha_2$是$AX=0$的基础解系。(实际上$k(\alpha_1-\alpha_2)$都是导出组的基础解系。)
\end{jie}

5. 若3阶矩阵$A$相似于$B$,矩阵$A$的特征值是1,2,3那么行列式$|2B+I|=$\underline{\hphantom{~~~~~~~~~~}}。(其中$I$是3阶单位矩阵)

\begin{jie}
$A$相似于$B$,所以$A$与$B$的特征值相等。所以$2B+I$的特征值为$2\lambda_i+1$,所以$|2B+I|=\prod\limits_{i=1}^{3}(2\lambda_i+1)=105$
\end{jie}

6. 设二次型$f(x_{1},x_{2},x_{3})=2x_{1}^{2}+x_{2}^{2}+x_{3}^{2}+2x_{1}x_{2}+2tx_{2}x_{3}$的秩为2,则$t=$\underline{\hphantom{~~~~~~~~~~}}。

\begin{jie}
二次型对应的矩阵为:
\begin{equation*}
A=
\begin{bmatrix}
  2 & 1 & 0\\
  1 & 1 & t\\
  0 & t &1
\end{bmatrix}
\end{equation*}
$A$的秩为2,即$|A|=0$,解得:$t=\pm\dfrac{1}{\sqrt{2}}$
\end{jie}

二、计算题

1. 计算行列式$D=
\begin{vmatrix}
  3 & 1 & -1 & 2 \\
  -5 & 1 & 3 & -4 \\
  2 & 0 & 1 & -1\\
   1 & -5 & 3 & -3
\end{vmatrix}
.$

\begin{jie}
\begin{align*}
D&\xlongequal{\substack{c_1+3c_3\\ c_2+c_3\\ c_4+c_3}}
\begin{vmatrix}
  0 & 0 & -1 & 0 \\
  4 & 4 & 3 & 2 \\
  5 & 1 & 1 & 1\\
   10 & -2 & 3 & 3
\end{vmatrix}=(-1)\times(-1)^{1+3}\begin{vmatrix}
  4 & 4 &  2 \\
  5 & 1 &  1\\
   10 & -2  & 3
\end{vmatrix}=-2
\begin{vmatrix}
  2 & 2 &  1 \\
  5 & 1 &  1\\
   10 & -2  & 3
\end{vmatrix}\xlongequal{\substack{c_1-2c_3\\ c_2-2c_3}}
-2\begin{vmatrix}
  0 & 0 &  1 \\
  3 & -1 &  1\\
   4 & -8  & 3
\end{vmatrix}\\ &=-2\times1\times(-1)^{1+3}\times[3\times(-8)-(-1)\times4]=40
\end{align*}
\end{jie}

2.解矩阵方程$(2I-B^{-1}A)X^{T}=B^{-1}$,其中$I$是3阶单位矩阵,$X^{T}$是3阶矩阵$X$的转置矩阵,$A=
\begin{bmatrix}
  1 & 2 & -3 \\
  0 & 1 & 2 \\
  0 & 0 & 1
\end{bmatrix},B=
\begin{bmatrix}
  1 & 2 & 0 \\
  0 & 1 & 2 \\
  0 & 0 & 1
\end{bmatrix}
$.

\begin{jie}
由题得:$(2I-B^{-1}A)X^{T}=B^{-1}$,所以:
\begin{align*}
X^{T}&=(2I-B^{-1}A)^{-1}B^{-1}=[B(2I-B^{-1}A)]^{-1}\\
&=(2B-A)^{-1}
\end{align*}
\begin{align*}
&C=2B-A=\begin{bmatrix}
  1 & 2 & 3 \\
  0 & 1 & 2 \\
  0 & 0 & 1
\end{bmatrix}
\end{align*}

$C^{-1}=\begin{bmatrix}
  1 & -2 & 1 \\
  0 & 1 & -2 \\
  0 & 0 & 1
\end{bmatrix}$
所以$X=C^{T}=\begin{bmatrix}
  1 & 0 & 0 \\
  -2 & 1 & 0 \\
  1 & -2 & 1
\end{bmatrix}$
\end{jie}

3. 求线性方程组
$
\begin{cases}
 2x_{1}-x_{2}+4x_{3}-3x_{4}=-4\\
 x_{1}+x_{3}-x_{4}=-3\\
 3x_{1}+x_{2}+x_{3}=1\\
 7x_{1}+7x_{3}-3x_{4}=3\\
\end{cases}
$
的通解。

\begin{jie}
增广矩阵
\begin{align*}
&\left[
\begin{array}{c:c}
 \begin{matrix}
   2 & -1 & 4 &-3 \\
   1 & 0 & 1 &-1 \\
   3&1&1&0\\
   7 & 0 & 7 &-3\\
 \end{matrix}
 &
  \begin{matrix}
   -4 \\
   -3\\
   1\\
   3\\
 \end{matrix}
\end{array}
\right]
\xrightarrow{\substack{r_{2}-\frac{1}{2}r_{1} \\ r_{3}-\frac{3}{2}r_{1} \\ r_{4}-\frac{7}{2}r_{1}}}
{
\left[
\begin{array}{c:c}
 \begin{matrix}
   2 & -1 & 4 &-3 \\
   0 & \frac{1}{2} & -1 &\frac{1}{2} \\
   0& \frac{5}{2} & -5 &\frac{9}{2} \\
   0 & \frac{7}{2} & -7 &\frac{15}{2}\\
 \end{matrix}
 &
  \begin{matrix}
   -4 \\
   -1\\
   7\\
   17\\
 \end{matrix}
\end{array}
\right]
}
\xrightarrow{\substack{r_{3}-5r_{2} \\ r_{4}-7r_{2}}}
{
\left[
\begin{array}{c:c}
 \begin{matrix}
   2 & -1 & 4 &-3 \\
   0 & \frac{1}{2} & -1 &\frac{1}{2} \\
   0&0&&0&2\\
   0 &0& 0 &4\\
 \end{matrix}
 &
  \begin{matrix}
   -4 \\
   -1\\
   12\\
   24\\
 \end{matrix}
\end{array}
\right]
}\\
\xrightarrow{r_4-2r_3}
&{
\left[
\begin{array}{c:c}
 \begin{matrix}
   2 & -1 & 4 &-3 \\
   0 & \frac{1}{2} & -1 &\frac{1}{2} \\
   0&0&&0&2\\
   0 &0& 0 &0\\
 \end{matrix}
 &
  \begin{matrix}
   -4 \\
   -1\\
   12\\
   0\\
 \end{matrix}
\end{array}
\right]
}
\xrightarrow{\substack{r_{1}\times\frac{1}{2} \\ r_{2}\times 2 \\ r_{3}\times \frac{1}{2}}}
{
\left[
\begin{array}{c:c}
 \begin{matrix}
   1 & -\frac{1}{2} & 2 &-\frac{3}{2} \\
   0 & 1 & -2 &1 \\
   0 &0& 0 &1\\
    0 &0& 0 &0\\
 \end{matrix}
 &
  \begin{matrix}
   -2 \\
   -2\\
   6\\ 0
 \end{matrix}
\end{array}
\right]
}
\xrightarrow{\substack{r_{2}-r_{3}\\r_{1}+\frac{3}{2} r_{3}}}
{
\left[
\begin{array}{c:c}
 \begin{matrix}
   1 & -\frac{1}{2} & 2 &0 \\
   0 & 1 & -2 &0 \\
   0 &0& 0 &1\\  0 &0& 0 &0\\
 \end{matrix}
 &
  \begin{matrix}
   7 \\
   -8\\
   6\\ 0
 \end{matrix}
\end{array}
\right]
}\\
\xrightarrow{r_{1}+\frac{1}{2} r_{2}}&
{
\left[
\begin{array}{c:c}
 \begin{matrix}
   1 & 0 & 1 &0 \\
   0 & 1 & -2 &0 \\
   0 &0& 0 &1\\  0 &0& 0 &0
 \end{matrix}
 &
  \begin{matrix}
   3 \\
   -8\\
   6\\ 0
 \end{matrix}
\end{array}
\right]
}
\end{align*}
由最简阶梯型矩阵可以看出:
\begin{equation*}
  x_{1}=3-x_{3}~~x_{2}=2x_{3}-8~~x_{3}=x_{3}~~x_{4}=6
\end{equation*}
令$x_{3}=C,C\in R$,则
\begin{equation*}
x=
 \begin{bmatrix}
   3-C \\
   2C-8 \\
   C\\
   6
 \end{bmatrix}
 =
  \begin{bmatrix}
   3 \\
   -8 \\
   0\\
   6
 \end{bmatrix}
 +
  \begin{bmatrix}
   -1 \\
   2 \\
   1\\
   0
 \end{bmatrix}C
 ~,~C\in R
\end{equation*}
\end{jie}

三、解答题

1. 设1为矩阵$A=
\begin{bmatrix}
  1 & 2 & 3 \\
  x & 1 & -1 \\
  1 & 1 & x
\end{bmatrix}
$的特征值,其中$x>1$.

(1)求$x$及$A$的其他特征值。

(2)判断$A$能否对角化,若能对角化,写出相应的对角矩阵$\Lambda$。

\begin{jie}
(1)设$\alpha_1$为特征值$1$对应的特征向量,所以$\alpha_1\neq 0$由题得:$A\alpha_1=\alpha_1$,即$(A-E)\alpha_1=0$,即$(A-E)x=0$有非零解。所以由存在唯一性定理:$|A-E|=0$,所以

\begin{equation*}
|A-E|=
\begin{vmatrix}
  0 & 2 & 3 \\
  x & 0 & -1 \\
  1 & 1 & x-1
\end{vmatrix}=-x
\begin{vmatrix}
 2 & 3 \\
1 & x-1
\end{vmatrix}+
\begin{vmatrix}
 2 & 3 \\
0 & -1
\end{vmatrix}=(2x-1)(x-2)=0
\end{equation*}
由题得:$x>1$,所以解得$x=2$。

(2)把$x=2$代入得:
\begin{align*}
|\lambda E-A|&=
\begin{vmatrix}
  \lambda-1 & -2 & -3 \\
  -2 & \lambda-1 & 1 \\
  -1 & -1 & \lambda-2
\end{vmatrix}\xlongequal{c_{1}-c_{2}}
\begin{vmatrix}
  \lambda+1 & -2 & -3 \\
  -(\lambda+1) & \lambda-1 & 1 \\
  0 & -1 & \lambda-2
\end{vmatrix}=(\lambda+1)
\begin{vmatrix}
  1 & -2 & -3 \\
  -1 & \lambda-1 & 1 \\
  0 & -1 & \lambda-2
\end{vmatrix}\xlongequal{r_{2}+r_{1}}
(\lambda+1)
\begin{vmatrix}
  1 & -2 & -3 \\
  0 & \lambda-3 & -2 \\
  0 & -1 & \lambda-2
\end{vmatrix}\\
&=(\lambda+1)[(\lambda-3)(\lambda-2)-2]=0
\end{align*}
解得:$\lambda_1=1,\lambda_2=4,\lambda_3=-1$。

因为$A$为三阶,并且有3个不同的特征值,所以可以相似对角化,
$
\Lambda=
\begin{bmatrix}
  1 & & \\
    & 4 &\\
    &&-1
\end{bmatrix}
$。\textcolor[rgb]{1.00,0.00,0.00}{(不唯一,只要对角线元素是这三个就可以)}
\end{jie}

2. 设$f(x_ {1},x_{2},x_{3})=2x_{1}^{2}+2x_{2}^{2}+3x_{3}^{2}+2x_{1}x_{2}$。

(1)写出该二次型的矩阵$A$;

(2)求正交矩阵$Q$使得$Q^{T}AQ=Q^{-1}AQ$为对角型矩阵;

(3)给出正交变换,化该二次型为标准型。

\begin{jie}
(1)由题得:
\begin{equation*}
A=
\begin{bmatrix}
  2 & 1 & 0 \\
  1  & 2 & 0\\
  0 & 0 & 3
\end{bmatrix}
\end{equation*}

(2)
\begin{equation*}
|\lambda E-A|=
\begin{vmatrix}
  \lambda-2 & -1 & 0 \\
  -1  & \lambda-2 & 0\\
  0 & 0 & \lambda-3
\end{vmatrix}=(\lambda-3)[(\lambda-2)^2-1]=0~~~\Rightarrow ~~~\lambda_1=\lambda_2=3,\lambda_3=1
\end{equation*}

$\lambda=3$时:
\begin{equation*}
[\lambda E-A]=
\begin{bmatrix}
  1 & -1 & 0 \\
  -1  & 1 & 0\\
  0 & 0 & 0
\end{bmatrix}~~~\Rightarrow~~~
\begin{cases}
 x_1=x_2\\
 x_3\in R
\end{cases}
\end{equation*}
分别取$[x_2,x_3]^T=[1,0]^T$和$[0,1]^T$得$\alpha_1=[1,1,0]^T,\alpha_2=[0,0,1]^T$。

$\lambda=1$时:
\begin{equation*}
[\lambda E-A]=
\begin{bmatrix}
  -1 & -1 & 0 \\
  -1  & -1 & 0\\
  0 & 0 & -2
\end{bmatrix}~~~\Rightarrow~~~
\begin{cases}
 x_1=-x_2\\
 x_3=0
\end{cases}
\end{equation*}
取$x_2=-1$得$\alpha_2=[1,-1,0]^T$。

因为对称矩阵对应于不同特征值的特征向量正交,所以$[\alpha_1,\alpha_2,\alpha_3]$为正交向量组。

单位化:
\begin{equation*}
\begin{cases}
\gamma_1=\dfrac{\alpha_1}{\|\alpha_1\|}=\left[\dfrac{1}{\sqrt{2}},\dfrac{1}{\sqrt{2}},0\right]^T\\[2mm]
\gamma_2=\dfrac{\alpha_2}{\|\alpha_2\|}=\left[0,0,1\right]^T\\[2mm]
\gamma_3=\dfrac{\alpha_3}{\|\alpha_3\|}=\left[\dfrac{1}{\sqrt{2}},-\dfrac{1}{\sqrt{2}},0\right]^T
\end{cases}~~~\Rightarrow~~~Q=
\begin{bmatrix}
\dfrac{1}{\sqrt{2}}&0&\dfrac{1}{\sqrt{2}}\\[2mm]
\dfrac{1}{\sqrt{2}}&0&-\dfrac{1}{\sqrt{2}}\\[2mm]
0&1&0
\end{bmatrix}
\end{equation*}
所以$f$可经正交变换$x=Qy$化为标准型:
\begin{equation*}
  f=3y_{1}^2+3y_{2}^2+y_{3}^2
\end{equation*}
\end{jie}

3. 已知$\alpha_ {1}=(1,4,0,2)^{T},\alpha_{2}=(2,7,1,3)^{T},\alpha_{3}=(0,1,-1,a)^{T}$及$\beta_{4}=(3,10,b,4)^{T}$.

(1)$a,b$为何值时,$\beta$不能表示成$\alpha_{1},\alpha_{2},\alpha_{3}$的线性组合?

(2)$a,b$为何值时,$\beta$可由$\alpha_{1},\alpha_{2},\alpha_{3}$线性表示?并写出该表达式。

\begin{jie}
记$A=[\alpha_{1},\alpha_{2},\alpha_{3}]$,则
\begin{align*}
[A|\beta]=&
\left[
\begin{array}{c:c}
\begin{matrix}
1 & 2 & 0 \\
  4 & 7 & 1 \\
  0 & 1 & -1\\
  2& 3&a
\end{matrix}&
\begin{matrix}
3  \\
10\\
b\\
4
\end{matrix}
\end{array}
\right]
\xrightarrow{\substack{r_{2}-4r_{1}\\ r_{4}-2r_{1}}}
{
\left[
\begin{array}{c:c}
\begin{matrix}
1 & 2 & 0 \\
  0 & -1 & 1 \\
  0 & 1 & -1\\
  0& -1&a
\end{matrix}&
\begin{matrix}
3  \\
-2\\
b\\
2
\end{matrix}
\end{array}
\right]
}
\xrightarrow{\substack{r_{3}+r_{2}\\ r_{4}-r_{2}}}
{
\left[
\begin{array}{c:c}
\begin{matrix}
1 & 2 & 0 \\
  0 & -1 & 1 \\
  0 & 0 & 0\\
  0& 0&a-1
\end{matrix}&
\begin{matrix}
3  \\
-2\\
b-2\\
0
\end{matrix}
\end{array}
\right]
}\\
\xrightarrow{r_{3}\leftrightarrow r_{2}}&
{
\left[
\begin{array}{c:c}
\begin{matrix}
1 & 2 & 0 \\
  0 & -1 & 1 \\
   0& 0&a-1\\
 0 & 0 & 0
\end{matrix}&
\begin{matrix}
3  \\
-2\\
0\\
b-2
\end{matrix}
\end{array}
\right]
}
\end{align*}

(1)可以看出$b\neq2,a\in R$时,$Ax=\beta$无解,即$\beta$不能表示成$\alpha_{1},\alpha_{2},\alpha_{3}$的线性组合。

(2)$b=2$时,$\beta$可由$\alpha_{1},\alpha_{2},\alpha_{3}$线性表示。

当$a\neq1$,$r(A)=r(A,\beta)=3$,此时:$Ax=\beta$有唯一解,即$\beta$可由$\alpha_{1},\alpha_{2},\alpha_{3}$线性表示的方法唯一。
\begin{align*}
\xrightarrow{r_{3}\times \frac{1}{a-1}}
{
\left[
\begin{array}{c:c}
\begin{matrix}
1 & 2 & 0 \\
  0 & -1 & 1 \\
   0& 0&1\\
 0 & 0 & 0
\end{matrix}&
\begin{matrix}
3  \\
-2\\
0\\
0
\end{matrix}
\end{array}
\right]
}
\xrightarrow{r_2-r_{3}}
{
\left[
\begin{array}{c:c}
\begin{matrix}
1 & 2 & 0 \\
  0 & -1 & 0 \\
   0& 0&1\\
 0 & 0 & 0
\end{matrix}&
\begin{matrix}
3  \\
-2\\
0\\
0
\end{matrix}
\end{array}
\right]
}\xrightarrow{r_1+2r_{2}}
{
\left[
\begin{array}{c:c}
\begin{matrix}
1 & 0 & 0 \\
  0 & -1 & 0 \\
   0& 0&1\\
 0 & 0 & 0
\end{matrix}&
\begin{matrix}
-1  \\
-2\\
0\\
0
\end{matrix}
\end{array}
\right]
}
\end{align*}
此时$Ax=\beta$的解为$x_{1}=-1,x_{2}=2,x_{3}=0$,所以$\beta=x_{1}\alpha_{1}+x_{2}\alpha_{2}+x_{3}\alpha_{3}=-\alpha_{1}+2\alpha_{2}+0\alpha_{3}=-\alpha_{1}+2\alpha_{2}$.

$a=1$时$r(A,\beta)=r(A)=2<3$,所以$\beta$可由$\alpha_{1},\alpha_{2},\alpha_{3}$线性表示的方法唯一。
\begin{equation*}
\xrightarrow{r_1+2r_{2}}
{
\left[
\begin{array}{c:c}
\begin{matrix}
1 & 0 & 2 \\
  0 & -1 & 1 \\
   0& 0&0\\
 0 & 0 & 0
\end{matrix}&
\begin{matrix}
-1  \\
-2\\
0\\
0
\end{matrix}
\end{array}
\right]}
\end{equation*}
所以解得$x_{1}=-1-2x_{3},x_{2}=x_{3}+2$,令$x_{3}=k,k\in R$,则$\beta=x_ {1}\alpha_{1}+x_{2}\alpha_{2}+x_{3}\alpha_{3}=-(1+2k)\alpha_{1}+(2+k)\alpha_{2}+k\alpha_{3}$
\end{jie}

四、证明题

1. 设$A,B$均为$n$阶方阵,证明:若$A,B$相似则$|A|=|B|$,举例说明反过来不成立。

\begin{zhengming}
若$A$与$B$相似,则依定义有:存在一个可逆矩阵$P$,使得$A=P^{-1}BP$,两边同时求行列式:
$|A|=|P^ {-1}BP|=|P^{-1}|\cdot|B|\cdot|P|=|B|\cdot|P^{-1}P|=|B|\cdot|E|=|B|$。

反过来描述:如果$|A|=|B|$,则$A$和$B$相似。

例如:$A
\begin{bmatrix}
  1 & 0\\
  0& 1
\end{bmatrix},|A|=1
,B=
\begin{bmatrix}
  1 & 1 \\
  0 & 1
\end{bmatrix},|B|=1
$,所以$|A|=|B|$,
但是:假设存在一个可逆矩阵$P$,$P^{-1}AP=P^{-1}EP=E\neq B$,即$|A|=|B|$,但是$A,B$不相似。
\end{zhengming}

2. 设$A$为$m\times n$实矩阵,证明$Ax=0$与$(A^{T}A)x=0$是同解方程,进一步得出$r(A)=r(A^{T}A)$。

\begin{jie}
(1)若$x_0$为$Ax=0$的解,则$Ax_0=0$,对等式两边同时左乘$A^T$:$A^TAx_0=0$,即$x_0$为$A^TAx=0$的解。

若$x_1$为$A^TAx=0$的解:则$A^TAx_1=0$,等式两边同时左乘$x_1^T$:$x_1^TA^TAx_1=(Ax_1)^T(Ax_1)=0$,所以$Ax_1=0$,所以$x_1$为$Ax=0$的解。(注:这里就认为$x$是一个列向量,所以$Ax$也是列向量,用\textcolor[rgb]{1.00,0.00,0.00}{向量内积}的性质。)

综上所述:$Ax=0$与$A^TAx=0$同解。

(2)$Ax=0$与$A^TAx=0$同解,则它们解的空间维数相同。又因为解的空间维数=未知量的个数-系数矩阵的秩。
两个方程的未知数个数相同,所以系数矩阵相同,即$r(A)=r(A^TA)$
\end{jie}
\newpage
\hphantom{~~}\hfill {\zihao{4}\heiti 2018-2019年第一学期} \hfill\hphantom{~~}

一、填空题

1.设$A$为5阶方阵满足$|A|=2$,$A^{*}$是$A$的伴随矩阵,则$|2A^{-1}A^{*}A^{T}|=$\underline{\hphantom{~~~~~~~~~~}}。

\begin{jie}
原式=
\begin{equation*}
2^{5}|A^{-1}|\cdot|A^*|\cdot|A^T|=2^{5}\cdot|A|^{-1}\cdot|A|^{5-1}\cdot|A|=2^9=512
\end{equation*}
\end{jie}

2. 已知向量组$\alpha_{1}=(1,3,1),\alpha_{2}=(0,1,1),\alpha_{3}=(1,4,k)$线性无关,则实数$k$满足的条件是\underline{\hphantom{~~~~~~~~~~}}。

\begin{jie}
$\alpha_{1},\alpha_{2},\alpha_{3}$线性无关,即$r(\alpha_{1},\alpha_{2},\alpha_{3})=3$,记$A=(\alpha_{1},\alpha_{2},\alpha_{3})$,则$|A|\neq0$
\begin{equation*}
  |A|=\begin{vmatrix}
       1& 0 & 1\\
       3 & 1 & 4\\
       1 & 1 & k
\end{vmatrix}\xlongequal{c_{3}-c_{1}}
\begin{vmatrix}
1& 0 & 0\\
3 & 1 & 1\\
1 & 1 & k-1
\end{vmatrix}=k-2\neq0~~~\Rightarrow k\neq 2
\end{equation*}
\end{jie}

3. 设$A$为$m$阶阵,存在非零的$m$维列向量$B$,使$AB=0$的充分必要条件是\underline{\hphantom{~~~~~~~~~~}}。

\begin{jie}
$B$非零,说明$Ax=0$有非零解,由存在唯一性定理:$|A|=0$,或$r(A)<m$。
\end{jie}

4. 设$A=(a_{ij})_{3\times 3}$,其特征值为$1,-1,2$,$A_{ij}$是元素$a_{ij}$的代数余子式,$A^{*}$是$A$的伴随矩阵,则$A^{*}$的主对角线元素之和即$A_{11}+A_{22}+A_{33}=$\underline{\hphantom{~~~~~~~~~~}}。

\begin{jie}
由题得:$|A|=\prod\limits_{i=1}^{3}\lambda_{i}=-2$,所以$A^*$的特征值为$\dfrac{|A|}{\lambda_{i}}$,所以$A^{*}$的主对角线元素之和为$trace(A^*)=\sum\limits_{i=1}^{3}\dfrac{|A|}{\lambda_{i}}=-1$。
\end{jie}

5. 若二次型$f(x_ {1},x_{2},x_{3})=x_{1}^{2}+4x_{2}^{2}+4x_{3}^{2}+2tx_{1}x_{2}-2x_{1}x_{3}+4x_{2}x_{3}$正定,则$t$应满足\underline{\hphantom{~~~~~~~~~~}}。

\begin{jie}
二次型矩阵$A=
\begin{bmatrix}
  1 & t & -1 \\
 t & 4 & 2\\
 -1 &2 &4
\end{bmatrix}
$,二次型正定,即$A$正定,即$A$的所有顺序主子是大于0.即
\begin{align*}
&D_1=1\\
&D_2=
\begin{vmatrix}
  1 & t\\
  t & 4
\end{vmatrix}=4-t^2>0~~~\Rightarrow ~~~-2<t<2\\
&D_3=\begin{vmatrix}
  1 & t & -1 \\
 t & 4 & 2\\
 -1 &2 &4
     \end{vmatrix}=8-4t-4t^2>0~~~-2<t<1
\end{align*}
综上所述:$-2<t<1$.
\end{jie}

6. 设3维列向量组$\alpha_{1},\alpha_{2},\alpha_{3}$线性无关,3阶方阵$A$满足$A\alpha_{1}=-\alpha_{1},A\alpha_{2}=\alpha_{2},A\alpha_{3}=\alpha_{2}+\alpha_{3}$。则行列式$|A|=$\underline{\hphantom{~~~~~~~~~~}}。

\begin{jie}
由题得:$A\alpha_ {1}=-\alpha_{1},A\alpha_{2}=\alpha_{2},A\alpha_{3}=\alpha_{2}+\alpha_{3}$所以
\begin{equation*}
A(\alpha_{1}~\alpha_{2}~\alpha_{3})=(-\alpha_{1}~~\alpha_{2}~~\alpha_{2}+\alpha_{3})
\end{equation*}
即
\begin{gather*}
|A(\alpha_{1}~\alpha_{2}~\alpha_{3})|=|A|\cdot|\alpha_{1}~\alpha_{2}~\alpha_{3}|=|-\alpha_{1}~~\alpha_{2}~~\alpha_{2}+\alpha_{3}|\\
|-\alpha_{1}~~\alpha_{2}~~\alpha_{2}+\alpha_{3}|\xlongequal{c_{3}-c_{2}}|-\alpha_{1}~~\alpha_{2}~~\alpha_{3}|=-|\alpha_{1}~\alpha_{2}~\alpha_{3}|\\
|A|=-1
\end{gather*}
\end{jie}

二、计算题

1.
已知$D=
\begin{vmatrix}
  1 & 1 & 1 & 1 \\
  -1 & 2 & 2 & 3 \\
  1 & 4 & 3 & 9 \\
  -1 & 8 & 5 & 27
\end{vmatrix}
$,求$A_{13}+A_{23}+A_{33}+A_{43}$,其中$A_{ij}$为元素$a_{ij}$的代数余子式。

\begin{jie}
-48.
\begin{align*}
A_{13}+A_{23}+A_{33}+A_{43}=
\begin{vmatrix}
  1 & 1 & \textcolor[rgb]{1.00,0.00,0.00}{1} & 1 \\
  -1 & 2 & \textcolor[rgb]{1.00,0.00,0.00}{1} & 3 \\
  1 & 4 & \textcolor[rgb]{1.00,0.00,0.00}{1} & 9 \\
  -1 & 8 & \textcolor[rgb]{1.00,0.00,0.00}{1} & 27
\end{vmatrix}
\end{align*}
可以看出该式为范德蒙行列式,其中$x_1=-1,x_2=2,x_3=1,x_4=3$,所以上式$=(x_4-x_3)(x_4-x_2)(x_4-x_1)(x_3-x_2)(x_3-x_2)(x_2-x_1)=-48$
\end{jie}

2. 已知
$
A=
\begin{bmatrix}
  1 & 3 & 1 \\
  1 & 1 & 0\\
  0 & 1 & 1
\end{bmatrix}
$,且$X$满足$AX=X+A$,求$X$。

\begin{jie}
由题得:$AX=X+A$,所以$(A-E)X=A$,所以$X=(A-E)^{-1}A$
\begin{align*}
&B=A-E=\begin{bmatrix}
  0 & 3 & 1 \\
  1 & 0 & 0\\
  0 & 1 & 0
\end{bmatrix}\\
[B|E]&\xrightarrow{\substack{r_{1}\Leftrightarrow r_{2}}}
{\left[
\begin{array}{c:c}
\begin{matrix}
 1 & 0 & 0\\
   0 & 3 & 1 \\
  0 & 1 & 0
\end{matrix} &
\begin{matrix}
 0 & 1 & 0\\
  1 & 0 & 0\\
  0 & 0 & 1
\end{matrix}
\end{array}
\right]
}\xrightarrow{\substack{r_{2}\Leftrightarrow r_{3}}}
{\left[
\begin{array}{c:c}
\begin{matrix}
 1 & 0 & 0\\
  0 & 1 & 0  \\
  0 & 3 & 1
\end{matrix} &
\begin{matrix}
 0 & 1 & 0\\
0 & 0 & 1  \\
  1 & 0 & 0
\end{matrix}
\end{array}
\right]
}\xrightarrow{\substack{r_{3}-3 r_{2}}}
{\left[
\begin{array}{c:c}
\begin{matrix}
 1 & 0 & 0\\
  0 & 1 & 0  \\
  0 & 0 & 1
\end{matrix} &
\begin{matrix}
 0 & 1 & 0\\
0 & 0 & 1  \\
  1 & 0 & -3
\end{matrix}
\end{array}
\right]
}
\end{align*}
所以$(A-E)^{-1}=\begin{bmatrix}
 0 & 1 & 0\\
0 & 0 & 1  \\
  1 & 0 & -3
\end{bmatrix}$,$X=(A-E)^{-1}A=\begin{bmatrix}
 0 & 1 & 0\\
0 & 0 & 1  \\
  1 & 0 & -3
\end{bmatrix}\begin{bmatrix}
  1 & 3 & 1 \\
  1 & 1 & 0\\
  0 & 1 & 1
\end{bmatrix}=\begin{bmatrix}
  1 & 1 & 0\\
  0 & 1 & 1\\
  1 & 0 & -2
\end{bmatrix}$
\end{jie}

3. 设矩阵
$
A=
\begin{bmatrix}
  1 & 1 & 1 & 1\\
  0 & 1 & -1& b\\
  2 & 3 & a & 3\\
  3 & 5 &1 &5
\end{bmatrix}
$,$A^{*}$是$A$的伴随矩阵,求$r(A),r(A^{*})$和$A$的列向量组的极大线性无关组。

\begin{jie}
\begin{align*}
A\xrightarrow{\substack{r_{3}-2 r_{1}\\ r_4-3r_1}}
{
\begin{bmatrix}
 1 & 1 & 1 & 1\\
  0 & 1 & -1& b\\
  0 & 1& a-2 & 1\\
  0 & 2 &-2 &2
\end{bmatrix}
}\xrightarrow{\substack{r_{2}\leftrightarrow r_4}}
{
\begin{bmatrix}
 1 & 1 & 1 & 1\\
  0 & 2 &-2 &2\\
  0 & 1& a-2 & 1\\
  0 & 1 & -1& b
\end{bmatrix}
}\xrightarrow{\substack{r_{3}-\frac{1}{2} r_2 \\r_{4}-\frac{1}{2} r_2}}
{
\begin{bmatrix}
 1 & 1 & 1 & 1\\
  0 & 2 &-2 &2\\
  0 & 0& a-1 & 0\\
  0 & 0 & 0& b-1
\end{bmatrix}
}
\end{align*}
\begin{equation*}
r(A^*)=
\begin{cases}
n,~r(A)=n;\\
1,~r(A)=n-1\\
0,~r(A)<n-1.
\end{cases}
\end{equation*}
上式中:$n$为$A$的阶数。

记$A=(\alpha_1,\alpha_2,\alpha_3,\alpha_4)$

(1)$a=1$且$b=1$时:$r(A)=2<4-1=3$,$r(A^*)=0$。极大线性无关组:$(\alpha_1,\alpha_2),(\alpha_1,\alpha_3),(\alpha_1,\alpha_4),(\alpha_2,\alpha_3),(\alpha_3,\alpha_4)$.

(2)$a=1$且$b\neq1$时:$r(A)=3=4-1$,$r(A^*)=1$。极大线性无关组:$(\alpha_1,\alpha_2,\alpha_4),(\alpha_1,\alpha_3,\alpha_4),(\alpha_2,\alpha_3,\alpha_4)$.

(3)$a\neq1$且$b=1$时:$r(A)=3=4-1$,$r(A^*)=1$。极大线性无关组:$(\alpha_1,\alpha_2,\alpha_3),(\alpha_1,\alpha_3,\alpha_4),(\alpha_2,\alpha_3,\alpha_4)$.

(4)$a\neq1$且$b\neq1$时:$r(A)=4$,$r(A^*)=4$。极大线性无关组:$(\alpha_1,\alpha_2,\alpha_3,\alpha_4)$.
\end{jie}

三、解答题

1. 设
$
\begin{cases}
\lambda x_{1}+x_{2}+x_{3}=\lambda-2\\
x_{1}+\lambda x_{2} +x_{3}=2\\
x_{1}+ x_{2} +\lambda x_{3}=2
\end{cases}
$
,$\lambda$为何值时,该方程组无解、唯一解、无穷解?并且在有唯一解时求出解;有无穷多解时,求出全部解并用向量表示。

\begin{jie}
\textcolor[rgb]{1.00,0.00,0.00}{(系数矩阵是方阵,也可以用行列式来做这个题。具体看14-15年期末试题的第四题,推荐这种方法)}

增广矩阵
\begin{align*}
&\left[
\begin{array}{c:c}
\begin{matrix}
\lambda & 1 & 1 \\
1& \lambda & 1 \\
1 & 1&\lambda  \\
\end{matrix}
&
\begin{matrix}
\lambda-2 \\
2 \\
2 \\
\end{matrix}
\end{array}
\right]
\xrightarrow{r_{1}\Leftrightarrow r_{2}}
{
\left[
\begin{array}{c:c}
\begin{matrix}
1& \lambda & 1 \\
\lambda & 1 & 1 \\
1 & 1&\lambda  \\
\end{matrix}
&
\begin{matrix}
2 \\
\lambda-2 \\
2 \\
\end{matrix}
\end{array}
\right]
}
\xrightarrow{\substack{r_{2}-\lambda r_{1}\\ r_{3}- r_{1} }}
{
\left[
\begin{array}{c:c}
\begin{matrix}
1& \lambda & 1 \\
0 & 1-\lambda^{2} & 1-\lambda \\
0 & 1-\lambda &\lambda-1  \\
\end{matrix}
&
\begin{matrix}
2 \\
-\lambda-2 \\
0 \\
\end{matrix}
\end{array}
\right]
}\\
\xrightarrow{r_{2}\Leftrightarrow r_{3}}&
{
\left[
\begin{array}{c:c}
\begin{matrix}
1& \lambda & 1 \\
0 & 1-\lambda &\lambda-1  \\
0 & 1-\lambda^{2} & 1-\lambda \\
\end{matrix}
&
\begin{matrix}
2 \\
0 \\
-\lambda-2 \\
\end{matrix}
\end{array}
\right]
}
\xrightarrow{r_{3}-(\lambda+1) r_{2}}
{
\left[
\begin{array}{c:c}
\begin{matrix}
1& \lambda & 1 \\
0 & 1-\lambda &\lambda-1  \\
0 & 0 & (\lambda-1)(-2-\lambda) \\
\end{matrix}
&
\begin{matrix}
2 \\
0 \\
-\lambda-2 \\
\end{matrix}
\end{array}
\right]
}
\end{align*}
讨论:

(1)解不存在:即存在矛盾方程(增广矩阵主元列在最右列)。即对于$r_{3}$
\begin{equation*}
  \begin{cases}
    (\lambda-1)(-2-\lambda)=0\\
    -\lambda-2\neq 0
  \end{cases}~~~
  \Rightarrow~~~\lambda=1
\end{equation*}

(2)存在唯一解:主元列三个元素都不为0.即
\begin{equation*}
  \begin{cases}
    1\neq 0\\
    1-\lambda\neq 0 \\
    (\lambda-1)(-2-\lambda)\neq 0
  \end{cases}
  ~~~\Rightarrow~~~\lambda\neq 1\text{且}\lambda\neq -2
\end{equation*}
$\lambda\neq 1\text{且}\lambda\neq -2$,继续对阶梯矩阵进行初等行变换
\begin{equation*}
\xrightarrow{\substack{ r_{2}\times \frac{1}{1-\lambda}\\  r_{3}\times \frac{1}{(\lambda-1)(-2-\lambda)}}}
{
\left[
\begin{array}{c:c}
\begin{matrix}
1& \lambda & 1 \\
0 & 1 & -1  \\
0 & 0 & 1 \\
\end{matrix}
&
\begin{matrix}
2 \\
0 \\
\frac{1}{\lambda-1} \\
\end{matrix}
\end{array}
\right]
}
\xrightarrow{\substack{ r_{2}+r_{3}\\  r_{1}-r_{3}}}
{
\left[
\begin{array}{c:c}
\begin{matrix}
1& \lambda & 0 \\
0& 1 & 0  \\
0 & 0 & 1 \\
\end{matrix}
&
\begin{matrix}
\frac{2\lambda-3}{\lambda-1} \\
\frac{1}{\lambda-1} \\
\frac{1}{\lambda-1} \\
\end{matrix}
\end{array}
\right]
}
\xrightarrow{ r_{1}-\lambda r_{3}}
{
\left[
\begin{array}{c:c}
\begin{matrix}
1& 0 & 0 \\
0 & 1 & 0  \\
0 & 0 & 1 \\
\end{matrix}
&
\begin{matrix}
\frac{\lambda-3}{\lambda-1} \\
\frac{1}{\lambda-1} \\
\frac{1}{\lambda-1} \\
\end{matrix}
\end{array}
\right]
}
\end{equation*}
所以方程组存在唯一解时:$\lambda\neq 1$且$\lambda\neq -2$,解为
\begin{equation*}
\mathbf{x}=
\begin{bmatrix}
\frac{\lambda-3}{\lambda-1} \\
\frac{1}{\lambda-1} \\
\frac{1}{\lambda-1}
\end{bmatrix}
~,~~~\lambda\neq1\text{且}\lambda\neq -2
\end{equation*}

(3)存在无穷解:至少存在一个自由变量。由阶梯矩阵可以看出
\begin{equation*}
  \begin{cases}
    (\lambda-1)(-2-\lambda)=0\\
    -2-\lambda=0
  \end{cases}
  ~~~\Rightarrow~~~\lambda=-2
\end{equation*}
把$\lambda=-2$代入阶梯矩阵:
\begin{equation*}
\left[
\begin{array}{c:c}
\begin{matrix}
1& -2 & 1 \\
0 & 3 & -3  \\
0 & 0 & 0 \\
\end{matrix}
&
\begin{matrix}
2 \\
0 \\
0 \\
\end{matrix}
\end{array}
\right]
\xrightarrow{ r_{2}\times\frac{1}{3}}
{\left[
\begin{array}{c:c}
\begin{matrix}
1& -2 & 1 \\
0 & 1 & -1  \\
0 & 0 & 0 \\
\end{matrix}
&
\begin{matrix}
2 \\
0 \\
0 \\
\end{matrix}
\end{array}
\right]
}
\xrightarrow{ r_{1}+2r_{2}}
{\left[
\begin{array}{c:c}
\begin{matrix}
1& 0 & -1 \\
0 & 1 & -1  \\
0 & 0 & 0 \\
\end{matrix}
&
\begin{matrix}
2 \\
0 \\
0 \\
\end{matrix}
\end{array}
\right]
}
\end{equation*}
由最简阶梯型矩阵可以看出:
\begin{equation*}
  x_{1}=2+x_{3}~~x_{2}=x_{3}~~x_{3}=x_{3}
\end{equation*}
令$x_{3}=C,C\in R$,则
\begin{equation*}
x=
 \begin{bmatrix}
   2+C \\
   C \\
   C
 \end{bmatrix}
 =
  \begin{bmatrix}
   2 \\
   0 \\
   0
 \end{bmatrix}
 +
  \begin{bmatrix}
   1 \\
   1 \\
   1
 \end{bmatrix}C
 ~,~C\in R
\end{equation*}

\end{jie}

2. 设实二次型$f(x_{1},x_{2},x_{3})=4x_{1}x_{2}-4x_{1}x_{3}+4x_{2}^{2}+8x_{2}x_{3}-3x_{3}^{2}$。

(1)写出该二次型的矩阵$A$;

(2)求正交矩阵$P$,使得$P^{-1}AP$为对角型矩阵;

(3)给出正交变换,将该二次型化为标准型;

(4)写出二次型的秩,正惯性指标和负惯性指标。

\begin{jie}
(1)由题得:
\begin{equation*}
  A=\begin{bmatrix}
      0 & 2 & -2 \\
      2& 4 & 4\\
      -2 & 4 & 3
    \end{bmatrix}
\end{equation*}

(2)
\begin{align*}
|\lambda E-A|=
\begin{vmatrix}
  \lambda & -2 & 2 \\
      -2& \lambda-4 & -4\\
      2 & -4 & \lambda-3
\end{vmatrix}=(\lambda-1)(\lambda^2-36)=0~~~\Rightarrow~~~\lambda_1=1,\lambda_2=6,\lambda_3=-6
\end{align*}

$\lambda_1=1$时:
\begin{equation*}
[\lambda E-A]=
\begin{bmatrix}
 1 & -2 & 2 \\
      -2& -3 & -4\\
      2 & -4 & -2
\end{bmatrix}~~~\Rightarrow~~~
\begin{cases}
 x_1=-2x_3\\
 x_2=0
\end{cases}
\end{equation*}
取$x_3=1$得$\alpha_1=[-2,0,1]^T$。

$\lambda_2=6$时:
\begin{equation*}
[\lambda E-A]=
\begin{bmatrix}
 6 & -2 & 2 \\
      -2& 2 & -4\\
      2 & -4 & 9
\end{bmatrix}~~~\Rightarrow~~~
\begin{cases}
 x_1=\dfrac{1}{2}x_3\\[2mm]
 x_3=\dfrac{5}{2}x_3
\end{cases}
\end{equation*}
取$x_3=2$得$\alpha_2=[1,5,2]^T$。

$\lambda_3=-6$时:
\begin{equation*}
[\lambda E-A]=
\begin{bmatrix}
 -6 & -2 & 2 \\
      -2&-10 & -4\\
      2 & -4 &-3
\end{bmatrix}~~~\Rightarrow~~~
\begin{cases}
 x_1=\dfrac{1}{2}x_3\\[2mm]
 x_3=-\dfrac{1}{2}x_3
\end{cases}
\end{equation*}
取$x_3=2$得$\alpha_3=[1,-1,2]^T$。
因为对称矩阵对应于不同特征值的特征向量正交,所以$[\alpha_1,\alpha_2,\alpha_3]$为正交向量组。

单位化:
\begin{equation*}
\begin{cases}
\gamma_1=\dfrac{\alpha_1}{\|\alpha_1\|}=\left[-\dfrac{2}{\sqrt{5}},0,-\dfrac{1}{\sqrt{5}}\right]^T\\[2mm]
\gamma_2=\dfrac{\alpha_2}{\|\alpha_2\|}=\left[\dfrac{1}{\sqrt{30}},\dfrac{5}{\sqrt{30}},\dfrac{2}{\sqrt{30}}\right]^T\\[2mm]
\gamma_3=\dfrac{\alpha_3}{\|\alpha_3\|}=\left[\dfrac{1}{\sqrt{6}},-\dfrac{1}{\sqrt{6}},\dfrac{2}{\sqrt{6}}\right]^T
\end{cases}~~~\Rightarrow~~~P=
\begin{bmatrix}
-\dfrac{2}{\sqrt{5}}&\dfrac{1}{\sqrt{30}}&\dfrac{1}{\sqrt{6}}\\[2mm]
0&\dfrac{5}{\sqrt{30}}&-\dfrac{1}{\sqrt{6}}\\[2mm]
\dfrac{1}{\sqrt{5}}&\dfrac{2}{\sqrt{30}}&\dfrac{2}{\sqrt{6}}
\end{bmatrix}
\end{equation*}

所以$P$即为所求的正交矩阵,$P^{-1}AP=\Lambda=
\begin{bmatrix}
  1 & & \\
    & 6&\\
    &&-6
\end{bmatrix}$

(3)由(2)得:$f$可经正交变换$x=Py$化为标准型:
\begin{equation*}
  f=y_{1}^2+6y_{2}^2-6y_{3}^2
\end{equation*}

(4)计算得$|A|=
-2\begin{vmatrix}
    2 & -2 \\
   4 & -3
  \end{vmatrix}-2\begin{vmatrix}
    2 & -2 \\
   4 & 4
  \end{vmatrix}=-36\neq0
$,所以二次型满秩,即$r(A)=3$。由$(3)$得,正惯性指标为$2$,负惯性指标为$1$。
\end{jie}

四、证明题

1. 设$n$阶矩阵$A$满足$A^{2}+3A-4I=0$,其中$I$为$n$阶单位矩阵。

(1)证明:$A,A+3I$可逆,并求他们的逆;

(2)当$A\neq I$时,判断$A+4I$是否可逆并说明理由。

\begin{jie}
(1)由题得:$A^{2}+3A-4I=0$,所以$A(A+3I)=4I$,所以$A,A+3I$可逆,$A$的逆为$\dfrac{1}{4}(A+3I)$,$A+3I$的逆为$\dfrac{1}{4}A$。

(2)不可逆,理由:

由题得:
$A^{2}+3A-4I=(A+4I)(A-I)=0$,假设$A+4I$可逆,则等式两端同时左乘$(A+4I)^{-1}$得$A-I=0$,即$A=I$与题目中$A\neq I$矛盾,所以假设不成立。即$A+4I$不可逆。
\end{jie}

2. 若同阶矩阵$A$与$B$相似,即$A\~{}B$,证明$A^{2}\~{}B^{2}$。反过来结论是否成立并说明理由。

\begin{zhengming}
若$A$与$B$相似,则依定义有:存在一个可逆矩阵$P$,使得$A=P^{-1}BP$,所以:
$A^2=P^ {-1}B\textcolor[rgb]{1.00,0.00,0.00}{P\cdot P^ {-1}}BP=P^ {-1}B^{2}P$。所以$A^{2}\~{}B^{2}$。

反过来描述:如果$A^{2}\~{}B^{2}$,则$A$和$B$相似。

不成立。理由如下:

例如:$A
\begin{bmatrix}
  1 & 0\\
  0& 1
\end{bmatrix},A^2=
\begin{bmatrix}
  1 & 0\\
  0& 1
\end{bmatrix}
,B=
\begin{bmatrix}
  0 & 1 \\
  1 & 0
\end{bmatrix},B^2=
\begin{bmatrix}
  1 & 0\\
  0& 1
\end{bmatrix}
$,所以$A^2=B^2$,由相似的性质$A^2\~{}B^{2}$
但是:假设存在一个可逆矩阵$P$,$P^{-1}AP=P^{-1}EP=E\neq B$,即$A^2\~{}B^{2}$,但是$A,B$不相似。
\end{zhengming}

3. 设$\lambda_{1},\lambda_{2}$是$A$的两个互异的特征值,$\alpha_{11},\cdots,\alpha_ {1s}$是对应于$\lambda_{1}$的线性无关的特征向量,$\alpha_ {21},\cdots,\alpha_ {2t}$是对应于$\lambda_{2}$的线性无关的特征向量,证明:向量组$\alpha_{11},\cdots,\alpha_{1s},\alpha_{21},\cdots,\alpha_{2t}$线性无关。

\begin{zhengming}
由题得:$A\alpha_ {1i}=\lambda_{1}\alpha_ {1i},(1\leq i\leq s)$,$A\alpha_ {2j}=\lambda_{1}\alpha_ {1j},(1\leq j\leq t)$。

设
\begin{equation*}
k_1\alpha_ {11}+\cdots+k_s\alpha_ {1s}+k_{s+1}\alpha_ {21}+\cdots+k_{s+t}\alpha_{2t}=0\tag{$1$}
\end{equation*}

要证明向量组$\alpha_{11},\cdots,\alpha_{1s},\alpha_{21},\cdots,\alpha_{2t}$线性无关,只需证明$k_1=k_2=\cdots=k_s=k_{s+1}=\cdots=k_{s+t}=0$即可。

在$(1)$式左边同乘$A$:
\begin{equation*}
  k_1A\alpha_ {11}+\cdots+k_sA\alpha_ {1s}+k_{s+1}A\alpha_ {21}+\cdots+k_{s+t}A\alpha_{2t}=\lambda_{1}(k_1\alpha_ {11}+\cdots+k_s\alpha_ {1s})+\lambda_{2}(k_{s+1}\alpha_ {21}+\cdots+k_{s+t}\alpha_{2t})=0\tag{$2$}
\end{equation*}
$(2)-\lambda_{2}(1)$得:$(\lambda_1-\lambda_2)(k_1\alpha_ {11}+\cdots+k_s\alpha_ {1s})=0$,因为$\lambda_{1},\lambda_{2}$是$A$的两个互异的特征值,所以$\lambda_1-\lambda_2\neq0$,所以$k_1\alpha_ {11}+\cdots+k_s\alpha_ {1s}=0$,又因为$\alpha_ {11},\cdots,\alpha_ {1s}$是对应于$\lambda_{1}$的线性无关的特征向量,所以:$k_1=k_2=\cdots=k_{s}=0$。代入到$(1)$式得:

$k_{s+1}\alpha_ {21}+\cdots+k_{s+t}\alpha_{2t}=0$,因为$\alpha_ {21},\cdots,\alpha_ {2t}$是对应于$\lambda_{2}$的线性无关的特征向量,所以$k_{s+1}=k_{s+2}=\cdots=k_{s+t}=0$

综上所述:$k_1=k_2=\cdots=k_{s}=k_{s+1}=k_{s+2}=\cdots=k_{s+t}=0$,所以向量组$\alpha_{11},\cdots,\alpha_{1s},\alpha_{21},\cdots,\alpha_{2t}$线性无关。
\end{zhengming}
\newpage
\hphantom{~~}\hfill {\zihao{4}\heiti 2019-2020年第一学期} \hfill\hphantom{~~}

一、填空题

1. 设$A$是3阶方阵,$E$是3阶单位矩阵,已知$A$的特征值为$1,1,2$,则$\left|\left(\left(\dfrac{1}{2}A\right)^{*}\right)^{-1}-2A^{-1}+E\right|= $\underline{\hphantom{~~~~~~~~~~}}。

\begin{jie}
由题得:$|A|=\prod\limits_{i=1}^{3}\lambda_{i}=2,A^*$的特征值为$\dfrac{|A|}{\lambda}=\dfrac{2}{\lambda}$.

由伴随矩阵的性质:$\left(\dfrac{1}{2}A\right)^*=\left(\dfrac{1}{2}\right)^{3-1}A^*=\dfrac{A^*}{4}$,所以$\left(\left(\dfrac{1} {2}A\right)^{*}\right)^{-1}-2A^{-1}+E $的特征值为
\begin{equation*}
\left(\dfrac{1} {4}\cdot\dfrac{2}{\lambda_{i}}\right)^{-1}-2\lambda_{i}^{-1}+1=2\lambda_{i}-\dfrac{2}{\lambda_i}+1
\end{equation*}
所以:
\begin{equation*}
  \left|\left(\left(\dfrac{1} {2}A\right)^{*}\right)^{-1}-2A^{-1}+E\right|=\prod_{i=1}^{3}\left(2\lambda_{i}-\dfrac{2}{\lambda_i}+1\right)=4
\end{equation*}
\end{jie}

2. 已知$A=
\begin{bmatrix}
  1 & -2 & 3k\\
 -1 & 2k & -3\\
 k & -2 & 3
\end{bmatrix}
$的秩为2,则$k=$\underline{\hphantom{~~~~~~~~~~}}。

\begin{jie}
若$k=0$,则
\begin{equation*}
A=\begin{bmatrix}
  1 & -2 & 0\\
 -1 & 0 & -3\\
 0 & -2 & 3
\end{bmatrix}
\xrightarrow{r_{2}+r_{1}}
{
\begin{bmatrix}
  1 & -2 & 0\\
 0 & -2 & -3\\
 0 & -2 & 3
\end{bmatrix}
}\xrightarrow{r_{3}-r_{2}}
{
\begin{bmatrix}
  1 & -2 & 0\\
 0 & -2 & -3\\
 0 & 0 & 6
\end{bmatrix}
}
\end{equation*}
所以$r(A)=3\neq2$,即$k\neq0$.
对$A$接着进行化简:
\begin{equation*}
  A\xrightarrow{\substack{r_{2}+r_{1} \\ r_{3}-kr_{1}}}
{
\begin{bmatrix}
  1 & -2 & 3k\\
 0 & 2k-2 & 3k-3\\
 0 & 2k-2 & 3-3k^{2}
\end{bmatrix}
}=B
\end{equation*}
若$k=1$,则
\begin{equation*}
B=
\begin{bmatrix}
  1 & -2 & 3\\
 0 & 0 & 0\\
 0 & 0 & 0
\end{bmatrix}
\end{equation*}
$r(A)=1\neq2$,所以$k\neq1$,继续对$A$进行化简:
\begin{equation*}
  B\xrightarrow{\substack{r_{2}\times\frac{1}{k-1} \\ r_{3}\times\frac{1}{k-1}}}
{
\begin{bmatrix}
  1 & -2 & 3k\\
 0 & 2 & 3\\
 0 & 2 & -3-3k
\end{bmatrix}
}
\end{equation*}
如果要使$r(A)=2$,则
\begin{equation*}
  \frac{2}{2}=\frac{3}{-3-3k}~~~\Rightarrow k=-2
\end{equation*}

也可以使用$|A|=0$来做。
\end{jie}

3.
记$A=
\begin{bmatrix}
  0 & 0 & 1 & 2 \\
  0 & 0 & 2 & 3 \\
  1 & 1 & 0 & 0  \\
  2& 3 & 0 & 0
\end{bmatrix}
$,则$A^{-1}$\underline{\hphantom{~~~~~~~~~~}}。

\begin{jie}
由题得:$A=
\begin{bmatrix}
  0 & A_{12} \\
  A_{21} & 0
\end{bmatrix}
$,其中$
A_{12}=
\begin{bmatrix}
  1 & 2 \\
  2 & 3
\end{bmatrix},
A_{21}=
\begin{bmatrix}
  1 & 1 \\
  2 & 3
\end{bmatrix}
$,设$A$的逆矩阵为$A^{-1}=
\begin{bmatrix}
  a & b \\
  c & d
\end{bmatrix}
$,其中$a,b,c,d$为$2$阶方阵,则
\begin{equation*}
AA^{-1}=\begin{bmatrix}
  0 & A_{12} \\
  A_{21} & 0
\end{bmatrix}\begin{bmatrix}
  a & b \\
  c & d
\end{bmatrix}=\begin{bmatrix}
  A_{12}c &  A_{12}d\\
  A_{21}a & A_{21}b
\end{bmatrix}
=\begin{bmatrix}
  E_{2} & 0\\
  0 & E_{2}
\end{bmatrix}
\end{equation*}
所以:$c=A_{12}^{-1},d=0,a=0,b=A_{21}^{-1}$.
\begin{align*}
&[A_{12}|E_{2}]\xrightarrow{\substack{r_{2}-2r_{1}}}
{\left[
\begin{array}{c:c}
\begin{matrix}
  1 & 2 \\
  0 & -1
\end{matrix} &
\begin{matrix}
1& 0\\
-2 &1
\end{matrix}
\end{array}
\right]
}\xrightarrow{\substack{r_{1}+2r_{2}}}
{\left[
\begin{array}{c:c}
\begin{matrix}
  1 & 0 \\
  0 & -1
\end{matrix} &
\begin{matrix}
-3 & 2\\
-2 &1
\end{matrix}
\end{array}
\right]
}\xrightarrow{\substack{r_{2}\times(-1)}}
{\left[
\begin{array}{c:c}
\begin{matrix}
  1 & 0 \\
  0 & 1
\end{matrix} &
\begin{matrix}
-3 & 2\\
2 &-1
\end{matrix}
\end{array}
\right]
}\\
&[A_{21}|E_{2}]\xrightarrow{\substack{r_{2}-2r_{1}}}
{\left[
\begin{array}{c:c}
\begin{matrix}
  1 & 1 \\
  0 & 1
\end{matrix} &
\begin{matrix}
1& 0\\
-2 &1
\end{matrix}
\end{array}
\right]
}\xrightarrow{\substack{r_{1}-r_{2}}}
{\left[
\begin{array}{c:c}
\begin{matrix}
  1 & 0 \\
  0 & 1
\end{matrix} &
\begin{matrix}
3& -1\\
-2 &1
\end{matrix}
\end{array}
\right]
}
\end{align*}
所以$A^{-1}=\begin{bmatrix}
  a & b \\
  c & d
\end{bmatrix}=
\begin{bmatrix}
  0 & A_{21}^{-1} \\
  A_{12}^{-1} & 0
\end{bmatrix}=
\begin{bmatrix}
  0 & 0 & 3 & -1\\
  0 & 0 & -2 & 1 \\
  -3& 2 & 0 & 0  \\
  2& -1 & 0 & 0
\end{bmatrix}
$
\end{jie}

4. 若线性方程组
$
\begin{cases}
x_{1}+x_{2}=-a_{1}\\
x_{2}+x_{3}=a_{2}\\
x_{3}+x_{4}=-a_{3}\\
x_{4}+x_{1}=a_{4}
\end{cases}
$
有解,$a_{1},a_{2},a_{3},a_{4}$应满足的条件是\underline{~~\textcolor[rgb]{1.00,0.00,0.00}{$a_{1}+a_{2}+a_{3}+a_{4}=0$}~~}。

\begin{jie}
增广矩阵
\begin{align*}
&\left[
\begin{array}{c:c}
\begin{matrix}
1 & 1 & 0 & 0 \\
0 & 1 & 1 & 0 \\
0 & 0 & 1 & 1 \\
1 & 0 & 0 & 1 \\
\end{matrix}
&
\begin{matrix}
-a_{1} \\
a_{2} \\
-a_{3} \\
a_{4} \\
\end{matrix}
\end{array}
\right]
\xrightarrow{r_{4}-r_{1}}
{
\left[
\begin{array}{c:c}
\begin{matrix}
1 & 1 & 0 & 0 \\
0 & 1 & 1 & 0 \\
0 & 0 & 1 & 1 \\
0 & -1 & 0 & 1 \\
\end{matrix}
&
\begin{matrix}
-a_{1} \\
a_{2} \\
-a_{3} \\
a_{1}+a_{4} \\
\end{matrix}
\end{array}
\right]
}
\xrightarrow{r_{4}+r_{2}}
{
\left[
\begin{array}{c:c}
\begin{matrix}
1 & 1 & 0 & 0 \\
0 & 1 & 1 & 0 \\
0 & 0 & 1 & 1 \\
0 & 0 & 1 & 1 \\
\end{matrix}
&
\begin{matrix}
-a_{1} \\
a_{2} \\
-a_{3} \\
a_{1}+a_{2}+a_{4} \\
\end{matrix}
\end{array}
\right]
}\\
\xrightarrow{r_{4}-r_{3}}&
{
\left[
\begin{array}{c:c}
\begin{matrix}
1 & 1 & 0 & 0 \\
0 & 1 & 1 & 0 \\
0 & 0 & 1 & 1 \\
0 & 0 & 0 & 0 \\
\end{matrix}
&
\begin{matrix}
-a_{1} \\
a_{2} \\
-a_{3} \\
a_{1}+a_{2}+a_{3}+a_{4} \\
\end{matrix}
\end{array}
\right]
}
\end{align*}
若方程有解:$a_{1}+a_{2}-a_{3}+a_{4}=0$
\end{jie}

5. 已知$n$阶方阵$A$对应于特征值$\lambda$的全部的特征向量为$c\alpha$,其中$c$为非零常数,设$n$阶方阵$P$可逆,则$P^{-1}AP$对应于特征值$\lambda$的全部的特征向量为\underline{\hphantom{~~~~~~~~~~}}。

\begin{jie}
由题得:$A(c\alpha)=\lambda (c\alpha)$等式两边同时左乘$P^{-1}$:
\begin{equation*}
P^{-1}A\textcolor[rgb]{1.00,0.00,0.00}{E}(c\alpha)=P^{-1}A\textcolor[rgb]{1.00,0.00,0.00}{PP^{-1}}(c\alpha)=(P^{-1}AP)(P^{-1}c\alpha)=\lambda (P^{-1}c\alpha)
\end{equation*}
所以$P^ {-1}AP$对应于特征值$\lambda$的全部的特征向量为$P^{-1}c\alpha=cP^{-1}\alpha$
\end{jie}

6. 已知实对称矩阵$A=
\begin{bmatrix}
  2 & 0 & 1 \\
  0 & 3 & 3\\
  1 & 3 & x
\end{bmatrix}
$的正惯性指数为3,则$x$的取值范围为\underline{\hphantom{~~~~~~~~~~}}。

\begin{jie}
$A$为实对称矩阵,且$A$的正惯性指数为3,所以$A$正定,所以$A$的所有顺序主子式大于0.所以$|A|=2(3x-9)-3>0~~~\Rightarrow~~~x>3.5$
\end{jie}

二、计算题

1. 设$
A=
\begin{bmatrix}
  0 & 1 & 0 \\
  0 & 0 & 1\\
  0 & 0 & 0
\end{bmatrix}
$.求满足$AX=XA$的全部的矩阵$X$。

\begin{jie}
设$X=
\begin{bmatrix}
  a &b& c \\
  d & e &f\\
  g&h&i
\end{bmatrix}
$,
\begin{gather*}
AX=\begin{bmatrix}
  0 & 1 & 0 \\
  0 & 0 & 1\\
  0 & 0 & 0
\end{bmatrix}\begin{bmatrix}
  a &b& c \\
  d & e &f\\
  g&h&i
\end{bmatrix}=
\begin{bmatrix}
d & e & f\\
g & h & i\\
0& 0 & 0
\end{bmatrix}\\
XA=
\begin{bmatrix}
  a &b& c \\
  d & e &f\\
  g&h&i
\end{bmatrix}\begin{bmatrix}
  0 & 1 & 0 \\
  0 & 0 & 1\\
  0 & 0 & 0
\end{bmatrix}=
\begin{bmatrix}
  0 & a & b\\
  0 & d & e\\
  0 & g & h
\end{bmatrix}
\end{gather*}
$AX=XA$,即
\begin{align*}
\begin{bmatrix}
d & e & f\\
g & h & i\\
0& 0 & 0
\end{bmatrix}=\begin{bmatrix}
  0 & a & b\\
  0 & d & e\\
  0 & g & h
\end{bmatrix}~~~~\Rightarrow
\begin{cases}
d=0~~a=e\hphantom{=0}~~~b=f\\
g=0~~h=d=0~~i=e=a\\
0=0~~g=0\hphantom{=0}~~~h=0
\end{cases}
\end{align*}
所以$x=
\begin{bmatrix}
  a & b & c \\
  0& a & b\\
  0 & 0 & a
\end{bmatrix}
$,其中$a,b,c$是任意常数。
\end{jie}

2. 求线性方程组$
\begin{cases}
x_{1}+3x_{2}+2x_{3}+3x_{4}=0\\
2x_{1}+4x_{2}+x_{3}+3x_{4}=0\\
2x_{1}+4x_{2}+4x_{4}=0\\
\end{cases}
$的一个基础解系。

\begin{jie}
由题得:增广矩阵
\begin{align*}
A&=
\begin{bmatrix}
1 & 3 & 2 &3\\
2 & 4 & 1 & 3\\
2 & 4 & 0 & 4
\end{bmatrix}
\xrightarrow{\substack{r_{2}-2r_1\\ r_3-2r_1}}
{
\begin{bmatrix}
1 & 3 & 2 &3\\
0 & -2 & -3 & -3\\
0 & -2 & -4 & -2
\end{bmatrix}
}
\xrightarrow{\substack{r_{3}-r_2}}
{
\begin{bmatrix}
1 & 3 & 2 &3\\
0 & -2 & -3 & -3\\
0 & 0 & -1 & 1
\end{bmatrix}
}
\xrightarrow{\substack{r_{1}+2r_3 \\ r_2-3r_3}}
{
\begin{bmatrix}
1 & 3 & 0 & 5\\
0 & -2 & 0 & -6\\
0 & 0 & -1 & 1
\end{bmatrix}
}\\
&
\xrightarrow{\substack{r_{2}\times\left(-\frac{1}{2}\right) \\ r_3\times\left(-1\right)}}
{
\begin{bmatrix}
1 & 3 & 0 & 5\\
0 & 1 & 0 & 3\\
0 & 0 & 1 & -1
\end{bmatrix}
}\xrightarrow{\substack{r_{1}-3r_2}}
{
\begin{bmatrix}
1 & 0 & 0 & -4\\
0 & 1 & 0 & 3\\
0 & 0 & 1 & -1
\end{bmatrix}
}
\end{align*}
所以$x_{1}=4x_4,x_2=-3x_4,x_3=x_4$,令$x_4=1$,得基础解系:$\xi=[4,-3,1,1]^T$。
\end{jie}

3. 记$2n$阶方阵$
A_{n}=
\begin{bmatrix}
  a_{n} & ~  & ~ & ~ & ~ & ~ & ~ & b_{n}\\
  ~ & a_{n-1}  & ~ & ~ & ~ & ~ & b_{n-1} & ~\\
  ~ & ~ & \ddots & ~ & ~ & \iddots  & ~ & ~\\
  ~ & ~ & ~ & a_{1}&b_{1} & ~ & ~ & ~\\
  ~ & ~ & ~ & c_{1}&d_{1} & ~ & ~ & ~\\
  ~ & ~ & \iddots & ~ & ~ & \ddots  & ~ & ~\\
  ~ & c_{n-1}  & ~ & ~ & ~ & ~ & d_{n-1} & ~\\
  c_{n} & ~  & ~ & ~ & ~ & ~ & ~ & d_{n}
\end{bmatrix}
$.

(1)求$|A_{1}|,|A_{2}|$

(2)求$|A_{n}|$。

\begin{jie}
(1)
\begin{align*}
|A_{1}|&=
\begin{vmatrix}
  a_{1} & b_{1} \\
  c_{1} & d_{1}
\end{vmatrix}=a_{1}d_{1}-c_{1}b_{1}\\
|A_{2}|&=\begin{vmatrix}
a_{2}& 0& 0&b_{2}\\
0&  a_{1} & b_{1}& 0 \\
0&  c_{1} & d_{1}& 0\\
c_{2}& 0& 0&d_{2}
\end{vmatrix}=a_{2}\begin{vmatrix}
  a_{1} & b_{1}& 0 \\
  c_{1} & d_{1}& 0\\
 0& 0&d_{2}
\end{vmatrix}-b_{2}
\begin{vmatrix}
0&  a_{1} & b_{1} \\
0&  c_{1} & d_{1}\\
c_{2}& 0& 0
\end{vmatrix}=a_{2}d_{2}
\begin{vmatrix}
  a_{1} & b_{1}\\
  c_{1} & d_{1}\\
\end{vmatrix}-b_{2}c_{2}\begin{vmatrix}
  a_{1} & b_{1}\\
  c_{1} & d_{1}\\
\end{vmatrix}=(a_{2}d_{2}-b_{2}c_{2})|A_{1}|\\
&=(a_{2}d_{2}-b_{2}c_{2})(a_{1}d_{1}-c_{1}b_{1})
\end{align*}

(2)用数学归纳法:

由(1)得:
\begin{gather*}
n=1:~~\left|A_ {1}\right|=a_{1}d_{1}-c_{1}b_{1}=\prod\limits_{i=0}^{1}(a_{i}d_{i}-c_{i}b_{i})\\
n=2:~~\left|A_ {2}\right|=(a_{2}d_{2}-c_{2}b_{2})\left|A_ {1}\right|=\prod\limits_{i=0}^{2}(a_{i}d_{i}-c_{i}b_{i})
\end{gather*}
则$n=k-1$时,有$|A_{n}|=\prod\limits_{i=0}^{k-1}(a_{i}d_{i}-c_{i}b_{i})$.\\
当$n=k$时,按第一列展开,得:
\begin{align*}
\left|A_ {k}\right|&=a_{11}A_{11}+a_{2k1}A_{2k1}\\
&=a_{k}\begin{bmatrix}
  a_{k-1}  & ~ & ~ & ~ & ~ & b_{k-1} & ~\\
   ~ & \ddots & ~ & ~ & \iddots  & ~ & ~\\
   ~ & ~ & a_{1}&b_{1} & ~ & ~ & ~\\
  ~ & ~ & c_{1}&d_{1} & ~ & ~ & ~\\
   ~ & \iddots & ~ & ~ & \ddots  & ~ & ~\\
   c_{k-1}  & ~ & ~ & ~ & ~ & d_{k-1} & ~\\
   0  & ~ & ~ & ~ & ~ & ~ & d_{k}
\end{bmatrix}+c_{k}(-1)^{2k+1}
\begin{bmatrix}
  0  & ~ & ~ & ~ & ~ & ~ & b_{k}\\
   a_{k-1}  & ~ & ~ & ~ & ~ & b_{k-1} & ~\\
   ~ & \ddots & ~ & ~ & \iddots  & ~ & ~\\
   ~ & ~ & a_{1}&b_{1} & ~ & ~ & ~\\
  ~ & ~ & c_{1}&d_{1} & ~ & ~ & ~\\
   ~ & \iddots & ~ & ~ & \ddots  & ~ & ~\\
   c_{k-1}  & ~ & ~ & ~ & ~ & d_{k-1} & ~
\end{bmatrix}\\
&=a_{k}d_{k}(-1)^{2k-2+1+2k-2+1}\begin{bmatrix}
  a_{k-1}  & ~ & ~ & ~ & ~ & b_{k-1} \\
   ~ & \ddots & ~ & ~ & \iddots  & ~ \\
   ~ & ~ & a_{1}&b_{1} & ~ & ~ \\
  ~ & ~ & c_{1}&d_{1} & ~ & ~ \\
   ~ & \iddots & ~ & ~ & \ddots  & ~ \\
   c_{k-1}  & ~ & ~ & ~ & ~ & d_{k-1}
\end{bmatrix}-c_{k}d_{k}(-1)^{1+2k-2+1}\begin{bmatrix}
  a_{k-1}  & ~ & ~ & ~ & ~ & b_{k-1} \\
   ~ & \ddots & ~ & ~ & \iddots  & ~ \\
   ~ & ~ & a_{1}&b_{1} & ~ & ~ \\
  ~ & ~ & c_{1}&d_{1} & ~ & ~ \\
   ~ & \iddots & ~ & ~ & \ddots  & ~ \\
   c_{k-1}  & ~ & ~ & ~ & ~ & d_{k-1}
\end{bmatrix}\\ &=(a_{k}d_{k}-c_{k}d_{k})|A_{k-1}|=\prod_{i=1}^{k}(a_{i}d_{i}-c_{i}b_{i})
\end{align*}
\end{jie}

三、解答题

1. 设向量组$\alpha_{1}=(1,-4,-3)^{T},\alpha_{2}=(-3,6,7)^{T},\alpha_{3}=(-4,-2,6)^{T},\alpha_{4}=(3,3,-4)^{T}$,求向量组的秩,并写出一个极大线性无关组,并将其余向量由极大无关组线性表示出。

\begin{jie}
由题得:
\begin{align*}
(\alpha_ {1},\alpha_{2},\alpha_{3},\alpha_{4})&=
\begin{bmatrix}
  1 & -3 & -4 & 3 \\
  -4 & 6 & -2 & 3\\
  -3 & 7 & 6 & -4
\end{bmatrix}
\xrightarrow{\substack{r_{2}+4r_1 \\ r_3+3r_1}}
{
\begin{bmatrix}
  1 & -3 & -4 & 3 \\
  0 & -6 & -18 & 15\\
  0 & -2 & -6 & 5
\end{bmatrix}
}
\xrightarrow{\substack{r_{3}-\frac{1}{3}r_2}}
{
\begin{bmatrix}
  1 & -3 & -4 & 3 \\
  0 & -6 & -18 & 15\\
  0 & 0 & 0 & 0
\end{bmatrix}
}\\ &\xrightarrow{\substack{r_{2}\times\left(-\frac{1}{6}\right)}}
{
\begin{bmatrix}
  1 & -3 & -4 & 3 \\
  0 & 1 & 3 & -2.5\\
  0 & 0 & 0 & 0
\end{bmatrix}
}
\xrightarrow{\substack{r_{1}+3r_2}}
{
\begin{bmatrix}
  1 & 0 & 5 & -4.5 \\
  0 & 1 & 3 & -2.5\\
  0 & 0 & 0 & 0
\end{bmatrix}
}
\end{align*}
所以$r(\alpha_ {1},\alpha_{2},\alpha_{3},\alpha_{4})=2$,极大线性无关组有两个向量:$(\alpha_ {1},\alpha_{2}),(\alpha_ {1}\alpha_{3}),(\alpha_ {1},\alpha_{4}),(\alpha_{2},\alpha_{3}),(\alpha_{2},\alpha_{4}),(\alpha_{3},\alpha_{4})$.\textcolor[rgb]{1.00,0.00,0.00}{(任写一个即可)}

以$(\alpha_ {1},\alpha_{2})$为例:$\alpha_3=5\alpha_1+3\alpha_2,\alpha_4=-4.5\alpha_1-2.5\alpha_2$。
\end{jie}

2. 已知3阶方阵$
A=
\begin{bmatrix}
  -1 & a+2 & 0\\
  a-2 & 3 & 0\\
 8 & -8 & -1
\end{bmatrix}
$可以相似对角化且$A$得到特征方程有一个二重根,求$a$的值。其中$a\leq 0$.

\begin{jie}
由题得:
\begin{align*}
|\lambda E-A|&=
\begin{vmatrix}
  \lambda+1 & -(a+2) & 0\\
  2-a & \lambda-3 & 0\\
 -8 & 8 & \lambda+1
\end{vmatrix}=(\lambda+1)[(\lambda+1)(\lambda-3)+(2+a)(2-a)]=(\lambda+1)[(\lambda-1)^2-a^2]=0\\
&\Rightarrow ~~\lambda_{1}=-1,\lambda_2=1+a,\lambda_{3}=1-a.
\end{align*}
依题意:有二重根且可以相似对角化且$a\leq0$.

讨论:

(1)$\lambda_1=\lambda_2$,即$-1=1+a,a=-2\leq 0$,此时$\lambda_{3}=1-a=3$,代入到${\lambda E-A}$得:
\begin{align*}
[\lambda E-A]=
\begin{bmatrix}
  \lambda+1 & 0 & 0\\
  4 & \lambda-3 & 0\\
 -8 & 8 & \lambda+1
\end{bmatrix}
\end{align*}
对于重根$-1$:
\begin{align*}
[\lambda E-A]=
\begin{bmatrix}
  0 & 0 & 0\\
  4 & -4 & 0\\
 -8 & 8 & 0
\end{bmatrix}
\rightarrow
\begin{bmatrix}
  1 & -1 & 0\\
  0 & 0 & 0\\
 0 & 0 & 0
\end{bmatrix}~~~\Rightarrow~~~\alpha_1=
\begin{bmatrix}
0\\
0\\
1
\end{bmatrix}~~\alpha_2=
\begin{bmatrix}
1\\
1\\
0
\end{bmatrix}
\end{align*}
对于根$3$:
\begin{align*}
[\lambda E-A]=
\begin{bmatrix}
  4 & 0 & 0\\
  4 & 0 & 0\\
 -8 & 8 & 4
\end{bmatrix}
\rightarrow
\begin{bmatrix}
  1 & 0 & 0\\
  0 & 2 & 1\\
 0 & 0 & 0
\end{bmatrix}~~~\Rightarrow~~~\alpha_3=
\begin{bmatrix}
0\\
-1\\
2
\end{bmatrix}
\end{align*}
可以看出$\alpha_{1},\alpha_{2},\alpha_3$线性无关,即$A$可相似对角化,即$a=-2$符合题意。

(2)$\lambda_1=\lambda_3$,即$-1=1-a,a=2>0$,不符合题意。

(3)$\lambda_{2}=\lambda=3$,即$1+a=1-a,a=0$。此时$\lambda_{2}=\lambda_3=1$.把$a=0$代入到$[\lambda E-A]$得:
\begin{equation*}
[\lambda E-A]=
\begin{bmatrix}
  \lambda+1 & -2 & 0\\
  2 & \lambda-3 & 0\\
 -8 & 8 & \lambda+1
\end{bmatrix}
\end{equation*}
对于重根$1$:
\begin{align*}
[\lambda E-A]=
\begin{bmatrix}
  2 & -2 & 0\\
  2 & -2 & 0\\
 -8 & 8 & 2
\end{bmatrix}
\rightarrow
\begin{bmatrix}
  1 & -1 & 0\\
  0 & 0 & 1\\
 0 & 0 & 0
\end{bmatrix}~~~\Rightarrow~~~\alpha_1=
\begin{bmatrix}
1\\
1\\
0
\end{bmatrix}
\end{align*}
对于重根$1$,其代数重数与几何重数不相等,所以不能相似对角化。

综上所述:$a=-2$.
\end{jie}

3. 设三元二次型$f(x_{1},x_{2},x_{3})=4x_{2}^{2}+4x_{3}^{2}-2x_{1}x_{2}+4x_{1}x_{3}$.

(1)写出该二次型的矩阵$A$;

(2)用正交变换$x=Qy$把该二次型化为标准型。

\begin{jie}
(1)由题得:
\begin{equation*}
  A=
  \begin{bmatrix}
    0 & -1 & 2\\
    -1 & 4 & 0\\
    2 & 0 & 4
  \end{bmatrix}
\end{equation*}

(2)
\begin{align*}
|\lambda E-A|=
\begin{vmatrix}
    \lambda & 1 & -2\\
    1 & \lambda-4 & 0\\
    -2 & 0 & \lambda-4
\end{vmatrix}=-2[2(\lambda-4)]+(\lambda-4)[\lambda(\lambda-4)-1]=(\lambda-4)(\lambda-5)(\lambda+1)~~~\Rightarrow ~~~\lambda_1=4,\lambda_2=5,\lambda_3=-1
\end{align*}

$\lambda_1=4$时:
\begin{equation*}
[\lambda E-A]=
\begin{bmatrix}
  4 & 1 & -2\\
    1 & 0 & 0\\
    -2 & 0 & 0
\end{bmatrix}~~~\Rightarrow~~~
\begin{cases}
 x_1=0\\
 x_2=2x_3
\end{cases}
\end{equation*}
取$x_3=1$得$\alpha_1=[0,2,1]^T$。

$\lambda_2=5$时:
\begin{equation*}
[\lambda E-A]=
\begin{bmatrix}
  5 & 1 & -2\\
    1 & 1 & 0\\
    -2 & 0 & 1
\end{bmatrix}~~~\Rightarrow~~~
\begin{cases}
 x_1=\dfrac{1}{2}x_3\\[2mm]
 x_2=-\dfrac{1}{2}x_3
\end{cases}
\end{equation*}
取$x_3=2$得$\alpha_2=[1,-1,2]^T$.

$\lambda_3=-1$时:
\begin{equation*}
[\lambda E-A]=
\begin{bmatrix}
  -1 & 1 & -2\\
    1 & -5 & 0\\
    -2 & 0 & -5
\end{bmatrix}~~~\Rightarrow~~~
\begin{cases}
 x_1=-\dfrac{5}{2}x_3\\[2mm]
 x_2=-\dfrac{1}{2}x_3
\end{cases}
\end{equation*}
取$x_3=-2$得$\alpha_3=[5,1,-2]^T$.因为对称矩阵对应于不同特征值的特征向量正交,所以$[\alpha_1,\alpha_2,\alpha_3]$为正交向量组。

单位化:
\begin{equation*}
\begin{cases}
\gamma_1=\dfrac{\alpha_1}{\|\alpha_1\|}=\left[0,\dfrac{2}{\sqrt{5}},\dfrac{1}{\sqrt{5}}\right]^T\\[2mm]
\gamma_2=\dfrac{\alpha_2}{\|\alpha_2\|}=\left[\dfrac{1}{\sqrt{6}},-\dfrac{1}{\sqrt{6}},\dfrac{2}{\sqrt{6}}\right]^T\\[2mm]
\gamma_3=\dfrac{\alpha_3}{\|\alpha_3\|}=\left[\dfrac{5}{\sqrt{30}},\dfrac{1}{\sqrt{30}},-\dfrac{2}{\sqrt{30}}\right]^T
\end{cases}~~~\Rightarrow~~~Q=
\begin{bmatrix}
0&\dfrac{1}{\sqrt{6}}&\dfrac{5}{\sqrt{30}}\\[2mm]
\dfrac{2}{\sqrt{5}}&-\dfrac{1}{\sqrt{6}}&\dfrac{1}{\sqrt{30}}\\[2mm]
\dfrac{1}{\sqrt{5}}&\dfrac{2}{\sqrt{6}}&-\dfrac{2}{\sqrt{30}}
\end{bmatrix}
\end{equation*}
所以$f$可经正交变换$x=Qy$化为标准型:
\begin{equation*}
  f=4y_{1}^2+5y_{2}^2-y_{3}^2
\end{equation*}
\end{jie}

四.证明题

1. 设$A$为$m$阶正定矩阵,$B$为$m\times n$实矩阵,$B^{T}$为$B$的转置矩阵,试证:$B^{T}AB$为正定矩阵的充分必要条件是$B$的秩$r(B)=n$。

\begin{zhengming}
必要性:如果$B^TB$正定,则存在任意非零实列向量$x\neq 0$,使得$x^TB^TBx>0$,即$(Bx)^TA(Bx)>0$,所以$Bx\neq 0$。所以$Bx=0$只有零解,即$r(B)=n$。

充分性:如果$B$的秩为$r(B)=n$,则线性方程组$Bx=0$只有零解,所以存在任意非零实列向量$x$,使得$Bx\neq 0$。又因为$A$为正定矩阵,由正定矩阵的定义得:$(Bx)^TABx>0$,即$x^TB^TABx=x^T(B^TAB)x>0$。
因为$x$为任意非零实列向量,所以依正定矩阵的定义,矩阵$(B^TAB)$正定。
\end{zhengming}

2. 设$\alpha,\beta$是$n$维列向量,证明$r(\alpha\alpha^{T}+\beta\beta^{T})\leq 2$。

\begin{zhengming}
由秩的性质:
\begin{equation*}
r(\alpha\alpha^ {T}+\beta\beta^{T})\leq r(\alpha\alpha^ {T})+r(\beta\beta^{T})\leq \min(r(\alpha),r(\alpha^T))+\min(r(\beta),r(\beta^T))\leq 1+1=2
\end{equation*}
\end{zhengming}
\newpage
\hphantom{~~}\hfill {\zihao{4}\heiti 2020-2021年第一学期} \hfill\hphantom{~~}

一、填空题(每小题3分,共计15分)

1.行列式$
\begin{vmatrix}
  1 & 2 & 3 & 0\\
  0 & 0 & 2 & 0\\
  3 & 0 & 4 & 5\\
  0 & 0 & 0 &1
\end{vmatrix}
$=\underline{~~~~~~~\textcolor[rgb]{1.00,0.00,0.00}{$12$}~~~~~~~}。

\begin{jie}
由题得,按最后一行展开

原式$=
\begin{vmatrix}
  1 & 2 & 3 \\
  0 & 0 & 2 \\
  3 & 0 & 4
\end{vmatrix}
=-2
\begin{vmatrix}
  1 & 2  \\
  3 & 0
\end{vmatrix}
=-2\times(-6)=12$。
\end{jie}

2.已知矩阵
$
A=\begin{bmatrix}
    1 & 2 &a\\
    -a & -1 &a-1\\
    3& 1 & -2
  \end{bmatrix}
$的秩为2,则$a$=\underline{~~~~\textcolor[rgb]{1.00,0.00,0.00}{$1$或$3$}~~~~}。

\begin{jie}
由题得:
\begin{equation*}
A\xrightarrow{\substack{ r_1 \leftrightarrow r_3 }}
{
\begin{bmatrix}
 3& 1 & -2\\
    -a & -1 &a-1\\
    1 & 2 &a
  \end{bmatrix}
}\xrightarrow{\substack{ r_2+\frac{a}{3}r_1 \\ r_3 - \frac{1}{3}r_1 }}
{
\begin{bmatrix}
 3& 1 & -2\\%1  1/3   -2/3   a   a/3  -2a/3
    0 & \dfrac{a}{3}-1 &\dfrac{a}{3}-1\\
    0 & \dfrac{5}{3} &a+\dfrac{2}{3}
  \end{bmatrix}
}
\end{equation*}

要使秩为2:第二行为0或第二第三行的非零元素成比例。

第二行为0即:
\begin{equation*}
  \begin{cases}
\dfrac{a}{3}-1=0\\
\dfrac{a}{3}-1=0
  \end{cases}~~~~~\Rightarrow~~~~~
  a=3
\end{equation*}

第二第三行的非零元素成比例即:
\begin{equation*}
\dfrac{\frac{a}{3}-1}{\frac{5}{3}}=\dfrac{\frac{a}{3}-1}{a+\dfrac{2}{3}}~~~~~\Rightarrow~~~~~a=1
\end{equation*}
综上所述:$a=1$或$3$。
\end{jie}


3.已知$A,B,C,D,H$为$n$阶实矩阵,$I$是同阶单位矩阵,且$ABCDH=I$,则$C^{-1}$=\underline{~~~~~\textcolor[rgb]{1.00,0.00,0.00}{$DHAB$}~~~~~}。

\begin{jie}
等式两边同左乘$(AB)$的逆和右乘$(DH)$的逆:$C=(AB)^{-1}(DH)^{-1}$,该等式两边同时求逆:
$C^{-1}=((AB)^{-1}(DH)^{-1})^{-1}=((DH)^{-1})^{-1}\cdot((AB)^{-1})^{-1}=DHAB$

\textcolor[rgb]{1.00,0.00,0.00}{注:可逆矩阵的性质:$(AB)^{-1}=B^{-1}A^{-1}$}
\end{jie}

4.已知$A$是3阶方阵,$I$是同阶单位矩阵,且$|A-I|=0,|A+I|=0,|I-2A|=0$,则$|2A^2+2A-I|$=\underline{~~~~\textcolor[rgb]{1.00,0.00,0.00}{$-1.5$}~~~~}。

\begin{jie}
特征多项式的定义:$|\lambda I-A|=0$,也即$|A-\lambda I|=0$.所以由题得:
$\lambda_1=1,\lambda_2=-1,|I-2A|=|-2(A-\frac{1}{2}I)|=0~~\Rightarrow~~\lambda_3=0.5$,

所以$2A^2+2A-I$的特征值为:$2\lambda^2+2\lambda-1=3,-1,0.5$,所以$|2A^2+2A-I|=3\times(-1)\times0.5=-1.5$
\end{jie}

5.向量组$\alpha_1,\alpha_2,\cdots,\alpha_n(n\geq2)$中,每个向量都能被其余向量线性表出是其向量组相关的\underline{~~~\textcolor[rgb]{1.00,0.00,0.00}{充分不必要}~~~}条件。

\begin{jie}
从左往右:(每个向量都能被其余向量线性表出$~~~\Rightarrow~~~$向量组相关)

向量组相关的定义是只要该向量组中有一个向量能被其余向量线性表出即可,所以满足充分性。

从右往左:(向量组相关$~~~\Rightarrow~~~$每个向量都能被其余向量线性表出)

例如:$\alpha_1=(1,0)^T,\alpha_2=(0,0)^T$,该向量组相关,但$\alpha_1$不能由$\alpha_2$表出,不满足必要性。
\end{jie}

\hphantom{1}

二、计算题(每小题10分,共30分)

1.若行列式$
\begin{vmatrix}
  1 & 2 & 1 & 2\\
  3 & 4&5 & 6\\
  1 & 2 &3 & 4\\
  -1 & -1 &-2&-2
\end{vmatrix}$,求$M_{21}+M_{22}+M_{23}+M_{24}$,其中$M_{ij}$是$a_{ij}$的余子式。

\begin{jie}
由代数余子式的定义:
\begin{gather*}
A_{21}=(-1)^{2+1}M_{21}=-M_{21}~~~~A_{22}=(-1)^{2+2}M_{22}=M_{22}\\
A_{23}=(-1)^{2+3}M_{23}=-M_{23}~~~~A_{24}=(-1)^{2+4}M_{24}=M_{24}
\end{gather*}
所以:
\begin{equation*}
M_{21}+M_{22}+M_{23}+M_{24}=-A_{21}+A_{22}-A_{23}+A_{24}=
\begin{vmatrix}
  1 & 2 & 1 & 2\\
  \textcolor[rgb]{1.00,0.00,0.00}{-1} & \textcolor[rgb]{1.00,0.00,0.00}{1}&\textcolor[rgb]{1.00,0.00,0.00}{-1} & \textcolor[rgb]{1.00,0.00,0.00}{1}\\
  1 & 2 &3 & 4\\
  -1 & -1 &-2&-2
\end{vmatrix}=
\begin{vmatrix}
 1 & 2 & 1 & 2\\
  0 & 3&0 & 3\\
  0 & 0 &2 & 2\\
  0 & 1 &-1&0
\end{vmatrix}=
\begin{vmatrix}
 3&0 & 3\\
 0 &2 & 2\\
 1 &-1&0
\end{vmatrix}=\textcolor[rgb]{1.00,0.00,0.00}{0}
\end{equation*}
\end{jie}

2.已知向量$\alpha=(1,-1,2)^{T},\beta=(\dfrac{1}{2},\dfrac{1}{2},-1)^{T}$,记$A=\alpha\beta^T$,求$A^{2021}$。

\begin{jie}
$k=\beta^T\alpha=(\dfrac{1}{2},\dfrac{1}{2},-1)
\begin{pmatrix}
1\\ -1\\ 2
\end{pmatrix}=-2$

\begin{gather*}
A^{2}=\alpha\beta^T\alpha\beta^T=\alpha(\beta^T\alpha)\beta^T=\alpha k\beta^T=k\alpha\beta^T=kA=k^{2-1}A\\
A^3=A\cdot A^2=A\cdot kA=kA^{2}=kkA=k^2A=k^{3-1}A\\
\cdots\cdots\\
A^n=k^{n-1}A
\end{gather*}
\begin{equation*}
A^{2021}=k^{2021-1}A=(-2)^{2020}\alpha\beta^T=\textcolor[rgb]{1.00,0.00,0.00}{2^{2020}
\begin{pmatrix}
\frac{1}{2}&\frac{1}{2}&-1 \\
-\frac{1}{2}&-\frac{1}{2}&1\\
1&1&-2
\end{pmatrix}}
\end{equation*}
\end{jie}

3.已知矩阵$A=
\begin{pmatrix}
  2 & 0 &1 \\
  1 & 0 &5\\
  3 & 2 &5
\end{pmatrix}
,B=\begin{pmatrix}
  1& 1 &2 \\
  1 & -1 &2\\
  3 & 2 &4
\end{pmatrix}$满足$AXB=BXB+I$,其中$I$是3阶单位矩阵,求$X$。

\begin{jie}
由题得:
\begin{equation*}
AXB=BXB+I~~\Rightarrow~~AXB-BXB=I~~\Rightarrow(A-B)XB=I~~\Rightarrow~~X=(A-B)^{-1}B^{-1}
\end{equation*}

分别求两个逆(过程略):
\begin{equation*}
B^{-1}=
\begin{pmatrix}
  1 & 0 & 0\\
 \frac{1}{2}&-\frac{1}{2}&0\\
 \frac{5}{4}&\frac{1}{4}&-\frac{1}{2}
\end{pmatrix}~~~~
A-B=
\begin{pmatrix}
  1 & -1 & -1 \\
  0 & 1 & 3\\
  0 & 0 & 1
\end{pmatrix}~~~(A-B)^{-1}=
\begin{pmatrix}
 1& 1&-2\\
  0 & 1 & -3\\
  0&0&1
\end{pmatrix}~~~\Rightarrow~~~(A-B)^{-1}B^{-1}=
\textcolor[rgb]{1.00,0.00,0.00}{\begin{pmatrix}
  -4 & -1 &2 \\
 -\frac{13}{4} & -\frac{5}{4} &\frac{3}{2}\\
 \frac{5}{4}&\frac{1}{4}&-\frac{1}{2}
\end{pmatrix}}
\end{equation*}
\end{jie}

\hphantom{1}

三、解答题(每小题13分,共计39分)

1.求向量组$\alpha_1=(1,1,2,0)^T,\alpha_2=(-2,-1,-2,2)^T,\alpha_3=(3,4,4,-4)^T,\alpha_1=(-1,-1,0,3)^T$的秩以及一个极大无关组,并用极大无关组表示其余向量。

\begin{jie}
由题得:
\begin{align*}
(\alpha_1,\alpha_2,\alpha_3,\alpha_4)&
\xrightarrow{\substack{ r_2-r_1 \\ r_3-2r_1 }}
{
\begin{bmatrix}
  1 & -2& 3&-1 \\
  0 & 1 & 1 &0\\
  0 &2&-2&2\\
  0&2&-4&3
\end{bmatrix}
}\xrightarrow{\substack{ r_3-2r_2 \\ r_4-2r_2 }}
{
\begin{bmatrix}
  1 & -2& 3&-1 \\
  0 & 1 & 1 &0\\
  0 &0&-4&2\\
  0&0&-6&3
\end{bmatrix}
}\xrightarrow{\substack{ r_4-\frac{3}{2}r_3 }}
{
\begin{bmatrix}
  1 & -2& 3&-1 \\
  0 & 1 & 1 &0\\
  0 &0&-4&2\\
  0&0&0&0
\end{bmatrix}
}\\
&\xrightarrow{\substack{ r_3\times\left(-\frac{1}{4}\right) }}
{
\begin{bmatrix}
  1 & -2& 3&-1 \\
  0 & 1 & 1 &0\\
  0 &0&1&-\frac{1}{2}\\
  0&0&0&0
\end{bmatrix}
}\xrightarrow{\substack{ r_1-3r_3\\ r_2-r_3}}
{
\begin{bmatrix}
  1 & -2& 0&\frac{1}{2} \\
  0 & 1 & 0 &\frac{1}{2}\\
  0 &0&1&-\frac{1}{2}\\
  0&0&0&0
\end{bmatrix}
}\xrightarrow{\substack{ r_1+2r_2}}
{
\begin{bmatrix}
  1 & 0& 0&\frac{3}{2} \\
  0 & 1 & 0 &\frac{1}{2}\\
  0 &0&1&-\frac{1}{2}\\
  0&0&0&0
\end{bmatrix}
}
\end{align*}
所以该向量组的秩为\textcolor[rgb]{1.00,0.00,0.00}{3},极大线性无关组为:$\textcolor[rgb]{1.00,0.00,0.00}{(\alpha_1,\alpha_2,\alpha_3)},(\alpha_1,\alpha_2,\alpha_4),(\alpha_1,\alpha_3,\alpha_4),(\alpha_2,\alpha_3,\alpha_4) $(依题意,任写出其中一个即可)

用$(\alpha_1,\alpha_2,\alpha_3)$表示$\alpha_4$:由最简阶梯型矩阵可以看出:$\textcolor[rgb]{1.00,0.00,0.00}{\alpha_4=\dfrac{3} {2}\alpha_1+\dfrac{1}{2}\alpha_2-\dfrac{1}{2}\alpha_3}$。
\end{jie}

2.判断线性方程组
$
\begin{cases}
x_1+2x_2+3x_3+2x_4=1\\
x_1+2x_2+4x_3+5x_4=2\\
2x_1+4x_2+ax_3+x_4=1\\
-x_1-2x_2-3x_3+7x_4=8\\
\end{cases}
$何时无解?何时有解?并在有无穷多组解时求出其通解。

\begin{jie}
由题可列增广矩阵并进行高斯消元:
\begin{align*}
(A|b)=
\begin{bmatrix}
1 & 2 & 3 & 2 &1 \\
1 & 2 & 4 & 5 &2 \\
2 & 4 & a & 1 &1 \\
-1 & -2 & -3 & 7 &8 \\
\end{bmatrix}&
\xrightarrow{\substack{ r_2-r_1\\ r_3-2r_1 \\ r_4+r_1}}
{
\begin{bmatrix}
1 & 2 & 3 & 2 &1 \\
0 & 0 & 1 & 3 &1 \\
0 & 0 & a-6 & -3 &1 \\
0 & 0 & 0& 9 &9
\end{bmatrix}
}\xrightarrow{\substack{ r_3-(a-6)r_2 \\ r_4\div 9}}
{
\begin{bmatrix}
1 & 2 & 3 & 2 &1 \\
0 & 0 & 1 & 3 &1 \\
0 & 0& 0 & 15-3a & 5-a \\
0 & 0 & 0& 1 &1
\end{bmatrix}
}\\
&
\xrightarrow{\substack{ r_3-(15-3a)r_4}}
{
\begin{bmatrix}
1 & 2 & 3 & 2 &1 \\
0 & 0 & 1 & 3 &1 \\
0 & 0& 0 & 0 & 2a-10 \\
0 & 0 & 0& 1 &1
\end{bmatrix}
}\xrightarrow{\substack{ r_3\leftrightarrow r_4}}
{
\begin{bmatrix}
1 & 2 & 3 & 2 &1 \\
0 & 0 & 1 & 3 &1 \\
 0 & 0 & 0& 1 &1\\
0 & 0& 0 & 0 & 2a-10
\end{bmatrix}
}
\end{align*}

由阶梯型矩阵可以看出:当$r(A)\neq r(A|b)$,即$2a-10\neq 0~~~\Rightarrow~~~\textcolor[rgb]{1.00,0.00,0.00}{a\neq 5}$时,该\textcolor[rgb]{1.00,0.00,0.00}{线性方程组无解},当$r(A)=r(A|b)$,即$2a-10=0~~~\Rightarrow~~~\textcolor[rgb]{1.00,0.00,0.00}{a=5}$时,该\textcolor[rgb]{1.00,0.00,0.00}{线性方程组有解,且有无穷多组解}。

把$a=5$代入上式继续化简:
\begin{align*}
\xrightarrow{\substack{ r_2-3r_3 \\ r_1-2r_3}}
{
\begin{bmatrix}
1 & 2 & 3 & 0 &-1 \\
0 & 0 & 1 & 0 &-2 \\
 0 & 0 & 0& 1 &1\\
0 & 0& 0 & 0 & 0
\end{bmatrix}
}\xrightarrow{\substack{ r_1-3r_2}}
{
\begin{bmatrix}
1 & 2 & 0 & 0 &5 \\
0 & 0 & 1 & 0 &-2 \\
 0 & 0 & 0& 1 &1\\
0 & 0& 0 & 0 & 0
\end{bmatrix}
}
\end{align*}
由最简阶梯型矩阵可以看出:$x_1=5-2x_2,x_3=-2,x_4=1$。

所以该方程组的一个特解为:$(5,0,-2,1)^T$,其导出组的基础解系为:$(-2,1,0,0)^T$,所以原方程组的通解为:

$\textcolor[rgb]{1.00,0.00,0.00}{(5,0,-2,1)^T+k(-2,1,0,0)^T,(k\in R)}$。
\end{jie}

3.已知矩阵$
A=
\begin{pmatrix}
  a & -2&-2 \\
  -2 & 1 & -b\\
  -2 & -b &1
\end{pmatrix}
$与$
\begin{pmatrix}
  -3 & 0 &0 \\
  0 & 3 & 0\\
  0 & 0 &3
\end{pmatrix}
$相似,求$a,b$的值以及$A$的全部的特征向量。

\begin{jie}
\begin{equation*}
|A|=a
\begin{vmatrix}
  1 & -b \\
  -b & 1
\end{vmatrix}+2
\begin{vmatrix}
  -2 & -b \\
  -2 & 1
\end{vmatrix}-2
\begin{vmatrix}
  -2 & 1 \\
  -2 & -b
\end{vmatrix}=a(1-b^2)+2(-2-2b)-2(2b+2)=a-ab^2-8b-8
\end{equation*}
相似矩阵具有相同的特征值,所以由特征值的性质有:
\begin{equation*}
\begin{cases}
-3\times3\times3=a-ab^2-8b-8\\
-3+3+3=a+1+1
\end{cases}~~~~\Rightarrow~~~~
\begin{cases}
a=1\\
b=2
\end{cases}~~~\text{或}~~~
\begin{cases}
a=1\\
b=-10
\end{cases}
\end{equation*}

$a=1,b=-10$时,$A=
\begin{pmatrix}
  1 & -2&-2 \\
  -2 & 1 & 10\\
  -2 & 10 &1
\end{pmatrix}$,所以
\begin{equation*}
|\lambda E-A|=
\begin{vmatrix}
  \lambda-1 & 2&2 \\
  2 & \lambda-1 & -10\\
 2 & -10 &\lambda-1
\end{vmatrix}=(\lambda-1)[(\lambda-1)^2-100]-4[2(\lambda-1)+20]=0
\end{equation*}
代入$\lambda=3$得:左式等于-278不等于右式,即$a=1,b=-10$不合题意。

$a=1,b=2$时,$A=
\begin{pmatrix}
  1 & -2&-2 \\
  -2 & 1 & -2\\
  -2 & -2 &1
\end{pmatrix}$,所以
\begin{equation*}
|\lambda E-A|=
\begin{vmatrix}
  \lambda-1 & 2&2 \\
  2 & \lambda-1 & 2\\
 2 & 2 &\lambda-1
\end{vmatrix}=-(3-\lambda)[(\lambda-1)(\lambda+1)-8]=0~~~~\Rightarrow~~~\lambda_1=\lambda_2=3,\lambda_3=-3
\end{equation*}
符合题意。

$\lambda=3$时:
\begin{equation*}
[\lambda E-A]=
\begin{bmatrix}
  2 & 2&2 \\
  2 & 2 & 2\\
 2 & 2 &2
\end{bmatrix}\rightarrow
\begin{bmatrix}
  1 & 1&1 \\
  0 & 0 & 0\\
 0 & 0 &0
\end{bmatrix}~~~~~\Rightarrow~~~~~x_1=-x_2-x_3
\end{equation*}\
分别取$[x_2,x_3]^T=[1,0]^T,[0,1]^T$得基础解系:$\alpha_1=[-1,1,0]^T,\alpha_2=[-1,0,1]^T$,所以$\lambda=3$对应的特征向量为$k_1\alpha_1+k_2\alpha_2$($k_1,k_2$不全为0)。

$\lambda=-3$时:
\begin{equation*}
[\lambda E-A]=
\begin{bmatrix}
  -4 & 2&2 \\
  2 & -4 & 2\\
 2 & 2 &-4
\end{bmatrix}\rightarrow
\begin{bmatrix}
  1 & 0&-1 \\
  0 & 1 & -1\\
 0 & 0 &0
\end{bmatrix}~~~~~\Rightarrow~~~~~
\begin{cases}
x_1=x_3\\
x_2=x_3
\end{cases}
\end{equation*}
取$x_3=1$得基础解系:$\alpha_3=[1,1,1]^T$,所以$\lambda=-3$对应的特征向量为$k_3\alpha_3$($k_3\neq 0$)。

综上所述:$\textcolor[rgb]{1.00,0.00,0.00}{a=1,b=2},A$的全部特征向量为:$\textcolor[rgb]{1.00,0.00,0.00}{k_1[-1,1,0]^T+k_2[-1,0,1]^T}$ (\textcolor[rgb]{1.00,0.00,0.00}{$k_1,k_2$不全为0}),\textcolor[rgb]{1.00,0.00,0.00}{$k_3[1,1,1]^T$($k_3\neq 0$)}。
\end{jie}

四、证明题(共计16分)

1.(10分)已知$n$阶方阵$A$满足$A^2+5A+6I=0$,证明$A-2I$可逆,并求出其逆矩阵(用$A$的多项式表示).

\begin{zhengming}
由题得:
\begin{align*}
&A^2+5A+6I=A(A-2I)+2A+5A+6I=A(A-2I)+7(A-2I)+20I=(A+7I)(A-2I)+20I=0\\
&\Rightarrow~~~~-\dfrac{A+7I}{20}(A-2I)=I
\end{align*}
所以$A-2I$可逆,且$(A-2I)^{-1}=\textcolor[rgb]{1.00,0.00,0.00}{-\dfrac{A+7I}{20}}$。
\end{zhengming}

2.(6分)已知向量组$\alpha_1,\alpha_2,\alpha_3,\alpha_4$线性无关,证明$\alpha_1+\alpha_2,\alpha_2+\alpha_3,\alpha_3+\alpha_4,\alpha_4+t\alpha_1$线性相关的充分必要条件是$t=1$。

\begin{zhengming}
充分性:

$\alpha_1+\alpha_2,\alpha_2+\alpha_3,\alpha_3+\alpha_4,\alpha_4+t\alpha_1$线性相关,则存在一组不全为0的数$k_i(i=1,2,3,4)$,使得
\begin{gather*}
k_1(\alpha_1+\alpha_2)+k_2(\alpha_2+\alpha_3)+k_{3}(\alpha_3+\alpha_4)+k_4(\alpha_4+t\alpha_1)=0\\
\Rightarrow\\
(k_1+tk_4)\alpha_1+(k_1+k_2)\alpha_2+(k_2+k_3)\alpha_3+(k_3+k_4)\alpha_{4}=0
\end{gather*}

因为$\alpha_1,\alpha_2,\alpha_3,\alpha_4$线性无关,所以
\begin{equation*}
\begin{cases}
k_1+tk_4=0\\
k_1+k_2=0\\
k_2+k3=0\\
k_3+k_4=0\\
k_1,k_2,k_3,k_4\text{不全为}0
\end{cases}~~~\Rightarrow~~~
k_1=\textcolor[rgb]{1.00,0.00,0.00}{-tk_4}=-k_2=k_3=\textcolor[rgb]{1.00,0.00,0.00}{-k_4}\neq 0~~~\Rightarrow~~~t=1
\end{equation*}

必要性:

$t=1$,所以$(\alpha_1+\alpha_2)+(\alpha_3+\alpha_{4})=(\alpha_{2}+\alpha_{3})+(\alpha_4+t\alpha_1)$,即$\alpha_1+\alpha_2,\alpha_2+\alpha_3,\alpha_3+\alpha_4,\alpha_4+t\alpha_1$线性相关。

证毕。
\end{zhengming}
\end{document}
