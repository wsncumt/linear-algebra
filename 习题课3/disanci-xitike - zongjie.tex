\documentclass{article}
\usepackage[space,fancyhdr,fntef]{ctexcap}
\usepackage[namelimits,sumlimits,nointlimits]{amsmath}
\usepackage[bottom=25mm,top=25mm,left=25mm,right=15mm,centering]{geometry}
\usepackage{xcolor}
\usepackage{paralist}%列表宏包
\usepackage{arydshln}%234页,虚线表格宏包
\pagestyle{fancy} \fancyhf{}
\fancyhead[OL]{~~~班序号:\hfill 学院:\hfill 学号:\hfill 姓名:王松年~~~ \thepage}
%\usepackage{parskip}
%\usepackage{indentfirst}
\usepackage{graphicx}%插图宏包,参见手册318页
\begin{document}

\newcounter{num} \renewcommand{\thenum}{\arabic{num}.} \newcommand{\num}{\refstepcounter{num}\text{\thenum}}

\hphantom{~~}\hfill {\zihao{3}\heiti 第三次习题课} \hfill\hphantom{~~}

\hphantom{~~}\hfill {\zihao{4}\heiti 知识点} \hfill\hphantom{~~}

\num 初等矩阵:对单位矩阵做一次初等变换得到的矩阵。

\num 初等矩阵的分类:
\begin{asparaenum}[(1)]
\item 交换其中的两行或两列;
\item 第$i$行(列)乘上一个非零常数;
\item 第$j$行(列)的$l$倍加到第$i$行(列)。
\end{asparaenum}

\num 矩阵的等价:定义和性质(三条)。

\num 标准型矩阵:$
\begin{bmatrix}
  E_{r} & 0 \\
  0& 0
\end{bmatrix}
$,(r表示单位矩阵的阶数,$r\geq 0$)一个矩阵一定可以通过初等变换化为标准型矩阵。

\num 矩阵的秩。定义。\\
性质(很重要):
\begin{asparaenum}[(1)]
\item 矩阵的秩$\leq$ 矩阵的行数与列数的最小值。即:$R(A_{m\times n})\leq \min\{m,n\}$。
\item 转置不改变矩阵的秩。即$R(A)=R(A^{T})$。
\item 等价矩阵的秩相同。即$A\~{}B$,则$R(A)=R(B)$。
\item $A,B$是同型矩阵(行数和列数相同),$R(A)=R(B)$当且仅当$A\~{}B$。
\item 子矩阵的秩$\leq$矩阵的秩,矩阵的秩$\leq$所有子矩阵的秩之和。
即:$\max\{R(A),R(B)\}\leq R(A,B)\leq R(A)+R(B)$。
\item 矩阵之和的秩$\leq$矩阵的秩之和。即:$R(A+B)\leq R(A)+R(B)$。
\item 矩阵乘积的秩$\leq$乘积因子的秩之最小值。即:$R(AB)\leq \min\{R(A),R(B)\}$。
\item 若$A_{m\times n}B_{n\times l}=0$,则$R(A)+R(B)\leq n$。
\item 左乘或右乘可逆矩阵秩不变。(左乘列满秩矩阵不改变矩阵的秩,右乘行满秩矩阵不改变矩阵的秩。)
\end{asparaenum}

\num 矩阵相抵。课本100页,定理3.2.3

\num 存在唯一性定理。
\end{document}  