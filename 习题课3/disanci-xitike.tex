\documentclass{article}
\usepackage[space,fancyhdr,fntef]{ctexcap}
\usepackage[namelimits,sumlimits,nointlimits]{amsmath}
\usepackage[bottom=25mm,top=25mm,left=25mm,right=15mm,centering]{geometry}
\usepackage{xcolor}
\usepackage{arydshln}%234页,虚线表格宏包
\pagestyle{fancy} \fancyhf{}
\fancyhead[OL]{~~~班序号:\hfill 学院:\hfill 学号:\hfill 姓名:王松年~~~ \thepage}
%\usepackage{parskip}
%\usepackage{indentfirst}
\usepackage{graphicx}%插图宏包,参见手册318页
\begin{document}

\newcounter{num} \renewcommand{\thenum}{\arabic{num}.} \newcommand{\num}{\refstepcounter{num}\text{\thenum}}

\hphantom{~~}\hfill {\zihao{3}\heiti 第三次习题课} \hfill\hphantom{~~}

\hphantom{~~}\hfill {\zihao{4}\heiti 群文件《期中$\&$期末试题》} \hfill\hphantom{~~}

{\heiti \zihao{4} 期中试题}

\num 2015-2016 ~二.5

设$A=
\begin{bmatrix}
  3 & 4 & 1\\
  0 & 2 & 0\\
  5 & 1 & 3
\end{bmatrix}
,B=
\begin{bmatrix}
  2 & -1 & 3\\
  0 & 3 & 1\\
  0 & 0 & 0
\end{bmatrix}
$,求$AB$的秩$r(AB)$。\\

\num 2016-2017 ~一.3

设方程组$
\begin{cases}
 2x_{1}-x_{2}+x_{3}=0\\
 x_{1}+kx_{2}-x_{3}=0\\
 kx_{1}+x_{2}+x_{3}=0
\end{cases}
$有非零解,则$k=$
\underline{\hphantom{~~~~~~~~~~}}。\\


\num 2016-2017 ~一.6

设$A=
\begin{bmatrix}
  1 & -1 & 2 & 1\\
  -1 & a & 2 & 1\\
  3 & 1 & b & -1\\
\end{bmatrix},r(A)=2
$,则$a+b=$\underline{\hphantom{~~~~~~~~~~~~~~~}}。\\

\num 2016-2017 ~二.4(2015-2016的期末试题一大题第2,原题)

设$
A=\begin{bmatrix}
    1 & 1 & 1 & 1\\
    0 & 2 & 2 & 2\\
    0 & 0 & 3 & 3\\
    0 & 0 & 0 & 4
  \end{bmatrix}
$,求$A^{2}-2A$的秩$r(A^{2}-2A)$。\\

\num 2017-2018~一.5

设$
A=\begin{bmatrix}
   2 & 4 & 1 & 0\\
   1 & 0 & 3 & 2\\
   -1 & 5 & -3 & 1\\
   0 & 1 & 0 & 2
  \end{bmatrix},B=
  \begin{bmatrix}
   1& 1& 1 & 1\\
   1 & 1 & 2 & 2\\
   a+1 & 2 & 3 &4\\
   1 & a & 1 & a
  \end{bmatrix}
$,若$r(A)=r(B)$,则$a$应满足什么条件。\\

\num 2018-2019~一.6(1)

设$A=
\begin{bmatrix}
  1 & 0 & 1\\
  -1 & -1 & 1\\
  0 & 2 & a
\end{bmatrix},
B=
\begin{bmatrix}
  1 & 0 & 1\\
  0 & -1 & 2\\
   0 & 0 & 0
\end{bmatrix}
$。

(1)问$a$为何值时,矩阵$A$和$B$等价。

(2)当$A$和$B$等价时,求可逆矩阵$P$,使得$PA=B$。\\

{\heiti \zihao{4} 期末试题}

\num 2014-2015~七

设$A$为$n$阶矩阵,且$A^{2}-A-2I=0$。

(1)证明:$r(A-2I)+r(A+I)=n$.\\

\num 2017-2018~一.2

设$
A=
\begin{bmatrix}
  1 & 1 & 1 \\
  0 & 1 & 1 \\
  2 & 3 & 2
\end{bmatrix},
B=\begin{bmatrix}
    1\\ 2\\ 0
  \end{bmatrix}
  \begin{bmatrix}
  1 & 2&3
  \end{bmatrix}
$,则$r(A+AB)=$\underline{\hphantom{~~~~~~~~~~}}。\\

\num 2019-2020~一.2

已知$A=
\begin{bmatrix}
  1 & -2 & 3k\\
 -1 & 2k & -3\\
 k & -2 & 3
\end{bmatrix}
$的秩为2,则$k=$\underline{\hphantom{~~~~~~~~~~}}。
\end{document}  