\documentclass{article}
\usepackage[space,fancyhdr,fntef]{ctexcap}
\usepackage[namelimits,sumlimits,nointlimits]{amsmath}
\usepackage[bottom=25mm,top=25mm,left=25mm,right=15mm,centering]{geometry}
\usepackage{xcolor}
\usepackage{arydshln}%234页,虚线表格宏包
\pagestyle{fancy} \fancyhf{}
\fancyhead[OL]{~~~班序号:\hfill 学院:\hfill 学号:\hfill 姓名:王松年~~~ \thepage}
%\usepackage{parskip}
%\usepackage{indentfirst}
\usepackage{graphicx}%插图宏包,参见手册318页
\begin{document}

\newcounter{num} \renewcommand{\thenum}{\arabic{num}.} \newcommand{\num}{\refstepcounter{num}\text{\thenum}}

\newenvironment{jie}{\kaishu\zihao{-5}\color{blue}{\noindent\em 解:}\par}{\hfill $\diamondsuit$\par}

\newenvironment{zhengming}{\kaishu\zihao{-5}\color{blue}{\noindent\em 证明:}\par}{\hfill $\diamondsuit$\par}

\hphantom{~~}\hfill {\zihao{3}\heiti 第三次习题课} \hfill\hphantom{~~}

\hphantom{~~}\hfill {\zihao{4}\heiti 群文件《期中$\&$期末试题》} \hfill\hphantom{~~}

{\heiti \zihao{4} 期中试题}

\num 期中2016-2017 一4.

设$f(x)=ax^{2}+bx+c$,$A$为$n$阶方阵,定义$f(A)=aA^{2}+bA+cI$,如果$
A=
\begin{bmatrix}
  1 & 0 & 0 & 0 \\
  0 & 1& 0& 0 \\
  2 & 0 & 1& 0\\
   0 & 0 &0 &1
\end{bmatrix},f(x)=x^{2}-x-1,
$则$f(A)=$\underline{\hphantom{~~~~~~~~~~}}。

\begin{jie}
由题得:
\begin{equation*}
A=E+\begin{bmatrix}
  0 & 0 & 0 & 0 \\
  0 & 0& 0& 0 \\
  2 & 0 & 0& 0\\
   0 & 0 &0 &0
\end{bmatrix}=E+B
\end{equation*}
可以看出:$B^2=0$.所以$A^2=(E+B)^2=E+B^2+2EB=E+2B$.

所以$f(A)=A^2-A-E=E+2B-E-B-E=B-E=
\begin{bmatrix}
  -1 & 0 & 0 & 0 \\
  0 & -1& 0& 0 \\
  2 & 0 & -1& 0\\
   0 & 0 &0 &-1
\end{bmatrix}
$
\end{jie}

\num 期中2017-2018 二1.判断是否成立并给出理由。

设$A,B$为同阶对称方阵,则$AB$一定是对称矩阵;

\begin{jie}
不成立。理由:

以2阶对称矩阵为例:(式中:$a,b,c,x,y,z$为任意实数。)
\begin{equation*}
A=\begin{bmatrix}
  a & b \\
  b & c
\end{bmatrix}~~~B=
\begin{bmatrix}
  x & y \\
  y & z
\end{bmatrix}~~AB=
\begin{bmatrix}
  ax+by & \textcolor[rgb]{1.00,0.00,0.00}{ay+bz}\\
  \textcolor[rgb]{1.00,0.00,0.00}{bx+cy} & by+cz
\end{bmatrix}
\end{equation*}
可以看出,由于$a,b,c,x,y,z$取值的任意性,所以$ay+bz\neq by+cz$。 (可以取$a=1,b=2,c=3,x=3,y=4,z=5$实际验证一下。)
\end{jie}

\num 期中2017-2018 二4.判断是否成立并给出理由。

设2阶矩阵$A=
\begin{bmatrix}
  a & b \\
  c & d
\end{bmatrix}
$,若$A$与所有的2阶矩阵均可以交换,则$a=d,b=c=0$。

\begin{jie}
成立,理由如下:

取任意二阶矩阵($x,y,z,w$为任意实数):$
B=
\begin{bmatrix}
  x & y \\
  z & w
\end{bmatrix}
$,则
\begin{gather*}
AB=\begin{bmatrix}
  a & b \\
  c & d
\end{bmatrix}
\begin{bmatrix}
  x & y \\
  z & w
\end{bmatrix}=
\begin{bmatrix}
  ax+bz & ay+bw \\
  cx+dz & cy+dw
\end{bmatrix}\\
BA=
\begin{bmatrix}
  x & y \\
  z & w
\end{bmatrix}
\begin{bmatrix}
  a & b \\
  c & d
\end{bmatrix}=
\begin{bmatrix}
  xa+cy & xb+yd \\
  az+cw & bz+dw
\end{bmatrix}
\end{gather*}
若$A$与$B$可交换,则有$AB=BA$,即:
\begin{gather}
ax+bz=xa+cy  \\
ay+bw=xb+yd\\
cx+dz=az+cw\\
cy+dw=bz+dw
\end{gather}
由(1)式和(4)式得:$bz=cy$,因为$z$和$y$为任意数,所以$b=c=0$,代入(2)式和(3)式:$ay=dy,az=dz$,所以$a=d$。
\end{jie}

\num 期中2018-2019 二2.

设$A$是$n$阶实对称矩阵,如果$A^{2}=0$,证明$A=0$.并举例说明,如果$A$不是实对称矩阵,上述命题不正确。

\begin{jie}
证明:
依题意设$A=
\begin{bmatrix}
  a_{11} & a_{12} & \cdots & a_{1n} \\
  a_{12} & a_{22} & \cdots & a_{2n} \\
  \vdots & \vdots & \ddots & \vdots \\
  a_{1n} & a_{2n} & \cdots & a_{nn}
\end{bmatrix}
$,所以$A^2$为:只看$A^2$对角线上的元素,$A^2$的第$k$行第$k$列的元素为$A$的第$k$行乘第$k$列:$a_{1k}^2+a_{2k}^2+\cdots+a_{kk}^2+a_{kk+1}^2+a_{kk+2}^2+\cdots+a_{kn}^2=0$,因为平方一定大于等于0,所以该式的每一项都为0,即$a_{1k},a_{2k},\cdots,a_{kk},a_{kk+1},a_{kk+2},\cdots,a_{kn}$为0.即$A$的第$k$行和第$k$列元素为0.\textcolor[rgb]{1.00,0.00,0.00}{(这一步看不懂的计算一下$A^2$的第一行第一列,第二行第二列的元素验证一下)}

因为$A^2$为0,即其对角线每个元素都为0,由上边的步骤可以推出$A$的每行每列元素都为0,即$A=0$。

举例:对于二阶矩阵$
A=\begin{bmatrix}
0& 0\\
1& 0
\end{bmatrix}
$,$A^2=0$,但$A$不是对称矩阵。
\end{jie}

\num 2015-2016 ~二.5

设$A=
\begin{bmatrix}
  3 & 4 & 1\\
  0 & 2 & 0\\
  5 & 1 & 3
\end{bmatrix}
,B=
\begin{bmatrix}
  2 & -1 & 3\\
  0 & 3 & 1\\
  0 & 0 & 0
\end{bmatrix}
$,求$AB$的秩$r(AB)$。

\begin{jie}
知识点:矩阵的性质的应用。总结里的性质9.

方法一:计算出$AB$,然后对$AB$进行高斯消元求出阶梯形矩阵,再求出秩。(步骤略,不讲,自己算)

方法二:

由题得:$r(B)=2$。
\begin{align*}
A\xrightarrow{ \substack{r_{1}\times\frac{1}{3} \\ r_{2}\times \frac{1}{2} \\ r_{3}\times \frac{1}{5}}}
{
\begin{bmatrix}
  1 & \frac{4}{3} & \frac{1}{5}\\
  0 & 1 & 0\\
  1 & \frac{1}{5} & \frac{3}{5}
\end{bmatrix}
}
\xrightarrow{r_{3}-r_{1}}
{
\begin{bmatrix}
  1 & \frac{4}{3} & \frac{1}{5}\\
  0 & 1 & 0\\
  0 & -\frac{17}{15} & \frac{4}{15}
\end{bmatrix}
}\xrightarrow{r_{3}+\frac{17}{15}r_{2}}
{
\begin{bmatrix}
  1 & \frac{4}{3} & \frac{1}{5}\\
  0 & 1 & 0\\
  0 & 0 & \frac{4}{15}
\end{bmatrix}
}
\end{align*}
所以$r(A)=3$,即A满秩。所以$r(AB)=r(B)=2$。

方法三:(行列式还没学,等学到了在来看这个方法)

由题得:
\begin{equation*}
  |A|=
  \begin{vmatrix}
     3 & 4 & 1\\
  0 & 2 & 0\\
  5 & 1 & 3
  \end{vmatrix}=2*
  \begin{vmatrix}
     3 &  1\\
  5 &  3
  \end{vmatrix}=2\times(3\times 3-1\times 5)=8\neq 0
\end{equation*}
所以$A$可逆,$r(AB)=r(B)=2$。

用任意一种方法做都可以,用自己最顺手的即可。
\end{jie}

\num 2016-2017 ~一.3.

设方程组$
\begin{cases}
 2x_{1}-x_{2}+x_{3}=0\\
 x_{1}+kx_{2}-x_{3}=0\\
 kx_{1}+x_{2}+x_{3}=0
\end{cases}
$有非零解,则$k=$
\underline{~~~~-1或4~~~~}。

\begin{jie}
如果方程组有非零解,则$r(A)<3$,当$k=0$时,可以得出$r(A)=3$;对系数矩阵进行高斯消元:
\begin{equation*}
  \begin{bmatrix}
    2 & -1 & 1 \\
    1 & k &-1\\
    k & 1 & 1
  \end{bmatrix}\xrightarrow{ \substack{r_{2}-\frac{1}{2} r_{1}\\ r_{3}-\frac{k}{2}r_{1} }}
{
\begin{bmatrix}
 2 & -1 & 1 \\
  0 & k+\frac{1}{2}  & -\frac{3}{2} \\
  0 & \frac{k}{2}+1  & 1-\frac{k}{2}
\end{bmatrix}
}
\end{equation*}
要使$r(A)<3$,则
\begin{equation*}
\dfrac{k+\frac{1}{2}}{\frac{k}{2}+1}=\dfrac{-\frac{3}{2}}{1-\frac{k}{2}}~~~\Rightarrow k_{1}=-1 ~k_{2}=4
\end{equation*}
\end{jie}

\num 2016-2017 ~一.6.

设$A=
\begin{bmatrix}
  1 & -1 & 2 & 1\\
  -1 & a & 2 & 1\\
  3 & 1 & b & -1\\
\end{bmatrix},r(A)=2
$,则$a+b=$\underline{~~~~~~$-3$~~~~~~}。

\begin{jie}
由题得:
\begin{equation*}
  A\xrightarrow{\substack{r_{2}+r_{1} \\ r_{3}-3r_{1}}}
{
    \begin{bmatrix}
  1 & -1 & 2 & 1\\
  0 & a-1 & 4 & 2\\
  0 & 4 & b-6 & -4\\
\end{bmatrix}
}
\end{equation*}
若$r(A)=2$,则$r_{2}=kr_{3}$其中$k\neq0$.(指的是非零元素成比例)即:
\begin{equation*}
  \frac{a-1}{4}=\frac{4}{b-6}=\frac{2}{-4}~~~\Rightarrow
  \begin{cases}
    a=-1\\ b=-2
  \end{cases}
\end{equation*}
\end{jie}

\num 2016-2017 ~二.4(2015-2016的期末试题一大题第2,原题)

设$
A=\begin{bmatrix}
    1 & 1 & 1 & 1\\
    0 & 2 & 2 & 2\\
    0 & 0 & 3 & 3\\
    0 & 0 & 0 & 4
  \end{bmatrix}
$,求$A^{2}-2A$的秩$r(A^{2}-2A)$。

\begin{jie}
方法一:计算出$A^{2}-2A$。然后对$A^{2}-2A$进行高斯消元求出阶梯形矩阵再求出秩。(步骤略,不讲,自己算)

方法二:

由题得:$A^{2}-2A=A(A-2E)$,可以看出$A$是满秩方阵,即$A$可逆。\textcolor[rgb]{1.00,0.00,0.00}{满秩方阵一定可逆},所以
$r(A^{2}-2A)=r(A(A-2E))=r(A-2E)$。
\begin{align*}
A-2E=
\begin{bmatrix}
-1 & 1 & 1 & 1\\
0 & 0 & 2 & 2\\
0 & 0 & 1 & 3\\
0 & 0 & 0 & 2
\end{bmatrix}
\xrightarrow{r_{2}\times\frac{1}{2}}
{
\begin{bmatrix}
-1 & 1 & 1 & 1\\
0 & 0 & 1 & 1\\
0 & 0 & 1 & 3\\
0 & 0 & 0 & 2
\end{bmatrix}
}\xrightarrow{r_{3}-r_{2}}
{
\begin{bmatrix}
-1 & 1 & 1 & 1\\
0 & 0 & 1 & 1\\
0 & 0 & 0 & 2\\
0 & 0 & 0 & 2
\end{bmatrix}
}\xrightarrow{r_{4}-r_{3}}
{
\begin{bmatrix}
-1 & 1 & 1 & 1\\
0 & 0 & 1 & 1\\
0 & 0 & 0 & 2\\
0 & 0 & 0 & 0
\end{bmatrix}
}
\end{align*}
所以r(A-2E)=3,所以$r(A^{2}-2A)=3$
\end{jie}

\num 2017-2018~一.5.

设$
A=\begin{bmatrix}
   2 & 4 & 1 & 0\\
   1 & 0 & 3 & 2\\
   -1 & 5 & -3 & 1\\
   0 & 1 & 0 & 2
  \end{bmatrix},B=
  \begin{bmatrix}
   1& 1& 1 & 1\\
   1 & 1 & 2 & 2\\
   a+1 & 2 & 3 &4\\
   1 & a & 1 & a
  \end{bmatrix}
$,若$r(A)=r(B)$,则$a$应满足什么条件。

\begin{jie}
由题得
\begin{align*}
A\xrightarrow{r_{1}\times \frac{1}{2}}&
{
\begin{bmatrix}
   1 & 2 & \frac{1}{2} & 0\\
   1 & 0 & 3 & 2\\
   -1 & 5 & -3 & 1\\
   0 & 1 & 0 & 2
  \end{bmatrix}
}\xrightarrow{ \substack{r_{2}-r_{1} \\ r_{3}+r_{1}}}
{
\begin{bmatrix}
   1 & 2 & \frac{1}{2} & 0\\
   0 & -2 & \frac{5}{2} & 2\\
   0 & 7 & -\frac{5}{2} & 1\\
   0 & 1 & 0 & 2
  \end{bmatrix}
}\xrightarrow{r_{2}\Leftrightarrow r_{4}}
{
\begin{bmatrix}
   1 & 2 & \frac{1}{2} & 0\\
   0 & 1 & 0 & 2\\
   0 & 7 & -\frac{5}{2} & 1\\
   0 & -2 & \frac{5}{2} & 2
  \end{bmatrix}
}\xrightarrow{\substack{r_{3}-7 r_{2} \\ r_{4}+2r_{2}}}
{
\begin{bmatrix}
   1 & 2 & \frac{1}{2} & 0\\
   0 & 1 & 0 & 2\\
   0 & 0 & -\frac{5}{2} & -13\\
   0 & 0 & \frac{5}{2} & 4
  \end{bmatrix}
}\\
\xrightarrow{r_{4}+ r_{3}}&
{
\begin{bmatrix}
   1 & 2 & \frac{1}{2} & 0\\
   0 & 1 & 0 & 2\\
   0 & 0 & -\frac{5}{2} & -13\\
   0 & 0 & 0 & -9
  \end{bmatrix}
}
\end{align*}
所以$r(A)=4$。
\begin{equation*}
  B\xrightarrow{\substack{r_{2}- r_{1} \\ r_{3}- r_{1} \\ r_{4}- r_{1} }}
  {
  \begin{bmatrix}
   1& 1& 1 & 1\\
   0 & 0 & 1 & 1\\
   a & 1 & 2 & 3 \\
   0 & a-1 & 1 & a
  \end{bmatrix}
  }=C
\end{equation*}
若$a=0$,则
\begin{equation*}
  C=
   \begin{bmatrix}
   1& 1& 1 & 1\\
   0 & 0 & 1 & 1\\
   0 & 1 & 2 & 3 \\
   0 & -1 & 1 & 0
  \end{bmatrix}
  \xrightarrow{\text{中间步骤略}}
  {
  \begin{bmatrix}
   1& 1& 1 & 1\\
   0 & 1 & -1 & 0\\
   0 & 0 & 1 & 1 \\
   0 & 0 & 0 & 0
  \end{bmatrix}
  }
\end{equation*}
可以看出此时$r(B)=3\neq r(A)$。所以\textcolor[rgb]{1.00,0.00,0.00}{$a\neq0$}。对$C$继续化简:
\begin{equation*}
  C\xrightarrow{r_{3}-ar_{1}}
  {
  \begin{bmatrix}
   1& 1& 1 & 1\\
   0 & 0 & 1 & 1\\
   0 & 1-a & 2-a & 3-a \\
   0 & a-1 & 0 & a-1
  \end{bmatrix}
  }=D
\end{equation*}
若$1-a=0$,即$a=1$,则:
\begin{equation*}
  D=\begin{bmatrix}
   1& 1& 1 & 1\\
   0 & 0 & 1 & 1\\
   0 & 0 & 1 & 2 \\
   0 & 0 & 0 & 0
  \end{bmatrix}
\end{equation*}
可以看出此时$r(A)<4$,所以\textcolor[rgb]{1.00,0.00,0.00}{$a\neq1$},继续对$D$进行化简(注意:此时$a\neq0$且$a\neq1$):
\begin{equation*}
D\xrightarrow{r_{4}+r_{3}}
{
     \begin{bmatrix}
   1& 1& 1 & 1\\
   0 & 0 & 1 & 1\\
   0 & 1-a & 2-a & 3-a \\
   0 & 0 & 2-a & 2
  \end{bmatrix}
}\xrightarrow{r_{2}\Leftrightarrow r_{3}}
{
 \begin{bmatrix}
   1& 1& 1 & 1\\
   0 & 1-a & 2-a & 3-a \\
   0 & 0 & 1 & 1\\
   0 & 0 & 2-a & 2
  \end{bmatrix}
}\xrightarrow{r_{4}-(2-a) r_{3}}
{
\begin{bmatrix}
   1& 1& 1 & 1\\
   0 & 1-a & 2-a & 3-a \\
   0 & 0 & 1 & 1\\
   0 & 0 & 0 & a
  \end{bmatrix}
}
\end{equation*}
因为$a\neq0$且$a\neq1$,所以此时$r(B)=4=r(A)$,所以$a$应满足的条件是$a\neq0$且$a\neq1$。
\end{jie}

\num 2018-2019~一.6(1)

设$A=
\begin{bmatrix}
  1 & 0 & 1\\
  -1 & -1 & 1\\
  0 & 2 & a
\end{bmatrix},
B=
\begin{bmatrix}
  1 & 0 & 1\\
  0 & -1 & 2\\
   0 & 0 & 0
\end{bmatrix}
$。

(1)问$a$为何值时,矩阵$A$和$B$等价。

(2)当$A$和$B$等价时,求可逆矩阵$P$,使得$PA=B$。

\begin{jie}
(1)
由题得:$r(B)=2$,若$A$和和$B$等价,则$r(A)=2$。
\begin{equation*}
 A\xrightarrow{r_{2}+ r_{1}}
{
\begin{bmatrix}
  1 & 0 & 1\\
  0 & -1 & 2\\
  0 & 2 & a
\end{bmatrix}
}\xrightarrow{r_{3}+2r_{2}}
{
\begin{bmatrix}
  1 & 0 & 1\\
  0 & -1 & 2\\
  0 & 0 & a+4
\end{bmatrix}
}
\end{equation*}
$a+4=0$即$a=-4$时,$r(A)=2$,所以$a=-4$。

(2)~~由(1)得:
\begin{equation*}A=
  \begin{bmatrix}
  1 & 0 & 1\\
  0 & -1 & 2\\
  0 & 2 & -4
  \end{bmatrix}\xrightarrow{r_{2}+r_{1}}
{
\begin{bmatrix}
  1 & 0 & 1\\
  0 & -1 & 2\\
  0 & 2 & 4
\end{bmatrix}
}\xrightarrow{r_{3}+2r_{2}}
{
\begin{bmatrix}
  1 & 0 & 1\\
  0 & -1 & 2\\
  0 & 0 & 0
\end{bmatrix}
}=B
\end{equation*}
由初等行变换得:$E(3~2(2))E(2~1(1))A=B$所以
\begin{equation*}
P=E(3~2(2))E(2~1(1))=
  \begin{bmatrix}
    1 & 0 & 0\\
    0 & 1& 0\\
    0 & 2& 1
  \end{bmatrix}
   \begin{bmatrix}
    1 & 0 & 0\\
    1 & 1& 0\\
    0 & 0& 1
  \end{bmatrix}=
   \begin{bmatrix}
    1 & 0 & 0\\
    1 & 1& 0\\
    2 & 2& 1
  \end{bmatrix}
\end{equation*}
\end{jie}

{\heiti \zihao{4} 期末试题}

\num 2014-2015~七

设$A$为$n$阶矩阵,且$A^{2}-A-2I=0$。

(1)证明:$r(A-2I)+r(A+I)=n$.

\begin{zhengming}
由题得:$A^{2}-A-2I=(A-2I)(A+I)=0$。

所以由矩阵秩的性质有:$r(A+2I)+r(A+I)\leq n$。

$r((A-2I)-(A+I))=r(-3I)\leq r(A-2I)+r(-(A+I))=r(A-2I)+r((A+I))$,即$n\leq r(A-2I)+r((A+I))$

所以$r(A-2I)+r(A+I)=n$.
\end{zhengming}

\num 2017-2018~一.2

设$
A=
\begin{bmatrix}
  1 & 1 & 1 \\
  0 & 1 & 1 \\
  2 & 3 & 2
\end{bmatrix},
B=\begin{bmatrix}
    1\\ 2\\ 0
  \end{bmatrix}
  \begin{bmatrix}
  1 & 2&3
  \end{bmatrix}
$,则$r(A+AB)=$\underline{\hphantom{~~~~~~~~~~}}。\\

\begin{jie}
\begin{equation*}
A\xrightarrow{r_{3}-2r_{1}}
{
\begin{bmatrix}
  1 & 1 & 1 \\
  0 & 1 & 1 \\
  0 & 1 & 0
\end{bmatrix}
}\xrightarrow{r_{3}-r_{2}}
{
\begin{bmatrix}
  1 & 1 & 1 \\
  0 & 1 & 1 \\
  0 & 0 & -1
\end{bmatrix}
}
\end{equation*}
$r(A)=3$,满秩。所以$r(A+AB)=r(A(E+B))=r(E+B)$
\begin{align*}
B=&\begin{bmatrix}
    1\\ 2\\ 0
  \end{bmatrix}
  \begin{bmatrix}
  1 & 2&3
  \end{bmatrix}
  =\begin{bmatrix}
     1 & 2 & 3\\
2 & 4 & 6\\
0 & 0 & 0
   \end{bmatrix}
  \\
E+B=&
\begin{bmatrix}
2 & 2 & 3\\
2 & 5 & 6\\
0 & 0 & 1
\end{bmatrix}\xrightarrow{r_{2}-r_{1}}
{
\begin{bmatrix}
2 & 2 & 3\\
0 & 3 & 3\\
0 & 0 & 1
\end{bmatrix}
}
\end{align*}
所以$r(E+B)=3$,所以$r(A+AB)=r(E+B)=3$
\end{jie}

\num 2019-2020~一.2

已知$A=
\begin{bmatrix}
  1 & -2 & 3k\\
 -1 & 2k & -3\\
 k & -2 & 3
\end{bmatrix}
$的秩为2,则$k=$\underline{\hphantom{~~~~~~~~~~}}。

\begin{jie}
若$k=0$,则
\begin{equation*}
A=\begin{bmatrix}
  1 & -2 & 0\\
 -1 & 0 & -3\\
 0 & -2 & 3
\end{bmatrix}
\xrightarrow{r_{2}+r_{1}}
{
\begin{bmatrix}
  1 & -2 & 0\\
 0 & -2 & -3\\
 0 & -2 & 3
\end{bmatrix}
}\xrightarrow{r_{3}-r_{2}}
{
\begin{bmatrix}
  1 & -2 & 0\\
 0 & -2 & -3\\
 0 & 0 & 6
\end{bmatrix}
}
\end{equation*}
所以$r(A)=3\neq2$,即$k\neq0$.
对$A$接着进行化简:
\begin{equation*}
  A\xrightarrow{\substack{r_{2}+r_{1} \\ r_{3}-kr_{1}}}
{
\begin{bmatrix}
  1 & -2 & 3k\\
 0 & 2k-2 & 3k-3\\
 0 & 2k-2 & 3-3k^{2}
\end{bmatrix}
}=B
\end{equation*}
若$k=1$,则
\begin{equation*}
B=
\begin{bmatrix}
  1 & -2 & 3\\
 0 & 0 & 0\\
 0 & 0 & 0
\end{bmatrix}
\end{equation*}
$r(A)=1\neq2$,所以$k\neq1$,继续对$A$进行化简:
\begin{equation*}
  B\xrightarrow{\substack{r_{2}\times\frac{1}{k-1} \\ r_{3}\times\frac{1}{k-1}}}
{
\begin{bmatrix}
  1 & -2 & 3k\\
 0 & 2 & 3\\
 0 & 2 & -3-3k
\end{bmatrix}
}
\end{equation*}
如果要使$r(A)=2$,则
\begin{equation*}
  \frac{2}{2}=\frac{3}{-3-3k}~~~\Rightarrow k=-2
\end{equation*}
\end{jie}
\end{document}  