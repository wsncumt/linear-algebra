\documentclass{article}
\usepackage[space,fancyhdr,fntef]{ctexcap}
\usepackage[namelimits,sumlimits,nointlimits]{amsmath}
\usepackage[bottom=25mm,top=25mm,left=25mm,right=15mm,centering]{geometry}
\usepackage{xcolor}
\usepackage{arydshln}%234页,虚线表格宏包
\usepackage{lastpage}
\pagestyle{fancy} \fancyhf{}
\fancyhead[C]{\kaishu 四川大学线性代数三历年期中测试答案解析}
\fancyfoot[C]{第\thepage 页}
%\usepackage{parskip}
%\usepackage{indentfirst}
\usepackage{graphicx}%插图宏包,参见手册318页
\usepackage{mathdots}%反对角省略号
\usepackage{extarrows}%等号上加文字
\begin{document}

\newcounter{num} \renewcommand{\thenum}{\arabic{num}.} \newcommand{\num}{\refstepcounter{num}\text{\thenum}}

\newenvironment{jie}{\kaishu\zihao{-5}\color{blue}{\noindent\em 解:}\par}{\hfill $\diamondsuit$\par}

\newenvironment{zhengming}{\kaishu\zihao{-5}\color{blue}{\noindent\em 证明:}\par}{\hfill $\diamondsuit$\par}

\hphantom{~~}\hfill {\zihao{2}\heiti 四川大学《线性代数三》历年期中测试答案解析} \hfill\hphantom{~~}

\hphantom{~~}\hfill {\zihao{5}\kaishu 王松年} \hfill\hphantom{~~}

{\zihao{3}\heiti 写在前面:}

\num 答案里说的课本指的是由四川大学数学学院编写的《线性代数》(中国人民大学出版社)。

\num 仅供内部参考使用,请勿将此文档上传到百度文库以及其同类网站上。

\num 由个人整理,如果发现错误或有更好的解题方法请发送邮件至$1315030237@qq.com$.

\hphantom{~~}\hfill {\zihao{4}\heiti 2015-2016年第一学期} \hfill\hphantom{~~}

一、填空题

1.已知$A=
\begin{bmatrix}
  3 & 0 & 4 & 0 \\
  2 & 2 & 2 & 2 \\
  0 & -7 & 0 & 0 \\
  5 & 3 & -2 & 2 \\
\end{bmatrix}
$,则代数余子式之和$A_{41}+A_{42}+A_{43}+A_{44}=$\underline{~~~~0~~~~}。

\begin{jie}
由题得:
\begin{align*}
A_{41}+A_{42}+A_{43}+A_{44}=
\begin{vmatrix}
  3 & 0 & 4 & 0 \\
  2 & 2 & 2 & 2 \\
  0 & -7 & 0 & 0 \\
  \textcolor[rgb]{1.00,0.00,0.00}{1} & \textcolor[rgb]{1.00,0.00,0.00}{1} & \textcolor[rgb]{1.00,0.00,0.00}{1} & \textcolor[rgb]{1.00,0.00,0.00}{1} \\
\end{vmatrix}=a_{3,2}A_{32}=
(-1)^{3+2}\times (-7)\times
\begin{vmatrix}
  3 &  4 & 0 \\
  2 &  2 & 2 \\
  1 &  1 & 1 \\
\end{vmatrix}=0
\end{align*}
\end{jie}

2.设$f(x)=
\begin{vmatrix}
  2x & x & 1 & 2\\
  1 & x & 1 & -1\\
  3 & 2 & x & 1\\
  1 & 1 & 1 & x\\
\end{vmatrix}
$,则$x^{3}$的系数为\underline{\hphantom{~~~~~~~~~~}}。

\begin{jie}
方法一:求出对应的行列式,然后写出$x^3$的系数。(此方法太过繁琐,容易出错,不推荐使用)

用$Matlab$计算出来的结果为:$f(x)=2x^4-x^3-7x^2+12x-8$.(仅供参考)
\\

方法二:使用定义,课本106-108页。

思路:使用行列式的定义来做。仅找出与$x^{3}$有关的项。这里取列按照自然排列,行由自己指定(也可以取行按照自然排列,列由自己指定)。

第一列中:取第一行,第二列第三列第四列无论怎么取都不可能构成$x^{3}$。

\textcolor[rgb]{1.00,0.00,0.00}{(注意:在行列式的定义式中,每一项中的几个元素必须来自不同的行数和列数,如:对于此题来说,列按自然排列,第一个元素取第一列中的第一行,那么第二个元素只能从剩下三列中的剩下三行来取)}

第一列中:取第二行,第二列取第一行,第三列取第三行,第四列取第四行。即$(-1)^{\tau(2134)}a_{\textcolor[rgb]{1.00,0.00,0.00}{2}\textcolor[rgb]{0.00,0.00,0.00}{1}}a_{\textcolor[rgb]{1.00,0.00,0.00}{1}\textcolor[rgb]{0.00,0.00,0.00}{2}}a_{\textcolor[rgb]{1.00,0.00,0.00}{3}\textcolor[rgb]{0.00,0.00,0.00}{3}}a_{\textcolor[rgb]{1.00,0.00,0.00}{4}\textcolor[rgb]{0.00,0.00,0.00}{4}}=(-1)^{1}1*x*x*x=-x^{3}$(注意下标,列(黑色)是自然排列,行(红色)是上边分析得来的)

第一列取第三行第四行都不能构成$x^{3}$。

\textcolor[rgb]{0.50,0.00,0.00}{
验证:$x^{2}$的系数
}

列取自然排列,行按下述几个取时构成$x^2$:$1324~~3124~~3214~~4231~~4132$
即:
\begin{align*}
&(-1)^{\tau(1324)}a_{11}a_{32}a_{23}a_{44}+(-1)^{\tau(3124)}a_{31}a_{12}a_{23}a_{44}+\\
&(-1)^{\tau(3214)}a_{31}a_{22}a_{13}a_{44}+(-1)^{\tau(4132)}a_{41}a_{12}a_{33}a_{24}+(-1)^{\tau(4231)}a_{41}a_{22}a_{33}a_{14}\\
=&(-4+3-3-1-2)x^2=-7x^2
\end{align*}
可以看出和方法1算的结果一样。
\end{jie}

3.
设$a,b,c$满足方程
$
\begin{vmatrix}
  1 & a & b & c\\
  a & 1 & 0 & 0\\
  b & 0 & 1 & 0\\
  c & 0 & 0 & 1\\
\end{vmatrix}=1
$,则$abc=$\underline{\hphantom{~~~~~~~~~~}}。

\begin{jie}
对行列式做如下变换:第一行减去$c$倍的第四行,第一行减去$b$倍的第三行,第一行减去$a$倍的第二行,得到如下行列式:
\begin{equation*}
\begin{vmatrix}
  1 & a & b & c\\
  a & 1 & 0 & 0\\
  b & 0 & 1 & 0\\
  c & 0 & 0 & 1\\
\end{vmatrix}=\begin{vmatrix}
  1-a^{2}-b^{2}-c^{2} & 0 & 0 & 0\\
  a & 1 & 0 & 0\\
  b & 0 & 1 & 0\\
  c & 0 & 0 & 1\\
\end{vmatrix}=1-a^{2}-b^{2}-c^{2}=1
\end{equation*}
即$a^{2}+b^{2}+c^{2}=0$,对于实数,任何数的平方都大于等于0,所以可以推出$a=0,b=0,c=0$,所以$abc=0$。
\end{jie}

4.
设$A=
\begin{bmatrix}
  2 & 0 \\
  1 & 4
\end{bmatrix}
$,若$B=2BA-3I$,其中$I$为单位矩阵,则$|B|=$\underline{\hphantom{~~~~~~~~~~}}。

\begin{jie}
由题得:$B=2BA-3I~~~\Rightarrow B(2A-I)=3I$,所以$|B(2A-I)|=|3I|=3^{2}=9$,$|2A-I|=
\begin{vmatrix}
  3 & 0 \\
  2 & 7
\end{vmatrix}=21
$,所以$|B(2A-I)|=|B||2A-I|=21|B|=9$,解得$|B|=\dfrac{3}{7}$
\end{jie}

5.
若$A$为4阶方阵,$A^{*}$为$A$的伴随矩阵,$|A|=\dfrac{1}{2}$,则$\left|\left(\dfrac{1}{4}A\right)^{-1}-A^{*}\right|=$\underline{\hphantom{~~~~~~~~~~}}。

\begin{jie}
$\left|\left(\dfrac{1}{4}A\right)^{-1}-A^{*}\right|=\left|\left(\dfrac{1}{4}A\right)^{-1}-|A|A^{-1}\right|=\left|4A^{-1}-\frac{1}{2}A^{-1}\right|=\left|\frac{7}{2}A^{-1}\right|=\left(\frac{7}{2}\right)^{4}|A|^{-1}=\frac{7^4}{8}$
\end{jie}

6.设$A=
\begin{bmatrix}
  1 & 0 & 0\\
  1 & 1 & 0\\
  1 & 2 & 3
\end{bmatrix}
$,则$(A^*)^{-1}=$\underline{\hphantom{~~~~~~~~~~}}。

\begin{jie}
$A^*=|A|A^{-1}$,所以$(A^*)^{-1}=(|A|A^{-1})^{-1}=|A|^{-1}(A^{-1})^{-1}=|A|^{-1}A$

$|A|=1\times 1\times 3=3$
所以:$(A^*)^{-1}=\dfrac{1}{3}A=\begin{bmatrix}
  \frac{1}{3} & 0 & 0\\
  \frac{1}{3} & \frac{1}{3} & 0\\
  \frac{1 }{3}& \frac{2}{3} & 1
\end{bmatrix}$
\end{jie}

7.设$A=
\begin{bmatrix}
  1 & 0 & 0 & 0\\
  0 & 1 & 0 & 0\\
  2 & 0 & 1 & 0\\
  0 & 0 & 0 & 1
\end{bmatrix},B=
\begin{bmatrix}
  1 & 1 & 2 & 3\\
  0 & 1 & 1 & -4\\
  1 & 2 & 3 & -1\\
  2 & 3 & -1 & -1
\end{bmatrix}
$,则$ABA^{-1}=$\underline{\hphantom{~~~~~~~~~~}}。

\begin{jie}
由题得:$A=
\begin{bmatrix}
  A_{11} & 0 \\
   A_{21}& A_{22}
\end{bmatrix}
$,其中$A_{11}=A_{22}=
\begin{bmatrix}
  1 & 0 \\
  0 & 1
\end{bmatrix}=E_{2}
,A_{21}=
\begin{bmatrix}
  2 & 0 \\
  0 & 0
\end{bmatrix}
$.所以
\begin{equation*}
  A^{-1}=
  \begin{bmatrix}
    E_{2} & 0 \\
    -E_{2}A_{21}E_{2} & E_{2}
  \end{bmatrix}=
  \begin{bmatrix}
    1 & 0 & 0 & 0\\
  0 & 1 & 0 & 0\\
  -2 & 0 & 1 & 0\\
  0 & 0 & 0 & 1
  \end{bmatrix}
\end{equation*}
所以
\begin{align*}
ABA^{-1}=\begin{bmatrix}
  1 & 0 & 0 & 0\\
  0 & 1 & 0 & 0\\
  2 & 0 & 1 & 0\\
  0 & 0 & 0 & 1
\end{bmatrix}
\begin{bmatrix}
  1 & 1 & 2 & 3\\
  0 & 1 & 1 & -4\\
  1 & 2 & 3 & -1\\
  2 & 3 & -1 & -1
\end{bmatrix}
  \begin{bmatrix}
    1 & 0 & 0 & 0\\
  0 & 1 & 0 & 0\\
  -2 & 0 & 1 & 0\\
  0 & 0 & 0 & 1
  \end{bmatrix}=\begin{bmatrix}
  1 & 1 & 2 & 3\\
  0 & 1 & 1 & -4\\
  3 & 4 & 7 & 5\\
  2 & 3 & -1 & -1
\end{bmatrix}
  \begin{bmatrix}
    1 & 0 & 0 & 0\\
  0 & 1 & 0 & 0\\
  -2 & 0 & 1 & 0\\
  0 & 0 & 0 & 1
  \end{bmatrix}=\begin{bmatrix}
  -3 & 1 & 2 & 3\\
  -2 & 1 & 1 & -4\\
  -11 & 4 & 7 & 5\\
  4 & 3 & -1 & -1
\end{bmatrix}
\end{align*}
\end{jie}

8.设$A,B$均为$n$阶方阵,$|A|=2$,且$AB$可逆,则$r(B)=$\underline{\hphantom{~~~~~~~~~~}}。

\begin{jie}
课本97页命题3.2.4.

$|A|=2\neq0$,所以$A$可逆,又因为$AB$可逆,所以$B$可逆,即$B$满秩。$r(B)=n$
\end{jie}

二、解答题

1.计算行列式$
\begin{vmatrix}
  5 & 0 & 4 & 2\\
  1 & -1 & 2 & 1\\
  4 & 1 & 2 & 0\\
  1 & 1 & 1 & 1\\
\end{vmatrix}
$。

\begin{jie}
(过程不唯一)

对行列式做如下变换:把第二行加到第三行,把第二行加到第四行,由行列式的性质,此时行列式的值没有改变。即
\begin{align*}
\begin{vmatrix}
  5 & 0 & 4 & 2\\
  1 & -1 & 2 & 1\\
  4 & 1 & 2 & 0\\
  1 & 1 & 1 & 1\\
\end{vmatrix}=
\begin{vmatrix}
  5 & 0 & 4 & 2\\
  1 & -1 & 2 & 1\\
  5 & 0 & 4 & 1\\
  2 & 0 & 3 & 2\\
\end{vmatrix}
=(-1)^{2+2}\times (-1)\times
\begin{vmatrix}
  5 &  4 & 2\\
  5 &  4 & 1\\
  2 &  3 & 2\\
\end{vmatrix}=-\begin{vmatrix}
  5 &  4 & 2\\
  5 &  4 & 1\\
  2 &  3 & 2\\
\end{vmatrix}=-|A|
\end{align*}
对于$|A|$:把第二行的负一倍加到第一行
\begin{equation*}
-\begin{vmatrix}
  5 &  4 & 2\\
  5 &  4 & 1\\
  2 &  3 & 2\\
\end{vmatrix}=-
\begin{vmatrix}
  0 &  0 & 1\\
  5 &  4 & 1\\
  2 &  3 & 2\\
\end{vmatrix}=-(-1)^{1+3}\times(5\times 3-2\times 4)=-7
\end{equation*}
\end{jie}

2.计算行列式$
\begin{vmatrix}
  a^{2} & ab & b^{2}\\
  2a & a+b & 2b \\
  1 & 1 & 1
\end{vmatrix}
$。

\begin{jie}
对行列式做如下变换:把第二列的负一倍分别加到第一列和第三列上。得到$
\begin{vmatrix}
  a(a-b) & ab & b(b-a)\\
  a-b & a+b & b-a \\
  0 & 1 & 0
\end{vmatrix}
$

把第三列加到第一列上:
\begin{equation*}
\begin{vmatrix}
  (a-b)^{2} & ab & b(b-a)\\
  0 & a+b & b-a \\
  0 & 1 & 0
\end{vmatrix}=(-1)^{1+1}(a-b)^{2}[(a+b)0-(b-a)]=(a-b)^{3}
\end{equation*}
\end{jie}

3.若$(2I-C^{-1}B)A^{T}=C^{-1}$,
$B=
\begin{bmatrix}
  1 & 2 & -3 & -2\\
  0 & 1 & 2 & -3\\
  0 & 0 & 1 & 2\\
  0 & 0 & 0 & 1\\
\end{bmatrix},C=
\begin{bmatrix}
  1 & 2 & 0 & 1\\
  0 & 1 & 2 & 0\\
  0 & 0 & 1 & 2\\
  0 & 0 & 0 & 1\\
\end{bmatrix}
$,求$A$。

\begin{jie}
由题得:$(2I-C^{-1}B)A^{T}=C^{-1}$,等式两边左乘$(2I-C^{-1}B)^{-1}$:
\begin{align*}
A^{T}&=(2I-C^{-1}B)^{-1}C^{-1}=[C(2I-C^{-1}B)]^{-1}\\
&=(2C-B)^{-1}
\end{align*}
\begin{equation*}
D=2C-B=2\begin{bmatrix}
  1 & 2 & 0 & 1\\
  0 & 1 & 2 & 0\\
  0 & 0 & 1 & 2\\
  0 & 0 & 0 & 1\\
\end{bmatrix}-\begin{bmatrix}
  1 & 2 & -3 & -2\\
  0 & 1 & 2 & -3\\
  0 & 0 & 1 & 2\\
  0 & 0 & 0 & 1\\
\end{bmatrix}=
\begin{bmatrix}
  1 & 2 & 3 & 4\\
  0 & 1 & 2 & 3\\
  0 & 0 & 1 & 2\\
  0 & 0 & 0 & 1\\
\end{bmatrix}
\end{equation*}
所以
\begin{align*}
[D|E_{4}]&\xrightarrow{\substack{r_{1}-4r_{4}\\ r_{2}-3r_{4}\\ r_{3}-2r_{4}}}
{\left[
\begin{array}{c:c}
\begin{matrix}
  1 & 2 & 3 & 0\\
  0 & 1 & 2 & 0\\
  0 & 0 & 1 & 0\\
  0 & 0 & 0 & 1\\
\end{matrix} &
\begin{matrix}
  1 & 0 & 0 & -4\\
  0 & 1 & 0 & -3\\
  0 & 0 & 1 & -2\\
  0 & 0 & 0 & 1\\
\end{matrix}
\end{array}
\right]
}\xrightarrow{\substack{r_{1}-3r_{3}\\ r_{2}-2r_{3}}}
{\left[
\begin{array}{c:c}
\begin{matrix}
  1 & 2 & 0 & 0\\
  0 & 1 & 0 & 0\\
  0 & 0 & 1 & 0\\
  0 & 0 & 0 & 1\\
\end{matrix} &
\begin{matrix}
  1 & 0 & -3 & 2\\
  0 & 1 & -2 & 1\\
  0 & 0 & 1 & -2\\
  0 & 0 & 0 & 1\\
\end{matrix}
\end{array}
\right]
}\\
&\xrightarrow{\substack{r_{1}-2r_{2}}}
{\left[
\begin{array}{c:c}
\begin{matrix}
  1 & 0 & 0 & 0\\
  0 & 1 & 0 & 0\\
  0 & 0 & 1 & 0\\
  0 & 0 & 0 & 1\\
\end{matrix} &
\begin{matrix}
  1 & -2 & 1 & 0\\
  0 & 1 & -2 & 1\\
  0 & 0 & 1 & -2\\
  0 & 0 & 0 & 1\\
\end{matrix}
\end{array}
\right]
}
\end{align*}
所以$A^{T}=D^{-1}=(2C-B)^{-1}=\begin{bmatrix}
  1 & -2 & 1 & 0\\
  0 & 1 & -2 & 1\\
  0 & 0 & 1 & -2\\
  0 & 0 & 0 & 1\\
\end{bmatrix}$
,$
  A=(A^{T})^{T}=\begin{bmatrix}
  1 & 0 & 0 & 0\\
  -2 & 1 & 0 & 0\\
  1 & -2 & 1 & 0\\
  0 & 1 & -2 & 1\\
\end{bmatrix}
.$
\end{jie}

4.设$
A=
\begin{bmatrix}
  1 & -1 & 0 \\
  0 & 1 & -1\\
  0 & 0 & 1
\end{bmatrix},B=
\begin{bmatrix}
  2 & 1 & 3 \\
  0 & 2 & 1\\
  0 & 0 & 2
\end{bmatrix}
$,$A^{T}(BA^{-1}-I)^{T}X=B^{T}$,求$X$。

\begin{jie}
由题得:$A^{T}(BA^{-1}-I)^{T}=[(BA^{-1}-I)A]^{T}=(B-A)^{T}$,所以
\begin{equation*}
  X=((B-A)^{T})^{-1}B^{T}=((B-A)^{-1})^{T}B^{T}=[B(B-A)^{-1}]^{T}
\end{equation*}
\begin{equation*}
  B-A=\begin{bmatrix}
  2 & 1 & 3 \\
  0 & 2 & 1\\
  0 & 0 & 2
\end{bmatrix}-\begin{bmatrix}
  1 & -1 & 0 \\
  0 & 1 & -1\\
  0 & 0 & 1
\end{bmatrix}=
\begin{bmatrix}
  1 & 2 & 3 \\
  0 & 1 & 2\\
  0 & 0 & 1
\end{bmatrix}
\end{equation*}
(\textcolor[rgb]{0.00,1.00,0.50}{由上一题可以看出,B-A是上一题目中D的左上角三行三列的元素。其逆矩阵也应该是D逆矩阵左上角三行三列,这里直接用结论})
所以:
\begin{gather*}
( B-A)^{-1}=\begin{bmatrix}
  1 & -2 & 1 \\
  0 & 1 & -2 \\
  0 & 0 & 1
\end{bmatrix}~~~~~~
B( B-A)^{-1}=\begin{bmatrix}
  2 & 1 & 3 \\
  0 & 2 & 1\\
  0 & 0 & 2
\end{bmatrix}\begin{bmatrix}
  1 & -2 & 1 \\
  0 & 1 & -2 \\
  0 & 0 & 1
\end{bmatrix}=
\begin{bmatrix}
  2 & -3 & 3 \\
  0 & 2 & -3 \\
  0 & 0 & 2
\end{bmatrix}\\
X=\left[B(B-A)^{-1}\right]^{T}=
\begin{bmatrix}
  2 & 0 & 0 \\
  -3 & 2 & 0\\
  3 & -3 & 2
\end{bmatrix}
\end{gather*}
\end{jie}

5.设$A=
\begin{bmatrix}
  3 & 4 & 1\\
  0 & 2 & 0\\
  5 & 1 & 3
\end{bmatrix}
,B=
\begin{bmatrix}
  2 & -1 & 3\\
  0 & 3 & 1\\
  0 & 0 & 0
\end{bmatrix}
$,求$AB$的秩$r(AB)$。

\begin{jie}
由题得:$r(B)=2$。

\begin{equation*}
|A|=2\times
\begin{vmatrix}
3&1\\
5&3
\end{vmatrix}=8\neq0
\end{equation*}

所以$A$可逆,即A满秩,$r(A)=3$,所以$r(AB)=r(B)=2$。
\end{jie}

三、证明题

1.设$A$可逆,且$A^{*}B=A^{-1}+B$,证明$B$可逆,当$A=
\begin{bmatrix}
  2 & 6 & 0 \\
  0 & 2 & 6\\
  0 & 0 & 2
\end{bmatrix}
$时,求$B$。

\begin{zhengming}
由题得:$A^{*}B=A^{-1}+B$,即$(A^*-E)B=A^{-1}$,两边同时左乘$A$得:$A(A^*-E)B=E$,所以$B$可逆,其逆矩阵为$A(A^*-E)=(|A|E-A)$.

由题得:$|A|=2\times2\times 2=8$
\begin{equation*}
B=(|A|E-A)^{-1}=\begin{bmatrix}
  6 & -6 & 0 \\
  0 & 6 & -6\\
  0 & 0 & 6
\end{bmatrix}^{-1}=\frac{1}{6}\begin{bmatrix}
  1 & 1 & 1 \\
  0 & 1 & 1\\
  0 & 0 & 1
\end{bmatrix}
\end{equation*}
\end{zhengming}

2. \textcolor[rgb]{1.00,0.00,0.00}{(18-19学年二大题的第1题)}
设$A$为$n$阶方阵,$AA^{T}=I$,$|A|<0$,证明:$|A+I|=0$。

\begin{zhengming}
由行列式的性质得$|AB|=|A||B|,|A^{T}|=|A|$,所以由题得

$|AA^{T}|=|I|$,等号左边:$|AA^{T}|=|A||A^{T}|=|A||A|=|A|^{2}$,等号右边等于1,由题得$|A|<0$,所以$|A|=-1$.

$|A+I|=|A+AA^{T}|=|A(I+A^{T})|=|A||(I+A)^{T}|=-|I+A|$,所以$|A+I|=0$。\textcolor[rgb]{1.00,0.00,0.00}{(注:两个矩阵相加的转置等于两个矩阵分别转置后相加,即$A^{T}+B^{T}=(A+B)^{T}$)}
\end{zhengming}
\newpage
\hphantom{~~}\hfill {\zihao{4}\heiti 2016-2017年第一学期} \hfill\hphantom{~~}

一、填空题

1.设$M_{ij}$是$
\begin{vmatrix}
  0 & 4 & 0 \\
  2 & 2 & 2\\
  2 & 0 & 0
\end{vmatrix}
$的第$i$行第$j$列元素的余子式,则$M_{11}+M_{12}=$\underline{\hphantom{~~~~~~~~~~}}。

\begin{jie}
由题得:$M_ {11}+M_{12}=M_ {11}+M_{12}+0M_{13}=A_{11}-A_{12}+0A_{13}=\begin{vmatrix}
  \textcolor[rgb]{1.00,0.00,0.00}{1} & \textcolor[rgb]{1.00,0.00,0.00}{-1} & \textcolor[rgb]{1.00,0.00,0.00}{0} \\
  2 & 2 & 2\\
  2 & 0 & 0
\end{vmatrix}=2\times(-1)^{(3+1)}\times[(-1)\times 2-2\times0]=-4$
\end{jie}

2.计算行列式$
\begin{vmatrix}
  1 & 1 & 1 & 1 \\
  1 & 2 & 4& 8 \\
  1 & 3 & 9& 27\\
   1 & 4 &16 &64
\end{vmatrix}
=$\underline{\hphantom{~~~~~~~~~~}}。

\begin{jie}
经观察,该行列式为四阶范德蒙行列式,且$x_{1}=1,x_{2}=2,x_{3}=3,x_{4}=4$,所以原式$=(x_{4}-x_{3})(x_{4}-x_{2})(x_{4}-x_{1})(x_{3}-x_{2})(x_{3}-x_{1})(x_{2}-x_{1})=12$
\end{jie}

3.设方程组$
\begin{cases}
 2x_{1}-x_{2}+x_{3}=0\\
 x_{1}+kx_{2}-x_{3}=0\\
 kx_{1}+x_{2}+x_{3}=0
\end{cases}
$有非零解,则$k=$
\underline{~~~~-1或4~~~~}。

\begin{jie}
该方程组的系数矩阵为一个三阶方阵,由解得存在唯一性定理,如果$Ax=0$有非零解,则$|A|=0$。所以:
\begin{equation*}
|A|=
\begin{vmatrix}
2 & -1 & 1 \\
1 & k &-1\\
k & 1 & 1
\end{vmatrix}=(1+k)(4-k)=0~~~\Rightarrow k_{1}=-1 ~k_{2}=4
\end{equation*}
(上述行列式第一行减去第三行,第二行加上第三行后按第三列展开)
\end{jie}

4.设$f(x)=ax^{2}+bx+c$,$A$为$n$阶方阵,定义$f(A)=aA^{2}+bA+cI$,如果$
A=
\begin{bmatrix}
  1 & 0 & 0 & 0 \\
  0 & 1& 0& 0 \\
  2 & 0 & 1& 0\\
   0 & 0 &0 &1
\end{bmatrix},f(x)=x^{2}-x-1,
$则$f(A)=$\underline{\hphantom{~~~~~~~~~~}}。

\begin{jie}
由题得:
\begin{equation*}
A=E+\begin{bmatrix}
  0 & 0 & 0 & 0 \\
  0 & 0& 0& 0 \\
  2 & 0 & 0& 0\\
   0 & 0 &0 &0
\end{bmatrix}=E+B
\end{equation*}
可以看出:$B^2=0$.所以$A^2=(E+B)^2=E+B^2+2EB=E+2B$.

所以$f(A)=A^2-A-E=E+2B-E-B-E=B-E=
\begin{bmatrix}
  -1 & 0 & 0 & 0 \\
  0 & -1& 0& 0 \\
  2 & 0 & -1& 0\\
   0 & 0 &0 &-1
\end{bmatrix}
$
\end{jie}

5.若$A$为3阶方阵,$A^{*}$为$A$的伴随矩阵,$|A|=\dfrac{1}{2}$,则$\left|(3A)^{-1}-2A^{*}\right|=$\underline{\hphantom{~~~~~~~~~~}}。

\begin{jie}
$\left|(3A)^{-1}-2A^{*}\right|=\left|3^{-1}A^{-1}-2|A|A^{-1}\right|=\left|-\frac{2}{3}A^{-1}\right|=\left(-\frac{2}{3}\right)^{3}|A|^{-1}=-\frac{16}{27}$
\end{jie}

6.设$A=
\begin{bmatrix}
  1 & -1 & 2 & 1\\
  -1 & a & 2 & 1\\
  3 & 1 & b & -1\\
\end{bmatrix},r(A)=2
$,则$a+b=$\underline{~~~~~~$-3$~~~~~~}。

\begin{jie}
由题得:
\begin{equation*}
  A\xrightarrow{\substack{r_{2}+r_{1} \\ r_{3}-3r_{1}}}
{
    \begin{bmatrix}
  1 & -1 & 2 & 1\\
  0 & a-1 & 4 & 2\\
  0 & 4 & b-6 & -4\\
\end{bmatrix}
}
\end{equation*}
若$r(A)=2$,则$r_{2}=kr_{3}$其中$k\neq0$.(指的是非零元素成比例)即:
\begin{equation*}
  \frac{a-1}{4}=\frac{4}{b-6}=\frac{2}{-4}~~~\Rightarrow
  \begin{cases}
    a=-1\\ b=-2
  \end{cases}
\end{equation*}
\end{jie}

二、解答题

1.计算行列式$
\begin{vmatrix}
  1 & 2 & 3 & 4 \\
  2 & 3 & 4& 1 \\
  3 & 4 & 1& 2\\
   4 & 1 &2 &3
\end{vmatrix}
$。

\begin{jie}
步骤不唯一

把所有列加到第一列,原行列式变为
\begin{equation*}
\begin{vmatrix}
  10 & 2 & 3 & 4 \\
  10 & 3 & 4& 1 \\
  10 & 4 & 1& 2\\
   10 & 1 &2 &3
\end{vmatrix}=10\begin{vmatrix}
  1 & 2 & 3 & 4 \\
  1 & 3 & 4& 1 \\
  1 & 4 & 1& 2\\
   1 & 1 &2 &3
\end{vmatrix}
\end{equation*}
把第一行的负一倍分别加到第二行,第三行,第四行。$10
\begin{vmatrix}
  1 & 2 & 3 & 4 \\
  0 & 1 & 1& -3 \\
  0 & 2 & -2& -2\\
   0 & -1 &-1 &-1
\end{vmatrix}
$,把第四行加到第二行\begin{align*}
10
\begin{vmatrix}
  1 & 2 & 3 & 4 \\
  0 & 0 & 0& -4 \\
  0 & 2 & -2& -2\\
   0 & -1 &-1 &-1
\end{vmatrix}=10\times(-4)\times2\times(-1)\begin{vmatrix}
  1 & 2 & 3 & 4 \\
  0 & 0 & 0& 1 \\
  0 & 1 & -1& -1\\
   0 & 1 &1 &1
\end{vmatrix}=80\times1\times(-1)^{2+4}\begin{vmatrix}
  1 & 2 & 3  \\
  0 & 1 & -1\\
   0 & 1 &1
\end{vmatrix}=160
\end{align*}
\end{jie}

2.设$f(x)=
\begin{vmatrix}
  x-1 & 1 & -1 & 1 \\
  -1 & x+1 & -1& 1 \\
  -1 & 1 & x-1& 1\\
   -1 & 1 &-1 &x+1
\end{vmatrix}
$,求$f(x)=0$的根。

\begin{jie}
分别把第四行的负一倍加到第一行、第二行与第三行上
\begin{equation*}
\begin{vmatrix}
  x & 0 & 0 & -x \\
  0 & x & 0& -x \\
  0 & 0 & x& -x\\
   -1 & 1 &-1 &x+1
\end{vmatrix}
\end{equation*}
把前边三列全部加到第四列上:
\begin{equation*}
\begin{vmatrix}
  x & 0 & 0 & 0 \\
  0 & x & 0& 0 \\
  0 & 0 & x& 0\\
   -1 & 1 &-1 &x
\end{vmatrix}=x^{4}
\end{equation*}
所以$f(x)=x^{4}=0$的根为$x_{1}=x_{2}=x_{3}=x_{4}=0$.
\end{jie}

3.设$
A=
\begin{bmatrix}
  1& 0 &-1 \\
  1&3&0\\
  0&2&1
\end{bmatrix}
$,若矩阵$X$满足方程$AX+I=A^{2}+X$,求$X$。

\begin{jie}
由题得:$AX+I=A^{2}+X$,所以$(A-I)X=A^{2}-I=(A+I)(A-I)=(A-I)(A+I)$.
等式两边同时左乘$(A-I)^{-1}$:
\begin{equation*}
 (A-I)^{-1}(A-I)X=(A-I)^{-1}(A-I)(A+I) ~~~\Rightarrow~~~X=(A+I)= \begin{bmatrix}
       2&0&-1\\
  1&4&0\\
  0&2&2\\
    \end{bmatrix}
\end{equation*}
\end{jie}

4.(2015-2016的期末试题一大题第2,原题)

设$
A=\begin{bmatrix}
    1 & 1 & 1 & 1\\
    0 & 2 & 2 & 2\\
    0 & 0 & 3 & 3\\
    0 & 0 & 0 & 4
  \end{bmatrix}
$,求$A^{2}-2A$的秩$r(A^{2}-2A)$。

\begin{jie}
由题得:$A^{2}-2A=A(A-2E)$,可以看出$A$是满秩方阵,即$A$可逆。\textcolor[rgb]{1.00,0.00,0.00}{满秩方阵一定可逆},所以
$r(A^{2}-2A)=r(A(A-2E))=r(A-2E)$。
\begin{align*}
A-2E=
\begin{bmatrix}
-1 & 1 & 1 & 1\\
0 & 0 & 2 & 2\\
0 & 0 & 1 & 3\\
0 & 0 & 0 & 2
\end{bmatrix}
\xrightarrow{r_{2}\times\frac{1}{2}}
{
\begin{bmatrix}
-1 & 1 & 1 & 1\\
0 & 0 & 1 & 1\\
0 & 0 & 1 & 3\\
0 & 0 & 0 & 2
\end{bmatrix}
}\xrightarrow{r_{3}-r_{2}}
{
\begin{bmatrix}
-1 & 1 & 1 & 1\\
0 & 0 & 1 & 1\\
0 & 0 & 0 & 2\\
0 & 0 & 0 & 2
\end{bmatrix}
}\xrightarrow{r_{4}-r_{3}}
{
\begin{bmatrix}
-1 & 1 & 1 & 1\\
0 & 0 & 1 & 1\\
0 & 0 & 0 & 2\\
0 & 0 & 0 & 0
\end{bmatrix}
}
\end{align*}
所以r(A-2E)=3,所以$r(A^{2}-2A)=3$
\end{jie}

5.若$\left(\dfrac{1}{4}A^{*}\right)^{-1}BA^{-1}=2AB+I$,且
$A=
\begin{bmatrix}
  2 & 0& 0 & 0 \\
  1 & 1 & 0& 0 \\
  0 & 0 & 2& 1\\
   0& 0 &0 &1
\end{bmatrix}
$,求$B$。

\begin{jie}
由题得:$A=
\begin{bmatrix}
  A_{11} & 0 \\
  0 & A_{22}
\end{bmatrix}
$,其中$A_{11}=\begin{bmatrix}
              2 & 0 \\
              1 & 1
            \end{bmatrix}$,
$A_{22}=
\begin{bmatrix}
  2 & 1 \\
  0 & 1
\end{bmatrix}
$,所以有:
\begin{equation*}
|A_{11}|=|A_{22}|=2\times1=2;~~|A|=|A_{11}|\cdot|A_{22}|=4;~~A_{11}^{-1}=
\begin{bmatrix}
  \frac{1}{2} & 0 \\
  -\frac{1}{2} & 1
\end{bmatrix};~~A_{22}^{-1}=
\begin{bmatrix}
  \frac{1}{2} & -\frac{1}{2} \\
   0& 1
\end{bmatrix};~~A^{-1}=\begin{bmatrix}
  A_{11}^{-1} & 0 \\
  0 & A_{22}^{-1}
\end{bmatrix}
\end{equation*}
\begin{gather*}
\left(\dfrac{1}{4}A^{*}\right)^{-1}BA^{-1}=2AB+I ~~~\Rightarrow~~~
4(|A|A^{-1})^{-1}BA^{-1}=2AB+I ~~~ \Rightarrow~~~
\frac{4}{|A|}ABA^{-1}=2AB+I\\
ABA^{-1}=2AB+I~~~\Rightarrow~~~
AB=2ABA+A~~~\Rightarrow~~~
AB(E-2A)=A~~~\Rightarrow~~~
B(E-2A)=I~~~\Rightarrow~~
B=(E-2A)^{-1}
\end{gather*}
\begin{equation*}
  B=(E-2A)^{-1}=\begin{bmatrix}
  -3 & 0& 0 & 0 \\
  -2 & -1 & 0& 0 \\
  0 & 0 & -3& -2\\
   0& 0 &0 &-1
\end{bmatrix}^{-1}=
\begin{bmatrix}
  -\frac{1}{3} & 0& 0 & 0 \\
  \frac{2}{3} & -1 & 0& 0 \\
  0 & 0 & -\frac{1}{3}& \frac{2}{3}\\
   0& 0 &0 &-1
\end{bmatrix}
\end{equation*}
\end{jie}

三、证明题

1.设$A$满足$A^{2}-2A+4I=0$,证明$A+I$可逆,并求$(A+I)^{-1}$.

\textcolor[rgb]{1.00,0.00,0.00}{思路:题目让证明谁可逆,就凑出这个表达式与某个表达式的乘积等于单位矩阵。}

\begin{zhengming}
由题得:$A^{2}-2A+4I=0$,所以:
\begin{gather*}
  A^{2}\textcolor[rgb]{1.00,0.00,0.00}{+A-A}-2A+4I=0\\
  A(A+I)-3A\textcolor[rgb]{1.00,0.00,0.00}{-3I+3I}+4I=0\\
  A(A+I)-3(A+I)=-7I\\
  (A-3I)(A+I)=-7I\\
  -\frac{1}{7}(A-3I)(A+I)=I
\end{gather*}
所以$A+I$可逆,$(A+I)^{-1}=-\dfrac{1}{7}(A-3I)$.
\end{zhengming}

2.已知$A=(a_{ij})$是三阶的非零矩阵,设$A_{ij}$是$a_{ij}$的代数余子式,且对任意的$i,j$有$A_{ij}+a_{ij}=0$,求$A$ 的行列式。

\begin{jie}
因为$A_{ij}+a_{ij}=0$,所以可以推出$A+(A^*)^T=0$。即$(A^*)^T=-A$.\textcolor[rgb]{1.00,0.00,0.00}{(看不懂的用伴随矩阵的定义表达出伴随矩阵,代入该式看一下)}两边同时取行列式:

左边:$|(A^*)^T|=|A^*|=||A|A^{-1}|=|A|^n|A|^{-1}=|A|^{n-1}=|A|^2$

右边:$|-A|=(-1)^3|A|=-|A|$

所以$|A|^2=-|A|$,解得$|A|=-1$或$|A|=0$。

又因为$(A^*)^T=-A$,所以有$r((A^*)^T)=r(-A)$,即$r(A^*)=r(A)$,所以$r(A)=n=3$。\textcolor[rgb]{1.00,0.00,0.00}{注:此步看不懂的看课本121页的例3.3.23,记住这个例题的结论。}

$r(A)=3$,即满秩,满秩即(可逆$\&$行列式不为$0$),所以$|A|=-1$。
\end{jie}
\newpage
\hphantom{~~}\hfill {\zihao{4}\heiti 2017-2018年第一学期} \hfill\hphantom{~~}

一、解答题

1.计算行列式$
\begin{vmatrix}
  1 & 1 & 1 & 1 \\
  1 & 1 & 1& 2 \\
  1 & 1 & 3& 1\\
   1 & 4 &1 &1
\end{vmatrix}
.$

\begin{jie}
过程不唯一 。

把第一列的负一倍分别加到第二、三、四列上:
\begin{equation*}
\begin{vmatrix}
  1 & 1 & 1 & 1 \\
  1 & 1 & 1& 2 \\
  1 & 1 & 3& 1\\
   1 & 4 &1 &1
\end{vmatrix}=\begin{vmatrix}
  1 & 0 & 0 & 0 \\
  1 & 0 & 0& 1 \\
  1 & 0 & 2& 0\\
   1 & 3 &0 &0
\end{vmatrix}=1\times(-1)^{1+1}\begin{vmatrix}
 0 & 0& 1 \\
 0 & 2& 0\\
 3 &0 &0
\end{vmatrix}=(-1)^{\frac{3\times(3-1)}{2}}6=-6
\end{equation*}
\end{jie}

2.求方程$
\begin{vmatrix}
  1 & 2 & 1 & 1 \\
  1 & x & 2& 3 \\
  1 & 2 & x& 2\\
   0 & 0 &2 &x
\end{vmatrix}=0
$的根。

\begin{jie}
把第一行的负一倍分别加到第二、三行上:
\begin{align*}
\begin{vmatrix}
  1 & 2 & 1 & 1 \\
  1 & x & 2& 3 \\
  1 & 2 & x& 2\\
   0 & 0 &2 &x
\end{vmatrix}=\begin{vmatrix}
  1 & 2 & 1 & 1 \\
  0 & x-2 & 1& 2 \\
  0 & 0 & x-1& 1\\
   0 & 0 &2 &x
\end{vmatrix}=1\times(-1)^{1+1}\begin{vmatrix}
  x-2 & 1& 2 \\
   0 & x-1& 1\\
   0 &2 &x
\end{vmatrix}=(x-2)\times(-1)^{1+1}[(x-1)x-2\times1]=0
\end{align*}
解得:$x_{1}=x_{2}=2,x_{3}=-1$。
\end{jie}

3.设$\gamma_{1},\gamma_{2},\gamma_{3},\gamma_{4}$及$\beta$均为4维列向量。4阶矩阵$A=[\gamma_ {1}~\gamma_{2}~\gamma_{3}~\gamma_{4}],B=[\beta~\gamma_{2}~\gamma_{3}~\gamma_{4}]$,若$\left|A\right|=2,\left|B\right|=3$,求

(1)$\left|A+B\right|$;

(2)$\left|A^{2}+AB\right|$;

\begin{jie}
(1)$A+B=[\gamma_ {1}~\gamma_{2}~\gamma_{3}~\gamma_{4}]+[\beta~\gamma_{2}~\gamma_{3}~\gamma_{4}]=[\gamma_{1}+\beta ~~2\gamma_{2}~~2\gamma_{3}~~2\gamma_{4}]$,由行列式的性质:
\begin{align*}
|A+B|&=|\gamma_{1}+\beta ~~2\gamma_{2}~~2\gamma_{3}~~2\gamma_{4}|=|\gamma_{1}~~2\gamma_{2}~~2\gamma_{3}~~2\gamma_{4}|+|\beta ~~2\gamma_{2}~~2\gamma_{3}~~2\gamma_{4}|\\
&=2\times2\times2|\gamma_ {1}~\gamma_{2}~\gamma_{3}~\gamma_{4}|+2\times2\times2|\beta~\gamma_{2}~\gamma_{3}~\gamma_{4}|=8|A|+8|B|=8(|A|+|B|)=40
\end{align*}

(2)$\left|A^{2}+AB\right|=|A(A+B)|=|A||A+B|=2\times 40=80.$
\end{jie}

4.设$
A=
\begin{bmatrix}
  3&2&2 \\
  0&1&1\\
  0&0&3
\end{bmatrix}
,~B=
\begin{bmatrix}
  1&0&0 \\
  0&0&0\\
  0&0&-1
\end{bmatrix}
,~AX+2B=BA+2X$,求$X^{2017}$。

\begin{jie}
\textcolor[rgb]{1.00,0.00,0.00}{若存在可逆矩阵$P$,使得$A=P^{-1}BP$,则称矩阵$A$与矩阵$B$相似,记为$A\~{}B$}

由题得:$(A-2I)X=B(A-2I)$,等式两边同时乘$(A-2I)^{-1}$得:$X=(A-2I)^{-1}B(A-2I)$,令$P=A-2I$,则$X=P^{-1}BP$.所以
\begin{equation*}
  X^{2}=(P^{-1}BP)^{2}=P^{-1}B\textcolor[rgb]{0.00,0.50,1.00}{PP^{-1}}BP=P^{-1}B\textcolor[rgb]{0.00,0.50,1.00}{I}BP=P^{-1}B^{2}P
\end{equation*}
同理可以推出:$X^{n}=P^{-1}B^{n}P$.
(由此我们可以得出一条结论:\textcolor[rgb]{1.00,0.00,0.00}{若$A\~{}B$,则$A^{n}=P^{-1}B^{n}P$},以后做题可直接使用)
\textcolor[rgb]{1.00,0.00,0.00}{对于任意对角矩阵$C=
\begin{bmatrix}
 c_{11} &~&~&~\\
 ~&c_{22}&~&~\\
 ~&~&\ddots&~\\
 ~&~&~&c_{nn}
\end{bmatrix}
$},可以计算得:$C^{2}=
\begin{bmatrix}
 c_{11}^{2} &~&~&~\\
 ~&c_{22}^{2}&~&~\\
 ~&~&\ddots&~\\
 ~&~&~&c_{nn}^{2}
\end{bmatrix}$,所以\textcolor[rgb]{1.00,0.00,0.00}{$C^{n}=
\begin{bmatrix}
 c_{11}^{n} &~&~&~\\
 ~&c_{22}^{n}&~&~\\
 ~&~&\ddots&~\\
 ~&~&~&c_{nn}^{n}
\end{bmatrix}$}
所以:
\begin{equation*}
B^{2017}=
\begin{bmatrix}
1^{2017}&0&0\\
0&0&0\\
0&0&(-1)^{2017}
\end{bmatrix}=
\begin{bmatrix}
1&0&0\\
0&0&0\\
0&0&-1
\end{bmatrix}=B
\end{equation*}
(下边求逆的过程略)
\begin{equation*}
P=A-2I=
\begin{bmatrix}
1&2&2\\
0&-1&1\\
0&0&1\\
\end{bmatrix}~~~
P^{-1}=
\begin{bmatrix}
1 &2&-4\\
0&-1&1\\
0&0&1
\end{bmatrix}
\end{equation*}
所以:
\begin{align*}
X^{2017}&=P^{-1}B^{2017}P=P^{-1}BP=\begin{bmatrix}
1 &2&-4\\
0&-1&1\\
0&0&1
\end{bmatrix}
\begin{bmatrix}
  1&0&0 \\
  0&0&0\\
  0&0&-1
\end{bmatrix}
\begin{bmatrix}
1&2&2\\
0&-1&1\\
0&0&1\\
\end{bmatrix}\\
&=
\begin{bmatrix}
1&0&4\\
0&0&-1\\
0&0&-1\\
\end{bmatrix}
\begin{bmatrix}
1&2&2\\
0&-1&1\\
0&0&1\\
\end{bmatrix}=\begin{bmatrix}
1&2&6\\
0&0&-1\\
0&0&-1
\end{bmatrix}
\end{align*}
\end{jie}

5.设$
A=\begin{bmatrix}
   2 & 4 & 1 & 0\\
   1 & 0 & 3 & 2\\
   -1 & 5 & -3 & 1\\
   0 & 1 & 0 & 2
  \end{bmatrix},B=
  \begin{bmatrix}
   1& 1& 1 & 1\\
   1 & 1 & 2 & 2\\
   a+1 & 2 & 3 &4\\
   1 & a & 1 & a
  \end{bmatrix}
$,若$r(A)=r(B)$,则$a$应满足什么条件。

\begin{jie}
由题得
\begin{align*}
A\xrightarrow{r_{1}\times \frac{1}{2}}&
{
\begin{bmatrix}
   1 & 2 & \frac{1}{2} & 0\\
   1 & 0 & 3 & 2\\
   -1 & 5 & -3 & 1\\
   0 & 1 & 0 & 2
  \end{bmatrix}
}\xrightarrow{ \substack{r_{2}-r_{1} \\ r_{3}+r_{1}}}
{
\begin{bmatrix}
   1 & 2 & \frac{1}{2} & 0\\
   0 & -2 & \frac{5}{2} & 2\\
   0 & 7 & -\frac{5}{2} & 1\\
   0 & 1 & 0 & 2
  \end{bmatrix}
}\xrightarrow{r_{2}\Leftrightarrow r_{4}}
{
\begin{bmatrix}
   1 & 2 & \frac{1}{2} & 0\\
   0 & 1 & 0 & 2\\
   0 & 7 & -\frac{5}{2} & 1\\
   0 & -2 & \frac{5}{2} & 2
  \end{bmatrix}
}\xrightarrow{\substack{r_{3}-7 r_{2} \\ r_{4}+2r_{2}}}
{
\begin{bmatrix}
   1 & 2 & \frac{1}{2} & 0\\
   0 & 1 & 0 & 2\\
   0 & 0 & -\frac{5}{2} & -13\\
   0 & 0 & \frac{5}{2} & 4
  \end{bmatrix}
}\\
\xrightarrow{r_{4}+ r_{3}}&
{
\begin{bmatrix}
   1 & 2 & \frac{1}{2} & 0\\
   0 & 1 & 0 & 2\\
   0 & 0 & -\frac{5}{2} & -13\\
   0 & 0 & 0 & -9
  \end{bmatrix}
}
\end{align*}
所以$r(A)=4$。$r(B)=4$,$B$满秩即$B$可逆,即$|B|\neq0$:
\begin{equation*}
|B|\xlongequal{\substack{c_{2}-c_{1}\\ c_{3}-c_{1}\\ c_{4}-c_{1}}}
\begin{vmatrix}
0 & 1 & 1\\
1-a & 2-a & 3-a\\
a-1 & 0 & a-1
\end{vmatrix}\xlongequal{\substack{ c_{3}-c_{2}}}
\begin{vmatrix}
0 & 1 & 0\\
1-a & 2-a & 1\\
a-1 & 0 & a-1
\end{vmatrix}=-a(1-a)\neq0
\end{equation*}
所以$a\neq0$且$a\neq1$.
\end{jie}

二、判断下列命题是否成立并给出理由。

1. 设$A,B$为同阶对称方阵,则$AB$一定是对称矩阵;

\begin{jie}
不成立。理由如下:

以2阶对称矩阵为例:(式中:$a,b,c,x,y,z$为任意实数。)
\begin{equation*}
A=\begin{bmatrix}
  a & b \\
  b & c
\end{bmatrix}~~~B=
\begin{bmatrix}
  x & y \\
  y & z
\end{bmatrix}~~AB=
\begin{bmatrix}
  ax+by & \textcolor[rgb]{1.00,0.00,0.00}{ay+bz}\\
  \textcolor[rgb]{1.00,0.00,0.00}{bx+cy} & by+cz
\end{bmatrix}
\end{equation*}
可以看出,由于$a,b,c,x,y,z$取值的任意性,所以$ay+bz\neq by+cz$。 (可以取$a=1,b=2,c=3,x=3,y=4,z=5$实际验证一下。)
\end{jie}

2. 设$A,B$为$n$阶可逆方阵,则$(AB)^{*}=B^{*}A^{*}$.

\begin{jie}
成立。理由如下:

因为$A,B$为$n$阶可逆方阵,所以$AB$可逆,所以$(AB)^{*}=|AB|(AB)^{-1}=|A||B|B^{-1}A^{-1}=|A|B^{*}A^{-1}=B^{*}A^{*}$
\end{jie}

3. 若$A^{2}=B^{2}$,则$A=B$或$A=-B$。

\begin{jie}
不成立。理由如下:

理由:
\begin{align*}
 A=\begin{bmatrix}
  1 & 0\\
  0 & 1
   \end{bmatrix}~~~
 A^{2}=\begin{bmatrix}
         1 & 0\\
         0 & 1
       \end{bmatrix}\\
 B=\begin{bmatrix}
  0 & 1\\
  1 & 0
   \end{bmatrix}~~~
 B^{2}=\begin{bmatrix}
         1 & 0\\
         0 & 1
       \end{bmatrix}
\end{align*}
可以看出$A^{2}=B^{2}$,但是$A\neq B$,$A\neq -B$
\end{jie}

4. 设2阶矩阵$A=
\begin{bmatrix}
  a & b \\
  c & d
\end{bmatrix}
$,若$A$与所有的2阶矩阵均可以交换,则$a=d,b=c=0$。

\begin{jie}
成立。理由如下:

取任意二阶矩阵($x,y,z,w$为任意实数):$
B=
\begin{bmatrix}
  x & y \\
  z & w
\end{bmatrix}
$,则
\begin{gather*}
AB=\begin{bmatrix}
  a & b \\
  c & d
\end{bmatrix}
\begin{bmatrix}
  x & y \\
  z & w
\end{bmatrix}=
\begin{bmatrix}
  ax+bz & ay+bw \\
  cx+dz & cy+dw
\end{bmatrix}\\
BA=
\begin{bmatrix}
  x & y \\
  z & w
\end{bmatrix}
\begin{bmatrix}
  a & b \\
  c & d
\end{bmatrix}=
\begin{bmatrix}
  xa+cy & xb+yd \\
  az+cw & bz+dw
\end{bmatrix}
\end{gather*}
若$A$与$B$可交换,则有$AB=BA$,即:
\begin{gather}
ax+bz=xa+cy  \\
ay+bw=xb+yd\\
cx+dz=az+cw\\
cy+dw=bz+dw
\end{gather}
由(1)式和(4)式得:$bz=cy$,因为$z$和$y$为任意数,所以$b=c=0$,代入(2)式和(3)式:$ay=dy,az=dz$,所以$a=d$。
\end{jie}

5.若$AB=I$且$BC=I$,其中$I$为单位矩阵,则$A=C$。

\begin{jie}
成立。理由如下:

设$I$为n阶单位矩阵,那么由矩阵乘法的定义有如下关系:

$A$的列数等于$B$的行数;$A$的行数等于$I$的行数等于$n$;$B$的列数等于$I$的列数等于$n$;\\
$B$的列数等于$C$的行数;$B$的行数等于$I$的行数等于$n$;$C$的列数等于$I$的列数等于$n$.

即$A,B,C$均为$n$阶方阵,对于$n$阶方阵,如果$AB=I$,那么$A,B$可逆,所以$A$是$B$的逆矩阵,$C$是$B$的逆矩阵,由逆矩阵的唯一性可知$B$=$C$。
\end{jie}

6. 若$n$阶矩阵$A$满足$A^{3}=3A(A-I)$,则$I-A$可逆。

\begin{jie}
成立。理由如下:

由题得:$A^{3}=3A(A-I)$,即$-A^{3}+3A^{2}-3A=0$,$-A^{3}+3A^{2}-3A+I=I$,所以$(I-A)^{3}=I$,所以$I-A$可逆,其逆矩阵为$(I-A)^{-1}=(I-A)^{2}$
\end{jie}

\newpage
\hphantom{~~}\hfill {\zihao{4}\heiti 2018-2019年第一学期} \hfill\hphantom{~~}

一、解答题

1.计算行列式$
\begin{vmatrix}
  b^{2}+c^{2} & c^{2}+a^{2} & a^{2}+b^{2} \\
  a & b & c \\
  a^{2} & b^{2} & c^{2}
\end{vmatrix}
.$

\begin{jie}
把第三行加到第一行上:
\begin{equation*}
\begin{vmatrix}
  b^{2}+c^{2} & c^{2}+a^{2} & a^{2}+b^{2} \\
  a & b & c \\
  a^{2} & b^{2} & c^{2}
\end{vmatrix}=\begin{vmatrix}
  a^{2}+b^{2}+c^{2} & b^{2}+c^{2}+a^{2} & a^{2}+b^{2}+c^{2} \\
  a & b & c \\
  a^{2} & b^{2} & c^{2}
\end{vmatrix}=(a^{2}+b^{2}+c^{2})\begin{vmatrix}
  1 & 1 & 1 \\
  a & b & c \\
  a^{2} & b^{2} & c^{2}
\end{vmatrix}=(a^{2}+b^{2}+c^{2})|A|
\end{equation*}
可以看出$|A|$是三阶范德蒙行列式,所以原式$=(a^{2}+b^{2}+c^{2})(c-b)(c-a)(b-a)$
\end{jie}

2. 设$A,B$为3阶矩阵,且$|A|=3,|B|=2$,$A^{*}$为$A$的伴随矩阵。

(1)若交换$A$的第一行与第二行得矩阵$C$,求$|CA^{*}|$;

(2)若$|A^{-1}+B|=2$,求$|A+B^{-1}|$.

\begin{jie}
(1)交换交换$A$的第一行与第二行得矩阵$C$,所以$|C|=-|A|$,所以$|CA^{*}|=|C||A^{*}|=-|A|||A|A^{-1}|-|A||A|^{3}|A|^{-1}=-|A|^3=-27$。

(2) $|A+B^{-1}|=|EA+B^{-1}E|=|B^{-1}BA+B^{-1}A^{-1}A|=|B^{-1}(BA+A^{-1}A)|=|B^{-1}|\cdot|BA+A^{-1}A|=|B^{-1}|\cdot|(B+A^{-1})A|=|B^{-1}|\cdot|(B+A^{-1})|\cdot|A|=2^{-1}\times2\times3=3$。
\end{jie}

3. 已知3阶矩阵$A$的逆矩阵$
A^{-1}=
\begin{bmatrix}
  1 & 1 & 1 \\
  1 & 2 & 1 \\
  2 & 1 & 3
\end{bmatrix}
$,试求伴随矩阵$A^{*}$的逆矩阵。

\begin{jie}
由伴随矩阵的性质:$A^{*}=|A|A^{-1}$,所以$(A^*)^{-1}=(|A|A^{-1})^{-1}=|A|^{-1}A=|A^{-1}|A$

由题得:
\begin{equation*}|A^{-1}|=
  \begin{vmatrix}
  1 & 1 & 1 \\
  1 & 2 & 1 \\
  2 & 1 & 3
  \end{vmatrix}\xlongequal{\substack{c_{2}-c_{1}\\c_{3}-c_{1}}}
  \begin{vmatrix}
  1 & 0 & 0 \\
  1 & 1 & 0 \\
  2 & -1 & 1
  \end{vmatrix}=1
\end{equation*}

\begin{align*}
[A^{-1}|E]=&
\left[
\begin{array}{c:c}
\begin{matrix}
1 & 1 & 1 \\
  1 & 2 & 1 \\
  2 & 1 & 3
\end{matrix}&
\begin{matrix}
1 & 0 & 0 \\
  0 & 1 & 0 \\
  0 & 0 & 1
\end{matrix}
\end{array}
\right]
\xrightarrow{\substack{r_{2}-r_{1}\\ r_{3}-2r_{1}}}
{
\left[
\begin{array}{c:c}
\begin{matrix}
1 & 1 & 1 \\
  0 & 1 & 0 \\
  0 & -1 & 1
\end{matrix}&
\begin{matrix}
1 & 0 & 0 \\
  -1 & 1 & 0 \\
  -2 & 0 & 1
\end{matrix}
\end{array}
\right]
}\xrightarrow{r_{3}+r_{2}}
{
\left[
\begin{array}{c:c}
\begin{matrix}
1 & 1 & 1 \\
  0 & 1 & 0 \\
  0 & 0 & 1
\end{matrix}&
\begin{matrix}
1 & 0 & 0 \\
  -1 & 1 & 0 \\
  -3 & 1 & 1
\end{matrix}
\end{array}
\right]
}\\
\xrightarrow{r_{1}-r_{3}}&
{
\left[
\begin{array}{c:c}
\begin{matrix}
1 & 1 & 0 \\
  0 & 1 & 0 \\
  0 & 0 & 1
\end{matrix}&
\begin{matrix}
4 & -1 & -1 \\
  -1 & 1 & 0 \\
  -3 & 1 & 1
\end{matrix}
\end{array}
\right]
}
\xrightarrow{r_{1}-r_{2}}
{
\left[
\begin{array}{c:c}
\begin{matrix}
1 & 0 & 0 \\
  0 & 1 & 0 \\
  0 & 0 & 1
\end{matrix}&
\begin{matrix}
5 & -2 & -1 \\
  -1 & 1 & 0 \\
  -3 & 1 & 1
\end{matrix}
\end{array}
\right]
}
\end{align*}
所以$(A^{*})^{-1}=|A^{-1}|A=
\begin{bmatrix}
  5 & -2 & -1\\
  -1 & 1 & 0\\
  -3 & 1 & 1
\end{bmatrix}
$
\end{jie}

4. 设$n$阶行列式$D_{n}(n=1,2,\cdots):D_{1}=1,D_{2}=
\begin{vmatrix}
  1 & 1 \\
  1 & 1
\end{vmatrix},D_{3}=
\begin{vmatrix}
  1 & 1 & 0\\
  1 & 1 & 1\\
  0 & 1 & 1
\end{vmatrix},D_{4}=
\begin{vmatrix}
  1 & 1 & 0 & 0\\
  1 & 1 & 1 & 0\\
  0 & 1 & 1 & 1\\
  0 & 0 & 1 & 1
\end{vmatrix},\ldots\ldots,D_{n}=
\begin{vmatrix}
  1 & 1 & 0 & 0 & \cdots & 0\\
  1 & 1 & 1 & 0 & \cdots & 0\\
  0 & 1 & 1 & 1 & \cdots & 0\\
  \vdots & \vdots &\ddots & \ddots &\ddots &\vdots\\
  0 &\cdots & 0 & 1 & 1& 1\\
  0 &\cdots & 0 & 0 & 1& 1
\end{vmatrix}.
$

(1)给出$D_{n},D_{n-1},D_{n-2}$的关系;

(2)利用找到的递推关系及$D_{1}=1,D_{2}=0$,计算$D_{3},D_{4},\cdots,D_{8}$;

(3)求$D_{2018}$

\begin{jie}
(1)对$D_{n}$按第一列进行行列式展开($a_{11}$表示第一行第一列的元素,$A_{11}$表示第一行第一列元素对应的代数余子式):
\begin{align*}
D_{n}=a_{11}A_{11}+a_{21}A_{21}=1\times(-1)^{1+1}
\begin{vmatrix}
   1 & 1 & 0 & \cdots & 0\\
   1 & 1 & 1 & \cdots & 0\\
   \vdots &\ddots & \ddots &\ddots &\vdots\\
  0 &\cdots  & 1 & 1& 1\\
  0 &\cdots & 0  & 1& 1
\end{vmatrix}+1\times(-1)^{2+1}
\begin{vmatrix}
   1 & 0 & 0 & \cdots & 0\\
   1 & 1 & 1 & \cdots & 0\\
  \vdots &\ddots & \ddots &\ddots &\vdots\\
  0 &\cdots  & 1 & 1& 1\\
  0 &\cdots  & 0 & 1& 1
\end{vmatrix}=D_{n-1}-D_{n-2}
\end{align*}

(2)由(1)得:
\begin{gather*}
D_{3}=D_{3-1}-D_{3-2}=D_{2}-D_{1}=-1\\
D_{4}=D_{4-1}-D_{4-2}=D_{3}-D_{2}=-1\\
D_{5}=D_{5-1}-D_{5-2}=D_{4}-D_{3}=0\\
D_{6}=D_{6-1}-D_{6-2}=D_{5}-D_{4}=1\\
D_{7}=D_{7-1}-D_{7-2}=D_{6}-D_{5}=1\\
D_{8}=D_{8-1}-D_{8-2}=D_{7}-D_{6}=0
\end{gather*}

(3)由(2)可以看出,每6组数据为一个循环,即$D_{n}=D_{n+6}$,所以$2018\div6=336$余$2$。所以$D_{2018}=D_{2}=0$
\end{jie}

5. 已知矩阵$A=
\begin{bmatrix}
  1 &  0 & 0\\
  1 & 1 & 0\\
   1 & 1 & 1
\end{bmatrix},B=
\begin{bmatrix}
  0 & 1 & 1\\
  1 & 0 & 1 \\
  0 & 1 & 0
\end{bmatrix}
$,且矩阵$X$满足
\begin{equation*}
  AXA+BXB=AXB+BXA+I
\end{equation*}
其中$I$为3阶单位阵,求$X$。

\begin{jie}
由题得:$AXA+BXB=AXB+BXA+I$所以:
\begin{gather*}
  AXA+BXB-AXB-BXA=I\\
  (A-B)XA+(B-A)XB=(A-B)XA-(A-B)XB=I \\
  (A-B)X(A-B)=I\\
  (A-B)X=I(A-B)^{-1}\\
  X=(A-B)^{-1}I(A-B)^{-1}=((A-B)^{-1})^{2}
\end{gather*}
\begin{align*}
 &C=A-B=\begin{bmatrix}
  1 &  0 & 0\\
  1 & 1 & 0\\
   1 & 1 & 1
\end{bmatrix}-
\begin{bmatrix}
  0 & 1 & 1\\
  1 & 0 & 1 \\
  0 & 1 & 0
\end{bmatrix}=\begin{bmatrix}
  1 &  -1 & -1\\
  0 & 1 & -1\\
   0 & 0 & 1
\end{bmatrix}\\
&[C|E]\xrightarrow{\substack{r_{1}+r_{3}\\ r_{2}+r_{3}}}
{\left[
\begin{array}{c:c}
\begin{matrix}
 1 &  -1 & 0\\
  0 & 1 & 0\\
   0 & 0 & 1
\end{matrix} &
\begin{matrix}
  1 & 0 & 1\\
  0 & 1 & 1\\
  0 & 0 & 1
\end{matrix}
\end{array}
\right]
}\xrightarrow{\substack{r_{1}+r_{3}\\ r_{2}+r_{3}}}
{\left[
\begin{array}{c:c}
\begin{matrix}
 1 &  0 & 0\\
  0 & 1 & 0\\
   0 & 0 & 1
\end{matrix} &
\begin{matrix}
  1 & 1 & 2\\
  0 & 1 & 1\\
  0 & 0 & 1
\end{matrix}
\end{array}
\right]
}
\end{align*}
所以$C^{-1}=\begin{bmatrix}
  1 & 1 & 2\\
  0 & 1 & 1\\
  0 & 0 & 1
\end{bmatrix},C^{2}=\begin{bmatrix}
  1 & 2 & 5\\
  0 & 1 & 2\\
  0 & 0 & 1
\end{bmatrix}$
\end{jie}

6. 设$A=
\begin{bmatrix}
  1 & 0 & 1\\
  -1 & -1 & 1\\
  0 & 2 & a
\end{bmatrix},
B=
\begin{bmatrix}
  1 & 0 & 1\\
  0 & -1 & 2\\
   0 & 0 & 0
\end{bmatrix}
$。

(1)问$a$为何值时,矩阵$A$和$B$等价。

(2)当$A$和$B$等价时,求可逆矩阵$P$,使得$PA=B$。

\begin{jie}
(1)
由题得:$r(B)=2$,若$A$和$B$等价,则$r(A)=2$。
\begin{equation*}
 A\xrightarrow{r_{2}+ r_{1}}
{
\begin{bmatrix}
  1 & 0 & 1\\
  0 & -1 & 2\\
  0 & 2 & a
\end{bmatrix}
}\xrightarrow{r_{3}+2r_{2}}
{
\begin{bmatrix}
  1 & 0 & 1\\
  0 & -1 & 2\\
  0 & 0 & a+4
\end{bmatrix}
}
\end{equation*}
$a+4=0$即$a=-4$时,$r(A)=2$,所以$a=-4$。

(2)~~由(1)得:
\begin{equation*}A=
  \begin{bmatrix}
  1 & 0 & 1\\
  0 & -1 & 2\\
  0 & 2 & -4
  \end{bmatrix}\xrightarrow{r_{2}+r_{1}}
{
\begin{bmatrix}
  1 & 0 & 1\\
  0 & -1 & 2\\
  0 & 2 & 4
\end{bmatrix}
}\xrightarrow{r_{3}+2r_{2}}
{
\begin{bmatrix}
  1 & 0 & 1\\
  0 & -1 & 2\\
  0 & 0 & 0
\end{bmatrix}
}=B
\end{equation*}
由初等行变换得:$E(3~2(2))E(2~1(1))A=B$所以
\begin{equation*}
P=E(3~2(2))E(2~1(1))=
  \begin{bmatrix}
    1 & 0 & 0\\
    0 & 1& 0\\
    0 & 2& 1
  \end{bmatrix}
   \begin{bmatrix}
    1 & 0 & 0\\
    1 & 1& 0\\
    0 & 0& 1
  \end{bmatrix}=
   \begin{bmatrix}
    1 & 0 & 0\\
    1 & 1& 0\\
    2 & 2& 1
  \end{bmatrix}
\end{equation*}
\end{jie}

二、证明题

1. 若$n$阶实矩阵$Q$满足$QQ^{T}=I$,则称$Q$为正交矩阵。设$Q$为正交矩阵,则

(1)$Q$的行列式为1或-1.

(2)当$|Q|=1$且$n$为奇数时,证明$|I-Q|=0$,其中$I$是$n$阶单位矩阵;

(3)$Q$的逆矩阵$Q^{-1}$和伴随矩阵$Q^{*}$都是正交矩阵。

\begin{zhengming}
(1)由题得:$|QQ^{T}|=|I|=1$,由行列式的性质:$|QQ^{T}|=|Q|\cdot|Q^T|,|Q^T|=|Q|$,所以$|QQ^{T}|=|Q|^2=1$,解得$|Q|=1$或$|Q|=-1$.

(2)$|I-Q|=|QQ^{T}-Q|=|Q|\cdot|Q^T-I|=|Q^T-I|=|(Q^T)^T-I^T|=|Q-I|=|-(I-Q)|=(-1)^{n}|I-Q|$,因为$n$为奇数,所以$(-1)^n=-1$,即$|I-Q|=-|I-Q|$,所以$|I-
Q|=0$。

(3)因为$QQ^{T}=I$,两边同时左乘$Q^{-1}$:$Q^{T}=Q^{-1}$,两边同时右乘$(Q^{T})^{-1}$:$I=Q^{-1}(Q^{T})^{-1}=Q^{-1}(Q^{-1})^{T}$,所以$Q^{-1}$是正交矩阵。

由伴随矩阵的性质:$Q^*=|Q|Q^{-1},(Q^T)^*=(Q^*)^T=|Q^T|(Q^{T})^{-1}=|Q|(Q^{-1})^{T}$,所以$Q^*(Q^*)^T=|Q|Q^{-1}|Q|(Q^{-1})^{T}=|Q|^{2}Q^{-1}(Q^{-1})^{T}$

由(1)得$|Q|^{2}=1$,所以有$|Q|^{2}Q^{-1}(Q^{-1})^{T}=Q^{-1}(Q^{-1})^{T}=I$,所以$Q^{*}$是正交矩阵。
\end{zhengming}

2. 设$A$是$n$阶实对称矩阵,如果$A^{2}=0$,证明$A=0$.并举例说明,如果$A$不是实对称矩阵,上述命题不正确。

\begin{zhengming}

(1)依题意设$A=
\begin{bmatrix}
  a_{11} & a_{12} & \cdots & a_{1n} \\
  a_{12} & a_{22} & \cdots & a_{2n} \\
  \vdots & \vdots & \ddots & \vdots \\
  a_{1n} & a_{2n} & \cdots & a_{nn}
\end{bmatrix}
$,所以$A^2$为:只看$A^2$对角线上的元素,$A^2$的第$k$行第$k$列的元素为$A$的第$k$行乘第$k$列:$a_{1k}^2+a_{2k}^2+\cdots+a_{kk}^2+a_{kk+1}^2+a_{kk+2}^2+\cdots+a_{kn}^2=0$,因为平方一定大于等于0,所以该式的每一项都为0,即$a_{1k},a_{2k},\cdots,a_{kk},a_{kk+1},a_{kk+2},\cdots,a_{kn}$为0.即$A$的第$k$行和第$k$列元素为0.\textcolor[rgb]{1.00,0.00,0.00}{(这一步看不懂的计算一下$A^2$的第一行第一列,第二行第二列的元素验证一下)}

因为$A^2$为0,即其对角线每个元素都为0,由上边的步骤可以推出$A$的每行每列元素都为0,即$A=0$。

(2)举例:对于二阶矩阵$
A=\begin{bmatrix}
0& 0\\
1& 0
\end{bmatrix}
$,$A^2=0$,$A$不是对称矩阵,$A^2=0$但$A\neq0$。
\end{zhengming}
\end{document}
