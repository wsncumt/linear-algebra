\documentclass{article}
\usepackage[space,fancyhdr,fntef]{ctexcap}
\usepackage[namelimits,sumlimits,nointlimits]{amsmath}
\usepackage[bottom=25mm,top=25mm,left=25mm,right=15mm,centering]{geometry}
\usepackage{xcolor}
\usepackage{paralist}%列表宏包
\usepackage{arydshln}%234页,虚线表格宏包
\pagestyle{fancy} \fancyhf{}
\fancyhead[OL]{~~~班序号:\hfill 学院:\hfill 学号:\hfill 姓名:王松年~~~ \thepage}
%\usepackage{parskip}
%\usepackage{indentfirst}
\usepackage{graphicx}%插图宏包,参见手册318页
\usepackage{amssymb}
\usepackage{bbm}
\begin{document}

\newcounter{num} \renewcommand{\thenum}{\arabic{num}.} \newcommand{\num}{\refstepcounter{num}\text{\thenum}}

\hphantom{~~}\hfill {\zihao{3}\heiti 第十一次习题课} \hfill\hphantom{~~}

\hphantom{~~}\hfill {\zihao{4}\heiti 知识点} \hfill\hphantom{~~}

\num 只含有二次项的$n$元多项式
\begin{equation*}
f(x_1,x_2,\cdots,x_n)=a_{11}x_1^2+2a_{12}x_1x_2+\cdots+2a_{1n}x_1x_n+a_{22}x_2^2+2a_{23}x_2x_3+\cdots+2a_{2n}x_2x_n+\cdots+\cdots+a_{nn}x_n^2
\end{equation*}
称为$x_1,x_2,\cdots,x_n$的一个$n$元二次齐次多项式,简称为$x_1,x_2,\cdots,x_n$的一个$n$元二次型。

\num 作一个$n$阶对称矩阵,$x=
\begin{bmatrix}
x_1\\ x_2\\  \\x_n
\end{bmatrix}
$,可以验证$f(x_1,x_2,\cdots,x_n)=x^TAx$,一般用$f(x)=x^TAx$表示二次型。矩阵$A$称为二次型$f(x)$的矩阵。对称矩阵$A$与二次型$f(x)$是一一对应的,定义二次型$f(x)$的秩为$r(A)$。

\num 设$C_{m\times n}$,我们称$C:\mathbb{R}^n\rightarrow\mathbb{R}^m$,$\forall v\in \mathbb{R}^n,v\mapsto Cv$是从$\mathbb{R}^n$到$\mathbb{R}^m$的一个线性变换。

\num 设$C_{n\times n}$是可逆矩阵,$x\in \mathbb{R}^n$,此时称$x=Cy$是可逆线性变换,此时二次型
\begin{equation*}
f(x_1,x_2,\cdots,x_n)=x^TAx=(Cy)^TA(Cy)=y^TC^TACy=y^T(C^TAC)y
\end{equation*}
变为矩阵$B=C^TAC$的$y$的$n$元二次型。

若$y^T(C^TAC)y$形如$d_1y_1^2+\cdots+d_ry_r^2$,其中$d_1\cdots d_r\neq0$,则称$y^T(C^TAC)y$为$x^TAx$的一个标准型。

\num 设$A,B$为两个$n$阶矩阵,如果存在$n$阶可逆矩阵$C$,使得$C^TAC=B$则称矩阵$A$合同于矩阵$B$,或$A$与$B$合同。记为$A\simeq B$。

\num 如果线性变换$C$是正交矩阵,则称$x=Cy$为正交变换。

设$A$为实对称矩阵,则存在正交阵$Q$,使$Q^{-1}AQ$为对角矩阵,由于二次型的矩阵是一个实对称矩阵,则$Q^{T}AQ$为对角矩阵。

\num 二次型一定可以用正交变换化为标准型。

\num 对任意二次型$f(x)=x^TAx$,存在正交矩阵$Q$,经过正交变换$x=Qy$可化为标准型
\begin{equation*}
\lambda_{1}y_1^2+\lambda_{2}y_2^2+\cdots+\lambda_{1n}y_n^2
\end{equation*}
其中$\lambda_{1},\lambda_{2},\cdots,\lambda_{n}$是二次型$f(x)$的矩阵$A$的全部特征值。(注:非零特征值要放在前面)

\num 对任意一个实对称矩阵$A$,存在一个非奇异矩阵$C$,使$C^TAC$为对角阵。即任何一个实对称矩阵都与一个对角矩阵合同。(称这个对角阵为$A$的标准形)

\num 二次型可以通过非退化线性变换化为规范形且规范形唯一。

规范形中正项个数$p$称为二次型的正惯性指标,负项个数$r-p$称为二次型的副惯性指标,$r$是二次型的秩。

\num 任给$0\neq x\in \mathbb{R}^n$都有$f(x)=x^TAx>0$(或$<0$),则称$A$为正定矩阵(负定矩阵)。或称$f(x)=x^TAx>0$为正定(负定)二次型。

\num 半正定,半负定

\num 设$A$为正(负)定矩阵。如果$A,B$合同,则$B$也是正(负)定矩阵。合同矩阵具有相同的有定性。

\num 单位矩阵是正定的。负单位矩阵是负定的。
$I_{p,q}=
\begin{bmatrix}
I_p &~\\
~&-I_q
\end{bmatrix}
$不定。

$
\begin{bmatrix}
I_p &~\\
~&0
\end{bmatrix}
$半正定。

$
\begin{bmatrix}
-I_q &~\\
~&0
\end{bmatrix}
$半负定。

$
\begin{bmatrix}
I_p &~&~\\
~&-I_q&~\\
~&~&0
\end{bmatrix}
$不定。

\num 实对称矩阵$A$为

正定矩阵,当且仅当$A$的特征值全大于0。

负定矩阵,当且仅当$A$的特征值全小于0。

半正定矩阵,当且仅当$A$的特征值有正有0。

半负定矩阵,当且仅当$A$的特征值有负有0。

不定矩阵,当且仅当$A$的特征值有正有负。

\num 如果矩阵$A$正定,以下描述等价

(1)$A$的特征值全大于0。

(2)$A$的规范形为$E_n$。

(3)存在可逆矩阵$C$,$C^TAC=E_n$。

(4)存在可逆矩阵$B$,$B=C^{-1},A=B^TB$。

\num $k$阶主子式,$k$阶顺序主子式。

\num 实对称矩阵$A=(a_{ij_{n\times n}})$为正定矩阵当且仅当$A$的所有顺序主子式大于0。

\end{document}  