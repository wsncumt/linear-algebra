\documentclass{article}
\usepackage[space,fancyhdr,fntef]{ctexcap}
\usepackage[namelimits,sumlimits,nointlimits]{amsmath}
\usepackage[bottom=25mm,top=25mm,left=25mm,right=15mm,centering]{geometry}
\usepackage{xcolor}
\usepackage{arydshln}%234页,虚线表格宏包
\pagestyle{fancy} \fancyhf{}
\fancyhead[OL]{~~~班序号:\hfill 学院:\hfill 学号:\hfill 姓名:王松年~~~ \thepage}
%\usepackage{parskip}
%\usepackage{indentfirst}
\usepackage{graphicx}%插图宏包,参见手册318页
\usepackage{mathdots}%反对角省略号
\begin{document}

\newcounter{num} \renewcommand{\thenum}{\arabic{num}.} \newcommand{\num}{\refstepcounter{num}\text{\thenum}}

\newenvironment{jie}{\kaishu\zihao{-5}\color{blue}{\noindent\em 解:}\par}{\hfill $\diamondsuit$\par}

\newenvironment{zhengming}{\kaishu\zihao{-5}\color{blue}{\noindent\em 证明:}\par}{\hfill $\diamondsuit$\par}

\hphantom{~~}\hfill {\zihao{3}\heiti 第十一次习题课} \hfill\hphantom{~~}

\hphantom{~~}\hfill {\zihao{4}\heiti 群文件《期中$\&$期末试题》} \hfill\hphantom{~~}

{\heiti \zihao{4} 期末试题}

\num 期末2014-2015 一5.

已知实二次型$f(x_{1},x_{2},x_{3})=a(x_{1}^{2}+x_{2}^{2}+x_{3}^{2})+4x_{1}x_{2}+4x_{1}x_{3}+4x_{2}x_{2}$经正交变换$x=py$可化为标准形:$f=6y^{2}$,则$a=$\underline{\hphantom{~~~~~~~~~~}}。\\

\num 期末2014-2015 六.

设实二次型
\begin{equation*}
  f(x_{1},x_{2},x_{3})=X^{T}AX=ax_{1}^{2}+2x_{2}^{2}-2x_{3}^{2}+2bx_{1}x_{3}~~(b>0)
\end{equation*}
的矩阵$A$的特征值之和为$1$,特征值之积为-12。

(1)求$a,b$的值;

(2)利用正交变换将二次型$f$化为标准型,并写出所用正交变换。
\\

\num 期末2015-2016 一6.

若矩阵$
A=
\begin{bmatrix}
  1 & 1 & 2 \\
  1 & 2 & 3\\
  2 & 3 & \lambda
\end{bmatrix}
$正定,则$\lambda$满足的条件为\underline{\hphantom{~~~~~~~~~~}}。\\

\num 期末2015-2016 四1.

设$A$为$n$阶实对称矩阵,且满足$A^{2}-3A+2E=0$,其中$E$为单位矩阵,试证:

%(1)$A+2E$可逆;

(2)$A$为正定矩阵。\\

\num 期末2016-2017 一6.

设
$
B=
\begin{bmatrix}
  1 & 2 & 4 \\
  0 & 2 & 6\\
  0 & 0 & \lambda
\end{bmatrix}
$,已知二次型$f(x)=x^{T}Bx$是正定的,则$\lambda$的取值范围为\underline{\hphantom{~~~~~~~~~~}}。\\

\num 期末2016-2017 三2.

已知实对称矩阵$A
\begin{bmatrix}
  a & -1 & 4 \\
  -1 & 3 & b\\
  4 & b & 0
\end{bmatrix}
$与
$A
\begin{bmatrix}
  2 & ~ & ~ \\
  ~ & -4 & ~\\
  ~ & ~ & 5
\end{bmatrix}
$相似。

(1)求矩阵$A$化;

(2)求正交线性变换$x=Qy$,把二次型$f(x)=x^{T}Ax$化为标准型.\\

\num 期末2016-2017 四2.

已知$A,B$是同阶实对称矩阵。

(1)证明如果$A\~{}B$,则$A\simeq B$,也就是相似一定合同;

(2)举例说明反过来不成立。\\

\num 期末2017-2018 一6.

设二次型$f(x_{1},x_{2},x_{3})=2x_{1}^{2}+x_{2}^{2}+x_{3}^{2}+2x_{1}x_{2}+2tx_{2}x_{3}$的秩为2,则$t=$\underline{\hphantom{~~~~~~~~~~}}。\\

\num 期末2017-2018 三2.

设$f(x_ {1},x_{2},x_{3})=2x_{1}^{2}+2x_{2}^{2}+3x_{3}^{2}+2x_{1}x_{2}$。

(1)写出该二次型的矩阵$A$;

(2)求正交矩阵$Q$使得$Q^{T}AQ=Q^{-1}AQ$为对角型矩阵;

(3)给出正交变换,化该二次型为标准型。\\

\num 期末2018-2019 一5.

若二次型$f(x_ {1},x_{2},x_{3})=x_{1}^{2}+4x_{2}^{2}+4x_{3}^{2}+2tx_{1}x_{2}-2x_{1}x_{3}+4x_{2}x_{3}$正定,则$t$应满足\underline{\hphantom{~~~~~~~~~~}}。\\

\num 期末2018-2019 三2.

设实二次型$f(x_{1},x_{2},x_{3})=4x_{1}x_{2}-4x_{1}x_{3}+4x_{2}^{2}+8x_{2}x_{3}-3x_{3}^{2}$。

(1)写出该二次型的矩阵$A$;

(2)求正交矩阵$P$,使得$P^{-1}AP$为对角型矩阵;

(3)给出正交变换,将该二次型化为标准型;

(4)写出二次型的秩,正惯性指标和负惯性指标。\\

\num 期末2019-2020 一6.

已知实对称矩阵$A=
\begin{bmatrix}
  2 & 0 & 1 \\
  0 & 3 & 3\\
  1 & 3 & x
\end{bmatrix}
$的正惯性指数为3,则$x$的取值范围为\underline{\hphantom{~~~~~~~~~~}}。\\

\num 期末2019-2020 三3.

设三元二次型$f(x_{1},x_{2},x_{3})=4x_{2}^{2}+4x_{3}^{2}-2x_{1}x_{2}+4x_{1}x_{3}$.

(1)写出该二次型的矩阵$A$;

(2)用正交变换$x=Qy$把该二次型化为标准型。\\

\num 期末2019-2020 四1.

设$A$为$m$阶正定矩阵,$B$为$\times n$实矩阵,$B^{T}$为$B$的转置矩阵,试证:$B^{T}AB$为正定矩阵的充分必要条件是$B$的秩$r(B)=n$。\\

\num 期末2019-2020 四2.

设$\alpha,\beta$是$n$维列向量,证明$r(\alpha\alpha^{T}+\beta\beta^{T})\leq 2$。\\

\num 期末2017-2018 四2.

设$A$为$m\times n$实矩阵,证明$Ax=0$与$(A^{T}A)x=0$是同解方程,进一步得出$r(A)=r(A^{t}A)$。\\

\end{document}  