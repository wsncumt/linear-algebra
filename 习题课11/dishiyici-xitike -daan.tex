\documentclass{article}
\usepackage[space,fancyhdr,fntef]{ctexcap}
\usepackage[namelimits,sumlimits,nointlimits]{amsmath}
\usepackage[bottom=25mm,top=25mm,left=25mm,right=15mm,centering]{geometry}
\usepackage{xcolor}
\usepackage{arydshln}%234页,虚线表格宏包
\pagestyle{fancy} \fancyhf{}
\fancyhead[OL]{~~~班序号:\hfill 学院:\hfill 学号:\hfill 姓名:王松年~~~ \thepage}
%\usepackage{parskip}
%\usepackage{indentfirst}
\usepackage{graphicx}%插图宏包,参见手册318页
\usepackage{mathdots}%反对角省略号
\begin{document}

\newcounter{num} \renewcommand{\thenum}{\arabic{num}.} \newcommand{\num}{\refstepcounter{num}\text{\thenum}}

\newenvironment{jie}{\kaishu\zihao{-5}\color{blue}{\noindent\em 解:}\par}{\hfill $\diamondsuit$\par}

\newenvironment{zhengming}{\kaishu\zihao{-5}\color{blue}{\noindent\em 证明:}\par}{\hfill $\diamondsuit$\par}

\hphantom{~~}\hfill {\zihao{3}\heiti 第十一次习题课} \hfill\hphantom{~~}

\hphantom{~~}\hfill {\zihao{4}\heiti 群文件《期中$\&$期末试题》} \hfill\hphantom{~~}

{\heiti \zihao{4} 期末试题}

\num 期末2014-2015 一5.

已知实二次型$f(x_{1},x_{2},x_{3})=a(x_{1}^{2}+x_{2}^{2}+x_{3}^{2})+4x_{1}x_{2}+4x_{1}x_{3}+4x_{2}x_{2}$经正交变换$x=py$可化为标准形:$f=6y^{2}$,则$a=$\underline{\hphantom{~~~~~~~~~~}}。

\begin{jie}
任意二次型$x^TAx$经过正交变换化为标准型时,标准型中平方项的系数即为二次型矩阵$A$的特征值,即$6,0,0$是$A$的特征值,而$A$的对角线元素是$a,a,a$,由特征值性质$trace(A)=a+a+a=\sum_{i=1}^{3}\lambda=6$,所以$a=2$。
\end{jie}

\num 期末2014-2015 六.

设实二次型
\begin{equation*}
  f(x_{1},x_{2},x_{3})=X^{T}AX=ax_{1}^{2}+2x_{2}^{2}-2x_{3}^{2}+2bx_{1}x_{3}~~(b>0)
\end{equation*}
的矩阵$A$的特征值之和为$1$,特征值之积为-12。

(1)求$a,b$的值;

(2)利用正交变换将二次型$f$化为标准型,并写出所用正交变换。

\begin{jie}
由题得:$A=
\begin{bmatrix}
  a & 0 & b\\
  0 & 2 & 0\\
  b & 0 & -2
\end{bmatrix},|A|=2(-2a-b^2)
$

(1)由特征值的性质有:
\begin{equation*}
\begin{cases}
trace(A)=a+2-2=1\\
\prod\limits_{i=1}^{3}\lambda_i=-12=|A|=2(-2a-b^2)\\
b>0
\end{cases}~~~\Rightarrow~~~
\begin{cases}
a=1\\
b=2
\end{cases}
\end{equation*}

(2)
\begin{equation*}
  |\lambda E-A|=
  \begin{vmatrix}
  \lambda-1& 0 & -2\\
  0 & \lambda-2 & 0\\
  -2 & 0 & \lambda+2
  \end{vmatrix}=(\lambda-2)[(\lambda-1)(\lambda+2)-4]=0~~~~\Rightarrow~~~~\lambda_1=\lambda_2=2,\lambda_3=-3
\end{equation*}

$\lambda_1=\lambda_2=2$时:
\begin{equation*}
  [\lambda E-A]=
  \begin{vmatrix}
  1& 0 & -2\\
  0 & 0 & 0\\
  -2 & 0 & 4
  \end{vmatrix}~~~\Rightarrow~~~
  \begin{cases}
   x_1=2x_3\\
   x_2\in R
  \end{cases}
\end{equation*}
分别取$[x_2,x_3]^T=[1,0]^T$和$[0,1]^T$得:$\alpha_1=[0,1,0]^T,\alpha_{2}=[2,0,1]^T$,可以看出$\alpha_1$与$\alpha_2$正交。

$\lambda_3=-3$时:
\begin{equation*}
  [\lambda E-A]=
  \begin{vmatrix}
  -4& 0 & -2\\
  0 & -5 & 0\\
  -2 & 0 & -1
  \end{vmatrix}~~~\Rightarrow~~~
  \begin{cases}
   x_1=-\dfrac{1}{2}x_3\\
   x_2=0
  \end{cases}
\end{equation*}
取$x_3=-2$得:$\alpha_3=[1,0,-2]^T$。因为对称矩阵对应于不同特征值的特征向量正交,所以$[\alpha_1,\alpha_2,\alpha_3]$为正交向量组。

单位化:
\begin{equation*}
\begin{cases}
\gamma_1=\dfrac{\alpha_1}{\|\alpha_1\|}=[0,1,0]^T\\
\gamma_2=\dfrac{\alpha_2}{\|\alpha_2\|}=\left[\dfrac{2}{\sqrt{5}},0,\dfrac{1}{\sqrt{5}}\right]^T\\[2mm]
\gamma_3=\dfrac{\alpha_3}{\|\alpha_3\|}=\left[\dfrac{1}{\sqrt{5}},0,-\dfrac{2}{\sqrt{5}}\right]^T
\end{cases}~~~\Rightarrow~~~Q=
\begin{bmatrix}
  0 & \dfrac{2}{\sqrt{5}}& \dfrac{1}{\sqrt{5}} \\
  1 & 0& 0\\
  0&  \dfrac{1}{\sqrt{5}}& -\dfrac{2}{\sqrt{5}}
\end{bmatrix}
\end{equation*}
所以$f$可经正交变换$x=Qy$化为标准型:
\begin{equation*}
  f=2y_{1}^2+2y_{2}^2-3y_{3}^2
\end{equation*}
\end{jie}

\num 期末2015-2016 一6.

若矩阵$
A=
\begin{bmatrix}
  1 & 1 & 2 \\
  1 & 2 & 3\\
  2 & 3 & \lambda
\end{bmatrix}
$正定,则$\lambda$满足的条件为\underline{\hphantom{~~~~~~~~~~}}。

\begin{jie}
由题得:$A$为对称矩阵,如果$A$正定,则$|A|>0$,所以$|A|=\lambda-5>0~~\Rightarrow~~\lambda>5$.
\end{jie}

\num 期末2015-2016 三3.

设$
A=
\begin{bmatrix}
  1 & 0 & 1 \\
  0 & 1 & 1 \\
  -1 & 0 & a \\
  0 & a & -1
\end{bmatrix},A^{T}
$为矩阵$A$的转置,已知$r(A)=2$,且二次型$f(x)=x^{T}A^{T}Ax$.

(1)求$a$;

(2)写出二次型$f(x)$的矩阵$B=A^{T}A$;

(3)求正交变换$x=Qy$将二次型$f(x)$化为标准型,并写出所用的正交变换。

\begin{jie}
(1)由题得:
\begin{equation*}
A
\xrightarrow{\substack{r_{3}+r_{1} \\r_4-ar_2}}
{
\begin{bmatrix}
  1 & 0 & 1 \\
  0 & 1 & 1 \\
  0 & 0 & a+1 \\
  0 & 0 & -1-a
\end{bmatrix}
}
\end{equation*}
$r(A)=2$,所以$1+a=0$且$-1-a=0$解得:$a=-1$。

(2)
\begin{equation*}
B=A^TA=
\begin{bmatrix}
1 & 0 &-1 & 0\\
0 & 1 & 0 &-1\\
1 & 1 & -1 &-1
\end{bmatrix}
\begin{bmatrix}
  1 & 0 & 1 \\
  0 & 1 & 1 \\
  -1 & 0 & -1 \\
  0 & -1 & -1
\end{bmatrix}=\begin{bmatrix}
                2 & 0&2 \\
                0 & 2&2\\
                2& 2&4
              \end{bmatrix}
\end{equation*}

(3)
\begin{equation*}
|\lambda E-B|=
\begin{vmatrix}
\lambda-2 & 0 &-2\\
0&\lambda-2& -2\\
-2&-2&\lambda-4
\end{vmatrix}=\lambda(\lambda-2)(\lambda-6)~~~\Rightarrow~~~\lambda_1=0,\lambda_2=2,\lambda_3=3
\end{equation*}

$\lambda_1=0$时:
\begin{equation*}
  [\lambda E-A]=
  \begin{vmatrix}
-2 & 0 &-2\\
0&-2& -2\\
-2&-2&-4
  \end{vmatrix}~~~\Rightarrow~~~
  \begin{cases}
   x_1=-x_3\\
   x_2=-x_3
  \end{cases}
\end{equation*}
取$x_3=1$得:$\alpha_1=[1,1,-1]^T$.

$\lambda_2=2$时:
\begin{equation*}
  [\lambda E-A]=
  \begin{vmatrix}
0 & 0 &-2\\
0&0& -2\\
-2&-2&-2
  \end{vmatrix}~~~\Rightarrow~~~
  \begin{cases}
   x_1=-x_2\\
   x_3=0
  \end{cases}
\end{equation*}
取$x_2=1$得:$\alpha_2=[1,-1,0]^T$.

$\lambda_3=6$时:
\begin{equation*}
  [\lambda E-A]=
  \begin{vmatrix}
4 & 0 &-2\\
0&4& -2\\
-2&-2&2
  \end{vmatrix}~~~\Rightarrow~~~
  \begin{cases}
   x_1=\frac{1}{2}x_3\\
   x_2=\frac{1}{2}x_3
  \end{cases}
\end{equation*}
取$x_3=2$得:$\alpha_3=[1,1,2]^T$.因为对称矩阵对应于不同特征值的特征向量正交,所以$[\alpha_1,\alpha_2,\alpha_3]$为正交向量组。

单位化:
\begin{equation*}
\begin{cases}
\gamma_1=\dfrac{\alpha_1}{\|\alpha_1\|}=\left[\dfrac{1}{\sqrt{3}},\dfrac{1}{\sqrt{3}},-\dfrac{1}{\sqrt{3}}\right]^T\\[2mm]
\gamma_2=\dfrac{\alpha_2}{\|\alpha_2\|}=\left[\dfrac{1}{\sqrt{2}},-\dfrac{1}{\sqrt{2}},0\right]^T\\[2mm]
\gamma_3=\dfrac{\alpha_3}{\|\alpha_3\|}=\left[\dfrac{1}{\sqrt{6}},\dfrac{1}{\sqrt{6}},\dfrac{2}{\sqrt{6}}\right]^T
\end{cases}~~~\Rightarrow~~~Q=
\begin{bmatrix}
\dfrac{1}{\sqrt{3}}&\dfrac{1}{\sqrt{2}}&\dfrac{1}{\sqrt{6}}\\[2mm]
\dfrac{1}{\sqrt{3}}&-\dfrac{1}{\sqrt{2}}&\dfrac{1}{\sqrt{6}}\\[2mm]
-\dfrac{1}{\sqrt{3}}&0&\dfrac{2}{\sqrt{6}}
\end{bmatrix}
\end{equation*}
所以$f$可经正交变换$x=Qy$化为标准型:
\begin{equation*}
  f=2y_{2}^2+6y_{3}^2
\end{equation*}
\end{jie}

\num 期末2015-2016 四1.

设$A$为$n$阶实对称矩阵,且满足$A^{2}-3A+2E=0$,其中$E$为单位矩阵,试证:

%(1)$A+2E$可逆;

(2)$A$为正定矩阵。

\begin{zhengming}
对于$n$阶实对称矩阵,如果$A$为正定矩阵,则$A$的全部特征值大于0.
设$A$的特征值为$\lambda$。由题得:
\begin{equation*}
\lambda^2-3\lambda+2=0~~~\Rightarrow~~~\lambda_1=1>0,\lambda_2=2>0
\end{equation*}
所以$A$为正定矩阵。
\end{zhengming}

\num 期末2016-2017 一6.

设
$
B=
\begin{bmatrix}
  1 & 2 & 4 \\
  0 & 2 & 6\\
  0 & 0 & \lambda
\end{bmatrix}
$,已知二次型$f(x)=x^{T}Bx$是正定的,则$\lambda$的取值范围为\underline{\hphantom{~~~~~~~~~~}}。

\begin{jie}
由题得:
\begin{equation*}
f(x)=x^{T}Bx=[x_1,x_2,x_3]\begin{bmatrix}
  1 & 2 & 4 \\
  0 & 2 & 6\\
  0 & 0 & \lambda
\end{bmatrix}
\begin{bmatrix}
x_1\\ x_2\\ x_3
\end{bmatrix}=\begin{bmatrix}
x_1&  2x_1+2x_2& 4x_1+6x_2+\lambda x_3
\end{bmatrix}\begin{bmatrix}
x_1\\ x_2\\ x_3
\end{bmatrix}=x_1^2+2x_2^2+\lambda x_3^2+2x_1x_2+4x_1x_3+6x_2x_3
\end{equation*}
所以其对应的二次型矩阵为:
$A=\begin{bmatrix}
  1 & 1 & 2 \\
  1 & 2 & 3\\
  2 & 3 & \lambda
\end{bmatrix}$,$A$为对称矩阵,如果$A$正定,则$|A|>0$,所以$|A|=\lambda-5>0~~\Rightarrow~~\lambda>5$.
\end{jie}

\num 期末2016-2017 三2.

已知实对称矩阵$A=
\begin{bmatrix}
  a & -1 & 4 \\
  -1 & 3 & b\\
  4 & b & 0
\end{bmatrix}
$与
$B=
\begin{bmatrix}
  2 & ~ & ~ \\
  ~ & -4 & ~\\
  ~ & ~ & 5
\end{bmatrix}
$相似。

(1)求矩阵$A$;

(2)求正交线性变换$x=Qy$,把二次型$f(x)=x^{T}Ax$化为标准型.

\begin{jie}
对于对角矩阵,其特征值为对角线上的元素。因为$A$与$B$相似,所以$A$与$B$有相同的特征值。

(1)由特征值的性质
\begin{equation*}
\begin{cases}
trace(A)=a+3+0=\sum\limits_{i=1}^{3}\lambda_i=2-4+5=3\\
|A|=(-1)\times(-1)^{1+2}
\begin{vmatrix}
  -1 & b\\
  4 & 0
\end{vmatrix}+4
\begin{vmatrix}
  -1 & 3\\
  4  & b
\end{vmatrix}=-8b-48=\prod\limits_{i=1}^{3}\lambda_i=2\times(-4)\times5=-40
\end{cases}~~\Rightarrow~~\begin{cases}a=0\\ b=-1\end{cases}
\end{equation*}

(2)

$\lambda_1=2$时:
\begin{equation*}
[\lambda E-A]=
\begin{bmatrix}
  2 & 1 & -4\\
  1 & -1 &1\\
  -4 & 1 & 2
\end{bmatrix}~~~\Rightarrow~~~
\begin{cases}
 x_1=x_3\\
 x_2=2x_3
\end{cases}
\end{equation*}
取$x_3=1$得$\alpha_1=[1,2,1]^T$。

$\lambda_2=-4$时:
\begin{equation*}
[\lambda E-A]=
\begin{bmatrix}
  -4 & 1 & -4\\
  1 & -7 &1\\
  -4 & 1 & -4
\end{bmatrix}~~~\Rightarrow~~~
\begin{cases}
 x_1=-x_3\\
 x_2=0
\end{cases}
\end{equation*}
取$x_3=-1$得$\alpha_2=[1,0,-1]^T$。

$\lambda_3=5$时:
\begin{equation*}
[\lambda E-A]=
\begin{bmatrix}
  5 & 1 & -4\\
  1 & 2 &1\\
  -4 & 1 & 5
\end{bmatrix}~~~\Rightarrow~~~
\begin{cases}
 x_1=x_3\\
 x_2=-x_3
\end{cases}
\end{equation*}
取$x_3=1$得$\alpha_2=[1,-1,1]^T$。因为对称矩阵对应于不同特征值的特征向量正交,所以$[\alpha_1,\alpha_2,\alpha_3]$为正交向量组。

单位化:
\begin{equation*}
\begin{cases}
\gamma_1=\dfrac{\alpha_1}{\|\alpha_1\|}=\left[\dfrac{1}{\sqrt{6}},\dfrac{2}{\sqrt{6}},\dfrac{1}{\sqrt{6}}\right]^T\\[2mm]
\gamma_2=\dfrac{\alpha_2}{\|\alpha_2\|}=\left[\dfrac{1}{\sqrt{2}},0,-\dfrac{1}{\sqrt{2}}\right]^T\\[2mm]
\gamma_3=\dfrac{\alpha_3}{\|\alpha_3\|}=\left[\dfrac{1}{\sqrt{3}},-\dfrac{1}{\sqrt{3}},\dfrac{1}{\sqrt{3}}\right]^T
\end{cases}~~~\Rightarrow~~~Q=
\begin{bmatrix}
\dfrac{1}{\sqrt{6}}&\dfrac{1}{\sqrt{2}}&\dfrac{1}{\sqrt{3}}\\[2mm]
\dfrac{2}{\sqrt{6}}&0&-\dfrac{1}{\sqrt{3}}\\[2mm]
\dfrac{1}{\sqrt{6}}&-\dfrac{1}{\sqrt{2}}&\dfrac{1}{\sqrt{3}}
\end{bmatrix}
\end{equation*}
所以$f$可经正交变换$x=Qy$化为标准型:
\begin{equation*}
  f=2y_{1}^2-4y_{2}^2+5y_{3}^2
\end{equation*}
\end{jie}

\num 期末2016-2017 四2.

已知$A,B$是同阶实对称矩阵。

(1)证明如果$A\~{}B$,则$A\simeq B$,也就是相似一定合同;

(2)举例说明反过来不成立。

\begin{zhengming}
(1)因为$A\~{}B$,所以$A,B$具有相同的特征值,记为$\lambda_i,(1\leq i\leq n)$。对于实对称矩阵$A$存在正交矩阵$Q$,使得$Q^{-1}AQ$为对角矩阵。即存在正交矩阵$Q_{1}$,使得$Q_1^{-1}AQ_1=\Lambda=
\begin{bmatrix}
  \lambda_1 & & \\
   & \ddots &\\
   &&\lambda_n
\end{bmatrix}
$,对于正交矩阵$Q_1$,有$Q_{1}^{-1}=Q_1^T$,即$Q_1^{T}AQ_1=\Lambda$,所以$A$合同于$\Lambda$,同理$B$合同于$\Lambda$,所以$A$合同于$B$。

(2)反过来描述:$A,B$是同阶实对称矩阵,$A\simeq B$,则$A\~{}B$。

由惯性定理(\textcolor[rgb]{1.00,0.00,0.00}{157页})知:
如果:$A=
\begin{bmatrix}
  1 &   \\
    & 2
\end{bmatrix},B=\begin{bmatrix}
  1 &   \\
    & 3
\end{bmatrix}
$,$A$,$B$为对角阵,且$A\simeq B$,但$A$和$B$的特征值不同,即$A$与$B$不相似。
\end{zhengming}

\num 期末2017-2018 一6.

设二次型$f(x_{1},x_{2},x_{3})=2x_{1}^{2}+x_{2}^{2}+x_{3}^{2}+2x_{1}x_{2}+2tx_{2}x_{3}$的秩为2,则$t=$\underline{\hphantom{~~~~~~~~~~}}。

\begin{jie}
二次型对应的矩阵为:
\begin{equation*}
A=
\begin{bmatrix}
  2 & 1 & 0\\
  1 & 1 & t\\
  0 & t &1
\end{bmatrix}
\end{equation*}
$A$的秩为2,即$|A|=0$,解得:$t=\pm\dfrac{1}{\sqrt{2}}$
\end{jie}

\num 期末2017-2018 三2.

设$f(x_ {1},x_{2},x_{3})=2x_{1}^{2}+2x_{2}^{2}+3x_{3}^{2}+2x_{1}x_{2}$。

(1)写出该二次型的矩阵$A$;

(2)求正交矩阵$Q$使得$Q^{T}AQ=Q^{-1}AQ$为对角型矩阵;

(3)给出正交变换,化该二次型为标准型。

\begin{jie}
(1)由题得:
\begin{equation*}
A=
\begin{bmatrix}
  2 & 1 & 0 \\
  1  & 2 & 0\\
  0 & 0 & 2
\end{bmatrix}
\end{equation*}

(2)
\begin{equation*}
|\lambda E-A|=
\begin{vmatrix}
  \lambda-2 & -1 & 0 \\
  -1  & \lambda-2 & 0\\
  0 & 0 & \lambda-2
\end{vmatrix}=(\lambda-2)[(\lambda-2)^2-1]=0~~~\Rightarrow ~~~\lambda_1=1,\lambda_2=2,\lambda_3=3
\end{equation*}

$\lambda_1=1$时:
\begin{equation*}
[\lambda E-A]=
\begin{bmatrix}
  -1 & -1 & 0 \\
  -1  & -1 & 0\\
  0 & 0 & -1
\end{bmatrix}~~~\Rightarrow~~~
\begin{cases}
 x_1=-x_2\\
 x_3=0
\end{cases}
\end{equation*}
取$x_2=-1$得$\alpha_1=[1,-1,0]^T$。

$\lambda_2=2$时:
\begin{equation*}
[\lambda E-A]=
\begin{bmatrix}
  0 & -1 & 0 \\
  -1  & 0 & 0\\
  0 & 0 & 0
\end{bmatrix}~~~\Rightarrow~~~
\begin{cases}
 x_1=x_2=0\\
 x_3\in R
\end{cases}
\end{equation*}
取$x_3=1$得$\alpha_2=[0,0,1]^T$。

$\lambda_3=3$时:
\begin{equation*}
[\lambda E-A]=
\begin{bmatrix}
  1 & -1 & 0 \\
  -1  & 1 & 0\\
  0 & 0 & 1
\end{bmatrix}~~~\Rightarrow~~~
\begin{cases}
 x_1=x_2\\
 x_3=0
\end{cases}
\end{equation*}
取$x_2=1$得$\alpha_3=[1,1,0]^T$。
因为对称矩阵对应于不同特征值的特征向量正交,所以$[\alpha_1,\alpha_2,\alpha_3]$为正交向量组。

单位化:
\begin{equation*}
\begin{cases}
\gamma_1=\dfrac{\alpha_1}{\|\alpha_1\|}=\left[\dfrac{1}{\sqrt{2}},-\dfrac{1}{\sqrt{2}},0\right]^T\\[2mm]
\gamma_2=\dfrac{\alpha_2}{\|\alpha_2\|}=\left[0,0,1\right]^T\\[2mm]
\gamma_3=\dfrac{\alpha_3}{\|\alpha_3\|}=\left[\dfrac{1}{\sqrt{2}},\dfrac{1}{\sqrt{2}},0\right]^T
\end{cases}~~~\Rightarrow~~~Q=
\begin{bmatrix}
\dfrac{1}{\sqrt{2}}&0&\dfrac{1}{\sqrt{2}}\\[2mm]
-\dfrac{1}{\sqrt{2}}&0&\dfrac{1}{\sqrt{2}}\\[2mm]
0&1&0
\end{bmatrix}
\end{equation*}
所以$f$可经正交变换$x=Qy$化为标准型:
\begin{equation*}
  f=y_{1}^2+2y_{2}^2+3y_{3}^2
\end{equation*}
\end{jie}

\num 期末2018-2019 一5.

若二次型$f(x_ {1},x_{2},x_{3})=x_{1}^{2}+4x_{2}^{2}+4x_{3}^{2}+2tx_{1}x_{2}-2x_{1}x_{3}+4x_{2}x_{3}$正定,则$t$应满足\underline{\hphantom{~~~~~~~~~~}}。

\begin{jie}
二次型矩阵$A=
\begin{bmatrix}
  1 & t & -1 \\
 t & 4 & 2\\
 -1 &2 &4
\end{bmatrix}
$,二次型正定,即$A$正定,即$A$的所有顺序主子是大于0.即
\begin{align*}
&D_1=1\\
&D_2=
\begin{vmatrix}
  1 & t\\
  t & 4
\end{vmatrix}=4-t^2>0~~~\Rightarrow ~~~-2<t<2\\
&D_3=\begin{vmatrix}
  1 & t & -1 \\
 t & 4 & 2\\
 -1 &2 &4
     \end{vmatrix}=8-4t-4t^2>0~~~-2<t<1
\end{align*}
综上所述:$-2<t<1$.
\end{jie}

\num 期末2018-2019 三2.

设实二次型$f(x_{1},x_{2},x_{3})=4x_{1}x_{2}-4x_{1}x_{3}+4x_{2}^{2}+8x_{2}x_{3}-3x_{3}^{2}$。

(1)写出该二次型的矩阵$A$;

(2)求正交矩阵$P$,使得$P^{-1}AP$为对角型矩阵;

(3)给出正交变换,将该二次型化为标准型;

(4)写出二次型的秩,正惯性指标和负惯性指标。

\begin{jie}
(1)由题得:
\begin{equation*}
  A=\begin{bmatrix}
      0 & 2 & -2 \\
      2& 4 & 4\\
      -2 & 4 & 3
    \end{bmatrix}
\end{equation*}

(2)
\begin{align*}
|\lambda E-A|=
\begin{vmatrix}
  \lambda & -2 & 2 \\
      -2& \lambda-4 & -4\\
      2 & -4 & \lambda-3
\end{vmatrix}=(\lambda-1)(\lambda^2-36)=0~~~\Rightarrow~~~\lambda_1=1,\lambda_2=6,\lambda_3=-6
\end{align*}

$\lambda_1=1$时:
\begin{equation*}
[\lambda E-A]=
\begin{bmatrix}
 1 & -2 & 2 \\
      -2& -3 & -4\\
      2 & -4 & -2
\end{bmatrix}~~~\Rightarrow~~~
\begin{cases}
 x_1=-2x_3\\
 x_2=0
\end{cases}
\end{equation*}
取$x_3=1$得$\alpha_1=[-2,0,1]^T$。

$\lambda_2=6$时:
\begin{equation*}
[\lambda E-A]=
\begin{bmatrix}
 6 & -2 & 2 \\
      -2& 2 & -4\\
      2 & -4 & 9
\end{bmatrix}~~~\Rightarrow~~~
\begin{cases}
 x_1=\dfrac{1}{2}x_3\\[2mm]
 x_3=\dfrac{5}{2}x_3
\end{cases}
\end{equation*}
取$x_3=2$得$\alpha_2=[1,5,2]^T$。

$\lambda_3=-6$时:
\begin{equation*}
[\lambda E-A]=
\begin{bmatrix}
 -6 & -2 & 2 \\
      -2&-10 & -4\\
      2 & -4 &-3
\end{bmatrix}~~~\Rightarrow~~~
\begin{cases}
 x_1=\dfrac{1}{2}x_3\\[2mm]
 x_3=-\dfrac{1}{2}x_3
\end{cases}
\end{equation*}
取$x_3=2$得$\alpha_3=[1,-1,2]^T$。
因为对称矩阵对应于不同特征值的特征向量正交,所以$[\alpha_1,\alpha_2,\alpha_3]$为正交向量组。

单位化:
\begin{equation*}
\begin{cases}
\gamma_1=\dfrac{\alpha_1}{\|\alpha_1\|}=\left[-\dfrac{2}{\sqrt{5}},0,-\dfrac{1}{\sqrt{5}}\right]^T\\[2mm]
\gamma_2=\dfrac{\alpha_2}{\|\alpha_2\|}=\left[\dfrac{1}{\sqrt{30}},\dfrac{5}{\sqrt{30}},\dfrac{2}{\sqrt{30}}\right]^T\\[2mm]
\gamma_3=\dfrac{\alpha_3}{\|\alpha_3\|}=\left[\dfrac{1}{\sqrt{6}},-\dfrac{1}{\sqrt{6}},\dfrac{2}{\sqrt{6}}\right]^T
\end{cases}~~~\Rightarrow~~~P=
\begin{bmatrix}
-\dfrac{2}{\sqrt{5}}&\dfrac{1}{\sqrt{30}}&\dfrac{1}{\sqrt{6}}\\[2mm]
0&\dfrac{5}{\sqrt{30}}&-\dfrac{1}{\sqrt{6}}\\[2mm]
\dfrac{1}{\sqrt{5}}&\dfrac{2}{\sqrt{30}}&\dfrac{2}{\sqrt{6}}
\end{bmatrix}
\end{equation*}

所以$P$即为所求的正交矩阵,$P^{-1}AP=\Lambda=
\begin{bmatrix}
  1 & & \\
    & 6&\\
    &&-6
\end{bmatrix}$

(3)由(2)得:$f$可经正交变换$x=Py$化为标准型:
\begin{equation*}
  f=y_{1}^2+6y_{2}^2-6y_{3}^2
\end{equation*}

(4)计算得$|A|=
-2\begin{vmatrix}
    2 & -2 \\
   4 & -3
  \end{vmatrix}-2\begin{vmatrix}
    2 & -2 \\
   4 & 4
  \end{vmatrix}=-36\neq0
$,所以二次型满秩,即$r(A)=3$。由$(3)$得,正惯性指标为$2$,负惯性指标为$1$。
\end{jie}

\num 期末2019-2020 一6.

已知实对称矩阵$A=
\begin{bmatrix}
  2 & 0 & 1 \\
  0 & 3 & 3\\
  1 & 3 & x
\end{bmatrix}
$的正惯性指数为3,则$x$的取值范围为\underline{\hphantom{~~~~~~~~~~}}。

\begin{jie}
$A$为实对称矩阵,且$A$的正惯性指数为3,所以$A$正定,所以$A$的所有顺序主子式大于0.所以$|A|=2(3x-9)-3>0~~~\Rightarrow~~~x>3.5$
\end{jie}

\num 期末2019-2020 三3.

设三元二次型$f(x_{1},x_{2},x_{3})=4x_{2}^{2}+4x_{3}^{2}-2x_{1}x_{2}+4x_{1}x_{3}$.

(1)写出该二次型的矩阵$A$;

(2)用正交变换$x=Qy$把该二次型化为标准型。

\begin{jie}
(1)由题得:
\begin{equation*}
  A=
  \begin{bmatrix}
    0 & -1 & 2\\
    -1 & 4 & 0\\
    2 & 0 & 4
  \end{bmatrix}
\end{equation*}

(2)
\begin{align*}
|\lambda E-A|=
\begin{vmatrix}
    \lambda & 1 & -2\\
    1 & \lambda-4 & 0\\
    -2 & 0 & \lambda-4
\end{vmatrix}=-2[2(\lambda-4)]+(\lambda-4)[\lambda(\lambda-4)-1]=(\lambda-4)(\lambda-5)(\lambda+1)~~~\Rightarrow ~~~\lambda_1=4,\lambda_2=5,\lambda_3=-1
\end{align*}

$\lambda_1=4$时:
\begin{equation*}
[\lambda E-A]=
\begin{bmatrix}
  4 & 1 & -2\\
    1 & 0 & 0\\
    -2 & 0 & 0
\end{bmatrix}~~~\Rightarrow~~~
\begin{cases}
 x_1=0\\
 x_2=2x_3
\end{cases}
\end{equation*}
取$x_3=1$得$\alpha_1=[0,2,1]^T$。

$\lambda_2=5$时:
\begin{equation*}
[\lambda E-A]=
\begin{bmatrix}
  5 & 1 & -2\\
    1 & 1 & 0\\
    -2 & 0 & 1
\end{bmatrix}~~~\Rightarrow~~~
\begin{cases}
 x_1=\dfrac{1}{2}x_3\\[2mm]
 x_2=-\dfrac{1}{2}x_3
\end{cases}
\end{equation*}
取$x_3=2$得$\alpha_2=[1,-1,2]^T$.

$\lambda_3=-1$时:
\begin{equation*}
[\lambda E-A]=
\begin{bmatrix}
  -1 & 1 & -2\\
    1 & -5 & 0\\
    -2 & 0 & -5
\end{bmatrix}~~~\Rightarrow~~~
\begin{cases}
 x_1=-\dfrac{5}{2}x_3\\[2mm]
 x_2=-\dfrac{1}{2}x_3
\end{cases}
\end{equation*}
取$x_3=-2$得$\alpha_3=[5,1,-2]^T$.因为对称矩阵对应于不同特征值的特征向量正交,所以$[\alpha_1,\alpha_2,\alpha_3]$为正交向量组。

单位化:
\begin{equation*}
\begin{cases}
\gamma_1=\dfrac{\alpha_1}{\|\alpha_1\|}=\left[0,\dfrac{2}{\sqrt{5}},\dfrac{1}{\sqrt{5}}\right]^T\\[2mm]
\gamma_2=\dfrac{\alpha_2}{\|\alpha_2\|}=\left[\dfrac{1}{\sqrt{6}},-\dfrac{1}{\sqrt{6}},\dfrac{2}{\sqrt{6}}\right]^T\\[2mm]
\gamma_3=\dfrac{\alpha_3}{\|\alpha_3\|}=\left[\dfrac{5}{\sqrt{30}},\dfrac{1}{\sqrt{30}},-\dfrac{2}{\sqrt{30}}\right]^T
\end{cases}~~~\Rightarrow~~~Q=
\begin{bmatrix}
0&\dfrac{1}{\sqrt{6}}&\dfrac{5}{\sqrt{30}}\\[2mm]
\dfrac{2}{\sqrt{5}}&-\dfrac{1}{\sqrt{6}}&\dfrac{1}{\sqrt{30}}\\[2mm]
\dfrac{1}{\sqrt{5}}&\dfrac{2}{\sqrt{6}}&-\dfrac{2}{\sqrt{30}}
\end{bmatrix}
\end{equation*}
所以$f$可经正交变换$x=Qy$化为标准型:
\begin{equation*}
  f=4y_{1}^2+5y_{2}^2-y_{3}^2
\end{equation*}
\end{jie}

\num 期末2019-2020 四1.

设$A$为$m$阶正定矩阵,$B$为$m\times n$实矩阵,$B^{T}$为$B$的转置矩阵,试证:$B^{T}AB$为正定矩阵的充分必要条件是$B$的秩$r(B)=n$。

\begin{jie}
必要性:如果$B^TB$正定,则存在任意非零实列向量$x\neq 0$,使得$x^TB^TBx>0$,即$(Bx)^TA(Bx)>0$,所以$Bx\neq 0$。所以$Bx=0$只有零解,即$r(B)=n$。

充分性:如果$B$的秩为$r(B)=n$,则线性方程组$Bx=0$只有零解,所以存在任意非零实列向量$x$,使得$Bx\neq 0$。又因为$A$为正定矩阵,由正定矩阵的定义得:$(Bx)^TABx>0$,即$x^TB^TABx=x^T(B^TAB)x>0$。
因为$x$为任意非零实列向量,所以依正定矩阵的定义,矩阵$(B^TAB)$正定。
\end{jie}

\num 期末2019-2020 四2.

设$\alpha,\beta$是$n$维列向量,证明$r(\alpha\alpha^{T}+\beta\beta^{T})\leq 2$。

\begin{zhengming}
由秩的性质:
\begin{equation*}
r(\alpha\alpha^ {T}+\beta\beta^{T})\leq r(\alpha\alpha^ {T})+r(\beta\beta^{T})\leq \min(r(\alpha),r(\alpha^T))+\min(r(\beta),r(\beta^T))\leq 1+1=2
\end{equation*}
\end{zhengming}

\num 期末2017-2018 四2.

设$A$为$m\times n$实矩阵,证明$Ax=0$与$(A^{T}A)x=0$是同解方程,进一步得出$r(A)=r(A^{T}A)$。

\begin{jie}
(1)若$x_0$为$Ax=0$的解,则$Ax_0=0$,对等式两边同时左乘$A^T$:$A^TAx_0=0$,即$x_0$为$A^TAx=0$的解。

若$x_1$为$A^TAx=0$的解:则$A^TAx_1=0$,等式两边同时左乘$x_1^T$:$x_1^TA^TAx_1=(Ax_1)^T(Ax_1)=0$,所以$Ax_1=0$,所以$x_1$为$Ax=0$的解。(注:这里就认为$x$是一个列向量,所以$Ax$也是列向量,用\textcolor[rgb]{1.00,0.00,0.00}{向量内积}的性质。)

综上所述:$Ax=0$与$A^TAx=0$同解。

(2)$Ax=0$与$A^TAx=0$同解,则它们解的空间维数相同。又因为解的空间维数=未知量的个数-系数矩阵的秩。
两个方程的未知数个数相同,所以系数矩阵相同,即$r(A)=r(A^TA)$
\end{jie}

\num 期末2016-2017 三3.

在对观测数据拟合的时候经常遇到线性方程组$Ax=b$是矛盾方程的情形,是没有解的。此时我们转而解$A^{T}Ax=A^{T}b$,我们称$A^{T}Ax=A^{T}b$是原线性方程组的正规方程组。称正规方程组的解为原方程组的最小二乘解。设
$
A=
\begin{bmatrix}
  1 & 1 & 0\\
  1 & 1 & 0\\
  1 & 0 & 1\\
  1 & 1 & 1
\end{bmatrix},b=
\begin{bmatrix}
1 \\ 3 \\ 8 \\ 2
\end{bmatrix}
$.

(1)证明$Ax=b$无解;

(2)求$Ax=b$的最小二乘解。

\begin{jie}
(1)
由题得:
\begin{equation*}
[A|B]\xrightarrow{\substack{r_2-r_1\\r_3-r_1 \\ r_4-r_1}}
{
\begin{bmatrix}
  1 & 1 & 0 & 1\\
  0 & 0 & 0 &2 \\
  0 & -1 & 1 & 7\\
  0 & 0& 1 &1
\end{bmatrix}
}
\end{equation*}
可以看出$r(A)=3\neq r(A|B)=4$,所以$Ax=b$无解。

(2)
\begin{gather*}
  A^TA=\begin{bmatrix}
  1 & 1 & 1& 1\\
  1 & 1 & 0& 1\\
  0 & 0 & 1& 1
\end{bmatrix}\begin{bmatrix}
  1 & 1 & 0\\
  1 & 1 & 0\\
  1 & 0 & 1\\
  1 & 1 & 1
\end{bmatrix}=
\begin{bmatrix}
  4 & 3 & 2\\
  3 & 3 & 1\\
  2 & 1 & 2
\end{bmatrix}\\
A^Tb=\begin{bmatrix}
  1 & 1 & 1& 1\\
  1 & 1 & 0& 1\\
  0 & 0 & 1& 1
\end{bmatrix}\begin{bmatrix}
1 \\ 3 \\ 8 \\ 2
\end{bmatrix}=
\begin{bmatrix}
14 \\ 6 \\ 10
\end{bmatrix}
\end{gather*}
高斯消元步骤略(考试必须写上)。
最后解得:$x_1=8,x_2=-6,x_3=0$
\end{jie}


\end{document}  